\documentclass[12pt]{article}
\usepackage[spanish]{babel}
\usepackage{csquotes}
\usepackage{setspace}
\doublespacing

\begin{document}

\section{Citas}

Según Johnsonbaugh (2005):

\begin{displayquote}
Una \textit{gráfica} (o \textit{gráfica no dirigida}) $G$ consiste en un conjunto $V$ de \textit{vértices} (o \textit{nodos}) y un conjunto $E$ de \textit{aristas} (o \textit{arcos}) tal que cada arista $e \in E$ se asocia con un par no ordenado de vértices. Si existe una arista única $e$ asociada con los vértices $u$ y $w$, se escribe $e = (v,w)$ o $e = (w,v)$. En este contexto, $(v,w)$ denota una arista entre $v$ y $w$ en una gráfica no dirigida y \textit{no} es un par ordenado. [...] Se dice que una arista $e$ en una gráfica (no dirigida o dirigida) que se asocia con el par de vértices $v$ y $w$ es \textit{incidente} sobre $v$ y $w$, y se dice que $v$ y $w$ son \textit{incidentes sobre e} y son \textit{vértices adyacentes}. (p.~320)
\end{displayquote}

\section*{Referencia}
Johnsonbaugh, R. (2005). \textit{Matemáticas discretas} (6.\,ª ed.). Pearson Educación.

\end{document}
