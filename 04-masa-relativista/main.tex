\documentclass[12pt, a4paper]{article}
\usepackage{amsmath}
\usepackage{amssymb}
\usepackage{amsfonts}
\usepackage{graphicx}
\usepackage[spanish]{babel}
%\usepackage[right=3cm, left=2.5cm, top=3cm, bottom=3.5cm]{geometry}
%\usepackage[style=apa]{biblatex}
%\addbibresource{references.bib}

\title{Masa Relativista}
\author{Torres Tarrillo Rober Esbel}

\begin{document}

\maketitle
\section{ Introducción }
Hay dos formas de medir la masa de un cuerpo: poniendo el cuerpo en un
campo gravitatorio y pesarlo, a la cual se denomina masa gravitatoria;
o aplicando una fuerza sobre un cuerpo y midiendo la aceleración que
produce, está es la masa inercial y es en la que nos vamos a centrar.

Para medir esta ultima, se debe medir la aceleración, pero para medir
la aceleración se debe hacer medidas de distancia, velocidades y tiempo,
pero estas medidas dependen dependen del movimiento relativo del observador
y la partícula que se debe medir, por lo que podemos suponer que se obtendrá
una masa inercial diferente para cada sistema de referencia inercial.

Llamaremos \textit{masa propia} o \textit{masa en reposo} a la masa $m_0$
medida en el sistema de referencia de la partícula.

\section{Momento lineal Relativista}
El momento lineal se define como $\Vec{p} = m \Vec{v}$, por la ley de
conservación del momento lineal y el primer postulado de  Einstein, este
debe permanece invariante ante las transformaciones de Lorentz, pero tal
como está definido al hacer dichas transformaciones el momento lineal no
se conserva, por lo que para solucionar este problema podemos redefinir
el momento lineal como
\[\Vec{p} =m\Vec{v} = \gamma m_0 \Vec{v}\]

Para encontrar el valor de $\gamma$ supongamos que dos partículas de la
misma masa en reposo $m'_1 = m'_2$, sufren una colisión elástica en un sistema de
referencia $S'$ como se muestra en la figura~\ref{fig:1}.

\begin{figure}[ht]\label{fig:1}
    \centering
    \includegraphics[scale=0.5]{img/f1.png}
    \caption{colisión en el sistema $S'$.}	
\end{figure}
Del gráfico se tiene que $p'_{0x} = p'_{fx}$ $p'_{0y} = p'_{fy}$,
las componentes $X'$ de las velocidades son iguales y permanecen inalteradas
y las componentes $Y'$, también iguales, cambian de sentido en la colisión.
Para evitar problemas de la medida \textit{simultánea} de los omentos
lineales, consideramos que todas las magnitudes citadas son medidas durante
el impacto y que las partículas no interaccionan a distancia.

Ahora tomemos como sistema $S$ uno respecto del cual $S'$ se desplaza
a velocidad $V=v'_{2x}$, en la dirección común de los ejes $X$ y $X'$,
como en la figura~\ref{fig:2}
\begin{figure}[ht]\label{fig:2}
    \centering
    \includegraphics[scale=0.5]{img/f2.png}
    \caption{colisión en el sistema $S$.}	
\end{figure}
\begin{equation}
    v_{1x} = \frac{-v'_{1x} + V}{1 + \frac{V}{c^2} (-v'_x)}
\end{equation}
por ser: $V=v'_{2x} = |-v'_{1x}|$; la partícula 1 se mueve perpendicularmente
al eje $X$. Para la partícula 2, se puede comprobar mediante
las relaciones de transformación de velocidades que $v_{2y} < v'_{2y}$,
por tanto, el observador $S$ ve la trayectoria de 2 más achatada hacia el eje
$X$ de lo que la ve $S'$ respecto de $X'$.

La conservación del momento lineal en el eje $X$ es evidente. En el eje $Y$
existe las siguientes variaciones de $\Vec{P}$:

\begin{gather*}
    \delta p_1 = -m_1v_{1y} - (m_1v_{1y}) = -2m_1v_{1y}\\
    \delta p_2 = m_2v_{2y} - (-m2v_{2y}) = 2m_2v_{2y} 
\end{gather*}

La conservación de $p_y$ implica: $m_1v_{1y} = m_2v_{2y}$. Como vamos a ver,
esas dos velocidades son distintas, por lo tanto a las dos partículas se les debe
medir distinta masa en $S$, a pesar de tener la misma masa en reposo.
Usando las expresiones para la transformación de velocidades, podemos poner:
\begin{gather*}
    v_{1y}=\frac{v'_{1y}}{\sqrt{1-\beta^2}} \quad \text{y} \quad
    v_{2y}=v'_{2y}\frac{1-\frac{v_{2x}v'{2x}}{c^2}}{\sqrt{1-\beta^2}}
\end{gather*}

como $v'_{1y} = v'_{2y}$, en valor absoluto, se tiene
\begin{gather*}
    \frac{m_2}{m_1} = \frac{1}{1 - \frac{v_{2x}v'_{2x}}{c^2}}
\end{gather*}

Por otra parte
\begin{gather*}
    v'_{2x} = V = \frac{v_{2x} - V}{1 - \frac{V}{c^2}v_{2x}}
    \Rightarrow V - \frac{V^2v_{2x}}{c^2} = v_{2x} -V
    \Rightarrow v_{2x} = 2V - V^2 \frac{v_{2x}}{x^2}
\end{gather*}

Multiplicando ambos miembros por $v_2x$, dividiendo por $c^2$ y restando $1$
resulta
\begin{gather*}
    1-\frac{v_{2x}^2}{c^2} = {\left(1  - \frac{Vv_{2x}^2}{c^2}\right)}^2
\end{gather*}

como $V = v'_{2x}$, se tiene
\[1 - \frac{v'_{2x}v_{2x}}{c^2} = \sqrt{1 - \frac{v^2_{2x}}{c^2}}\]

con sto ultimo tenemos
\[\frac{m_2}{m_1} = \frac{1}{\sqrt{1 - \frac{v^2_{2x}}{c^2}}}\]

ahora si hacemos $v_1y = 0 \Rightarrow v_2y = 0$. Así
que, en este caso tenemos $m_1 = m_{01} = m_{02} = m_0, v_{2y}=0 \wedge v_2 = v_{2x}$,
la partícula 2 se mueve paralela al eje $X$ y para ella la expresión anterior es
$m_2 = \frac{m_{02}}{1 - \frac{v^2_2}{c^2}}$

dejando los subindices tenemos
\[m = \frac{m_0}{\sqrt{1 - \frac{v^2}{c^2}}} = \gamma m_0\]

En esta ultima expresión $m = m_{rel}$ es la masa relativista, con
esto el momento lineal $\Vec{p} = \gamma m \Vec{v}$ es invariante
ante las transformaciones de Lorentz.

\section{Problemas de la masa Relativista}
Hay muchos detractores al considerar la masa relativista, dado que
no sería una masa como tal, por ejemplo en la segunda ley de Newton
se tiene que $\Vec{F} = m\Vec{a}$, tomando la masa relativista
$\Vec{F} = m_{rel}\Vec{a}$, pero esto no es así, dado que si tomamos
la fuerza como la variación del momento lineal se tiene

\begin{gather*}
    \vec{F} = \frac{d}{dt} \Vec{p} = \frac{d}{dt} \gamma m_0\Vec{v}
    = {\left(\frac{d}{dt}\gamma\right)} m_0\Vec{v} + \gamma m_0\frac{d}{dt}\Vec{v}
\end{gather*}

Claramente $\Vec{F} \neq m_{rel}\Vec{a}$. Cuando $v \to 0, \gamma \to 1$
y se cumple que $F = m_0\Vec{a}$, pero cuanto $v$ tiende a ser muy
grande esto ya no se cumple. Podemos interpretar esto a medida
que la velocidad se esta haciendo mas grande, la masa inercial aumenta, por
lo que debemos aplicar una fuerza cada vez mayor para poder tener
una aceleración constante.

\section*{Referencias}
Burbano, S., Enrique, B. y Gracia, C. (s.f). Física General. Madrid: Tebar.

Young, H. y Freedman, A. (2013). Física universitaria con física moderna(13ava Ed, Vol. 2). México: PEARSON.

\end{document}

