\documentclass[12pt, a4paper]{article}
\usepackage{amsmath}
\usepackage{amssymb}
\usepackage[utf8]{inputenc}    % Soporte para caracteres especiales
\usepackage{amsfonts}
\usepackage[right=3cm, left=2.5cm, top=3cm, bottom=3.5cm]{geometry}
\usepackage{graphicx}
%\usepackage{hyperref}          % Enlaces y referencias
%\usepackage[style=apa]{biblatex}
%\usepackage{csquotes}          % Manejo de citas (opcional, útil para biblatex)
%\addbibresource{references.bib}
%\usepackage[spanish]{babel}
\title{Ejercicio}
\author{Torres Tarrillo, María Licela}

\begin{document}

\pagestyle{empty}

Cierta población de pacientes en tratamiento tiene un tamaño
inicial de 100 personas. Cada persona representa una unidad de atención,
y el número de pacientes cambia con una tasa de crecimiento natural
específica de 0.25 (es decir, debido a nuevos ingresos o mejores), en el tiempo medido en años.

Simultáneamente, los pacientes son dados de alta (o egresados del programa) a una tasa constante de $h$
personas por año.

El número de personas bajo seguimiento $y(t)$ satisface la siguiente ecuación diferencial
y condiciones iniciales:

\begin{equation}
    \label{eq:problem}
    \frac{dy}{dt} = 0.25y - h, \qquad y(0) = 100
\end{equation}

Determine $y(t)$ la cantidad de pacientes bajo el seguimiento de el tiempo, para los siguientes casos:
(a) $h = 20$, (b) $h = 25$, (c) $h = 30$.


\vspace{12pt}
\textit{Solución:} Analizando el problema se tiene lo siguiente:
\begin{itemize}
    \item $t$ es el tiempo medido en años.
    \item $y(t)$ es la cantidad de pacientes bajo el seguimiento de el tiempo.
    \item $h$ cantidad de pacientes dados de alta por año.
    \item $dy/dt$ es la tasa de crecimiento de pacientes.
    \item $y(0)$ es la cantidad de pacientes iniciales.
\end{itemize}

Iniciemos por resolver la ecuación diferencial~\ref{eq:problem}, esta es una ecuación
diferencial de primer orden que tiene la siguiente forma

\begin{equation}
    \label{eq:lineal}
    \frac{dy}{dt} + p(t)y = q(t)
\end{equation}

tomando la ecuación~\ref{eq:problem} y dando forma tenemos:

\begin{equation}
    \label{eq:lineal:problem}
    \frac{dy}{dt} - 0.25y = -h
\end{equation}

Comparando las ecuaciones~\ref{eq:lineal} y~\ref{eq:lineal:problem} tenemos:

\begin{equation}
    \label{eq:lineal:coefficients}
    p(t) = -0.25, \qquad q(t) = -h
\end{equation}

La solución a la ecuación diferencial~\ref{eq:lineal} es:
\begin{equation}
    \label{eq:lineal:solution}
    y(t) = \frac{1}{I(t)}\left[
        \int I(t)q(t) dt + c 
    \right]
\end{equation}

donde $I(t)$ es el factor integrante y está dado por

\begin{equation}
    I(t) = e^{\int p(t) dt}
\end{equation}

por lo que usando~\ref{eq:lineal:coefficients} tenemos: 

\begin{equation}
    \label{eq:lineal:solution:coefficients}
    I(t) = e^{\int -0.25 dt} = e^{-0.25t}
\end{equation}

Remplazando en la ecuación~\ref{eq:lineal:solution} tenemos:

\begin{align*}
    \label{eq:lineal:solution:final}
    y(t) = \frac{1}{e^{-0.25t}}\left[
        \int e^{-0.25t}(-h) dt + c 
    \right]
    =e^{0.25t}
    \left[
        \frac{-h}{-0.25}e^{-0.25t} + c
    \right]
\end{align*}

Expandiendo los términos tenemos

\begin{equation}
    \label{eq:sol:const}
    \Rightarrow y(t)=\frac{h}{0.25} + c e^{0.25t}
\end{equation}

Ahora tenemos que encontrar el valor de la constante, para eso usamos el hecho que
$y(0)=100$, de aquí tenemos qué

\begin{equation}
    y(0) = \frac{h}{0.25} + c e^{0.25(0)} = \frac{h}{0.25} + c = 100
\end{equation}

De la ultima igualdad de la derecha tenemos

\begin{equation}
    \label{eq:const}
    c = 100 - \frac{h}{0.25}
\end{equation}

Remplazando está ultima ecuación en la ecuación~\ref{eq:sol:const} tenemos:
\begin{equation}
    \label{eq:sol}
    y(t) = \frac{h}{0.25} + \left(100 - \frac{h}{0.25}\right) e^{0.25t}
\end{equation}

Ahora remplacemos para cada uno de los casos: (a) $h = 20$:

\begin{equation*}
    \label{eq:sol:a}
    y(t) = \frac{20}{0.25} + \left(100 - \frac{20}{0.25}\right) e^{0.25t}
\end{equation*}

Simplificando tenemos

\begin{equation}
    \label{eq:sol:a:final}
    y(t) = 80 + 20e^{0.25t}
\end{equation}

(b) $h = 25$:

\begin{equation*}
    \label{eq:sol:b}
    y(t) = \frac{25}{0.25} + \left(100 - \frac{25}{0.25}\right) e^{0.25t}
\end{equation*}

Simplificando tenemos

\begin{equation}
    \label{eq:sol:b:final}
    y(t) = 100
\end{equation}

(c) $h = 30$:

\begin{equation*}
    \label{eq:sol:c}
    y(t) = \frac{30}{0.25} + \left(100 - \frac{30}{0.25}\right) e^{0.25t}
\end{equation*}

Simplificando tenemos

\begin{equation}
    \label{eq:sol:c:final}
    y(t) = 120 - 20e^{0.25t}
\end{equation}

Analizando cada una de las partes tenemos:

\begin{itemize}
    \item (a) $y(t) = 80 + 20e^{0.25t}$. Esto es cuando se dan de alta 20 pacientes por años
    la cantidad de pacientes crece dado que el $e^{0.25t}$ es mayor a 1 y su coeficiente es mayor que cero.
    \item (b) $y(t) = 100$. Esto es cuando se dan de alta 25 pacientes por año entonces la cantidad de pacientes no crece.
    \item (c) $y(t) = 120 - 20e^{0.25t}$. Esto es cuando se dan de alta 30 pacientes por año entonces la cantidad de pacientes decrece dado que el $e^{0.25t}$ es menor a 1 y su coeficiente es menor que cero.
\end{itemize}

Usando Geogebra podemos gráficar las tres funciones (ver figura~\ref{fig:graph}):

\begin{figure}[h]
    \label{fig:graph}
    \centering
    \includegraphics[width=0.8\textwidth]{1.jpg}
    \caption{Graficando las funciones (color verde)~\ref{eq:sol:a:final}, (color azul)~\ref{eq:sol:b:final} y (color rojo)~\ref{eq:sol:c:final}}
\end{figure}

\end{document}