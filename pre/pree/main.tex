\documentclass[twocolumn, 12pt, a4paper]{article}
\usepackage{amsmath}
\usepackage{amssymb}
\usepackage[utf8]{inputenc}    % Soporte para caracteres especiales
\usepackage{amsfonts}
\usepackage{graphicx}
\usepackage{eso-pic}
\usepackage{float}
\usepackage[right=3cm, left=2.5cm, top=3cm, bottom=3.5cm]{geometry}
%\usepackage{hyperref}          % Enlaces y referencias
%\usepackage[style=apa]{biblatex}
%\usepackage{csquotes}          % Manejo de citas (opcional, útil para biblatex)
%\addbibresource{references.bib}
%\usepackage[spanish]{babel}

\newcommand\BackgroundPic{
    \put(0,0){%
        \parbox[b][\paperheight]{\paperwidth}{%
            \includegraphics[width=\paperwidth,height=\paperheight]{img/back.pdf}%
        }%
    }
}

\newcommand*{\exer}[8]{
    \item #1
    \begin{center}
        \includegraphics[width=0.3\textwidth]{img/f#2.png}
    \end{center}
    \begin{tabular*}{0.4\textwidth}{@{\extracolsep{\fill}}lll}    % 3 columnas sin líneas
                (A) $#3$ #8 & (B) $#4$ #8 & (C) $#5$ #8\\
                (D) $#6$ #8 &  & (E) $#7$ #8
    \end{tabular*}
}

\newcommand{\T}[1]{°#1}

\newcommand*{\ex}[7]{
    \item #1.
        
    \vspace{12pt}
    \begin{tabular*}{0.4\textwidth}{@{\extracolsep{\fill}}lll}    % 3 columnas sin líneas
                (A) $#2$ #7 & (B) $#3$ #7 & (C) $#4$ #7\\
                (D) $#5$ #7 &  & (E) $#6$ #7
    \end{tabular*}
}

\begin{document}
\AddToShipoutPicture*{\BackgroundPic}
    \begin{enumerate}
        \ex{Se tiene 2 litros de agua a 10 °C en un recipiente de capacidad calorífica despreciable. Determine qué cantidad de agua a 100 °C se debe de agregar al recipiente para que la temperatura final de equilibrio sea de 20 °C}{1}{2}{0.25}{1.5}{2.5}{L}
        \exer{El calor que recibe 10 g de un líquido hace que su
        temperatura cambie del modo que se indica en el gráfico
        Q versus T. Se pide encontrar el valor de su calor
        específico en cal/g°C.}{1}{0.2}{0.25}{0.3}{0.4}{0.7}{ }
        \ex{ En un recipiente de capacidad calorífica despreciable,
        se mezclan 20; 30 y 50g de agua a 80°C, 50°C y 10°C
        respectivamente. Hallar la temperatura de equilibrio}{31}{21}{30}{36}{69}{°C}
        \exer{Una muestra de mineral de 10 g de masa recibe calor
        de modo que su temperatura tiene un comportamiento
        como el mostrado en la figura. Determinar los calores
        latentes específicos de fusión y vaporización en cal/g}{2}{3;8}{10;15}{8;15}{6;15}{7;10}{ }
        \ex{Se tiene 5 g de hielo a -10°C, hallar el calor total
        suministrado para que se convierta en vapor de agua a
        100°C}{3 625}{7 200}{4 000}{5 250}{5 800}{cal}
        \ex{ Tenemos 2 g de agua a 0°C. ¿Qué cantidad de calor se
        le debe extraer para convertirlo en hielo a 0°C}{80}{160}{200}{250}{300}{cal}
        \ex{ Se dispara una bala de 5g contra un bloque de hielo,
        donde inicia su penetración con una velocidad de
        300m/s, se introduce una distancia de 10cm,
        fundiéndose parte del hielo. ¿Qué cantidad de hielo se
        convierte en agua; en gramos? (el hielo debe estar a
        0°C)}{0.535}{0.672}{0.763}{0.824}{0.763}{ }
        \ex{ Hallar el calor que libera 2g de vapor de agua que se
        encuentra a 120°C de manera que se logre obtener
        agua a 90°C.}{800}{880}{1100}{1120}{1200}{cal}
        \ex{En un vaso lleno de agua a 0°C se deposita un cubo de
        hielo de 40 g a -24°C, si no hay pérdida de calor al
        ambiente. ¿Qué cantidad de agua se solidificará en
        gramos?}{3}{6}{12}{15}{0}{ }
        \ex{Dos varillas de igual lon gitud y de coeficientes $\alpha_1 = 18\mu$°C${}^{-1}$ y $\alpha_2 = 15\mu$°C${}^{-1}$ se someten a un calentamiendo de manera que experimentan la misma dilataci'on. Si el primero se calento en 80 °C, en cuanto debe aumentar su temperatura el segundo en °C}{94}{95}{96}{97}{98}{ }
        \ex{Determine la longitud en cm de una barilla de $\alpha_1 = 8\times 10^{-5} \T{C}^{-1}$ para que al ser calentada lo mismo que otra varilla de $\alpha_2 = 32\times 10^{-6} \T{C}^{-1}$ y de 125 cm se dilaten exactamente lo mismo}{10}{20}{30}{40}{50}{ }
        \ex{En cuantos $cm^2$ se dilata una lámina metálica que se caliente en 240 \T{C} si se sabe que el calentarse en 300 \T{C} la dilatación aumentaría en 8 $cm^2$}{32}{33}{34}{35}{36}{ }
        \ex{Determine en $cm^3$ el cambio en volumen de un bloque metálico de 5 cm, 15 cm y 10 cm cuando la temperatura cambia de 10\T{C} a 80 \T{C} $\gamma = 400\mu\T{C}^{-1}$}{12}{15}{16}{18}{20}{ }
        \ex{Para calentar cierta cantidad de gas de 20°C hasta 100°C
        se requieren 400 cal siempre que su volumen
        permanezca constante. ¿Cuánto aumentará su energía
        interna en el proceso?}{200}{400}{500}{350}{250}{cal}
        \ex{En un proceso isobárico, 2 moles de un gas
        monoatómico reciben 831 J de calor. Determinar el
        incremento de la temperatura del sistema.}{10}{20}{30}{40}{50}{K}
        \exer{Sabiendo que el trabajo realizado por un gas en el
        proceso ABC es 500 J, hallar $P_1$.}{3}{13}{12.5}{18}{16.5}{6}{Pa}
        \exer{Isobáricamente a la presión de 400 Pa, el volumen de
        un gas ideal se extiende hasta triplicarse, si en este
        proceso el gas desarrolla un trabajo de 80 J, encuentre
        el volumen inicial que ocupa el gas.}{4}{0.05}{0.2}{0.1}{0.5}{5}{$\text{m}^3$}
        \exer{Un gas ideal posee una energía interna de  450 J en el
        estado 1. Si dicho gas efectúa una expansión isobárica
        del estado 1 al estado 2. ¿Cuál será la energía interna
        que tendrá el gas al final del proceso si en total ganó
        500 J de calor?}{5}{520}{530}{540}{550}{560}{J}
        \exer{Un gas realiza un proceso tal como se indica en la
        figura. ¿Qué trabajo realizó el gas al pasar del estado 1
        al estado 2?}{6}{5.5}{4.5}{3.5}{2.5}{1.5}{kJ}
        \exer{El gráfico representa el volumen de un gas ideal en
        función de su temperatura a presión constante de 3N/
        m2. Si durante la transformación de A hacia B el gas
        absorbió 5 calorías. ¿Cuál fue la variación de su energía
        interna?}{7}{2}{2}{17.93}{17.93}{4}{J}
        \exer{Dos líquidos no miscibles están en el tubo "U" que se
        muestra. Determinar la relación entre las presiones
        hidrostáticas en los puntos A y B.}{8}{1/3}{2/3}{1}{4/3}{3/2}{ }
        \exer{Determine la lectura del manómetro M, si se está
        ejerciendo una fuerza F=210N sobre el émbolo
        ingrávido el cual permanece en reposo. }{9}{11}{10}{1}{2}{9}{kPa}
        \exer{ Un cubo de 2m de arista cuyo peso es 90kN flota tal
        como se muestra en la figura. La esfera tiene la mitad
        de su volumen en el agua y su peso es 30 kN. ¿Cuál es
        su volumen? g=10m/s2.}{10}{8}{10}{4}{15}{9}{$\text{m}^3$}
        \exer{ Una barra uniforme de 20 kg y 10m de longitud, cuya
        densidad relativa es 0,5 puede girar alrededor de un
        eje que pasa por uno de sus extremos situado debajo
        del agua (ver figura). ¿Qué peso N debe colocarse al
        otro extremo de la barra para que queden sumergidos
        8m de ésta?}{11}{313.6}{588}{2744}{117.6}{27.44}{N}
        \exer{ Una barra uniforme de 3,6 m de longitud y de masa
        12kg está sujeta en el extremo B por una cuerda
        flexible y lastrada en el extremo A por un masa de
        6kg. La barra flota como indica la figura con la mitad
        de su longitud sumergida. Puede despreciarse el
        empuje sobre el lastre. Hallar la tensión en la cuerda.}{12}{19.6}{29.4}{39.2}{88.2}{59.8}{N}
        \exer{ Determinar la presión hidrostática en el fondo del
        recipiente mostrado que contiene agua y que sube
        con una aceleración de 2m/s2.}{13}{1}{2}{3}{4}{6}{kPa}
        \exer{ Los radios de los émbolos 1 y 2 de áreas A1 y A2 son
        de 4cm y 20cm respectivamente. Determine la masa
        m1 (en kg) que equilibra el sistema. Considere
        m2= 2000 kg.}{14}{80}{40}{800}{30}{20}{ }
    \end{enumerate}
\end{document}