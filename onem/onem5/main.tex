\documentclass[12pt, a4paper]{article}
\usepackage{amsmath}
\usepackage{amssymb}
\usepackage[utf8]{inputenc}    % Soporte para caracteres especiales
\usepackage{amsfonts}
\usepackage[right=3cm, left=2.5cm, top=3cm, bottom=3.5cm]{geometry}
\usepackage{graphicx}
%\usepackage{hyperref}          % Enlaces y referencias
%\usepackage[style=apa]{biblatex}
%\usepackage{csquotes}          % Manejo de citas (opcional, útil para biblatex)
%\addbibresource{references.bib}
%\usepackage[spanish]{babel}
\title{Triángulos, Polígonos y Cuadriláteros}
\author{Torres Tarrillo Rober}

\begin{document}

\maketitle

\begin{enumerate}
    \item En un triángulo rectángulo $ABC$, recto en $B$,
    la bisectriz del ángulo exterior $A$ interseca a la
    prolongación de la altura $BH$ en $F$. Si $AB + AH = 4$
    y $HF = 3$. Hallar $BH$.

    \item En la figura: $DA = BD$, m$\angle AMD = 50^\circ$
    y $AE = BC + EC$. Hallar la m$\angle ABC$.
    
    \begin{figure*}[h]
        \centering
        \includegraphics[width=0.4\textwidth]{1.png}
    \end{figure*}
    
    \item En un triángulo $ABC$, recto en $B$, la altura $\overline{BH}$
    mide 16 y $Q$ es un punto de $\overline{BC}$, tal que $AQ = 19$ y
    m$\angle BAQ = \mathrm{m}\angle ACB$. Halar la distancia de $Q$
    a $\overline{AC}$.

    \item En la figura mostrada $AB = BC$; $\overline{AB}\perp \overline{BD}$,
    $\overline{AC}\perp \overline{AE}$ y $AC = AE$. Hallar el valor de $x$.

    \begin{figure*}[h]
        \centering
        \includegraphics[width=0.3\textwidth]{2.png}
    \end{figure*}
    
    \item En un $\triangle ABC$, m$\angle A = 78^\circ$.
    La meditariz de $\overline{AC}$ interior seca a $\overline{BC}$ 
    en el punto $D$. Calcular la m$\angle C$, sabiendo que $AB = DC$.

    \item En un polígono regular las mediatrices de
    de dos lados consecutivos forman un ángulo que mide $22.5^\circ$.
    Determine la medida de su ángulo externo.

    \item En un polígono convexo de $n$ lados, desde $(n - 10)$
    vértices consecutivos se pueden trazar $3n + 9$ diagonales.
    Entonces el número total de diagonales del polígono es:

    \item Calcular el número  de lados de un polígono
    equiángulo $VERANO\dots$, si las mediatrices de
    $\overline{VE}$ y $\overline{NO}$ forman un ángulo de $36^\circ$.
    
    \item En un polígono regular su ángulo interior es $160^\circ$.
    Calcular la suma de las medidas de los ángulos internos convexos
    de la estrella formada al prolongar en ambos sentidos.

    \item En la figura, $ABCD$ es un trapecio, $AM = MD$,
    $AE = EC, EQ = QM$, $\overline{AB}\parallel \overline{QR}$ y
    $BC = 18$. Hallar $QR$.

    \begin{figure}[h]
        \centering
        \includegraphics[width=0.4\textwidth]{3.png}
    \end{figure}

    \item En un trapezoide $ABCD$, m$\angle A = 60^\circ$,
    $AB = 8 \sqrt{3}$, $CD = 20\sqrt{2}$ y m$\angle D = 45^\circ$.
    Hallar la distancia del punto medio $M$ de $\overline{BC}$ a $AD$.

    \item En el gráfico, $2CD = 3AB, BC = 8 $m y $AD = 6$ m, calcule $AB$.

    \begin{figure}[h]
        \centering
        \includegraphics[width=0.4\textwidth]{4.png}
    \end{figure}
    
    \item En un trapecio $ABCD$, $\overline{BC}\parallel \overline{AD}$,
    $BC = 4$ cm, $AD = 20$ cm y $\mathrm{m}\angle A = \mathrm{m} \angle
    D= 90^\circ$, calcular la distancia entre los puntos medios de las bases.

    \item En un romboide $ABCD$, $AB > BC$, las mediatrices de los lados
    $\overline{AB}$ y $\overline{BC}$ se intersecan  en el punto $R$,
    ubicado en la prolongación de $\overline{AD}$.
    Calcular m$\angle RCD$; si m$\angle BAD = 48^\circ$.

    \item En un cuadrilátero $ABCD$, m$\angle ADB = 90^\circ$,
    $\mathrm{m}\angle BCA = m \angle ACD = 15^\circ$ y m$\angle CAD = 30^\circ$.
    Hallar m$\angle BAC$

    \item Se tiene un cuadrilátero $ABCD$, donde se cumple:
    $AB = AD$, m$\angle BAD = 60^\circ$, m$\angle BCA = 30^\circ$, 
    m$\angle CAD = \theta$ ($0 < \theta < 60^\circ$). Hallar m$\angle BDC$

        
\end{enumerate}


\end{document}