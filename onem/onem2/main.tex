\documentclass[12pt, a4paper]{article}
\usepackage{amsmath}
\usepackage{amssymb}
\usepackage[utf8]{inputenc}    % Soporte para caracteres especiales
\usepackage{amsfonts}
\usepackage[right=3cm, left=2.5cm, top=3cm, bottom=3.5cm]{geometry}
%\usepackage{hyperref}          % Enlaces y referencias
%\usepackage[style=apa]{biblatex}
%\usepackage{csquotes}          % Manejo de citas (opcional, útil para biblatex)
%\addbibresource{references.bib}
%\usepackage[spanish]{babel}
\title{Título del documento}
\author{Autor del documento}

\begin{document}

\section{ONEM NIVEL 1 - FASE 2}
\begin{enumerate}
    \item Se tiene dos fracciones equivalentes tales que sus numeradores suman 15
    y sus denominadores son 2 y 4, respectivamente. Hallar el mayor de los numeradores.

    \item Un colegio contrató a un técnico para trabar durante 15 días con un salario de 40 nuevos soles diarios,
    pero cada día que el técnico llega tarde solo gana 20 nuevos soles. Si el técnico trabajó los 15 días
    y terminó ganando 410 nuevos soles, ¿Cuántos días llegó tarde?.

    \item Se tiene dos cuadrados de lados enteros tales que la suma de sus áreas es 650. ¿Cuántos valores distintos puede toar la suma de sus perímetros?
    \item Los enteros positivos $a, b, c$ forman, en ese orden, una progresión aritmética de razón $r, (r > 0)$. Si $a$ es múltiplo de $3, b$ es múltiplo de 7 y $c$ es múltiplo de 9, ¿Cuál es el menor valor que puede tomar $a + b + c$?
    \item En un salón de clases de 50 alumnos, 24 no trajeron el libro de comunicación y 28 no trajeron el libro de matemática.Si 14 estudiantes no trajeron el libro de matemática ni el de comunicación. ¿Cuántos estudiantes trajeron solamente un libro?
    \item Raul reparte su herencia entres sus tres hijas de tal forma que la primera le toca los 4/15 del total, a la segunda los 3/5 y a la tercera S/. 1800. ¿Cuál fue el total de la herencia?
    \item Si $m$ y $n$ son números enteros tales que $m + n = 5$, entonces $2m - n$ no puede ser igual a:
    (a) -5, (b) 1, (c) -2, (d) 2, (e) 7.

    \item Omar tiene cierto número de rosas y quire regalarlas a sus amigas. Si regala 8 rosas a cada una le sobran 15, pero si quisiera regalar 11 rosas a cada una le faltarían 3. ¿Cuántas rosas tiene Omar?
    \item Dos números son tales que el triple del mayor excede a un tercio del menor en 176; y cinco veces el menor excede a tres octavos del mayor en 216. Hallar la diferencia positiva de los números.
    \item sea $P(x)$ un polinomio tal que:
    \begin{gather*}
        xP(x + 1) = P(x^2 + 1), \forall x \in \mathbb{R}
    \end{gather*}

    Si $P(x) = 2$. Hallar el valor de $P(5)$

    \item sean
    \begin{equation*}
        \frac{1}{a + 1} = 2,
        \frac{1}{b + 2} = 3,
        \frac{1}{c + 3} = 6,
    \end{equation*} 
    
    Hallar
    \begin{equation*}
        \frac{1}{a + b + c}
    \end{equation*}
\end{enumerate}



\end{document}