\documentclass[12pt, a4paper]{article}
\usepackage{amsmath}
\usepackage{amssymb}
\usepackage[utf8]{inputenc}    % Soporte para caracteres especiales
\usepackage{amsfonts}
\usepackage[right=3cm, left=2.5cm, top=3cm, bottom=3.5cm]{geometry}
\usepackage{graphicx}
%\usepackage{hyperref}          % Enlaces y referencias
%\usepackage[style=apa]{biblatex}
%\usepackage{csquotes}          % Manejo de citas (opcional, útil para biblatex)
%\addbibresource{references.bib}
%\usepackage[spanish]{babel}

\begin{document}
\section*{ONEM NIVEL III\: Fase 2 - COLEGIO SIGMA}

\begin{enumerate}
    \item Si $\theta$ es un ángulo del primer cuadrante tal que
    $\tan \theta = 1/6$, halla el valor de la siguiente expresión:
    \begin{equation*}
        \sqrt{37}\left(
            \sqrt{
                \frac{\tan \theta + \cot\theta + 2}{\tan \theta + \cot \theta}
            }
            - \frac{\cos\theta}{2}
        \right)
    \end{equation*}
    
    \item En la figura $O$ En la siguiente figura O es el centro de las circunferencias que contienen a los arcos AB y CD. 
    La longitud del arco AB es 8 unidades, la longitud del arco CD es 12 unidades y la longitud del 
    segmento AC es 4 unidades. Halla el área del sector circular AOB.

    \begin{figure}[h]
        \centering
        \includegraphics[width=0.5\textwidth]{1.png}
    \end{figure}

    \item si $\alpha$ y $\beta$ son ángulos agudos complementarios tales
    que $\sec\alpha = m + 1.6$ y $\csc\beta = 3m - 0.4$. Calcule $5\tan\alpha + 13 \cos\beta$.

    \item ¿Cuántos metros mide la hipotenusa de un triángulo rectángulo $ABC$,
    recto en $B$, si se sabe ue su área es de 18 $\mathrm{m}^2$
    y que $\tan A + \tan C  = 4\cot C \cdot \cot A$.
    
    \item Hallar el valor de
    \begin{equation*}
        2(\cos^2 1^\circ + \cos^2 2^\circ + \cos^2 3^\circ + \cdots + \cos^2 90^\circ )
    \end{equation*}

    \item si $\sin \theta + 2 \tan \theta = 5$, hallar el valor
    de ${(2\sec \theta + 1)}^2 - {(\cos \theta + 2)}^2$.

    \item Si $(1 + \sec x)(1 + \csc x) = 20/3$, halla $100 (\sec x - 1)(\csc x - 1)$.
    
    \item $f(x) = \sin(2x) + 2|\sin(x) + \cos(x)$, sean $M$ y $m$ el
    mayor y menor valor de $f(x)$, respectivamente. Calcule $(M + m)^2$.
    
    \item si en un triángulo $ABC$ se cumple que $AB = 9$ y $BC/CA = 40/41$,
    ¿Cuál es el mayor valor posible del área del triángulos $ABC$.

    \item Sea $x$ el ángulo agudo tal que $\csc x = \cot x = \sqrt{3}$.
    Hallar $\sec^2 x$.

    \item Sea $ABC$ un triángulo tal que $\angle ABC = 3\angle ACB$
    y $AC = 2AB$. Hallar la medida del ángulo $\angle CAB$.

    \item En un triángulo $ABC$ se cumple que $\cot A - \cot B = 2$ y
    $ 3\tan A + \tan B = -2$. Calcule $3\tan C$.

    \item Calcule el área del triángulo por por los ejes cartesianos y la recta de ecuación
    $2x + 3y - 12 = 0$.

    \item El área de un triángulo $ABC$, recto en $B$, es $640 \mathrm{cm}^2$
    y además $\tan a + \sec A = 7/2$. ¿Cuántos centímetros mide la hipotenusa?
    
    \item Sea $ABC$ un triángulo rectángulo de hipotenusa $AC$. Se ubica el punto
    $D$ en el cateo $BC$ de modo que $CD/DB = 3$. Si $\angle DAB = \alpha$ y
    $\angle ACB = \beta$, halla el valor de $\cot\alpha\cdot \cot \beta$.
    
   \item $\theta$ es un ángulo  para el cual $\tan\theta = 3$.
   ademas
   
   \begin{gather*}
    \sin\theta + \tan\theta + \sec\theta = a\\
    \cos\theta + \cot \theta + \csc\theta = b
   \end{gather*}
   
   halla el valor de
   \begin{equation*}
    \frac{a - b}{b + 1}
   \end{equation*}
   
   En el gráfico se cumple qu $BC = CD$, $\angle CBA = 74^\circ$ y
   $\angle BAD = 83^\circ$. Halla el valor de $x$.

   \begin{figure}[h]
    \centering
    \includegraphics[width=0.5\textwidth]{2.png}
   \end{figure} 

\end{enumerate}

\end{document}