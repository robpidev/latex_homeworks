\documentclass[12pt, a4paper]{article}
\usepackage{amsmath}
\usepackage{amssymb}
\usepackage[utf8]{inputenc}    % Soporte para caracteres especiales
\usepackage{amsfonts}
\usepackage[right=3cm, left=2.5cm, top=3cm, bottom=3.5cm]{geometry}
\usepackage{graphicx}
%\usepackage{hyperref}          % Enlaces y referencias
%\usepackage[style=apa]{biblatex}
%\usepackage{csquotes}          % Manejo de citas (opcional, útil para biblatex)
%\addbibresource{references.bib}
%\usepackage[spanish]{babel}
\title{Fracciones, Probabilidades y estadística}
\author{Rober Torres}

\begin{document}
\maketitle

\begin{enumerate}
    \item Si gasté los 2/3 de lo que no gasté, entonces lo que no gasté representa:
    \begin{enumerate}
        \item 3/5 de mi dinero.
        \item 3/2 de mi dinero.
        \item 1/3 de mi dinero.
        \item 2/5 de mi dinero.
        \item 2/5 de mi dinero.
        \item 4/5 de mi dinero
    \end{enumerate}

    \item Un niño tiene 100 soles ahorrados.  Con la cuarta parte
    compra un juguete; con la tercera parte del resto compra
    lapiceros, y con la mitad que le queda compra fruta.
    Los ahorros iniciales se han reducido a:

    \begin{enumerate}
        \item s/. 10
        \item S/. 5
        \item S/. 25
        \item S/. 20
        \item S/. 20
        \item S/. 15
    \end{enumerate}
    
    \item Si los 4/7 de los alumnos de un salón de clase no exceden los 12
    12 años de edad y 15 alumnos son mayores de 12.
    ¿Cuántos alumnos tiene el salón?

    \begin{enumerate}
        \item 21
        \item 23
        \item El problema no tiene solución
        \item 35
        \item 26
    \end{enumerate}

    \item De un cilindro lleno de agua, se extrae la quinta parte.
    ¿Qué fracción se debe sacar para que quede solo 6/10 de su capacidad
    inicial?

    \item Encontrar el número racional entre 2/14 y 41/52 cuya distancia
    al primero sea el doble de la distancia al segundo.

\textbf{Enunciado}
    Se analizan las notas de 20 alumnos en el curso de Aritmética
 recogiéndose los siguientes datos:
 4, 13, 13, 10, 9, 9, 6, 10, 11, 7
 15, 16, 12, 10, 7, 11, 2, 8, 4, 3

    \item ¿cuántos estudiantes aprobaron el curso según
    los datos originales?

    \item Calcular la moda para los datos sin agrupar.
    \item Calcular la media para los datos sin agrupar.
    \item Calcular la mediana para los datos sin agrupar.
    
    \item El profesor Lau  tiene 6 hijos, de los cuales 3 son trillizos
 y 2 mellizos.  Si al calcular la media, mediana y moda
 de estas edades resultaron 10; 11 y 12
 respectivamente.
 Halle la diferencia entre la máxima y mínima edad.
 
 \item Dada la siguiente distribución de frecuencia. Hallar:
 $f_1 + f_3 + F_4$
 
 \begin{figure}[h]
    \centering
    \includegraphics[width=0.5\textwidth]{1.png}
 \end{figure}

 \item Lance al aire dos monedas imparciales y registre el resultado. Encuentre la probabilidad 
de observar exactamente una cara en los dos tiros.

 \item Un plato contiene un dulce amarillo y dos rojos. Usted cierra los ojos, del plato escoge 
dos dulces, uno por uno y anota sus colores. ¿Cuál es la probabilidad de que ambos 
dulces sean rojos?

\item El chofer de un camión puede tomar tres rutas de la ciudad A a la ciudad B, cuatro de la 
ciudad B a la C y tres de la ciudad C a la D. Si, cuando viaja de A a D, el chofer debe ir 
de A a B a C a D, ¿cuántas rutas posibles de A a D hay?

\item Tres billetes de lotería se sacan de entre un total de 50. Si los billetes se han de distribuir 
a cada uno de tres empleados en el orden en que son sacados, el orden será importante. 
¿Cuántos eventos simples están asociados con el experimento?

\end{enumerate}

\end{document}