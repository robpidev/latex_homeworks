\section{Conjuntos} 
Definiremos un conjunto como una colección de elementos distintos
(cuyos elementos pueden ser números u objetos cualesquiera).
Los conjuntos serán denotados por una letra mayuscula ($A$, por ejemplo)
y sus elementos se denotaran entre llaves seprados por comas.
Por ejemplo para el junto $N$ cuyos elementos son $2, 3, 5$ y $7$, será
representado como

\begin{equation*}
    N = \{2, 3, 5, 5\}
\end{equation*}

Un conjunto puede estar dado \textit{por comprensión} si los elementos
no están explisitamente

\begin{equation}
    N = \{x : p(x)\}
\end{equation}

donde los dos puntos se lee ``tal que'' y $p(x)$ es una propoción que se
debe cumplir para $x$. Ejemplo:

\begin{equation}
    \label{set:comprension}
    B = \{x : x \text{ es natural y menor que } 3\}
\end{equation}

Si los elementos están dados explicitamente, entonces el conjunto
está \textit{por extención}. Por ejemplo el conjunto~\ref{set:comprension} 
por extensión será:

\begin{equation}
    B = \{1, 2, 3\}
\end{equation}


