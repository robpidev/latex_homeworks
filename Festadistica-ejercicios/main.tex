\documentclass[12pt, a4paper]{article}
\usepackage{amsmath}
\usepackage{amssymb}
\usepackage{amsfonts}
%\usepackage[right=3cm, left=2.5cm, top=3cm, bottom=3.5cm]{geometry}
%\usepackage[style=apa]{biblatex}
%\addbibresource{references.bib}

\title{Solucion a la guia 3 de ejercicios}
\author{Torres Tarrillo Rober E.}

\begin{document}

\maketitle
\tableofcontents

\section{Introducción}
Hola soy la indtroducción


\section{Funciones}
una funcion es una relacion entre dos conjuntos
que tiene una regla de correspondecia $y = f(x)$.
Por ejemplo se tiene la funcion $f(x) = x^2$. A
qui se tiene otra función:

\begin{equation*}
    f(x) = \frac{x^2}{\tan(e^{-x})}
\end{equation*}

Esta es una ecuacion usando gather

\begin{gather}
    f(x) = \tan^2(x^2) \\
    g(x) = \sin^2(x) + x^3
\end{gather}

Ecuacion alinead
\begin{align}
    f(x) &= {(x + 1)}^2 \\
    &= x^2 + 2x + 1
\end{align}

\subsection{Funcion inyectiva}
Una funcion inyectiva es aquella que tiene un solo valor para cada elemento de su dominio.

\begin{equation}
    \int_a^b f(x) dx = \lim_{n\to\infty}\sum_{i=0}^{n}f(x_i^*) \Delta x_i^*
\end{equation}

\section{Escribiendo ecuaciones}
\subsection{Escribiendo ecuaciones lineales}
para escribir una ecuacion lineal basta con usar
solo los simbolos de toda la vida:

\begin{equation}
    ax + b = 0
\end{equation}

\begin{equation}
    x  + y = 3 + 2
\end{equation}

\subsection{Escribiendo fracciones}
Solo vasta con escribir el comando \\frac
\begin{equation}
   \frac{x + a}{c + d} 
\end{equation}

\subsection{Escribir exponentes}
Utilizamos este simbolo
\begin{equation}
    x^2, x^3, x^\frac{1}{2}, x^{a + b} 
\end{equation}

\subsection{Funciones especiales}
\begin{equation}
    \ln(x), \log(x), \sqrt(x), \sqrt[7]{x^2 + 1},
    \csc\left(\frac{x}{2}\right)
\end{equation}

\subsection{simbolos griegos}
\begin{equation}
    \alpha, \beta, \gamma, \delta,
    \pi, \phi
\end{equation}

Y las mayusculas empieza por letra amyuscula
\begin{equation}
    \Phi, \Gamma, \Delta, \Pi
\end{equation}

\subsection{Simbolos especiales}
\begin{equation}
    <, \leq, > \geq, \neq, \implies 
\end{equation}

\subsection{Escribiendo Matrices}

\begin{equation}
    M = \begin{bmatrix}
        a & b & d\\
        c & d & f\\
        g & h & \frac{a}{b}
     \end{bmatrix}
\end{equation}

\end{document}