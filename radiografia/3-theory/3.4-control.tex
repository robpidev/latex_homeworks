\subsection{Control de calidad en tomografía computarizada}

El control de calidad (CQ) en tomografía computarizada es un proceso sistemático destinado a asegurar que el equipo funcione dentro de parámetros óptimos y que las imágenes obtenidas mantengan la calidad diagnóstica necesaria con la menor dosis posible de radiación al paciente.  
Este proceso no solo evalúa el desempeño técnico del escáner, sino también la precisión de los parámetros de exposición, la calibración de los detectores, la uniformidad de la imagen y la correcta aplicación de los protocolos clínicos.

\subsubsection{Procedimientos recomendados por la ACR y la IAEA}

El \textbf{American College of Radiology (ACR)} y el \textbf{Organismo Internacional de Energía Atómica (OIEA / IAEA)} establecen guías detalladas para el aseguramiento de calidad en TC.  
Estos procedimientos están orientados a verificar periódicamente el rendimiento del equipo mediante pruebas estandarizadas, realizadas con fantomas específicos y bajo condiciones controladas.  

De acuerdo con los estándares de la \textbf{ACR CT Accreditation Program}, el control de calidad debe incluir:

\begin{itemize}
    \item \textbf{Prueba de calibración del número CT:} garantiza que el valor de referencia para el agua sea de \(0 \pm 5\) unidades Hounsfield (HU).
    \item \textbf{Uniformidad de imagen:} verifica que los valores HU no varíen más de 5 unidades entre el centro y la periferia del campo de visión.
    \item \textbf{Resolución espacial:} evalúa la capacidad del sistema para diferenciar estructuras finas, medida en pares de líneas por centímetro (lp/cm).
    \item \textbf{Resolución de bajo contraste:} mide la capacidad para detectar objetos con diferencias mínimas de densidad, típicamente del orden de 0.5\% respecto al fondo.
    \item \textbf{Espesor de corte:} compara el espesor nominal con el espesor efectivo determinado a partir del perfil de sensibilidad.
    \item \textbf{Linealidad y ruido electrónico:} analizan la relación entre valores HU y la densidad electrónica del material, así como la estabilidad de los detectores.
\end{itemize}

La \textbf{IAEA}, por su parte, en sus manuales de referencia —como el \textit{Quality Assurance Programme for Computed Tomography: Diagnostic and Therapy Applications (IAEA Human Health Series No. 19)}— recomienda la implementación de un programa de control de calidad estructurado que incluya:

\begin{enumerate}
    \item \textbf{Pruebas diarias:} verificación del calentamiento del tubo, calibración automática y chequeo de uniformidad básica.
    \item \textbf{Pruebas semanales:} control de ruido, resolución espacial y bajo contraste mediante fantomas específicos.
    \item \textbf{Pruebas mensuales o trimestrales:} evaluación de la dosis al paciente (CTDIvol y DLP), precisión geométrica y linealidad de HU.
    \item \textbf{Pruebas anuales:} evaluación integral del sistema, calibración de detectores, consistencia de reconstrucción y validación del software de imagen.
\end{enumerate}

Ambas instituciones subrayan la importancia de documentar cada prueba y comparar los resultados con los valores de referencia proporcionados por el fabricante o por los organismos de acreditación.

\subsubsection{Fantomas utilizados para control de calidad}

Los \textbf{fantomas} son dispositivos físicos diseñados para simular las propiedades de absorción y dispersión de los tejidos humanos, permitiendo la evaluación objetiva del desempeño del tomógrafo.  
Entre los más empleados se destacan:

\begin{itemize}
    \item \textbf{CATPHAN:} fantoma modular ampliamente utilizado para pruebas de calibración y evaluación de parámetros físicos. Incluye módulos para analizar uniformidad, resolución espacial, bajo contraste, geometría y precisión del espesor de corte.
    \item \textbf{ACR Phantom:} diseñado por el American College of Radiology, consta de cuatro módulos cilíndricos que permiten evaluar: calibración de números CT, uniformidad, resolución espacial y contraste. Es el estándar principal para programas de acreditación.
    \item \textbf{Fantoma de agua o de cabeza:} utilizado para pruebas de uniformidad y calibración básica, especialmente en verificaciones diarias o rápidas.
\end{itemize}

El uso sistemático de estos fantomas permite identificar variaciones en la calidad de imagen antes de que afecten la práctica clínica, garantizando consistencia diagnóstica y seguridad radiológica.

\subsubsection{Reglas y lineamientos de la OIEA}

El \textbf{Organismo Internacional de Energía Atómica (OIEA / IAEA)} establece un conjunto de reglas y lineamientos universales para el aseguramiento de calidad en radiodiagnóstico, recogidos en documentos como el \textit{IAEA Safety Standards Series No. GSR Part 3} y el \textit{Radiation Protection and Safety in Medical Uses of Ionizing Radiation (IAEA Safety Report No. 104)}.

Estas reglas se basan en tres pilares fundamentales:

\begin{enumerate}
    \item \textbf{Justificación:} toda exposición médica debe justificarse por el beneficio diagnóstico que aporta, evitando estudios innecesarios o repetitivos.
    \item \textbf{Optimización:} toda práctica debe realizarse con la mínima dosis posible compatible con una calidad de imagen diagnóstica adecuada, conforme al principio \textbf{ALARA}.
    \item \textbf{Limitación de dosis ocupacional y poblacional:} se deben establecer límites de exposición para el personal y el público, garantizando la seguridad dentro de los valores establecidos por la \textit{ICRP (International Commission on Radiological Protection)}.
\end{enumerate}

Además, la OIEA recomienda que cada institución cuente con un \textbf{Programa de Garantía de Calidad (PGC)} formalmente establecido, que incluya:

\begin{itemize}
    \item Protocolos escritos de control de calidad y mantenimiento preventivo.
    \item Designación de un físico médico responsable del aseguramiento de calidad.
    \item Registros continuos de las pruebas de desempeño del equipo.
    \item Revisión anual de los resultados y acciones correctivas documentadas.
    \item Capacitación periódica del personal técnico y médico.
\end{itemize}

El cumplimiento de estas normas asegura la estabilidad funcional del tomógrafo, la calidad diagnóstica de las imágenes y la protección del paciente y del personal frente a la radiación ionizante.

En conclusión, el control de calidad en TC no solo mantiene el rendimiento técnico del equipo, sino que constituye un elemento esencial de la \textbf{seguridad radiológica y la gestión hospitalaria moderna}. Los lineamientos de la ACR y la OIEA garantizan una práctica diagnóstica confiable, segura y alineada con los más altos estándares internacionales.
