\section{Marco Teórico}
\subsection{Principios de la tomografía computarizada (TC)}

La tomografía computarizada (TC) es una técnica de diagnóstico por imágenes que utiliza rayos X para obtener cortes transversales del cuerpo humano. A diferencia de la radiografía convencional, donde las estructuras se superponen en una sola proyección bidimensional, la TC reconstruye digitalmente volúmenes tridimensionales a partir de múltiples proyecciones adquiridas desde distintos ángulos alrededor del paciente.

El principio físico fundamental de la TC se basa en la atenuación diferencial de los rayos X al atravesar distintos tejidos. Cada material presenta un coeficiente de atenuación lineal diferente, dependiendo de su número atómico, densidad y la energía del haz incidente. Los detectores registran la intensidad transmitida después de atravesar el cuerpo, y estos datos son utilizados por el sistema computacional para reconstruir la imagen.

\subsubsection{Formación de imagen por rayos X}

En el proceso de formación de imagen, el tubo de rayos X emite un haz colimado que gira alrededor del paciente, mientras que una serie de detectores mide la intensidad del haz que emerge. La intensidad registrada en cada detector \( I \) está relacionada con la intensidad inicial \( I_0 \) mediante la ley de Beer-Lambert:

\[
I = I_0 e^{-\int \mu(x) \, dx}
\]

donde \( \mu(x) \) es el \textbf{coeficiente de atenuación lineal} del material atravesado, que representa la probabilidad de que los fotones sean absorbidos o dispersados por unidad de longitud.

En términos discretos, el sistema puede representarse como una serie de medidas lineales conocidas como \textit{proyecciones}, que representan la suma de las atenuaciones a lo largo de cada trayectoria del haz. Cada proyección contiene información parcial del objeto, y al combinar múltiples proyecciones desde diferentes ángulos, se puede reconstruir la distribución espacial de los coeficientes de atenuación.

\subsubsection{Atenuación y reconstrucción de imagen}

La reconstrucción de la imagen en TC consiste en estimar la distribución del coeficiente de atenuación \(\mu(x,y)\) a partir de las proyecciones adquiridas. Matemáticamente, este proceso se expresa mediante la \textbf{transformada de Radon}, que relaciona el conjunto de proyecciones con la función de atenuación del objeto:

\[
p(\theta, t) = \int_{-\infty}^{\infty} \int_{-\infty}^{\infty} \mu(x,y) \, \delta(x \cos \theta + y \sin \theta - t) \, dx \, dy
\]

donde \(p(\theta, t)\) representa la proyección del objeto para un ángulo \(\theta\) y una distancia \(t\) desde el centro de rotación.

La reconstrucción de \(\mu(x,y)\) se logra aplicando el proceso inverso de la transformada de Radon. Uno de los algoritmos clásicos para esta tarea es la \textbf{retroproyección filtrada} (\textit{Filtered Back Projection}, FBP), que consiste en dos pasos fundamentales:

\begin{enumerate}
    \item \textbf{Filtrado:} cada proyección se filtra en el dominio de la frecuencia mediante un filtro de alta frecuencia (como el filtro Ram-Lak) para corregir el efecto de suavizado que genera la simple retroproyección.
    \item \textbf{Retroproyección:} las proyecciones filtradas se distribuyen nuevamente sobre la imagen reconstruida en las direcciones correspondientes a sus ángulos de adquisición.
\end{enumerate}

El resultado final es una imagen que representa los coeficientes de atenuación en cada punto del plano, expresados en unidades de Hounsfield (HU), definidas como:

\[
HU = 1000 \cdot \frac{\mu - \mu_{\text{agua}}}{\mu_{\text{agua}}}
\]

donde \(\mu_{\text{agua}}\) es el coeficiente de atenuación del agua. Así, el agua se asigna a 0 HU, el aire a aproximadamente \(-1000\) HU y los huesos densos pueden alcanzar valores superiores a \(+1000\) HU.

Estos valores permiten distinguir distintos tejidos biológicos en función de su densidad y composición, otorgando a la TC su capacidad de discriminación anatómica y diagnóstica.
