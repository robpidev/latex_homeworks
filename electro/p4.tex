\section{Ley de circuitos de Ampére}
Los campos deducidos anteriormente satisfacen la siguiente ecuación
\begin{equation}
    \label{corriente:densidad:eq}
    \mathbf{\nabla} \cdot \mathbf{J} = 0
\end{equation}

Por lo que se puede se puede deducir una ecuación importante para el rotacional
de la ecuación~\ref{campo:magnetico:densidad}

\begin{gather*}
    \mathbf{\nabla_2\times B(r_2)}
    =\frac{\mu_0}{4\pi}
    \int_V\frac{\mathbf{J(r_1)\times\nabla_2(r_2 - r_1)}}{\mathbf{|r_2 - r_1|}^\text{3}}
    dv_1
\end{gather*}

Ahora utilizando
$\mathbf{A\times\nabla\times B = \nabla_B(A\cdot B) - (A\cdot\nabla)\times B}$,
y como $\nabla_2$ solo afecta a $\mathbf{r_2}$

\begin{gather*}
    \mathbf{\nabla_2\times B(r_2)}
    =  \frac{\mu_0}{4\pi}\int_V\left[
        \mathbf{
            J(r_1)\left(
                \nabla_2\cdot\frac{r_2 - r_1}{|r_2-r_1|^\text{3}}
            \right)
            -
            J(r_1)\cdot\nabla_2\frac{r_2-r_1}{|r_2-r_1|^\text{3}}
        }
    \right]dv_1
\end{gather*}

El primer término se puede expresar en función de la delta de Dirac,
el segundo término intercambiando los vectores y por simetría obtenemos

\begin{gather*}
    \mathbf{\nabla_2\times B(r_2)}
    =  \frac{\mu_0}{4\pi}\int_V\left[
        \mathbf{J(r_1)}4\pi\delta(\mathbf{r_2 - r_1})
        -
        \mathbf{J(r_1)\cdot\nabla_1\frac{\mathbf{r_1 - r_2}}{\mathbf{|r_1-r_2|^3}}}
    \right]
    dv_1
\end{gather*}

Al integrar el primer término obtenemos $\mu_0\mathbf{J(r_2)}$,
para el segundo término podemos usar el hecho que

\begin{gather*}
    \nabla_1\cdot\left(
        \mathbf{J(r_1)}\frac{x_1 - x_2}{|\mathbf{r_1 - r_2}|^3}
    \right)
    = \frac{x_1-x_2}{\mathbf{|r_1 - r_2|}^3}\nabla_1\cdot\mathbf{J}
    +\mathbf{J}\cdot\nabla_1\frac{x_1-x_2}{|\mathbf{r_1 - r_2}|^3}
\end{gather*}

de la misma forma par $y$ y $z$, al sumar estos términos se anula, esta es la
forma diferencial de la ley de Ampére.

\begin{equation}
    \label{ley:apere:diferencial}
    \mathbf{\nabla_1\times B(r_2)} = \mu_0\mathbf{J(r_2)}
\end{equation}

Integrando sobre una superficie $S$ tenemos

\begin{gather*}
    \int_S\mathbf{\nabla_1\times B(r_2)}\cdot d\mathbf{S} = \int_S\mu_0\mathbf{J(r_2)}\cdot d\mathbf{S}
\end{gather*}

Utilizando el teorema de Stokes tenemos
\begin{equation}
    \label{ley:apere:integral}
    \oint_c \mathbf{B}\cdot d\mathbf{l} = \mu_0 \int_S \mathbf{J}\cdot d\mathbf{S}
\end{equation}

esta es la ley de Ampére la cual es análoga a la ley de Gauss, no tiene significado
físico como tal, pero no ayuda a calcular los campos magnéticos.


\section{Referencias}

Reitsz, J., Milford, F. y Chisty, R. (Sin fecha). "Fundamentos de la teoría electromagnética", 4ta ed. Addison-Wesley Iberoamericana.

Young, H. y Freedman, A. (2013). Física universitaria con física moderna(13ava Ed, Vol. 2). México: PEARSON.

Wangsness, R. (2001). "Campos electromagnéticos". México: Limusa.

