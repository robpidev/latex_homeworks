\section{Introducción}
En este trabajo se deducirá la ley de Ampere tanto de forma
integral como diferencial, el cual se hará haciendo un breve repaso de
la fuerza electrostática y la fuerza magnética.

\section{Fuerza y campos electromagnéticos de partículas cargadas}
\subsection{Fuerza eléctrica}
De los datos experimentales se deduce que la fuerza eléctrica entre dos
partículas que se encuentran en reposo con cargas $q$ y $q_1$
sobre un mismo sistema de referencia es directamente proporcional a la
multiplicación de sus cargas e inversamente proporcional a su distancia
de separación elevada al cuadrado, utilizando un vector unitario
para la dirección esto se escribe:

\begin{equation}
    \label{fuerza:electrica}
    \mathbf{F}_e = \frac{1}{4\pi\epsilon_0}\frac{qq_1}{r^2}\frac{\mathbf{r}}{r}
\end{equation}

Donde $\epsilon_0$ es la constante eléctrica de permitividad en el vacío.

\subsection{Campo eléctrico}
Si se toma a $q$ como una carga de prueba, y se divide
a~\ref{fuerza:electrica} por esta carga da como resultado lo que se denomina
campo eléctrico de la carga $q$,

\begin{equation}
    \label{camp:electrico}
    \mathbf{E} = \frac{1}{4\pi\epsilon_0}\frac{q_1}{r^2}\frac{\mathbf{r}}{r}
\end{equation}

por lo que la fuerza ahora se puede escribir como
\begin{equation}
    \label{fuerza:campo:electrico}
    \mathbf{F_e} = q_1\mathbf{E}
\end{equation}

\subsection{Campo de inducción magnética}
Si la carga $q_1$ se encuentra en movimiento con respecto al sistema de
referencia entonces además de generar un campo eléctrico, esta genera un
campo magnético también llamado campo de inducción magnética.

De los experimentos realizados se observó lo siguiente:
(1) el campo magnético es directamente proporcional a la carga $q_1$, (2)
es perpendicular a la su velocidad $\mathbf{v_1}$ y su posición $\mathbf{r}$
e (3) inversamente proporcional a la distancia de la carga a donde se este
midiendo el campo, esto se puede escribir como sigue

\begin{equation}
    \label{campo:magnetico}
    \mathbf{B} =  \frac{\mu_0}{4\pi}\frac{q_1}{r^2}\mathbf{v_1}\times\frac{\mathbf{r}}{r}
\end{equation}

Donde $\mu_0$ es la constante de permitividad magnética en el vacío.

\subsection{Fuerza magnética}
De los datos experimentales la fuerza magnética sobre una partícula de 
carga $q$ que también se encuentra en movimiento con una velocidad
$\mathbf{v}_1$ es directamente proporcional a su carga y perpendicular
a su velocidad y el campo

\begin{equation}
    \label{fuerza:magnetica}
    \mathbf{F}_m = q\mathbf{v}\times\mathbf{B}
\end{equation}



\subsection{Fuerza de Lorentz}
Si se encuentra presente el campo eléctrico y el campo magnético
entonces la fuerza total será la denominada fuerza de Lorentz
\begin{equation}
    \label{fuerza:lorentz}
    \mathbf{F} = q(\mathbf{E} + \mathbf{v}\times\mathbf{B})
\end{equation}
