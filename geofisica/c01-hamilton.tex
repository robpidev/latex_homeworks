\section{Ecuaciones de Hamilton}
El concepto de un rayo es muy útil. Es una línea trazada en el espacio
que corresponde a la dirección del flujo de energía radiante. Los rayos
son una idealización geométrica; no tienen ancho. Como tal, un rayo es
una herramienta matemático en lugar de una entidad física. En la práctica,
podemos producir haces o lápices muy estrechos (como, por ejemplo, un
rayo láser), y podríamos imaginar un rayo como el límite inalcanzable
de la estrechez de dicho haz. Similar a las líneas de la geometría,
los rayos son una ficción conveniente. Existen en el mundo real como un
haz de luz; y los haces tienen ancho. De la misma manera, podemos pensar
en los rayos sísmicos como haces idealizados en la dirección del flujo
de energía sísmica.


Un medio es \textit{homogéneo} si sus propiedades no son función de la
posición, es decir, si sus propiedades no varían de punto a punto. En 
otras palabras, un medio homogéneo es idéntico en todo. De lo contrario,
se dice que el medio es \textit{no homogéneo} o, alternativamente,
\textit{heterogéneo}. Si un medio homogéneo fuera cortado en pedazos,
entonces cada pieza sería idéntica. Si las rocas sub superficiales
encontradas en la tierra fueran figurativamente cortadas en pedazos,
generalmente se observarían diferencias. Así, la sismología tiene que
lidiar con medios no homogéneos.
 

Un medio es \textit{isotrópico} si sus propiedades (como la densidad y
el módulo de Young) no dependen de una dirección particular.
Si estas propiedades dependen de la dirección, entonces podemos decir
que el medio es \textit{anisotrópico}. Más específicamente, un material
se dice que es anisotrópico si el valor de una medición de una propiedad
de la roca varía con la dirección.


La anisotropía sísmica puede definirse como la dependencia de la velocidad
según la dirección. Hay muchos tipos diferentes de anisotropía. De manera
particula Los tipos de anisotropía que poseen un eje de invarianza
rotacional son de interés; es decir, si la formación se rota alrededor de
dicho eje, el material sigue siendo indistinguible de lo que era antes.
Dos casos de anisotropía sísmica son de especial interés: uno es la
isotropía transversal vertical (VTI) y el otro es la isotropía transversal
horizontal (HTI). La isotropía transversal vertical tiene un eje vertical
de invarianza rotacional. Este tipo de anisotropía está asociado con la
estratificación y las lutitas, y se encuentra donde la gravedad es el
factor dominante. La isotropía transversal horizontal, también conocida
como anisotropía azimutal, tiene un eje horizontal de invarianza rotacional.
Este tipo de anisotropía está asociado con grietas y fracturas, y se
encuentra donde el estrés regional es el factor dominante.


El vidrio es un ejemplo de un medio homogéneo isotrópico. El espato de
Islandia es un ejemplo de un medio homogéneo anisotrópico. En el vidrio,
el rayo es ortogonal a las frentes de onda. En el espato de Islandia, hay
dos caminos de luz, conocidos como el rayo ordinario y el rayo
extraordinario. El rayo ordinario es ortogonal a las frentes de onda.
El rayo extraordinario no es ortogonal a las frentes de onda.


Los rayos en un medio homogéneo son líneas rectas. Los rayos en un medio no
homogéneo son líneas curvas. La línea curva no tiene que ser suave. Por
ejemplo, si el medio no homogéneo está compuesto por diferentes capas
homogéneas, el rayo será recto a través de cada capa individual pero se
doblará (por refracción) en cada interfaz. Los rayos en un medio
isotrópico son ortogonales a las frentes de onda; más específicamente,
los rayos son perpendiculares a las frentes de onda en cada punto de
intersección. En consecuencia, en un medio isotrópico, un rayo es paralelo
al vector de propagación $\mathbf{k}$. Excepto en casos especiales como
el rayo ordinario en el espato de Islandia, los rayos en un medio
anisotrópico no son ortogonales a las frentes de onda.


La anisotropía difiere de la propiedad de la roca llamada heterogeneidad
en que la anisotropía es la variación en los valores vectoriales con la
dirección en un punto mientras que la heterogeneidad es la variación en
los valores escalares o vectoriales entre dos o más puntos. La tabla
~\ref{table:1}
describe el comportamiento de los rayos y las frentes de onda para
varios medios.

\begin{table}[h]\label{table:1}
    \centering
    \begin{tabular}{c p{5.5cm} p{5.5cm}}
        \toprule
        & \textbf{Homogéneo} & \textbf{Heterogéneo} \\ 
        \midrule
        Isotrópico & Los rayos son lineas rectas.  & Los rayos son lineas curvadas.\\
        & Los frentes de onda son ortogonales a los rayos & Los frente de onda son ortogonales a los rayos. \\
        & Ejemplo: glass & Ejemplo: Modelo típico de la tierra.\\
        Anisotrópico & Los rayos son lineas rectas & Los rayos son lineas curvadas \\
        &Los frentes generalmente no son ortogonales a los rayos & Los frente de onda generalmente no son ortogonales a los rayos \\
        &Ejemplo: espato de de Islandia & Ejemplo: Modelo avanzado de la tierra\\ 
        \bottomrule
    \end{tabular}
    \caption{Comportamiento de rayos y frentes de onda para
    distintos medios.}
\end{table}

A menudo, la luz puede pensarse como rayos. Una onda de agua vista en
un estanque aparece como un conjunto de frentes de onda en movimiento.
Si se arroja una piedra al estanque, las crestas (frentes de onda)
forman un patrón de círculos concéntricos. La energía de la
perturbación se propaga hacia afuera radialmente desde el centro.
Es decir, la energía se propaga a lo largo de los rayos en ángulos
rectos a las frentes de onda. Si observamos cuidadosamente el
movimiento de la onda, observamos que las ondas más largas aparecen
en el exterior del patrón de círculos concéntricos en expansión.


Mientras observamos el progreso de una de estas crestas exteriores,
de repente veremos que desaparecen. Esto no es una ilusión, y la
siguiente cresta que viene desde atrás también desaparece. Más y más
crestas siguen viniendo desde atrás y desaparecen en el borde exterior.
En el interior del anillo, nuevas crestas siguen apareciendo desde el
agua central ahora calmada. La razón de este fenómeno es la siguiente.
El paquete de ondas representado por los anillos concéntricos se mueve
hacia afuera a la velocidad de grupo. La velocidad de grupo es la
velocidad a la que la energía de la perturbación se propaga hacia afuera.
Sin embargo, las crestas se mueven a la velocidad de fase. Examinando
la física, se puede demostrar que la velocidad de grupo es menor que
la velocidad de fase. Por lo tanto, la cresta de cada onda se mueve
más rápido que el paquete de ondas, y las crestas avanzan con respecto
al patrón concéntrico. Una vez que una cresta llega al borde exterior
del patrón, no puede continuar, porque aún no ha llegado energía allí,
y por lo tanto, la cresta simplemente desaparece.


En muchos problemas físicos, podemos encontrar la relación de dispersión
para la ecuación de onda que gobierna el movimiento ondulatorio.
En símbolos, podemos escribir esta relación de dispersión como la
ecuación matemática:

\begin{equation}
    \omega = \omega(\mathbf{k}, \mathbf{r})
\end{equation}

donde \( \omega \) es la frecuencia, \( \mathbf{r} \) es el vector de
posición y \( \mathbf{k} \) es el vector de número de onda. Podemos
pensar en \( \omega(\mathbf{r}, \mathbf{k}) \) como una superficie en el
espacio de seis dimensiones. Para un valor fijo de \( \mathbf{r} \),
la sub superficie es la superficie del número de onda en un espacio
tridimensional \( \mathbf{k} \). Como hemos visto en las ecuaciones
84 y 87 del Capítulo 5, el gradiente de esta superficie tridimensional
es la velocidad de grupo.

\begin{equation}
    \nabla_k\omega = \frac{\partial \omega}{\partial\mathbf{k}}
\end{equation}


Para un valor fijo de \( \mathbf{k} \), la sub superficie es una
superficie en el espacio tridimensional \( \mathbf{r} \).
El gradiente de esta superficie tridimensional es:

\begin{equation}
    \nabla_{r}\omega = \frac{\partial \omega}{\partial \mathbf{r}}
\end{equation}

Estos dos gradientes juegan un papel interesante en la teoría de Hamilton.

Ejemplo. Consideremos un medio estratificado verticalmente bidimensional:

\begin{equation}
    \mathbf{r} = (x,z), \quad \mathbf{k} = (k_x, k_z), \quad 
    \mathbf{v}(x,z) = \mathbf{v}(z)
\end{equation}

Por definición, la velocidad depende solo de la coordenada de profundidad
\( z \), y no de la coordenada horizontal \( x \). La relación de dispersión
es:

\begin{equation}
    \omega(\mathbf{k}, \mathbf{r}) = v(z) \sqrt{k_x^2 + k_z^2} = v(z) k
\end{equation}

Por lo tanto, \( \omega \) es una función de la forma
\( \omega(k_x, k_z, z) \). El vector de velocidad de grupo es:

\begin{equation}
    \mathbf{v}_g = \left( \frac{\partial \omega}{\partial k_x},
    \frac{\partial \omega}{\partial k_z} \right) 
    = \left(\frac{k_x v}{k}, \frac{k_z v}{k}\right)
    = (\alpha v(z), \gamma v(z))
\end{equation}

donde el numero de onda $k$ es $\sqrt{k_x^2 + k_z^2}$ y
\( a = \sin(u) \) y \( g = \cos(u) \) son los cosenos directores del vector
de número de onda \( \mathbf{k} \).


Como se muestra en la Figura~\ref{fig1:frente_onda}, \( \theta \)
es el ángulo que \( \mathbf{k} \) forma con el eje \( z \). Dado que la
velocidad de grupo es independiente de la frecuencia \( v \), no hay
dispersión.

\begin{figure}[h]\label{fig1:frente_onda}
    \centering
    \begin{tikzpicture}
        % \draw[help lines] (0, 0) grid (10, -10);
        \draw [->, color=purple] (0, 0) -- (5, 0) node[right] {$x$};
        \draw [->, color=purple] (0, 0) -- (0, -5) node[below] {$z$};
        \draw [dashed] (3, 0) -- (3, -4);
        \draw [dashed] (0, -4) -- (3, -4);
        \draw [line width=1.2pt] (0, 0) -- (3, -4);
        \node[below] at (1.5, -4) {$\alpha = \sin \theta$};
        \node[right] at (3, -2) {$\gamma = \cos \theta$};
        \draw (0, -0.5) arc (-90:-53:0.5);
        \node [right] at (0, -0.75) {$\theta$};
    \end{tikzpicture}
    \caption{Propagación bidimensional}
\end{figure}


El otro vector de gradiente es, según la ecuación 5:
\begin{equation}
    \frac{\partial \omega}{\partial \mathbf{r}}
    = \left( \frac{\partial \omega}{\partial x},
    \frac{\partial \omega}{\partial z} \right) = (0, v'(z)k)
\end{equation}

Este vector tiene un componente nulo en la dirección \( x \), por lo que
las variaciones solo pueden tener lugar en la dirección \( z \), como
podríamos esperar para un medio estratificado verticalmente.

Ahora, desarrollaremos la dualidad onda-partícula de Hamilton. Comenzamos
nuestro tratamiento con la ecuación de dispersión
\( \omega(\mathbf{k}, \mathbf{r}) \). El supuesto básico que hacemos para el
medio es que la frecuencia y el número de onda varían lentamente. Es decir,
asumimos (1) que la frecuencia \( \omega \) no cambia mucho en un período de
oscilación, y (2) que el vector de número de onda \( \mathbf{k} \) no
cambia mucho en magnitud y dirección en una distancia de una longitud
de onda.

Recordemos que una onda plana en un medio homogéneo se puede escribir en
la forma

\begin{equation}
    \exp(i\phi) = \exp(i(\omega t - \mathbf{k} \cdot \mathbf{r}))
\end{equation}

donde \( \omega \) y \( \mathbf{k} \) son constantes. La cantidad

\begin{equation}
    \phi = \omega t - \mathbf{k} \cdot \mathbf{r}
\end{equation}

se llama la fase. Para tal onda plana, vemos que

\begin{equation}
    d\phi = \omega \, dt - \mathbf{k} \cdot d\mathbf{r}
    \end{equation}

por lo que podemos escribir la fase como
\begin{equation}
    \phi = \int d\phi = \int \left(
         \omega - \mathbf{k} \cdot \frac{d\mathbf{r}}{dt}
    \right) \, dt 
\end{equation}


Ahora hacemos uso de nuestro supuesto básico de variación lenta de
\( \omega \) y \( \mathbf{k} \). Es decir, \( \omega \) y \( \mathbf{k} \)
son aproximadamente constantes con respecto al período y la longitud de onda,
respectivamente. Por lo tanto, asumimos que, para nuestro medio de variación
lenta, se mantiene la misma ecuación, a saber:

\begin{equation}
    \phi = \int_{t_0}^{t_1}
    \left( \omega(\mathbf{k}, \mathbf{r})
    - \mathbf{k} \cdot \frac{d\mathbf{r}}{dt} \right) dt
\end{equation}

En términos físicos, la cantidad dentro de los paréntesis en la ecuación 12 se llama el Hamiltoniano y la fase \(\phi\) se llama la acción. Según el principio de mínima acción de Hamilton, la solución requerida para la trayectoria del rayo se encuentra minimizando la acción; es decir, minimizando la integral anterior. La magnitud de la integral depende de la función matemática elegida para la trayectoria del rayo \(\mathbf{r}(t)\) como función del tiempo \(t\). Para examinar la relación entre la acción (es decir, la fase) \(\phi\) y la función \(\mathbf{r}(t)\), es conveniente calcular el cambio en \(\phi\) para la transición de una función arbitraria \(\mathbf{r}(t)\) a otra función arbitraria pero infinitamente cercana \(\mathbf{r}_1(t)\).

La Figura 2 muestra dos trayectorias concebibles, donde el tiempo \(t\) se 
traza a lo largo de la abscisa y \(\mathbf{r}\) se traza esquemáticamente 
en la ordenada. Asumimos que todas estas trayectorias pasan por los mismos 
puntos
\(\mathbf{r}_0 = \mathbf{r}(t_0)\) y \(\mathbf{r}_1 = \mathbf{r}(t_1)\) 
en el tiempo inicial \(t_0\) y el tiempo final \(t_1\), respectivamente. 
La distancia vertical entre dos trayectorias en un instante dado se llama 
la variación de \(\mathbf{r}\) y se denota por \(\delta\mathbf{r}\). 
En los puntos extremos \(t_0\) y \(t_1\), por supuesto,
\(\delta\mathbf{r} = 0\) 
porque todas las trayectorias coinciden en estos puntos según la suposición.
La razón por la que se utiliza el símbolo \(\delta\) es para dejar clara la 
diferencia entre la variación \(\delta\) y el diferencial \(d\). El diferencial
se toma para la misma trayectoria en varios instantes de tiempo, mientras
que la variación se toma para el mismo instante de tiempo entre diferentes
trayectorias.

La variación en la acción (fase) está dada por:

\begin{equation}
    \delta \phi = \delta \int_{t_0}^{t_1}
    \left( \omega(\mathbf{k}, \mathbf{r}) -
    \mathbf{k} \cdot \frac{d\mathbf{r}}{dt} \right) dt
\end{equation}

%% TODO: Insert figure 2 here
\begin{figure}[h]\label{fig2:variation}
   \centering
   \begin{tikzpicture}
       % \draw[help lines] (0, 0) grid (10, 10);
       \draw [->, color=purple] (0, 0) -- (10, 0) node[right] {$x$};
       \draw [->, color=purple] (0, 0) -- (0, 5) node[left] {$\mathbf{r}$};
       \draw [dashed] (0, 2) -- (4, 2);
       \draw [dashed] (4, 0) -- (4, 2);
       \draw [dashed] (0, 4) -- (8, 4);
       \draw [dashed] (8, 0) -- (8, 4);
       \node [left] at (0, 2) {$\mathbf{r}_1$};
       \node [left] at (0, 4) {$\mathbf{r}_0$};
       \node [below] at (4, 0) {$t$};
       \node [below] at (8, 0) {$t_0$};
       \draw plot [smooth, tension=1] coordinates {(4,2) (5.1,2.5) (6.8, 2.7) (8, 4)};
       \draw plot [smooth, tension=1] coordinates {(4,2) (5.3,3.2) (6.8, 3.4) (8, 4)};
       \draw [-> ] (5.5, 3.8) -- (5.5, 3.35);
       \draw [-> ] (5.5, 2.1) -- (5.5, 2.5);
       \node [right] at (5.5, 2.2) {$\delta \mathbf{r}$};
   \end{tikzpicture}
   \caption{Ilustración esquemática del variación de $\delta \mathbf{r}$}
\end{figure}

que es

\begin{equation}
    \delta\phi = \delta\int_{t_0}^{t_1}
    \left( \frac{\partial \omega}{\partial \mathbf{k}} \cdot \delta\mathbf{k}
    + \frac{\partial \omega}{\partial \mathbf{r}} \cdot \delta\mathbf{r}
    - \mathbf{k} \cdot \delta\left( \frac{d\mathbf{r}}{dt} \right) \right) dt.
\end{equation}

En la ecuación 14, las dos derivadas parciales de \(\omega\) son gradientes;
es decir,

\begin{equation}
    \frac{\partial \omega}{\partial \mathbf{k}} =
    \left( \frac{\partial \omega}{\partial k_x},
    \frac{\partial \omega}{\partial k_y},
    \frac{\partial \omega}{\partial k_z} \right) =
    \nabla_{\mathbf{k}} \omega =
    \text{gradiente de } \omega \text{ con respecto a } \mathbf{k},
\end{equation}

\begin{equation}
    \frac{\partial \omega}{\partial \mathbf{r}}
    = \left( \frac{\partial \omega}{\partial x},
    \frac{\partial \omega}{\partial y},
    \frac{\partial \omega}{\partial z} \right)
    = \nabla_{\mathbf{r}} \omega
    = \text{gradiente de } \omega \text{ con respecto a } \mathbf{r}.
\end{equation}

El último término dentro de los paréntesis en la ecuación 13 puede integrarse
por partes. Haciéndolo, obtenemos

\begin{equation}
    \int_{t_0}^{t_1}
    \left(-\mathbf{k} \cdot \delta\frac{d\mathbf{r}}{dt} \right) dt
    = {\left[ -\mathbf{k} \cdot \delta \mathbf{r} \right]}_{t_0}^{t_1}
    + \int_{t_0}^{t_1} \frac{d\mathbf{k}}{dt} \cdot \delta\mathbf{r} \, dt.
\end{equation}

Dado que todas las curvas \(\mathbf{r}(t)\) pasan por los mismos puntos finales
(como se indicó anteriormente), la parte integrada se vuelve cero. Así,
la variación en la fase es

\begin{equation}
    \delta\phi = \int_{t_0}^{t_1}\left(\delta \mathbf{k} \cdot
        \left(
            -\frac{-d\mathbf{r}}{dt} + \frac{\partial\omega}{\partial\mathbf{k}}
        \right)
        + \delta\mathbf{r}\cdot
        \left(
            \frac{\partial \mathbf{k}}{dt} + \frac{\partial \omega}{\partial \mathbf{r}}
        \right) 
    \right) \, dt.
\end{equation}

Las variables independientes ahora son \(\mathbf{k}\) y \(\mathbf{r}\).
Las variaciones \(d\mathbf{k}\) y \(d\mathbf{r}\) son completamente arbitrarias.
Así, para que \(d\phi\) sea cero, cada una de las siguientes dos ecuaciones debe ser
satisfecha:

\begin{equation}
    \frac{d\mathbf{r}}{dt} = \frac{\partial \omega}{\partial \mathbf{k}}
    \quad \text{y} \quad
    \frac{d\mathbf{k}}{dt} = -\frac{\partial \omega}{\partial \mathbf{r}}.
\end{equation}

Estas dos ecuaciones se llaman las ecuaciones de Hamilton.

Ahora, mostraremos que la frecuencia \(\omega\) se mantiene constante a lo largo
de una trayectoria del rayo. La tasa de cambio de la frecuencia
\(\omega(\mathbf{k}, \mathbf{r})\) con el tiempo para tasas de cambio
arbitrarias de \(\mathbf{k}\) y \(\mathbf{r}\) es la derivada total

\begin{equation}
    \frac{d\omega}{dt}
    = \frac{\partial \omega}{\partial \mathbf{k}}
    \cdot \frac{\partial \mathbf{k}}{\partial t}
    + \frac{\partial \omega}{\partial \mathbf{r}}
    \cdot \frac{\partial \mathbf{r}}{\partial t}.
\end{equation}

Sustituyendo las ecuaciones de Hamilton 19 en el lado derecho de la ecuación 20,
obtenemos cero. Por lo tanto, para las tasas de cambio a lo largo del rayo,
como se da en las ecuaciones de Hamilton, la derivada \( \frac{d\omega}{dt} \)
es cero. Por lo tanto, la frecuencia \( \omega \) permanece constante a lo largo
de un rayo.

Ahora enunciemos la dualidad de Hamilton entre ondas y partículas. Sea el vector
de coordenadas \( \mathbf{r} \) que representa las coordenadas tanto de la onda
como de la partícula. Entonces, el vector número de onda \( \mathbf{k} \)
de la onda corresponde al vector momento de la partícula. La frecuencia \( \omega \)
de la onda corresponde a la energía de la partícula. Como hemos dicho, la velocidad
de grupo

\begin{equation}
    \frac{d\mathbf{r}}{dt} = \frac{\partial \omega}{\partial \mathbf{k}},
\end{equation}

con la que el paquete de ondas viaja, corresponde a la velocidad con la que
la partícula viaja a lo largo del rayo (ver ecuación 19). Aunque la dualidad
onda-partícula fue establecida por Sir William Hamilton en 1825, la significación
física de esta dualidad no se comprendió hasta 1924, cuando el Príncipe Louis de
Broglie sugirió que un electrón tiene un carácter dual (es decir, un electrón es una
partícula con leyes de movimiento que son de carácter ondulatorio). Esta dualidad
onda-partícula para el electrón coincide con la dualidad onda-partícula para el fotón
desarrollada por Arthur Compton en 1923.

