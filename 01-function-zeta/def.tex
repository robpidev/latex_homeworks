\section{Teoría}

\subsection{Definición}
La función Zeta $\zeta(z)$ para $z\in\mathbb{C}$ con $\Re(z)>1$ de Riemann se
define de la siguiente forma

\begin{equation}
    \zeta(z) = \sum_{n=1}^\infty \frac{1}{n^z} = 1 + \frac{1}{2^z}
    + \frac{1}{3^z} + \frac{1}{4^z} + \cdots
\end{equation}

se define para $\Re(z) > 1$ dado que si $Re(z) < 1$, entonces
\begin{equation*}
    \zeta(z) = \sum_{n=1}^{\infty} \frac{1}{n^z} = \infty
\end{equation*}

Es decir la serie diverge, si $z = 1$, se tiene
\begin{equation*}
    \zeta(z) = \sum_{n=1}^{\infty} \frac{1}{n}
    =1 + \frac{1}{2} + \frac{1}{3} + \frac{1}{4} + \cdots = \infty
\end{equation*}

la cual es la serie armonica y es divergente.

\subsection{Producto de Euler}
Para $\Re(z) > 1$ y $p \in \mathbb{P}$, es decir $p$ pertenece
al conjunto de los números primos $\mathbb{P}$, se tine que
$\zeta(z)$ es igual al producto de Euler
\begin{equation}
    \zeta(z) = \sum_{n=1}^\infty \frac{1}{n^z} 
    = \prod_{p\in \mathbb{P}} \frac{1}{1-p^{-s}}
\end{equation}

\begin{proof}[Demostración]
    Se tiene que 
    \begin{align}
        &\zeta(z) = 1 + \frac{1}{2^s} + \frac{1}{3^s} + \frac{1}{4^s} + \frac{1}{5^s} + \frac{1}{6^s} + \frac{1}{7^s} + \cdots  \\
        &\frac{1}{2^s}\zeta(z) = \frac{1}{2^s} + \frac{1}{4^s} + \frac{1}{6^s} + \frac{1}{8^s} + \frac{1}{10^s} + \frac{1}{12^s} + \cdots
    \end{align}
    
    Haciendo $(3) - (4)$
    
    \begin{gather}
        \zeta(z) - \frac{1}{2}\zeta(z) =\left(1 - \frac{1}{2^s}\right)\zeta(z)
        = 1 + \frac{1}{3^s} + \frac{1}{5^s} + \frac{1}{7^s} \cdots
    \end{gather}
    
    se puede observar que se a eliminado eliminado todos los elementos de la suma cuyo denominador es
    múltiplos de $2$.
    
    De la misma forma si hacemos
    
    \begin{gather}
        \frac{1}{3^s}\left(1 - \frac{1}{2^s}\right)\zeta(z) 
        = \frac{1}{3^s} + \frac{1}{9^s} + \frac{1}{15^s} + \frac{1}{21^s} + \frac{1}{27^s} + \frac{1}{33^s} + \cdots
    \end{gather}
    
    si ahora hacemos $(6) - (y)$ nos queda
    
    
    \begin{gather}
        \left(1-\frac{1}{3^s}\right)\left(1 - \frac{1}{2^s}\right)\zeta(z)
        = 1 + \frac{1}{5^s} + \frac{1}{7^s} + \frac{1}{11^s} + \frac{1}{13^s}
        + \frac{1}{17^s} + \frac{1}{19^s} + \cdots
    \end{gather}
    
    Al seguir, haciendo todo este procedimiento llegamos a
    \begin{gather}
        \prod_{p\in\mathbb{P}}\left(1 - \frac{1}{p^-s}\right)\cdot
        \sum_{n=1}^{\infty} \frac{1}{n^z} = 1
    \end{gather}
    
    De aquí al dividir entre el primer miembro de la multiplicación se
    tiene
    \begin{gather}
        \sum_{n=1}^{\infty} \frac{1}{n^z}
        = \frac{1}{\prod_{p\in\mathbb{P}}\left(1 - \frac{1}{p^-s}\right)}
        = \prod_{p\in\mathbb{P}}\frac{1}{1 - p^{-s}}
    \end{gather}
\end{proof}

\subsection{Extensión analítica}
Dado que $\zeta(z)$ esta definido paro para $\Re(z) > 1$, al igual
que como se con el factorial de un número $n$ denotado $n!$ se puede, se puede
hacer una extensión analítica utilizando una función donde se extiende
para todo $z\in\mathbb{C}$ interpolando $n!$ para $n\in\mathbb{N}$ con
la función $\Gamma(z)$, para la función $\zeta(z)$ también se puede
hacer lo mismo, una función para su extensión analítica es la siguiente
para $\Re(z) < 1$

\begin{gather}
    \zeta(z) = 2^z\pi^{z - 1}\sin\left(\frac{\pi z}{2}\right)\Gamma(1-z)\zeta(1-z)
\end{gather}

A pesar de eso como se puede observar en la ecuación para $\Re(z) = 1$ $\zeta(z)$
sigue sin converger dado que

\begin{gather}
    \zeta(1) = 2^1\pi^{1 - 1}\sin\left(\frac{\pi 1}{2}\right)\Gamma(1-1)\zeta(1-1) = \zeta(0) = 1 + 1 +1 \cdots = \infty
\end{gather}

por lo que es validad para $z \in \mathbb{C} - \{1\}$
Uno de los aspectos importantes en este punto es donde se anula
la función, de manera trivial se puede observar que esta función
se anula $\pi z/2 = n pi \Rightarrow z = -2, -4, -6$, la cual está
en $\Re(z) <= 0$, pero para $0 < Re(z) < 1$, ¿donde se anula
esta función? con algunos cálculos se a llegado en que lo
hace en $z = 1/2$, pero como $z \in \mathbb{C}$ hay
más valores donde $\zeta(z)$ se anula en este intervalo, esto
se verá en la proxima sección.

\section{Aplicaciones}
\subsection{La hipótesis de Riemann}
El enunciado de la hipótesis de Riemann es la siguiente

\textit{La parte real de todo cero no trivial de la función
zeta de Riemann es $1/2$}

Este es uno de los 7 problemas del milenio aun sin resolver, el cual ahora
se puede entender a la perfección dado que se conoce la función zeta de
Riemann ademas que un punto no trivial donde se anula esta función esta función
es en $z = 1/2$ con cálculos computacionales se a encontrado
otros valores con $z \in \mathbb{C}$ donde se anula está función
donde $\Re(z) = 1/2$.

En el siguiente gráfico si interpretamos como los cambios de color
donde convergen en un punto en sentido antihorario la función se anula
se puede ver que están los ceros triviales y ademas en la recta
formada por los puntos deo $\Re(z) = 1/2$ en el plano complejo,
lo que da indicios de que esta hipótesis podría ser cierta

\begin{center}
    \includegraphics[scale=0.25]{img/f1.png}
\end{center}

\subsection{Logaritmo integral}
Sea $\pi(x)$ la función contadora de primos, es decir esta función
aproxima cuantos primos existen por debajo de $x$, una aproximación
a esta función es $x/\ln(x)$, a raíz de esta función existen otras
aproximaciones y una de estas es el logaritmo integral, esta es una función que aproxima cuantos
números primos existen menores que un $x$ dado y se define
de la siguiente manera
\begin{equation*}
    li(x) = \int_{0}^{x} \frac{1}{\ln(t)}dt
\end{equation*}

En el siguiente gráfico se puede observar las diversas aproximaciones
a la función contadora de primos.


\begin{center}
    \includegraphics[scale=0.5]{img/f2.png}
\end{center}

\subsection{Función de Riemann}
Es la función que mejor aproxima a la cantidad de
primos que existe de 0 asta $x$, esta función se define por
\begin{gather}
    R(x) = li(x) - \ln(2) - \sum_{\rho} li(x^\rho)
\end{gather}

donde $\rho$ es los ceros de la función $\zeta(z)$, mientra
mas valores se encuentre para $\rho$ mejor será la aproximación $R(x)$,
por lo que la hipótesis de riman es uno de los mas importares problemas matemáticos
en la actualidad. En la siguiente gráfica se puede observar la aproximación
de $R(x)$ con $50$ ceros no triviales

\begin{center}
    \includegraphics[scale=0.4]{img/f3.png}
\end{center}

\section{Referencias}
Arfken, G., Weber, H. y Harris, F. (2013). Methods for Physicists (7ma ed.). Oxford: Elsevier.

Mates Mike. (2021, 25 de noviembre). El Patrón de los Números Primos y la Hipótesis de Riemann [Archivo de video]. Recuperado de https://www.youtube.com/@MatesMike

Sáenz, E. [derivado]. (2019, 29 de mayo). La función Zeta de Riemann | La Hipótesis de Riemann - Parte 1. Recuperado de https://www.youtube.com/@Derivando

Sáenz, E. [derivado]. (2019, 5 de junio). La función Zeta de Riemann | La Hipótesis de Riemann - Parte 2. Recuperado de https://www.youtube.com/@Derivando
