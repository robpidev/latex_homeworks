\section{Introducción}
Al científico chino chang Heng
se le atribuye la inveción en el año 132 d.C, 
hace casi dos mi años, del primer sismoscopio funcional

\section{Teoría elástica}
Describe el comportamiento de materiales sometidos a tensiones y deformaciones dentro del límite elástico, utilizando ecuaciones matemáticas para relacionar fuerzas, desplazamientos y propiedades del material.

\subsection{Tensión (\textit{Stress})}
La tensión es la fuerza interna por unidad de área que actúa dentro de un material:
\[
\sigma = \frac{F}{A}
\]
Donde:
\begin{itemize}
    \item \(\sigma\): tensión (Pa o N/m\(^2\)).
    \item \(F\): fuerza aplicada (N).
    \item \(A\): área de la sección transversal (m\(^2\)).
\end{itemize}

En un sistema tridimensional, se representa mediante el \textbf{tensor de tensiones}:
\[
\sigma_{ij} = 
\begin{bmatrix}
\sigma_{xx} & \tau_{xy} & \tau_{xz} \\
\tau_{yx} & \sigma_{yy} & \tau_{yz} \\
\tau_{zx} & \tau_{zy} & \sigma_{zz}
\end{bmatrix}
\]
Donde:
\begin{itemize}
    \item \(\sigma_{xx}, \sigma_{yy}, \sigma_{zz}\): tensiones normales.
    \item \(\tau_{xy}, \tau_{yz}, \tau_{zx}\): tensiones cortantes.
\end{itemize}

y están dadas por
\begin{equation}
    \sigma_{ij} = \lim_{A_j \to 0 }\frac{F_{i}}{A_j}
\end{equation}

\subsection{Deformación (\textit{Strain})}
La deformación es el cambio relativo en la forma o tamaño de un material. En una dimensión:
\[
\varepsilon = \frac{\Delta L}{L}
\]
Donde:
\begin{itemize}
    \item \(\varepsilon\): deformación (sin unidades).
    \item \(\Delta L\): cambio de longitud (m).
    \item \(L\): longitud original (m).
\end{itemize}

En 3D, se utiliza el \textbf{tensor de deformaciones}:
\[
\varepsilon_{ij} = 
\begin{bmatrix}
\varepsilon_{xx} & \gamma_{xy} & \gamma_{xz} \\
\gamma_{yx} & \varepsilon_{yy} & \gamma_{yz} \\
\gamma_{zx} & \gamma_{zy} & \varepsilon_{zz}
\end{bmatrix}
\]
Donde:
\begin{itemize}
    \item \(\varepsilon_{xx}, \varepsilon_{yy}, \varepsilon_{zz}\): deformaciones normales.
    \item \(\gamma_{xy}, \gamma_{yz}, \gamma_{zx}\): deformaciones cortantes.
\end{itemize}

Donde

\begin{equation}
    \varepsilon_{ij} = \frac{\partial u_i}{\partial x_j}
\end{equation}

Además se tiene las siguientes relaciones

\begin{equation}
    \varepsilon_{yy} = -v\varepsilon \wedge \varepsilon_{zz}
    = - v \varepsilon_{xx}
\end{equation}

\subsection{Ley de Hooke Generalizada}
La relación entre tensiones y deformaciones para materiales elásticos lineales está dada por:
\[
\sigma_{ij} = \sum_{k=1}^3 \sum_{l=1}^3 C_{ijkl} \varepsilon_{kl}
\]
Donde:
\begin{itemize}
    \item \(C_{ijkl}\): tensor de rigidez elástica.
\end{itemize}

Para materiales isotrópicos, se simplifica utilizando el \textbf{módulo de elasticidad} (\(E\)) y el \textbf{coeficiente de Poisson} (\(\nu\)):
\[
\varepsilon_{xx} = \frac{1}{E} \left( \sigma_{xx} - \nu (\sigma_{yy} + \sigma_{zz}) \right)
\]

\subsection{Energía de Deformación}
La energía almacenada por unidad de volumen en un material elástico es:
\[
U = \frac{1}{2} \sigma_{ij} \varepsilon_{ij}
\]