\section{Introducción}
El estudio del gas de fotones constituye un tema central en la física
estadística y la mecánica cuántica. Los fotones, partículas elementales
de la radiación electromagnética, son bosones que obedecen la estadística
de Bose-Einstein. A diferencia de las partículas masivas, los fotones no
tienen masa en reposo ni número de partículas fijo, lo que los distingue
de otros sistemas termodinámicos. 

El interés por este sistema surge en parte de su relevancia histórica en
la resolución del problema de la radiación del cuerpo negro a finales del
siglo XIX y principios del siglo XX\@. Este problema llevó al desarrollo de
la teoría cuántica, comenzando con el trabajo pionero de Max Planck en 1900.
La descripción del gas de fotones es también crucial en la comprensión de
fenómenos como la radiación de fondo de microondas, que proporciona
evidencia clave del modelo del Big Bang en cosmología, así como en el
diseño de tecnologías modernas como láseres y celdas fotovoltaicas.

En este trabajo, se explorarán las propiedades fundamentales del gas de
fotones.