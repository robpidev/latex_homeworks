\section{El oscilador armónico simple}
El hamiltoniano en una dimensión está dado por
\begin{equation}
    H = \frac{1}{2}Kx^2 + \frac{p^2}{2m}
    = \frac{1}{2}Kx^2 - \frac{\hbar^2}{2m}\frac{d^2}{dx^2}
\end{equation}

donde $m$ es la masa de la partícula y $K$ la constante elástica.
Se tiene una solución no trivial a la ecuación de Schrödinger para
$n = 0,1,\ldots,\infty$ con energía.

\begin{equation}
    \label{eq:sho:energy}
    E_n = \hbar\omega\left(n + \frac{1}{2}\right)
\end{equation}

La frecuencia angular $\omega$ es identico al valor classico.
\begin{equation}
    \label{eq:angular:frecuency}
    \omega = \sqrt{\frac{K}{m}}
\end{equation}

Como la función de partición está dada por

\begin{equation*}
    Z = \sum_{n=0}^{\infty} \exp(-\beta E_n)
\end{equation*}

Con $\beta = 1/k_B T$ remplazando se tiene

\begin{equation}
   Z  = \sum_{0}^{\infty} \exp(-\beta\hbar\omega(n + 1/2))
      = \frac{\exp(-\beta\hbar\omega/2)}{1 - \exp(-\beta\hbar\omega)}
\end{equation}

Donde se a utilizado el hecho que
\begin{equation*}
    \sum_{n = 0}^{\infty} x^{-n}= \frac{1}{1 - x}
\end{equation*}

Utilizando física estadística para calcular la energía tenemos

\begin{align}
    U = - \frac{\partial}{\partial} \ln{Z}
     = \frac{1}{2}\hbar\omega + \frac{\hbar\omega}{\exp(\beta\hbar\omega) - 1}
\end{align}

Comparando con la energía media
\begin{gather}
    \langle E_n \rangle = \frac{1}{2}\hbar\omega + \hbar\omega\langle n \rangle
\end{gather}

Se puede ver que el número promedio de excitaciones del oscilador armónico es
\begin{gather}
    \langle n \rangle = \frac{1}{\exp(\beta\hbar\omega) - 1}
\end{gather}