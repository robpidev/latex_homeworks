\section{Gas de fotones}

\subsection{Cuerpo Negro}
En física, la expresión ``cuerpo negro'' se refiere a un objeto que absorbe
toda la radiación que incide sobre él y no refleja nada. Es, por supuesto,
una idealización, pero una que se puede aproximar muy bien en el laboratorio.

Un cuerpo negro no es realmente negro. Aunque no refleja la luz, puede
irradiar luz derivada de su energía térmica. Esto es, por supuesto,
necesario si el cuerpo negro ha de estar alguna vez en equilibrio térmico
con otro objeto.

Planck investigó las consecuencias de la suposición de que la luz sólo podía
aparecer en cantidades discretas dadas por la cantidad

\begin{equation}
    \label{eq:wave:energy}
    \Delta \epsilon_\omega = h\nu = \hbar\omega
\end{equation}

Dónde $\nu$ es la frecuencia, $\omega = 2\pi\nu$, $h$ constante de Planck
y $\hbar = h/2\pi$

\subsection{Modelo de solución}
Consideraremos una cavidad cúbica con dimensiones $L\times L\times L$.
Los lados están hechos de metal y contiene radiación electromagnética,
pero no importa. La radiación solo puede entrar y salir de la cavidad a
través de un orificio muy pequeño en un lado. Dado que la radiación debe
ser reflejada por las paredes muchas veces antes de regresar al agujero,
podemos suponer que ha sido absorbida en el camino, haciendo de este
objeto, o al menos del agujero, un cuerpo negro. Lo único que hay dentro
de la cavidad es radiación electromagnética a temperatura T.

Deseamos encontrar el espectro de frecuencia de la energía almacenada
en esa radiación, que también nos dará el espectro de frecuencia de la luz
emitida por el agujero.

\subsection{Dos tipos de cuantización}
Al analizar el modelo simple descrito en la sección anterior,
debemos ser conscientes de los dos tipos de cuantificación que
entran en el problema. Como resultado de las condiciones de contorno
debidas a las paredes metálicas del contenedor, se cuantifican las
frecuencias de las ondas estacionarias permitidas. Se trata de un efecto
totalmente clásico, similar a la cuantización de la frecuencia en las
vibraciones de una cuerda de guitarra.

La segunda forma de cuantización se debe a la mecánica cuántica,
que especifica que la energíaalmacenada en una onda electromagnética
con frecuencia angular $\omega$ viene en múltiplos de $\hbar \omega$
(generalmente llamados
fotones). La teoría de la electrodinámica nos da la ecuación de onda
para el campo eléctrico $\vec{E}(\vec{r})$ esta dado por

\begin{equation}
    \nabla^2 \vec E (\vec{r}, t)
    = \frac{1}{c^2} \partial_{t}^2 \vec{E}(\vec{r}, t)
\end{equation}

donde $c$ es la velocidad de la luz y $\partial_q^n = \partial^n/\partial_{q^n}$

Tomando en cuanta que

\begin{equation}
    \vec \nabla \equiv
    \left(\partial_x, \partial_y, \partial_z \right)
\end{equation}

Entonces

\begin{equation}
    \nabla^2 = \vec\nabla\cdot \vec\nabla
    = \partial_x^2 + \partial_y^2 + \partial_z^2
\end{equation}

Usando la simetría del modelo podemos encontrar la sigueinte solución

\begin{gather}
    \label{eq:electric:field:solution}
    E_x(\vec{r}, t) = E_{x,0}\sin(\omega t) \cos(k_x x) \sin(k_y y) \sin(k_z z)\\
    E_y(\vec{r}, t) = E_{y,0}\sin(\omega t) \sin(k_x x) \cos(k_y y) \sin(k_z z)\\
    E_z(\vec{r}, t) = E_{z,0}\sin(\omega t) \sin(k_x x) \cos(k_y y) \cos(k_z z)
\end{gather}

Dónde $E_{x_i, 0}$, son las aplitudes del campo electromagnético en la cavidad
correspondiente.

Por las condiciones de contotorno las ecuaciones anteriores las ecuaciones
~\ref{eq:electric:field:solution} a 11 se tiene que

\begin{gather}
    k_x L = n_x\pi\\
    k_y L = n_y\pi\\
    k_z L = n_z\pi
\end{gather}

Donde $n_x, n_y, n_z \in \mathbf{Z}^+$, ya que para los negativos tienen la
misma solución.

Utilizando la relación entre la frecuencia y número de onda
\begin{equation}
    k_x^2 + k_y^2 + k_z^2 = \frac{\omega^2}{c^2}
\end{equation}

Y remplazando las ecuaciones de 12 a 14

\begin{equation}
   {\left(\frac{n_x \pi}{L}\right)}^2
   + {\left(\frac{n_y \pi}{L}\right)}^2
   + {\left(\frac{n_z \pi}{L}\right)}^2
   = \frac{\omega^2}{c^2}
\end{equation}

tomando en centa la $\vec{n}$-dependencia para $\omega$ podemos escribir

\begin{equation}
    \omega_{\vec{n}}
    = (n_x^2 + n_y^2 + n_z^2){\left(\frac{\pi c}{L}\right)}^2
\end{equation}

Por lo que obtiene

\begin{equation}
    \omega_{\vec{n}}
    = \frac{\pi c}{L} \sqrt{n_x^2 + n_y^2 + n_z^2}
    = \frac{n \pi c}{L} = \omega_n
\end{equation}

Donde $n = |{\vec{n}}|$

\subsection{Espectro de energía del cuerpo negro}
El primer paso para calcular el espectro de energía del cuerpo negro es
encontrar la densidad de estados. A partir de las soluciones a la ecuación
de onda en la Sección anterior, expresamos los modos individuales en términos
de los vectores $\vec{n}$. Nótese que la densidad de puntos en el espacio 
$\vec{n}$ es uno, ya que los componentes de $\vec{n}$ son todos enteros.

\begin{equation}
    P_{\vec{n}}(\vec{n}) = 1
\end{equation}

Par encontrar la dencidad de estados $P_\omega(\omega)$,
solo se necesita integrar la ecuación anterior en el $\vec{n}$-espacio

\begin{equation}
    P_\omega(\omega) = 1 \frac{1}{8}
    \int_{0}^{\infty} 4\pi n^2 dn \delta(\omega - nc\pi / L) 
    = \pi {\left(\frac{L}{c\pi}\right)}^3 \omega^2
\end{equation}

El factor de 2 es para las dos polarizaciones de la radiación electromagnética,
y el factor de 1/8 corrige para contar los valores positivos y negativos
de los componentes de n. Dado que cada fotón con frecuencia $\omega$ tiene
energía $\hbar\omega$, la energía promedio se puede encontrar sumando todos los
números de fotones ponderados por el factor de Boltzmann $\exp(-\beta \omega \hbar)$.
Dado que esta suma es formalmente idéntica a la del oscilador armónico simple, podemos
escribir la respuesta.

\begin{equation}
    \langle \epsilon_\omega \rangle
    = \frac{\hbar\omega}{\exp(\beta\hbar\omega) - 1}
\end{equation}

Asi con las ecuaciones anteriores se tiene la dencidad de energía

\begin{equation}
    u_\omega = \frac{1}{V} \pi {\frac{L}{c\pi}}^3
    \omega^2 \frac{\hbar\omega}{\exp(\beta\hbar\omega) - 1}
    = \frac{\hbar}{\pi^2c^3}\omega^3(\exp(\beta\hbar\omega) - 1)^{-1}
\end{equation}

\subsection{Energía Total}
La energía mecánica cuántica total en la radiación de la cavidad a la temperatura $T$ se da sumando la energía media en cada modo.

\begin{equation}
    U = 2 \sum_{\vec{n}} \langle \epsilon_{\vec{n}} \rangle
    = 2 \sum_{\vec{n}} \hbar\omega_{\vec{n}}
    \frac{1}{\exp(\beta\hbar\omega) - 1}
\end{equation}

Esta ecuación se restringe al espacio octantante positivo para evitar
la doble contabilidad, y el factor de dos explicaciones para la doble
polarización asociada con todos los modos espaciales.
Volvemos a utilizar el hecho de que el espectro de frecuencias es
continuo para escribir la suma en una integral, que podemos evaluar
explícitamente utilizando la densidad de estados, $P_\omega(\omega)$, Encontrada

\begin{align}
    U &= \int_{0}^{\infty}\langle \epsilon\omega \rangle  P_\omega(\omega) d\omega\\
    &= \pi {\left(\frac{L}{c\pi}\right)}^3
    \int_{0}^{\infty}\langle \epsilon_\omega \rangle \omega^3 d\omega\\
    &=\pi {\left(\frac{L}{c\pi}\right)}^3
    \int_{0}^{\infty} \frac{\hbar\omega}{exp(\beta\hbar\omega - 1)} \omega^2 d\omega
\end{align}

Tomando $x = \beta \hbar \omega$

\begin{equation}
    U = \pi \beta^{-1}{\left(\frac{L}{\beta\hbar\pi c}\right)}
    \int_{0}^{\infty}dx \frac{x^3}{e^x - 1}
\end{equation}

Esta ultima integral es conocida y es $\pi^4/15$

Por lo tanto la energía está dada por

\begin{equation}
    U = u {\left(\frac{\pi^2}{15\hbar^3c^3}\right)}V^3{\left(k_B T\right)}^4
\end{equation}

\section{Conclusiones}
Quizás la ocurrencia más famosa de radiación de cuerpo negro es en la
radiación de fondo del universo, que fue descubierta en 1964 por Arno
Penzias (físico alemán que se convirtió en ciudadano estadounidense, 1933-,
Premio Nobel 1978) y Robert Wilson (astrónomo estadounidense, 1936-, Premio
Nobel 1978). Poco después del Big Bang, el universo contenía radiación
electromagnética a una temperatura muy alta. Con la expansión del universo,
el gas de los fotones se enfrió, de la misma manera que un gas de partículas
se enfriaría a medida que aumentara el tamaño del recipiente. Las mediciones
actuales muestran que la radiación de fondo del universo se describe a una temperatura de 2.725K.

Con esta ultima ecuación se puede calcular el resto de los potenciales termodinámicos
segun sea requerido.