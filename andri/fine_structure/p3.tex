\subsection{acoplamiento Spin-Orbita}
Imagínese qué el electrón se encuentra en orbita alrededor
del núcleo, desde punto de vista del electrón, el protón
está rotando al rededor de este (ver figura 1). Esta orbita
de carga positiva configura un campo magnético $\vec{B}$
en el marco del electrón, el cual ejerce un torque
en la rotación del electrón tendiendo a alinear este
momento magnético $\vec{\mu}$ a lo largo de la dirección del campo.
El Hamiltoniano es

\begin{gather}
    \label{eq:hamiltoniano:magnec:moment}
    H = - \vec{\mu} \cdot \vec{B}
\end{gather}

\begin{figure}[h]
    \centering
    \includegraphics[width=0.4\textwidth]{f1.png}
    \caption{Átomo de hidrógeno desde la perspectiva de un electrón}
\end{figure}

Para empezar necesitamos trazar el campo magnético del proton
$\vec{B}$ y el momento dipolar del electrón ($\vec{\mu}$)

El campo magnético del protón pude ser calculado utilizando la ley de Biot-Savart

\begin{equation}
    B = \frac{\mu_0I}{2r}
\end{equation}

Con una corriente efectiva $I = e/T$ donde $e$ es la
carga del electrón y $T$ es el periodo de orbita. Por otro
lado el momento orbital del electrón es $L = rmv = 2\pi mr^2/T$
Además, $\vec{B}$ y $\vec{L}$ apunta en la misma dirección
Así se tiene

\begin{gather}
    \vec{B} = \frac{1}{4\pi\epsilon_0}
        = \frac{e}{mc^2r^3} \vec{L}
\end{gather}

Donde sé a a usado $c = 1/\sqrt{\epsilon_0\mu_0}$
to eliminate $\mu_0$ en vez de $\epsilon_0$

\textbf{Momento dipolar magnético del electrón}. El momento dipolar magnético de una
carga rotando ir relacionado con el momento angular (Spin). El factor de proporcionalidad
es el radio de giro magnético. 
considere primero una carga una carga $q$ que rota con un periodo $T$ en un anillo
de radio $r$, así que se tiene.

\begin{equation}
    \mu = IA = \frac{q\pi r^2}{T}
\end{equation}

Si la masa del anillo es $m$, su momento angular es el momento de inercia $mr^2$
en términos de la velocidad angular $2\pi/T$

\begin{equation}
    S = \frac{2\pi m r^2}{T}
\end{equation}

Expresando el momento dipolar magnético $\vec{\mu}$ en términos de $\vec{S}$
dado que tienen la misma dirección, se tiene

\begin{equation}
    \vec{\mu} = {\left(\frac{q}{2m}\right)} \vec{S}
\end{equation}

Este fue un cálculo puramente clásico, sin embargo; El momento magnético
del electron es dos veces el valor clásico

\begin{equation}
    \vec{\mu}_e = - \frac{e}{m}\vec{S}
\end{equation}

El valor 2 `Extra'  es explicado por Dirac, in su teoría relativista del
electrón.

Remplazando en la ecuación~\ref{eq:hamiltoniano:magnec:moment} obtenemos
entonces

\begin{equation}
    H = \left(\frac{e^2}{4\pi\epsilon_0}\right)\frac{1}{m^2c^2r^3}\vec{S}\cdot\vec{L}
\end{equation}

Sin embargo estos cálculos tienen un serio problema dado qué se an analizado
para un marco de referencia inercial, el cual no lo es ya qué este es
acelerado a medida que el electrón orbita alrededor del núcleo,
por lo que sé hace una corrección cinética adecuada, 
conocida como la precesión de Thomas. En este contexto, arroja un
factor de $1/2$, así la corrección es:

\begin{equation}
    H'_{so} = \left(\frac{e^2}{8\pi\epsilon_0}\right)\frac{1}{m^2c^2r^3}
    \vec{S}\cdot\vec{L}
\end{equation}

Ahora en mecánica cuántica. En presencia del acoplamiento Spin-orbita,
el Hamiltoniano ya no conmuta con $\hat{L}$ y $\hat{S}$, así que
el spin y el momento angular orbital separadamente no se conserva.
sin embargo $H'_{so}$ lo hace con $L^2, S^2$ y el momento angular total.

\begin{equation}
    \hat{J}\equiv\hat{L}+\hat{S}
\end{equation}

Y estas ecuaciones se conservan. Puesto de otra manera los auto-estados
de $L_z$ y $S_z$ no son `Buenos' estados a usar en la teoría de perturbaciones,
pero los auto-estados de $L^2, S^2, J^2$ y $J_z$ lo son. vamos calcularlos

\begin{equation*}
    J^2 = (\hat{L}+\hat{S})\cdot(\hat{L}+\hat{s})
    = L^2 + S^2 + 2\hat{L}\cdot\hat{S}
\end{equation*}

Así que

\begin{equation}
    \hat{L}\cdot\hat{S} = \frac{1}{2}\left(J^2 - L^2 - S^2\right)
\end{equation}

Como estos valores se conoces, entonces

\begin{equation}
    \hat{L}\cdot\hat{S}
    = \frac{\hbar^2}{2}
    [j(j + 1) - l(l + 1) - s(s + 1)]
\end{equation}

en este caso tomando $s = 1/2$. Mientras el valor esperado de
$1/r^3$ es

\begin{equation}
    \bk{\frac{1}{r^3}}
    = \frac{1}{l(l + 1/2)(l + 1)n^3a^3}
\end{equation}

Con lo que se concluye que

\begin{equation}
    E^1_{so} = \bk{H'_{so}} = \frac{e^2}{8\pi\epsilon_0}\frac{1}{m^2c^2}
    \frac{\left(\hbar^2/2\right)[j(j+1) - l(l + 1) - 3/4]}{l(l + 1/2)(l + 1)n^3a^3}
\end{equation}

O expresado en los términos de $E_n$

\begin{equation}
    E^1_{so} = \frac{{(E_n)}^2}{mc^2}
        \frac{n[j(j+1) - l(l + 1) - 3/4]}{l(l + 1/2)(l + 1)}
\end{equation}

Como se puede observar la corrección relativista y el acoplamiento spin-orbita
son del mismo orden $(E_n^2/mc^2)$. Agregando ambos términos se obtiene
la formula de la estructura fina

\begin{equation}
    E^1_{fs} = E^1_r + E^1_{so}
\end{equation}

Teniendo en cuenta $j = l \pm 1/2$ y haciendo los cálculos por
separado para cada signo se llega al mismo
resultado de esta ultima expresión se puede reducir a

\begin{equation}
    E^1_{fs} = \frac{{(E_n)}^2}{2mc^2}
    \left(3 - \frac{4n}{j + 1/2}\right)
\end{equation}

Combinando estos resultados con la formula de Bohr,
se obtiene el resultado

\begin{equation}
    E_{nj} = - \frac{13.6eV}{n^2}
    \left[
        1 + \frac{\alpha^2}{n^2}
        \left(
            \frac{n}{j + 1/2} - 3/4
        \right)
    \right]
\end{equation}

Estos niveles de energía se pueden observar en la figura 2

\begin{figure}
    \centering
    \includegraphics[width=\textwidth]{f2.png}
    \caption{Niveles de energía del átomo de hidrógeno incluyendo la estructura fina (No está en escala)}
\end{figure}

\subsection{Corrección de Darwin}
El electrón relativista experimenta pequeñas fluctuaciones en su posición (efecto Zitterbewegung). Estas fluctuaciones generan un efecto adicional de interacción con el núcleo, que se traduce en una corrección a la energía.

a densidad de probabilidad cuántica asociada al electrón incluye un término adicional en la ecuación de Dirac, lo que lleva a un operador de potencial efectivo:

\begin{equation}
    H_D = \frac{\pi Z e^2 \hbar^2}{2m_e^2 c^2} \delta(\vec{r})
\end{equation}

Este término solo afecta a electrones con 
$l = 0$
(s-orbitales), ya que la densidad de probabilidad en 
$r = 0$ es máxima solo para estos estados.
