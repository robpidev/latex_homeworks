\section{Introducción}
En el estudio de los átomos, la mecánica cuántica ha revelado detalles
fascinantes sobre la estructura energética de estos sistemas. Uno de estos
descubrimientos es la estructura fina, un fenómeno que detalla la división
de los niveles de energía electrónicos en un átomo, que ocurre debido a
correcciones adicionales en la energía. Estas correcciones surgen de
factores cuánticos y relativistas, y son responsables de generar líneas
espectrales adicionales en los espectros atómicos que no pueden explicarse
únicamente con el modelo no relativista de Schrödinger.

En los átomos, la estructura fina resulta de tres aspectos, el primero
el acoplamiento espín-órbita que es la interacción entre el espín intrínseco
del electrón y su momento angular orbital. Segundo, las correcciones relativistas
que son los cambios en la energía de los electrones debido a efectos como
el aumento de masa relativista cuando los electrones se mueven a velocidades
cercanas a la luz. Y por las interacciones electrón-electrón que en átomos
multielectrónicos, la repulsión entre los electrones modifica los niveles
energéticos predichos por modelos más simples.

Estudiar la estructura fina es fundamental debido a que proporciona una
visión más detallada de cómo los electrones se comportan e interactúan
dentro de un átomo. Su estudio permite entender mejor las interacciones
internas del átomo, mejorar la precisión de los modelos cuánticos y además
explorar sistemas multielectrónicos. 

Por otro lado, sabemos que la estructura fina está directamente relacionada
con los espectros atómicos, ya que las divisiones de los niveles de energía
generan líneas espectrales adicionales. Estas líneas son observadas en
experimentos de espectroscopia y son clave para identificación de elementos,
la espectroscopía de precisión ,donde el análisis detallado de la estructura
fina mejora la resolución espectral y permite el estudio de propiedades
nucleares y electrónicas, también estas líneas espectrales funcionan en los
relojes atómicos en su precisión que es clave para la  navegación y
telecomunicaciones, y en el desarrollo tecnológico visto que fenómenos
como el efecto láser o los avances en computación cuántica están vinculados
con la manipulación de niveles de energía y transiciones asociadas a la
estructura fina.
