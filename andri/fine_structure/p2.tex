\section{Estructura fina de el átomo de hidrógeno}
El hamiltoniano para el átomo de hidrógeno está dado por

\begin{equation}
    \label{eq:hamiltoniano}
    H = - \frac{\hbar^2}{2m}\nabla^2
    - \frac{e^2}{4\pi \epsilon_0 r}
\end{equation}

Pero este no es del todo correcto dado que existe
perturbaciones como la gravedad y además se tiene la
corrección relativista y el acoplamiento de spin-orbita,
que se detallará a continuación.

\subsection{Corrección relativista}
Del primer termino del Hamiltoniano supone que se presenta
energía cinética:

\begin{equation}
    \label{eq:kinetic:energy}
    T = \frac{1}{2}mv^2 = \frac{p^2}{2m}
\end{equation}

Y utilizando el operador $p \to (\hbar /i) \nabla$ Por lo que
el operador de energía cinética será

\begin{equation}
    T = - \frac{\hbar^2}{2m}\nabla^2
\end{equation}

La ecuación~\ref{eq:kinetic:energy} es la expresión de la energía
cinética clásica, la relativista está dada por

\begin{equation}
    \label{eq:kinetic:energy:rel}
    T = \frac{mc^2}{\sqrt{1 - {(v/c)}^2}} - mc^2
\end{equation}

Y el momento relativista está dado por

\begin{equation}
    \label{eq:momentum:rel}
    p = \frac{mv}{\sqrt{1 - {(v/c)}^2}}
\end{equation}

Se puede ver que utilizando la ecuación~\ref{eq:momentum:rel} junto
a~\ref{eq:kinetic:energy:rel} se tiene que

\begin{equation}
    p^2c^2 + m^2c^4
    = \frac{m^2v^2c^2 + m^2c^4[1 - {(v/c)}^2]}{1 - {(v/c)}^2}
    = \frac{m^2c^4}{1 - {(v/c)}^2}
    = {(T + mc^2)}^2
\end{equation}

así que

\begin{equation}
    \label{eq:kinetic:momentum}
    T = \sqrt{p^2c^2 + m^2c^4} - mc^2
\end{equation}

Pero dado que para un átomo se tiene $p \ll mc$, 
por lo que utilizando series de Taylor se tiene la siguiente
aproximación

\begin{align}
    T &= mc^2\left[\sqrt{1 + {\left(\frac{p}{mc}\right)}^2}\right]
    = mc^2\left[
        1 + \frac{1}{2}{\left(\frac{p}{mc}\right)}^2
        - \frac{1}{8}{\left(\frac{p}{mc}\right)}^4
        \cdots - 1
    \right]\\
    &=
    \frac{p^2}{2m} - \frac{p^4}{8m^3c^2} + \cdots
\end{align}

Por lo que es evidente que la corrección relativista es

\begin{equation}
    \label{eq:relativistic:correction}
    H'_r = - \frac{p^4}{8m^3c^2}
\end{equation}

Para el primer orden de la teoría de perturbación,
la corrección $E_n$ está dado por el valor esperado del
estado no perturbado $H'$ está dado por

\begin{equation}
    E_r^1 = \langle H'_r \rangle
     = - \frac{1}{8m^3c^2} \langle \psi | p^4 | \psi \rangle
     = - \frac{1}{8m^3c^2}\langle p^2\psi | p^2 \psi \rangle
\end{equation}

Así la ecuación de Schrödinger (para los estados no perturbados)
dice

\begin{equation}
    p^2\psi = 2m(E - V)\psi
\end{equation}

y así

\begin{equation}
    E_r^1 = - \frac{1}{2mc^2}\langle{(E - V)^2} \rangle
    = - \frac{1}{mc^2}[E^2 - 2E\bk{V} + \bk{V^2}]
\end{equation}

Pero como para el átomo de hidrógeno $V(r) = -(1/4\pi\epsilon_0)e^2/r$, reemplazando
en la ecuación anterior

\begin{equation}
    E_r^1 = - \frac{1}{2mc^2}
    \left[
        E_n^2
        + 2E_n\left(\frac{e^2}{4\pi\epsilon_0}\right)\bk{\frac{1}{r}}
        + {\left(\frac{e^2}{4\pi\epsilon_0}\right)}^2\bk{\frac{1}{r^2}}
    \right]
\end{equation}

Donde $E_n$ es la energía de Bohr del estado en cuestión, los
resultados para los valores esperados tomando la función
$\psi_{nlm}$ se tiene

\begin{gather}
    \bk{\frac{1}{r}} = \frac{1}{n^2a}\\
    \bk{\frac{1}{r^2}} = \frac{1}{(l + 1/2)n^3a^2}
\end{gather}

Donde $a$ es el radio de Bohr, remplazando en la ecuación anterior
a esta se tiene que

\begin{gather}
    E_r^1 = - \frac{1}{2mc^2}
    \left[
        E_n^2 + 2E_n\left(\frac{e^2}{4\pi\epsilon_0}\right)\frac{1}{n^2a}
        + {\left(\frac{e^2}{4\pi\epsilon_0}\right)}^2
        \frac{1}{(l + 1/2)n^3a^2}
    \right]
\end{gather}

Si usamos el hecho que
\begin{equation}
    a \equiv \frac{4\pi\epsilon_0h^2}{me^2}
\end{equation}

Y remplazando, finalmente tenemos qué

\begin{equation}
    E_r^1 = - \frac{E_n^2}{2mc^2}
    \left[
        \frac{4n}{l + 1/2} - 3
    \right]
\end{equation}

Es evidente que la corrección relativista es menor
qué $E_n$
