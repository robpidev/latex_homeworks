\section{Capacidad de penetración y protección}
\subsection{Partículas Alfa (\( \alpha \))}
\subsubsection{Capacidad de Penetración}
\begin{itemize}
    \item \textbf{Puede penetrar:} No puede penetrar mucho. Se detiene rápidamente.
    \item \textbf{No puede penetrar:}
    \begin{itemize}
        \item Piel humana (se detiene en las capas exteriores de la piel).
        \item Una hoja de papel.
        \item Unos pocos centímetros de aire.
    \end{itemize}
\end{itemize}

\subsubsection{Protección}
Protegido con ropa común o barreras muy delgadas (como una hoja de papel).

\subsection{Partículas Beta (\( \beta^- \) y \( \beta^+ \))}

\subsubsection{Capacidad de Penetración}
\begin{itemize}
    \item \textbf{Puede penetrar:}
    \begin{itemize}
        \item Piel humana (puede penetrar algunos milímetros).
        \item Plásticos delgados, aluminio delgado (alrededor de unos pocos milímetros).
    \end{itemize}
    \item \textbf{No puede penetrar:}
    \begin{itemize}
        \item Placas gruesas de metal (algunos milímetros de aluminio o más).
        \item Vidrio grueso.
        \item Madera gruesa.
    \end{itemize}
\end{itemize}

\subsubsection{Protección}
    Protegido con láminas de plástico o metal fino.

\subsection{Rayos Gamma (\( \gamma \))}

\subsubsection{Capacidad de Penetración}
\begin{itemize}
    \item \textbf{Puede penetrar:}
    \begin{itemize}
        \item Casi cualquier material, pero la densidad y el grosor del material determinarán la profundidad de penetración.
        \item Tejidos humanos (requiere protección pesada para detener completamente).
        \item Varios centímetros de plomo o concreto.
    \end{itemize}
    \item \textbf{No puede penetrar:}
    \begin{itemize}
        \item Barreras extremadamente gruesas de materiales densos como plomo o concreto.
    \end{itemize}
\end{itemize}

\subsubsection{Protección}
Protegido con barreras gruesas de plomo, concreto o agua.

\subsection{Neutrones}

\subsubsection{Capacidad de Penetración}
\begin{itemize}
    \item \textbf{Puede penetrar:}
    \begin{itemize}
        \item Materiales con bajo número atómico, como hidrógeno (agua, plásticos).
        \item Tejidos humanos (requiere protección significativa).
    \end{itemize}
    \item \textbf{No puede penetrar:}
    \begin{itemize}
        \item Materiales ricos en hidrógeno son efectivos para moderar (reducir la energía) neutrones.
        \item Materiales específicos como boro, cadmio y ciertos polímeros absorben neutrones térmicos.
    \end{itemize}
\end{itemize}

\subsubsection{Protección}
Protegido con agua, parafina, polietileno, y combinaciones con boro o cadmio.

\section{Referencias}
Attix, H. (2004). Introduction to Radiological Physics And Radiation Dosimetry. Federal Republic of Germany: WILEY-VCH.  

Burbano, S., Enrique, B. y Gracia, C. (s.f). Física General. Madrid: Tebar.

Young, H. y Freedman, A. (2013). Física universitaria con física moderna(13ava Ed, Vol. 2). México: PEARSON.

Ramos, S. (2018). Relatividad para futuros físicos (1a Ed). México: Universidad Nacional Autónoma de México.
