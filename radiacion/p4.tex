\section{Interacción de la radiación con materiales}

\subsection{Interacción con el agua}

La interacción de la radiación ionizante con el agua es de gran importancia, dado que el agua es un componente principal de los organismos vivos y es ampliamente utilizada en muchas aplicaciones industriales y médicas. A continuación, se describen los mecanismos de interacción y se presentan algunos ejemplos prácticos.

\subsubsection{Mecanismos de interacción}
\begin{enumerate}
    \item \textbf{Ionización y Excitación:}
        La radiación ionizante (rayos gamma, partículas alfa y beta, neutrones) puede ionizar moléculas de agua $(H_2 O)$ arrancando electrones y formando radicales libres y iones.
        La molécula de agua puede ser ionizada para formar un ion positivo $H_2 O^+$ y un electrón libre $(e^-)$
        
        \begin{gather*}
            H_2O \to H_2 O^+ + e^-
        \end{gather*}

    \item \textbf{Radicales Libres: } Los electrones libres pueden interactuar con otras moléculas de agua, produciendo radicales libres, como el radical hidroxilo $(OH)$ y el radical hidrógeno$(H)$:
    \begin{gather*}
        H_2O^+ + H_2O \to H_3O^+ + OH\\
        e^- + H_2O \to H + OH^-
    \end{gather*}

    \item \textbf{Radiólisis del Agua en Reactores Nuclares: }
    En los reactores nucleares, la radiación ionizante puede causar la radiólisis del agua utilizada como moderador refrigerante. La radióliis produce gaces como el hidrógeno $(H_2)$ y el oxígeno $(O_2)$, que deben ser controlados para evitar la acumulación de gases inflamables y la corrosión del sistema: 
    \begin{gather*}
        2H_2O \to 2H_2 + O_2
    \end{gather*}

    \item \textbf{Disociación y Reacciones Químicas: }
    Los radicales libres son muy reactivos y pueden causar una serie de reacciones químicas adicionales, como la formación de presóxido de hidrógeno $(H_2 O_2)$:
    \begin{gather*}
        OH + OH \to H_2O_2
    \end{gather*}
\end{enumerate}

\subsubsection{Aplicaciones}
\begin{enumerate}
    \item \textbf{Radioterapia en Medicina:} En la radioterapia, se utiliza radiación ionizante para destruir células cancerosas. El agua dentro de las células tumorales se ioniza, produciendo radicales libres que dañan el ADN y otras estructuras celulares, lo que lleva a la muerte de las células cancerosas.
    \item \textbf{Esterilización y Conservación de Alimentos:} La radiación ionizante se utiliza para esterilizar equipos médicos y alimentos. La ionización del agua en microorganismos y parásitos presentes en los alimentos causa daños irreparables en su ADN, matándolos o inhibiendo su reproducción, lo que ayuda a conservar los alimentos por más tiempo y a mantenerlos seguros para el consumo.
    \item \textbf{Investigación Científica:} En experimentos de química radiolítica, se estudian las reacciones inducidas por la radiación en soluciones acuosas. Esto ayuda a entender los mecanismos de daño celular en organismos vivos y a desarrollar nuevos métodos para mitigar los efectos dañinos de la radiación.
\end{enumerate}

\subsubsection{Efectos Biológicos}
El daño causado por la radiación ionizante en el agua de las células biológicas puede tener varios efectos:

\begin{enumerate}
    \item \textbf{Daño al ADN:} Los radicales libres producidos por la ionización del agua pueden atacar y romper las cadenas de ADN, lo que puede llevar a mutaciones, cáncer o muerte celular.
    \item \textbf{Muerte Celular:} La acumulación de daño en componentes celulares críticos puede llevar a la apoptosis (muerte celular programada) o necrosis (muerte celular descontrolada).
    \item \textbf{Formación de productos Tóxicos:} Los productos de la radiolisis, como el peróxido de hidrógeno, pueden ser tóxicos para las células y contribuir al daño oxidativo.
\end{enumerate}

\subsection{Interacción con otros materiales}
La interacción de la radiación ionizante con diferentes materiales varía según el tipo de radiación y las propiedades del material. A continuación se describen los principales mecanismos de interacción para partículas alfa, partículas beta, rayos gamma y neutrones con materiales comunes como el aire, los tejidos biológicos y los metales.


\subsubsection{Interacción de partículas Alfa}

Las partículas alfa (\( \alpha \)) son núcleos de helio (\( \text{He}^{2+} \)) con una alta capacidad de ionización y baja capacidad de penetración.

\begin{enumerate}
    \item \textbf{Aire}
    \begin{itemize}
        \item Las partículas alfa ionizan fuertemente las moléculas del aire, perdiendo rápidamente energía.
        \item Tienen un alcance muy corto en el aire, generalmente solo unos pocos centímetros.
    \end{itemize}
    
    \item \textbf{Agua y Tejidos Biológicos:} 
    \begin{itemize}
        \item Ionizan las moléculas de agua y los componentes celulares en los tejidos.
        \item Tienen un alcance muy corto en los tejidos, generalmente unos pocos micrómetros.
    \end{itemize}
    
    \item \textbf{Metales}
    \begin{itemize}
        \item Las partículas alfa no penetran significativamente en los metales debido a su alta densidad y número atómico.
    \end{itemize}
\end{enumerate}

\subsubsection{Interacción de Partículas Beta}
Las partículas beta (\( \beta^- \) y \( \beta^+ \)) son electrones o positrones con una capacidad de ionización moderada y una mayor capacidad de penetración que las partículas alfa.

\begin{enumerate}
    \item \textbf{Aire}
    Las partículas beta ionizan las moléculas del aire, pero pueden viajar varios metros antes de perder toda su energía.
    
    \item  \textbf{Agua y Tejidos Biológicos}
    \begin{itemize}
        \item Las partículas beta ionizan las moléculas de agua y los componentes celulares, causando daños a nivel molecular.
        \item Pueden penetrar varios milímetros en los tejidos.
    \end{itemize}
    
    \item \textbf{Metales: } Las partículas beta pueden penetrar una cierta distancia en los metales, dependiendo de su energía, pero se ven frenadas y dispersadas por los electrones en el material. 
\end{enumerate}


\subsubsection{Interacción de Rayos Gamma}
\begin{enumerate}
    \item \textbf{Efecto fotoeléctrico: }
    Un fotón gamma puede transferir toda su energía a un electrón en el material, arrancándolo de su átomo. Este efecto es más probable en materiales con altos números atómicos y energías de fotón bajas a moderadas.
    
    \item \textbf{Dispersión Compton: }
    Un fotón gamma puede chocar con un electrón y transferirle parte de su energía, cambiando la dirección del fotón. Este efecto es significativo en una amplia gama de energías de fotón y es común en materiales de números atómicos bajos y medios.
    
    \item \textbf{Producción de pares: }
    Un fotón gamma de alta energía (mayor que 1.022 MeV) puede convertir su energía en un par electrón-positrón cerca de un núcleo. Este proceso es más probable en materiales con altos números atómicos y energías de fotón muy altas.
\end{enumerate}


\subsubsection{Iteracción de Neutrones}
Los neutrones no tienen carga eléctrica y pueden penetrar profundamente en los materiales. Interactúan principalmente a través de colisiones con núcleos atómicos.

\begin{enumerate}
    \item \textbf{Dispersión Elástica: }
    Un neutrón choca con un núcleo y transfiere parte de su energía cinética, causando que el núcleo retroceda.
    
    \item \textbf{Dispersión Inelástica: }Un neutrón es absorbido por un núcleo y luego reemitido, generalmente con menor energía, mientras el núcleo se excita y luego emite rayos gamma.
    
    \item \textbf{Captura de Neutrones}
    Un neutrón es absorbido por un núcleo, formando un núcleo más pesado y, a menudo, emitiendo rayos gamma. Esto es significativo en materiales como el boro y el cadmio, que son buenos absorbentes de neutrones.
    
    \item \textbf{Fisión Nuclear: }
    En materiales fisibles como el uranio-235 y el plutonio-239, un neutrón puede causar la división del núcleo en fragmentos más ligeros, liberando una gran cantidad de energía y más neutrones.

\end{enumerate}
