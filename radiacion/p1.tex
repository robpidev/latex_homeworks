\section{Radiación y sus tipos}
La radiación es la emisión y propagación de energía a través del espacio o un medio material en forma de ondas electromagnéticas o partículas. Existen varios tipos de radiación, clasificados en función de su origen, naturaleza y efectos sobre la materia. Principalmente se puede clasificar como: 
\subsection{Radiación electromagnética}

\begin{enumerate}
    \item \textbf{Luz visible}: Parte del espectro electromagnético que puede ser percibida por el ojo humano.
    \item \textbf{Rayos X y rayos gamma}:  Alta energía y frecuencias, capaces de penetrar profundamente en los materiales. Se utilizan en aplicaciones médicas y de diagnóstico.
    Los rayos gamma tienen  energías superiores Los rayos gamma tienen energías superiores a 100 keV, aunque pueden alcanzar energías mucho mayores, en el rango de (MeV).
    Tienen frecuencias myy altas, típicamente en el rango de $10^9$ a $10^{24}$ Hz y longitud de onda $\lambda$ menores a 10 pm.
    \item \textbf{Microondas y radiofrecuencias}: Baja energía y frecuencias, utilizadas en comunicaciones y dispositivos electrónicos.
\end{enumerate}

\subsection{Radiación de Partículas}
\begin{enumerate}
    \item \textbf{Partículas alfa}: $\alpha$: Núcleos de helio (2 protones y 2 neutrones). Tienen una alta masa y carga, lo que les da un poder de ionización alto pero baja penetración.
    \item \textbf{Partículas beta}: $\beta^-\wedge\beta^+$: Electrones $\beta^-$ o positrones $(\beta^+)$ Menos masivas que las partículas alfa, con mayor capacidad de penetración.
    \item \textbf{Neutrones}: Partículas neutras que pueden penetrar profundamente en los materiales y causar activación nuclear.
\end{enumerate}

\subsection{Radiación Ionizante y Radiación no Ionizante}
\subsubsection{Radiación Ionizante}
\begin{itemize}
    \item Capaz de ionizar átomos y moléculas, es decir, puede arrancar electrones de los átomos, creando iones. Esto puede causar daño biológico significativo.
    \item Ejemplos: rayos X, rayos gamma, partículas alfa y beta, neutrones. 
\end{itemize}

\subsubsection{Radiación no Ionizante}
\begin{itemize}
    \item No tiene suficiente energía para ionizar átomos o moléculas, pero puede excitar los electrones a niveles de energía más altos.
    \item Ejemplos: luz visible, microondas, ondas de radio, infrarrojo.
\end{itemize}

\section{Interacción de la radiación con la materia y sus efectos}
\subsection{Ionización y Excitación}
La radiación ionizante puede arrancar electrones de los átomos, creando iones que pueden interactuar químicamente con otros átomos y moléculas, causando daño en tejidos biológicos y materiales.

\subsection{Daño Biológico}
La radiación ionizante puede dañar el ADN y otras estructuras celulares, lo que puede resultar en cáncer, mutaciones genéticas y otros efectos adversos para la salud.


