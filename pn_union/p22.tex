\subsection{Configuraciones Físicas}
La Figura~\ref{fig:union-pn} muestra el esquema simplificado de una unión PN plana.
El área real del dispositivo está definida por la densidad de las líneas de campo que conectan los dos terminales.
Como primera aproximación, el dispositivo puede restringirse a la superficie de separación paralela a la superficie del semiconductor.
El dispositivo se puede fabricar en un semiconductor de tipo P.
Un área circular se define a través de un paso litográfico adecuado en la superficie.
En el área seleccionada, se crea una región de tipo N a través de la difusión de átomos N-dopantes hasta que se alcanza una concentración de donantes mayor que la de los aceptores.
Los métodos más utilizados para el dopaje son la implantación de iones y la deposición gaseosa.
En el primer caso, los átomos dopantes se ionizan y se aceleran hacia la superficie del semiconductor. En el segundo método, los átomos dopantes se vaporizan en la fase gaseosa y son absorbidos por el semiconductor a través de un proceso de solubilidad.

\begin{figure}[h]\label{fig:union-pn}
    \centering
    \includegraphics[width=\textwidth]{img/f5.png}
    \caption{Esquema principal de una unión PN en tecnología plana.
    El sustrato básico es un silicio de tipo P,
    por lo que el pocillo de tipo N se forma en un volumen donde la concentración de donantes domina la densidad de aceptores.
    Los contactos óhmicos están formados por implantes de N+ y P+.
    Como se muestra en la vista superior, en aras de la simetría,
    el contacto de tipo P forma un anillo alrededor del tipo N}
\end{figure}

La difusión permite que los átomos dopantes cubran un volumen relevante del semiconductor. Los átomos implantados o absorbidos se difunden a temperatura controlada.

\begin{figure}[ht]
    \centering
    \includegraphics[width=0.8\textwidth]{img/f6.png}
    \caption{
        Boceto ideal de un cruce PN\@.
        El dispositivo está polarizado por un voltaje VA aplicado a través de dos contactos óhmicos.
        En aras de la simplicidad, el material de tipo N se mantiene conectado a tierra.
        La interfaz entre los dos materiales fija el origen de la coordenada x a lo largo de la cual fluye la corriente
    }
\end{figure}