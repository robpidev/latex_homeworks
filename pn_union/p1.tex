\section{Introducción}
La unión PN es uno de los elementos principales de los dispositivos
basados en semiconductores tal como la fabricación de diodos,
transistores de unión bipolar y una gran cantidad de sensores
como los fotodetectores. 

La unión $PN$ ideal es una pieza homogénea de semiconductor
caracterizado por un cambio brusco en el carácter de los átomos dopantes
para $x < 0$ el material es de tipo $P$ dopado por una distribución
uniforme de aceptores $N_A$, mientras para que $x > 0$ es de tipo
$N$ dopado por una distribución uniforme de donantes $N_P$.

Las uniones $PN$ puede ser formadas por cualquier parte de semiconductores
$PN$. En este sentido, los casos destables son tanto
las homo-uniones, formadas por los mismos semiconductores,
como las hetero-uniones, formadas por diferentes semiconductores.
Aquí estudiaremos las homo-uniones que en términos de diagrama de bandas
una homo-unión está formada por dos materiales que comparten la
misma afinidad y banda próvida, pero con diferentes funciones de
trabajo. 
