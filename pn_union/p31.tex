\subsection{Corriente ideal}
El modelo de corriente ideal es una consecuencia del cambio en la
concentración de carga minoritaria en los límites de las zonas
neutras debido al voltaje aplicado. El cálculo se lleva a cabo
utilizando los supuestos principales de cuasiequilibrio y límite
de inyección bajo. De acuerdo con la aproximación de cuasi-equilibrio,
incluso bajo polarización, tanto las concentraciones de electrones
como las de huecos se pueden calcular a través de las ecuaciones
de equilibrio estadístico. Por otro lado, en cuanto a la hipótesis
del límite de inyección bajo, la concentración excesiva de carga
minoritaria en una región es mucho menor que la concentración de
carga mayoritaria correspondiente. En equilibrio, la densidad de
las cargas minoritarias es aproximadamente 11 órdenes de magnitud
menor que la densidad de las cargas mayoritarias. Por lo tanto,
la condición de baja inyección no se cumple hasta después de que
la carga excesiva alcance este nivel. Esta condición permite
considerar la versión simplificada de la función de recombinación
de generación

En el equilibrio ($V_A = 0$), el número de cargas mayoritarias
en las coordenadas $x_n$ y $-x_p$ es:

\begin{equation}
    n_{n0}(x_n) = N_D; \qquad p_{p0}(-x_p) = N_A
\end{equation}

Mientras que el número de cargas minoritarias es:

\newcommand{\ffd}{\exp\left(-\frac{q\phi_i}{kT}\right)}

\begin{equation}
    n_{p0}(-x_p) = n_{n0}(x_n)\ffd
    = N_D \ffd = \frac{n_i^2}{N_A}
\end{equation}

y

\begin{equation}
    p_{n0}(-x_p)\ffd = N_A \ffd = \frac{n_i^2}{N_D}
\end{equation}

El subíndice $p$ indica los lados de la unión,
mientras que el subíndice $0$ representa el equilibrio.

Con un sesgo insuficiente ($V_A=0$), la concentración de las tasas
minoritarias en la frontera de la zona neutra se transforma en:

\begin{equation}
    n_p(-x_p) = N_D \exp\left(
        - \frac{q(\phi_i - V_A)}{kT}
    \right)
    > \frac{n_i^2}{N_A}
\end{equation}

Junto a

\begin{equation}
    p_n(x_n) = N_A \exp\left(
        - \frac{q(\phi_i - V_A)}{kT}
    \right)
    > \frac{n_i^2}{N_A}
\end{equation}


La creación de este exceso de tasas minoritarias en la frontera con las
zonas neutras es, en realidad, el principal efecto de la tensión aplicada,
que mantiene constante su concentración.
Por lo tanto, a pesar de que estas cargas se eliminan por recombinación,
su concentración en $x_n$ y $x_p$ permanece constante.
Los gastos de concentración excesiva pueden escribirse como:

\begin{align*}
    n'_p &= n_p - n_{p0} = N_D \exp\left(
        - \frac{q(\phi_i - V_A)}{kT}
    \right)
    - N_d \ffd\\
    &= N_D \ffd \left[
        \exp\left(
            \frac{qV_A}{kT} - 1
        \right)
    \right]
\end{align*}

y

\begin{align*}
    p'_n &= n_p - n_{p0} = N_D \exp\left(
        - \frac{q(\phi_i - V_A)}{kT}
    \right)
    - N_A \ffd\\
    &= N_A \ffd \left[
        \exp\left(
            \frac{qV_A}{kT} - 1
        \right)
    \right]
\end{align*}


El exceso de carga minoritaria (electrones en la región de tipo P y huecos
en la de tipo N) que se crea en el borde de las zonas neutras,
el campo eléctrico en la zona neutra es nulo. En consecuencia,
la corriente recogida en el electrodo es corriente de difusión.
Esto significa que para que se pueda medir una corriente constante,
se debe establecer un gradiente constante de cargas minoritarias en
las zonas neutras. El gradiente de las cargas minoritarias es una
consecuencia tanto de la recombinación como de la difusión.
El exceso de cargas minoritarias activa los procesos de recombinación,
que tienden a restablecer el equilibrio. Bajo la hipótesis del límite
de inyección bajo, la función de recombinación de generación se puede
escribir de forma simplificada, de modo que la función U en los dos
lados de la unión es:

\begin{equation}
    U_n = {(R - G)}_n = \frac{n_p'}{\tau_n}; \qquad
    U_p = {(R - G)}_p = \frac{p'_n}{\tau_p}
\end{equation}

Calculemos ahora la corriente debida a los orificios inyectados en el lado de tipo N de la unión. Los cálculos son simétricos para los electrones en el otro lado del dispositivo. Para calcular el efecto de la recombinación, partamos de la ecuación de continuidad de los agujeros:

\newcommand{\pd}[2]{\frac{\partial#1}{\partial#2}}
\newcommand{\pdd}[3]{\frac{\partial^#1#3}{\partial#2^#1}}


\begin{equation}
    \pd{p_n}{t} = - \frac{1}{q}\pd{J_p}{x} - U  
\end{equation}

La concentración total de agujeros es $p_n = p'_n + p_{n0}$.
Dado que el campo eléctrico es nulo, la corriente es una corriente de
difusión: $J_p = - q D_p \pd{p_n}{x}$. Además, al ser uniforme en el
dopaje, los derivados de pn0 con respecto al tiempo y el espacio son
nulos. Reemplazando la función de combinación, la corriente de difusión
y la concentración total de agujeros en la ecuación de continuidad,
obtenemos:

\begin{equation}
    \pd{p'_n}{t} = D_p \pdd{2}{x}{p'_n} - \frac{p'_n}{\tau_p}
\end{equation}

Restrinjamos el análisis al caso del estado estacionario, es decir,
a la corriente continua. Al final de este capítulo se darán algunas
consideraciones sobre los fenómenos transitorios.
El estado estacionario significa que $\pd{p'_n}{t}$,
por lo que la ecuación de continuidad para los agujeros inyectados
en la región de tipo N se reduce a:

\begin{equation}
    \pdd{2}{x}{p_n'} = \frac{p_n'}{L_p^2}
\end{equation}

La cantidad $L_p = \sqrt{D_p\tau_p}$ se conoce como longitud de
recombinación. Es una medida de la profundidad de penetración del agujero
en el interior del semiconductor de tipo N. La solución de la ecuación
de continuidad es una combinación lineal de dos funciones exponenciales:

\begin{equation}
    p'_n(x) = A \exp\left(
        \frac{x - x_n}{\sqrt{D_t\tau_p}}
    \right)
    + B \exp\left(
        - \frac{x - x_n}{\sqrt{D_p\tau_p}}
    \right)
\end{equation}

Que también se puede escribir en funciones hiperbólicas

\begin{equation}
    p'_n(x) = A^* \sinh(\frac{x - x_n}{L_p})
    + B^* \cosh(\frac{x - x_n}{L_p})
\end{equation}

Donde  $A^* = (A - B)/2$ y $B^* = (A + B)/2$.
La corriente se calcula directamente a partir de la definición de
corriente de difusión. Es proporcional a la derivada del perfil
de concentración de exceso de pozo, que se calcula imponiendo
condiciones de contorno a la Ecuación anterior. La primera restricción
se refiere a la concentración excesiva del orificio en $x_n$,
mientras que la segunda está fijada por la posición del electrodo.
En el electrodo, se eliminan todas las cargas excesivas, de modo que
$p'_n(WB) = 0$. Esto significa que:

\begin{gather}
   p'_n(x = x_n) \to B^* = p'_n(x_n)\\ 
   p'_n(W_B) = 0 \to A^* = - \frac{p'_n(x_n)}{\tanh(\frac{W_B}{L_p})}
\end{gather}

Finalmente, la distribución de exceso de huecos esta dado por:

\begin{equation}
    p'_n(x) = p'_n(x_n)
    \left[
        \cosh(\frac{x - x_n}{L_p})
        - \frac{\sinh\left(\frac{x-x_n}{L_p}\right)}{
            \tanh\left(
                \frac{W_B}{L_p}
            \right)
        }
    \right]
\end{equation}

El comportamiento analítico de las cargas excesivas depende de la
distancia entre el electrodo y la capa de agotamiento con respecto a
la longitud de recombinación. La función aproxima una exponencial como
$W_B \gg  Lp$ y una función lineal como $W_B \ll L_p$.

\subsubsection{Diodo de Base larga}
La condición del diodo de base larga se cumple cuando la distancia entre
el electrodo y la capa de agotamiento es mucho mayor que la longitud de
difusión. De acuerdo con los símbolos de la Fig. 6, el diodo de base larga
corresponde a las siguientes condiciones: material de tipo N $W_B-x_n \gg L_p$
y en el material de tipo P  $-W_E+x_p \gg L_n$.
Las dos zonas son independientes, por lo que es posible hacer un diodo
donde las bases son largas y cortas respectivamente.
el total de corriente es la suma de corrientes $J_N$ y $J_P$ y
está dada por la ecuación

\begin{equation}
    J_{tot} = J_N + J_P 
    = qn_i^2 \left(
        \frac{D_n}{N_A L_n} + \frac{D_p}{N_D L_p}
    \right)
    \left[
        \exp(\frac{qV_A}{kT}) - 1
    \right]
\end{equation}

\subsubsection{Diodo de Base corta}

El otro límite geométrico se produce cuando la distancia entre el electrodo
y la capa de agotamiento es mucho menor que la longitud de difusión.
A saber: $W_B - x_n \ll L_p$ en el material de tipo N y $-W_E + x_p \ ll L_n$
en el material de tipo P. La solución se puede derivar directamente
de la Ec. (42), considerando que la condición de diodo base corta,
corresponde a un pequeño argumento de la función exponencial.
Por lo tanto, la solución de la ecuación de continuidad se puede
linealizar como: $p_n' = A + Bx$. Aplicando las condiciones de contorno:
$p'(x_n) = p'_{n0}$ y $p'_n(W_B) = 0$.

Con esto se tiene una solución similar a la subsection anterior

\begin{equation}
    J_{tot} = J_N + J_P 
    = qn_i^2 \left(
        \frac{D_n}{N_A W'_E} + \frac{D_p}{N_D W'_E}
    \right)
    \left[
        \exp(\frac{qV_A}{kT}) - 1
    \right]
\end{equation}

\section{Referencias}
\begin{itemize}
    \item Di Natale, C. (2023). Introduction to Electronic Devices. Springer.
    \item Veritasium en español. (2025, febrero 7). Por qué es casi imposible hacer luz LED azul [Video]. YouTube. https://youtu.be/KzTm5UmF0Xk
\end{itemize}