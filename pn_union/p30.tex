\section{La corriente en la union PN}

Para calcular la relación corriente-voltaje de una unión PN,
consideremos el dispositivo ideal esbozado en la Fig.4.8.
El dispositivo se compone de una región de carga espacial y
dos regiones neutras en los lados de tipo P y N.
Se proporcionan contactos óhmicos en cada extremo del dispositivo.
El modelo de diodo ideal es mono-dimensional y se calcula la densidad
de corriente $J$, de modo que la corriente medible es $I = JA$,
donde $A$ es el área de la sección transversal del dispositivo.


El voltaje aplicado se encuentra casi por completo a través de la capa de agotamiento.
Esta es una suposición fundamental para el estudio de la corriente a través del dispositivo.
Como consecuencia, los campos eléctricos en ambas zonas neutras son nulos,
por lo que ninguna corriente de deriva puede fluir a través de la masa del semiconductor.
Como resultado del voltaje aplicado, la energía de carga en el semiconductor de tipo P se desplaza en una cantidad $qV_A$,
mientras que la energía de carga en el semiconductor de tipo N permanece sin cambios.
Esto significa que el cambio en la energía de los electrones en el lado del tipo P es $-qV_A$.
De este modo, el potencial incorporado se convierte en $q\phi_i = q(\phi_i - VA)$,
de modo que todos los cuantizaciones que dependen del potencial incorporado
cambian como consecuencia.
Por ejemplo, el tamaño de la capa de agotamiento se convierte en:

\begin{equation}
    x_d = x_n + x_p =
    \sqrt{
        \frac{2(\phi_i - V_A) \e_s}{q}
        \left(
            \frac{1}{N_A} + \frac{1}{N_D}
        \right)
    }
\end{equation}

El potencial incorporado ($\phi_i$) mantiene todo el sistema en equilibrio
cuando la corriente total en cada sección del dispositivo es cero.
En la condición de equilibrio,
las concentraciones de cargas mayoritarias y minoritarias en las zonas
neutrales obedecen a la ley de acción de masas.


El cambio en la barrera de potencial incorporada depende del signo del voltaje aplicado.
La condición $V_A > 0$ se conoce como polarización directa, que corresponde al caso de reducción de la barrera potencial.
El cambio en el potencial hace que la ley ya no sea válida.
La reducción de barreras aumenta la concentración de cargas minoritarias en ambas zonas neutrales.
El potencial incorporado, de hecho, evita que las cargas mayoritarias se difundan en la otra región, donde su concentración es mucho menor.
Por lo tanto, con respecto a la condición de equilibrio, se observa un exceso de cargas minoritarias en las zonas neutras:

\begin{equation}
    n_p \to \frac{n_i^2}{N_A}+n;\quad
    p_n \to \frac{n_i^2}{N_D}+p'
\end{equation}

Dónde $n_p$ es la concentración de electrones y $p_n$ es la concentración
de huecos en el material de tipo P.

El exceso de cargos minoritarios es responsable de la corriente.
Sin embargo, en las regiones entre la capa de agotamiento y los electrodos,
no hay campo eléctrico, por lo que la recolección de carga en los
electrodos no es sencilla. Más bien, la corriente es impulsada por
fenómenos de recombinación.

Bajo la condición de polarización opuesta ($V_A < 0$), conocida como
polarización inversa, la barrera de potencial aumenta, de modo que se
favorece la transferencia de cargas minoritarias de un material a otro.
Debido al diferente número de cargos mayoritarios y minoritarios,
las dos corrientes son obviamente diferentes a nivel cuantitativo.
Esta misma diferencia conduce al carácter rectificador de la unión PN\@.
De hecho, bajo el sesgo directo, la gran población de cargos mayoritarios
se transfiere a la otra región, donde dan lugar a un exceso de cargos
menores, que a su vez se recombinan de acuerdo con la ley de recombinación
de Shockley-Hall-Read (SHR). Por el contrario, con arreglo al sesgo
inverso, la misma transferencia hacia la otra región se refiere a la
pequeña población de cargos minoritarios, lo que se traduce en un aumento
casi insignificante de los cargos mayoritarios. Eventualmente,
las propiedades de la unión PN dependen del destino de las cargas
minoritarias.

La corriente total a través del dispositivo se compone de dos
contribuciones principales. El componente dominante es la corriente
debido al aumento de las tasas minoritarias en la zona neutral.
El segundo término adicional, que es dominante en el sesgo inverso,
se debe a los procesos de generación y recombinación en la región
de carga espacial. La primera contribución a la corriente también
se conoce como corriente de diodo ideal.