\section{Unión $PN$ en el equilibrio}

La formación de la unión ideal puede considerarse como la unión
instantánea de dos materiales separados que se mantienen en
equilibrio a la misma temperatura. Después de unirse,
la combinación de materiales alcanza una nueva condición de equilibrio,
considerando tanto los electrones como los huecos.
Aquí, los electrones migran del material con la función
de trabajo más pequeña, hacia el material con la función de
trabajo más grande, y viceversa para el movimiento de
los agujeros, es decir los electrones parten del material
de tipo $N$ al de tipo $P$, mientras que los huecos
se transfieren del material del tipo $P$ al tipo $N$.
La trasferencia de cargas se vuelve uniforme 
en todo el dispositivo.

Inicialmente, las cargas inyectadas de un material en el otro
son cargas mayoritarias. Pero de lo contrario, se convierten
en cargas minoritarias a medida que llegan a las regiones de
destino, donde se recombinan rápidamente con las cargas
mayoritarias. El resultado de estos procesos es el agotamiento
de las cargas mayoritarias en ambos lados de los cruces.
Esto deja una densidad de cargas fijadas formados por átomos
dopantes ionizados en las proximidades de la unión, está 
densidad dá como consecuencia la formación de un campo eléctrico 
qué impiden que se transfieran más cargas, de modo
que se estable un equilibrio del sistema, donde este equilibrio
significa un equilibrio entre corrientes opuestas.

En el equilibrio todo el sistema está formado por dos regiones
de masa no perturbadas conectadas por una región de carga espacial
(particularmente una capa de agotamiento). La densidad
de carga total $Q$ en cualquier volumen es la suma de las
concentraciones de los 4 tipos de cargas disponibles:
$Q = q(p - n + N_D - N_A)$

Las densidades de carga de cada zona se enumeran en la siguiente table:

\vspace{12pt}
\begin{figure}[h]
    \includegraphics[width=\textwidth]{img/f1.png}
    \caption{Condición de carga en las 3 regiones
del semiconductor.
}
\end{figure}

Tanto el campo eléctrico como el potencial se calcula a partir
de la ecuación de Poisson resuelta en la región
espacial donde la carga total es distinta de 0:

\let\e\epsilon{}
\newcommand{\dif}{\frac{d}{dx}}
\newcommand{\diff}[1]{\frac{d^#1}{dx^#1}}
\newcommand{\E}{\mathcal{E}}


\begin{equation}
    \nabla^2 \phi = -\frac{\rho}{\e_s}
\end{equation}

Pero como se trabajará en una sola dimensión se tiene

\begin{equation}
    \diff{2}\phi = - \frac{\rho}{\e_s}
\end{equation}

Las características electrostáticas de la unión se calculan
mediante la hipótesis de agotamiento profundo, asumiendo
que la carga total en la capa de agotamiento solo es
aportada por los átomos donantes y aceptores

\begin{equation}
    \diff{2}\phi = - \frac{q}{\e_s}(N_D - N_A)
\end{equation}

La distribución de la densidad de carga bajo la hipótesis de
agotamiento profundo se muestra en la figura~\ref{fig:dis-carga}.

\begin{figure}[h]\label{fig:dis-carga}
    \centering
    \includegraphics[width=0.5\textwidth]{img/f2.png}
    \caption{distribución total de la carga bajo la aproximación
    de agotamiento profundo}
\end{figure}

De aquí se puede observar que $+Q$ y $-Q$ está dado por

\begin{equation}
    +Q = qN_D x_n \wedge -Q = q N_A x_p
\end{equation}

Dado que la carga total debe ser nula, se mantiene la siguiente
relación

\begin{equation}
    \label{eq:charge:neut}
    N_D x_n = N_A x_p
\end{equation}

La distribución de carga es la entrada utilizada para calcular
tanto el campo eléctrico como el potencial. El campo eléctrico
viene dad por

\begin{equation}
    \dif \E = - \frac{\rho}{\e_s}
\end{equation}

En el lado de tipo N de la capa de agotamiento $(0 < x < x_n)$,
según la hipótesis del dopaje constante, el campo eléctrico es:

\begin{equation}
    \int_{\E(x)}^{\E(x_n)} d \E = \int_{x}^{x_n} \frac{qN_D}{\e_s} dx
    \implies \E(x_n) - \E(x) = \frac{qN_D}{\e_s} (x_n - x)
\end{equation}

Las condiciones de contorno es $\E(x_n) = 0$, 
de modo que el campo eléctrico en la capa de agotamiento en lado
de tipo $N$ de la capa de agotamiento es

\begin{equation}
   \E_n(x) = q \frac{N_D}{\e_s} (x - x_n) 
\end{equation}

Similarmente, en la parte de tipo $P$ de la región de carga espacial, 
obtenemos

\begin{equation}
    \E_n(x) = - q \frac{N_A}{\e_s} (x + x_p)
\end{equation}

Dado que el material es homogéneo, la permitividad eléctrica
es constante y el campo eléctrico es continuo en la interface
$(\E_n(0) = \E_p(0))$ donde su valor absoluto es máximo

\begin{equation}
    \E_{\max} = \frac{-qN_A}{\e_s}x_p
    = \frac{-qN_D}{\e_s}x_n
\end{equation}

El potencial se calcula a partir del campo eléctrico
$d\phi = \E dx$ en los dos lados de la capa de agotamiento.

Las condiciones de contorno para el cálculo del potencial son: 
$\phi(x \geq x_n) =  \phi_n$ y 
$\phi (x \leq - x_p) = \phi_p$ donde $\phi_n$ $\phi_p$ son
los potenciales en la zona neutral y donde $\phi = E_F -E_i$:

\begin{gather}
   x < 0 \implies
   \phi(x) = -\phi_p + \frac{qN_a}{2\e_s}{(x - x_p)}^2 \\
   x>0 \implies
   \phi(x) =  \phi_n - \frac{qN_D}{2\e_s}{(x - x_n)}^2
\end{gather}

Nótese que el potencial en $x = 0$ es nulo solo si las concentraciones
de aceptores y donantes permanecen iguales $N_A = N_D$.
El potencial nulo se produce cuando el semiconductor se vuelve
intrínseco ($p = n = n_i$). Cabe destacar que
en el caso del dopaje simétrico el potencial se anula
dentro de la capa de agotamiento del semiconductor menos
dopado. Por lo tanto en este lado de la capa de agotamiento
la región situada entre la que obedece la condiciones $phi = 0$
y la interface se puebla con un mayor número de cargar minoritarias
que las cargas mayoritarias. Esta situación se conoce como
inversion y es primordial en las uniones metal-oxido-semiconductores.
La caída neta del potencial en toda la unión es el potencial 
incorporado. Se puede calcular a partir delos potenciales en las
zonas neutras no perturbadas, $\phi_i = \phi_n - \phi_p$, que
dependen de la concentración de las cargas mayoritarias
en las regiones respectivas: 

\begin{equation}
    \phi_n = \frac{kT}{q}\ln\left(\frac{N_D}{N_i}\right) \wedge
    \phi_p = -\frac{kT}{q} \ln \left(
        \frac{N_A}{n_i}
    \right)
\end{equation}

De los cuales  

\begin{equation}
    \phi_i = \frac{kT}{q}\left[
        \ln \left(\frac{N_D}{ni} \right)
        +
        \ln \left(\frac{N_A}{ni} \right)
    \right]
    =
    \frac{k T }{q}
        \ln \left(\frac{
            N_A N_D 
        }{n_i^2}
    \right)
\end{equation}

\begin{figure}[h]
    \centering
    \includegraphics[width=0.5\textwidth]{img/f3.png}
    \caption{Campo eléctrico y potencial en equilibrio}
\end{figure}

El tamaño total de la capa de agotamiento es una consecuencia
de la condición de la continuidad potencial en $x = 0$:

\begin{equation}
    \phi_n - q \frac{N_D}{2\e_s}x_n^2
    = \phi_p + q \frac{N_A}{2\e_s}x_p^2
\end{equation}

Introducción el potencial incorporado obtenemos

\begin{equation}
    \phi_i = \phi_n - \phi_p
     = \frac{q}{2\e_s} \left(
        N_D x^2_n + N_A x_p^2
     \right)
\end{equation}

Remplazando $x_n$ con la expresión de la ecuación~\ref{eq:charge:neut}
se obtiene

\begin{equation}
    x_p = \sqrt{
        \frac{2\phi_i \e_s}{q}
    }
    \frac{N_D}{\sqrt{N_D N_A^2 + N_A N_D^2}}
\end{equation}

De la misma forma


\begin{equation}
    x_n = \sqrt{
        \frac{2\phi_i \e_s}{q}
    }
    \frac{N_A}{\sqrt{N_D N_A^2 + N_A N_D^2}}
\end{equation}

Por lo que la región de agotamiento será

\begin{equation}
    x_d = x_p + x_n =
    \sqrt{\frac{2\phi_i \e_s}{q}
    \left(\frac{1}{N_A} + \frac{1}{N_D}\right)
    }
\end{equation}

