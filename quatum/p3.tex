\section{Construcción de estados acoplados}
El estado $|1, +1\rangle$ debe estar compuesto por $\alpha_1\alpha_2$
por que solo este estado corresponde a $M_S = +1$. De esto se deduce

\begin{equation}
    |1, +1\rangle = \alpha_1 \alpha_2
\end{equation}

Utilizando el operador de descenso $S_{-}$ que ahora se escribe como
\begin{equation}
    S_{-}|S, M_s\rangle =
    {S(S + 1) - M_S(M_S - 1)}^{1/2}\hbar|S, M_S - 1\rangle
    \implies S_{-} |1, +1\rangle = 2^{1/2}\hbar|1, 0\rangle
\end{equation}

Sin embargo, como $S_{-} = s_{1-} + s_{2-}$ también se puede escribir
como
\begin{equation}
    S_{-}|S, M_s\rangle = (S_{-} = s_{1-} + s_{2-})\alpha_1\alpha_2
    = \hbar(\alpha_1\beta_2 + \beta_1\alpha_2)
\end{equation}

La comparación te estos dos resultados resulta en

\begin{equation}
    |1, 0\rangle = \frac{1}{2^{1/2}}(\alpha_1\beta_2 + \beta_1\alpha_2)
\end{equation}

Como se encuentra en la tabla 1. El terser estado de el triple
se obtiene repitiendo el procedimiento
\begin{gather}
    S_{-}|1, 0\rangle = 2^{1/2}\hbar|1, -1\rangle = (s_{1-} + s_{2-})
    \frac{1}{2^{1/2}}(\alpha_1\beta_2 + \beta_1\alpha_2)=2^{1/2}\hbar\beta_1\beta_2\\
    \implies  |1, 0\rangle = \beta_1,\beta_2
\end{gather}

Es fácil de verificar que
\begin{equation}
    \ket{0, 0} = \frac{1}{2^{1/2}}(\alpha_1\beta_2-\beta_1\alpha_2)
\end{equation}
