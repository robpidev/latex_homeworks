\section{El modelo vectorial del momento angular acoplado}
El modelo vectorial de momentos angulares acoplados es un intento
de representar pictóricamente las características de los momentos
angulares acoplados que se a deducido de las relaciones de
conmutación.

Las características que deben expresar los diagramas vectoriales
de momentos acoplados son los siguientes:

\begin{itemize}
    \item La longitud del vector que representa el momento angular total
    es $\{{j(j+1)\}}^{1/2}$, con $j$ uno de los valores permitidos de
    las series de Clebsch-Gordan.

    \item Este vector debe estar en un ángulo indeterminado en un
    cono alrededor del eje $z$ (porque $j_x$ y $j_y$ no se pueden
    especificar si se ha especificado $j_z$).

    \item Las longitudes de los vectores de momento angular
    contribuyentes son ${\{j_1 (j_1 + 1)\}}^{1/2}$ y ${\{j_2 (j_2 + 1)}\}1/2$.
    Estas longitudes tienen valores definidos incluso cuando se
    especifica $j$.

    \item La proyección del momento angular total en el eje $z$ es
    $m_j$; En la imagen acoplada (en la que se especifica $j$),
    los valores de $m_{j1}$ y $m_{j2}$ son indefinidos, pero su suma es igual
    a $m_j$.

    \item En la imagen desacoplada (en la que no se especifica $j$),
    se pueden especificar los componentes individuales $m_{j1}$ y
    $m_{j2}$, y su suma es igual a $m_j$.
\end{itemize}

Las figuras 1 y 2 muestran estos puntos. La figura 1 muestra uno
de los estados de la imagen desacoplada: se especifican tanto
$m_{j1}$ como $m_j2$, pero no hay indicación de la orientación
relativa de $j1$ y $j2$, aparte del hecho de que se encuentran
sobre sus respectivos conos. POr lo tanto, el momento angular total
es indeterminado, ya que podría ser cualquiera de las resultantes
que muestran (a) o (b) o cualquier intermedia. (Nótese,
sin embargo, que la componente $z$ del momento angular) total
está bien definida ya que $m_j = m_{j1} + m_{j2}$. La figura 2
muestra uno de los estados de la imagen acoplada. Ahora bien,
la resultante, el momento angular total, tienen una magnitud
bien definida y la resultante en el eje $z$, pero las componentes
individuales $m_{j1}$ y $m_{j2}$ son indeterminadas.
el modelo vectorial es una visualización de orientaciones posibles
pero no específicos.


\begin{figure}[h!]
    \centering
    \includegraphics[width=0.46\linewidth]{img/1}
    \caption{Dos posibles estados del momento angular total que
    puede surgir de dos momentos contribuyentes especificados
    con números cuánticos $j_1$ y $j_2$. Las orientaciones relativas
    de los momentos contribuyentes en sus conos determinan
    la magnitud total.}

\end{figure}

\begin{figure}[h!]
        \centering
        \includegraphics[width=0.4\linewidth]{img/2}
        \caption{Si los dos momentos contribuyentes están
        bloqueados juntos de modo que dan lugar a un total
        especificado, las proyecciones de los momentos
        contribuyentes abarcan un rango (como se representa en las
        barras verticales) y, aunque se puede especificar
        su suma, no se puede especificar sus valores individuales.}
\end{figure}

Un ejemplo importante, es el caso de dos partículas con espín $s = 1/2$, como
dos electrones. Para cada partícula, $s = 1/2$ y $m_s = ± 1/2$.
En la imagen desacoplada, los electrones pueden estar en cualquiera
de los cuatro estados

\begin{gather*}
    \alpha_1\alpha_2 \qquad \alpha_1\beta_2 \qquad \beta_1\alpha_2
    \qquad \beta_1\beta_2
\end{gather*}

Estos cuatro estados se ilustran en la figura 3 los
momentos individuales se encuentra en posiciones no
especificadas en sus conos y el momento indeterminado.

Consideremos ahora la imagen acoplada. La condición
del triángulo (o la serie de Clebsch-Gorban) nos dice
que el espín total $S$ (Utilizaremos letras mayúsculas
para denotar los momentos angulares de las colecciones
de partículas) pueden tomar los valores 1 y 0.
Cuando $S = 0$, solo hay un valor posible
de su componente $z$, llamado 0, correspondiente
a $M_s = 0$, a este lo llamaremos singlete. Cuando
$S = 1$, $M_s = +1, 0, -1$, al cual llamaremos triplete.

\begin{figure}[h]
    \centering
    \includegraphics[width=0.4\linewidth]{img/3.png}
    \caption{Los cuatro estados desacoplados de un sistema que
    consta de dos partículas de espín $1/2$ (como electrones),
    representados por los conos en los que se encuentran
    los espines individuales}
\end{figure}

El modelo vectorial del triplete se puede ver en en la figura
4. Los conos se han dibujado a escala y varios puntos
deberían ser evidentes. Una es que para llegar a una
resultante correspondiente a $s = 1$ (de longitud 
$2^{1/2}$), utilizando vectores componentes
correspondientes a $s = 1/2$ (de longitud $1/2 \times 3^{1/2}$),
los vectores deben estar en un ángulo definido entre sí.
De hecho, deben estar en el mismo plano vertical, como
se muestra en la figura, ya que solo esa orientación
da como resultado un vector de longitud correcta. Nótese
que aunque se dice que los espines son paralelos 
en un estado de triplete (y se representa $\uparrow \uparrow$),
en realidad están en un ángulo agudo. Los
dos espines forma el mismo ángulo entre si en los tres estados;
que es necesario para que tengan el mismo resultado.

\begin{figure}[h]
    \centering
    \includegraphics[width=0.4\linewidth]{img/4.png}
    \caption{Tres de cuatro estados acoplados
    de un sistema formado por dos partículas de espín ${-1/2}$}
\end{figure}

El modelo vectorial del singlete debe representar un
estado en el que los vectores de momento angular de espín
se suman para dar una resultante cero (Figura 5).
De la figura se deduce claramente que los dos espines
son verdaderamente antiparallel $\uparrow \downarrow$
en este estado. Al igual que los estados tripletes,
solo se fija la orientación relativa de los vectores;
la orientación absoluta alrededor del eje $z$ es
completamente indeterminada.

\begin{figure}[h]
    \centering
    \includegraphics[width=0.4\linewidth]{img/5.png}
    \caption[short]{Estado resultante de dos partículas de
    espín $-1/2$. Este caso corresponde a S = 0}
\end{figure}