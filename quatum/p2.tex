\section{Relación entre estados acoplados}
De la sección anterior se puede deducir que el estado
$|j_1j_2, jm_j\rangle$ se puedo construir a partir de los
valores $m_{j1}$ y $m_{j2}$ tal que $m_{j1} + m_{j2} = m_j$.
Podemos suponer que el estado acoplado es la suma sobre todos
los estados desacoplados $|j_1m_1; j_2m_{j2}\rangle$ tal que
$m_{j} = m_{j1} + m_{j2}$, de aquí podemos deducir
que podemos escribir

\begin{equation}
    |j_1 j_2; jm_j\rangle = \sum_{m_{j1},m_{j2}}{C(m_{j1},m_{j2}) |j_1m_{j1};j_2m_{j2}\rangle}
\end{equation}

Donde el coeficiente $C(m_{j1}, m_{j2})$ es llamado \textbf{
    coeficiente de acoplamiento vectorial
}.

También son llamados \textit{Coeficientes de Clebsch-Gordan}, \textit{Coeficientes de Wigner}
Y en una forma ligeramente modificada los \textit{Símbolos} $3j$.

\section{Estados acoplados Singlete y triplete}
Se ilustrará el uso de los coeficientes de acoplamiento vectorial
considerando los estados singlete y triplete de dos partículas
de espín $-1/2$. Los valores se representan en la tabla 1.

\begin{table}[h!]
    \centering
    \begin{tabular}{|c|c|c|c|c|c|}
        \hline
        $m_{s1}$ & $m_{s2}$ & $|\textbf{1}, +\textbf{1} \langle$
        & $|\textbf{1, 0} \rangle $ & $|\textbf{1, -1}\rangle$ 
        & $|0, 0 \rangle$
        \\
        \hline
        $+1/2$ & $+1/2$ & 1 & 0 & 0 & 0\\
        $+1/2$ & $-1/2$ & 0 & $1/2^{1/2}$ & $1/2^{1/2}$ & 0 \\
        $-1/2$ & $+1/2$ & 0 & $1/2^{1/2}$ & $-1/2^{1/2}$ & 0 \\
        $-1/2$ & $-1/2$ & 0 & 0 &  & 0 \\
        \hline
    \end{tabular}
\end{table}

\begin{align*}
    |\textbf{1, +1} \rangle &= \alpha_1 \alpha_2\\
    |\textbf{1, 0} \rangle &= \frac{1}{2^{1/2}}\alpha_1 \beta_1 + \frac{1/}{2^{1/2}}\beta_1\alpha_2\\
    |\textbf{1, -1} \rangle &= \beta_1\beta_2\\
    |\textbf{0, 0} \rangle &= \frac{1}{2^{1/2}}\alpha_1 \beta_1 - \frac{1/}{2^{1/2}}\beta_1\alpha_2
\end{align*}

Para encontrar los coeficientes multiplicamos el lado izquierdo
de la ecuación (1) por $\langle 1_1m'_{j1};j_2m'_{j2}|$,
por ortogonalidad el único término que sobrevive por la derecha es cuando
$m_{j1} = m'_{j1} \wedge m_{j2}=m'_{j2}$ por lo tanto

\begin{equation}
\langle 1_1m'_{j1};j_2m'_{j2}|j_1 j_2; jm_j \rangle = C(m'_{j1}, m'_{j2})
\end{equation}

El coeficiente $C(m_{j1}, m_{j2})$ puede interpretarse como la medida
en que el estado acoplado $|j_1 j_2; jm_j\rangle$ se asemeja al estado
desacoplado
$|1_1m_{j1};j_2m_{j2} \rangle$

Alternativamente  $C(m_{j1}, m_{j2})$ puede interpretarse como la probabilidad
de que se encuentre los valores $m_{j1}$ y $m_{j2}$ si el estado
$|1_1m_{j1};j_2m_{j2} \rangle$ se inspecciona para determinar
sus valores individuales (en lugar de su suma).