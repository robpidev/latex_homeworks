\section{Dosis Absorbida}

\subsection{Unidad en el SI\@: Gray (Gy)}
\textbf{Definición} La dosis absorbida \(D\) se define como la energía \(E\) depositada por la radiación en una masa \(m\):
  \[
  D = \frac{E}{m}
  \]
  donde \(D\) es la dosis absorbida en grays (Gy), \(E\) es la energía en julios (J), y \(m\) es la masa en kilogramos (kg).  
\\[1em]
\noindent\textbf{Deducción}
  Si \(E = 1 \text{ J}\) y \(m = 1 \text{ kg}\),
  \[
  D = \frac{1 \text{ J}}{1 \text{ kg}} = 1 \text{ Gy}
  \]

\subsection{Unidad en el sistema estadounidense: Rad}
\textbf{Conversión}
  \[
  1 \text{ Gy} = 100 \text{ rad}
  \]
  Esto se deriva de la definición de rad, donde 1 rad = 0.01 Gy.

\section{Dosis Equivalente}

\subsection{Unidad en el SI\@: Sievert (Sv)}
\textbf{Definición} La dosis equivalente \(H\) se define como la dosis absorbida \(D\) multiplicada por un factor de calidad \(Q\) que depende del tipo de radiación:
  \[
  H = D \times Q
  \]
  donde \(H\) es la dosis equivalente en sieverts (Sv), \(D\) es la dosis absorbida en grays (Gy), y \(Q\) es un factor adimensional específico para el tipo de radiación.
\\[1em]
\textbf{Deducción}
  Si \(D = 1 \text{ Gy}\) y \(Q = 1\) (para radiación gamma o X),
  \[
  H = 1 \text{ Gy} \times 1 = 1 \text{ Sv}
  \]

El factor de calidad $Q$ no tiene límite de medida fijo;
en cambio, tiene valores establecidos que dependen del tipo y la
energía de radiación. Estos son algunos valores
\begin{itemize}
    \item Radiación gamma y rayos X: $Q=1$
    \item Electrones y positrones: $Q=1$
    \item Neutrones:
    \begin{itemize}
        \item $E < 10 keV: Q = 5$
        \item $10keV < E < 100 keV: Q = 10$
        \item $100keV < E < 2 MeV: Q = 20$
        \item $2MeV < E < 20 MeV: Q = 10$
    \end{itemize}
    \item Protones: $Q = 2$
    \item Partículas alfa y fragmentos de fisión: $Q = 20$
\end{itemize}
\subsection{Unidad en el sistema estadounidense: Rem}
\textbf{Conversión}
  \[
  1 \text{ Sv} = 100 \text{ rem}
  \]
  Esto se deriva de la definición de De dosis absorbida.

\section{Exposición}

\subsection{Unidad en el SI\@: Coulomb por kilogramo (C/kg)}
\textbf{Definición} La exposición \(X\) se define como la carga eléctrica total de iones de un signo producida en el aire por la radiación por unidad de masa:
  \[
  X = \frac{Q}{m}
  \]
  donde \(X\) es la exposición en coulombs por kilogramo (C/kg), \(Q\) es la carga en coulombs (C), y \(m\) es la masa de aire en kilogramos (kg).
\\[1em]
\textbf{Deducción}
  Si \(Q = 1 \text{ C}\) y \(m = 1 \text{ kg}\),
  \[
  X = \frac{1 \text{ C}}{1 \text{ kg}} = 1 \text{ C/kg}
  \]

\subsection{Unidad en el sistema estadounidense: Becquerel (Bq)}
\textbf{Conversión}
  \[
  1 \text{ R} \approx 2.58 \times 10^{-4} \text{ C/kg}
  \]

\section{Actividad Radiactiva}

\subsection{Unidad en el SI: Becquerel (Bq)}
\textbf{Definición} La actividad \(A\) se define como el número de desintegraciones por segundo:
  \[
  A = \frac{dN}{dt}
  \]
  donde \(A\) es la actividad en becquerels (Bq), \(dN\) es el número de desintegraciones, y \(dt\) es el tiempo en segundos (s).
\\[1em]
\textbf{Deducción}
  Si \(dN = 1\) y \(dt = 1 \text{ s}\),
  \[
  A = \frac{1}{1 \text{ s}} = 1 \text{ Bq}
  \]

\subsection{Unidad en el sistema estadounidense: Becquerel (Bq)}
\textbf{Conversión}
  \[
  1 \text{ Ci} = 3.7 \times 10^{10} \text{ Bq}
  \]
  Esto se basa en la definición de curie, que corresponde a la actividad de un gramo de radio-226.

\section{Ejemplos de Uso en Máquinas y Aplicaciones}

\begin{enumerate}
    \item \textbf{Aceleradores Lineales}
    \begin{itemize}
        \item Usan: Grays (Gy)
        \item Aplicación: Radioterapia para tratamiento del cáncer.
        \item Máquina: Acelerador lineal entrega dosis precisas de radiación a tejidos malignos.
    \end{itemize}

    \item \textbf{Dosímetros Personales}
    \begin{itemize}
        \item Usan: Sieverts (Sv) y rems
        \item Aplicación: Monitorear la dosis de radiación recibida por el personal.
        \item Máquina: Dosímetro mide la dosis equivalente recibida durante un año de tiempo.
    \end{itemize}

    \item \textbf{Cámaras de Ionización}
    \begin{itemize}
        \item Usan: Coulomb por kilogramo (C/kg) o roentgen (R)
        \item Aplicación: Medición de la exposición en ambientes radiológicos.
        \item Máquina: Câmara de ionización mide la ionización producida por radiación en el aire.
    \end{itemize}

    \item \textbf{Contadores Geiger}
    \begin{itemize}
        \item Usan: Becquerel (Bq) y curie (Ci)
        \item Aplicación: Conteo de la actividad radiactiva producida por radiación en el aire.
        \item Máquina: Contador Geiger mide la actividad radiactiva producida por radiación en el aire.
    \end{itemize}
\end{enumerate}

\section{Referencias}
\indent International Atomic Energy Agency (IAEA). (2007). Dosimetry in Diagnostic Radiology: An International Code of Practice (IAEA Technical Reports Series No. 457). Vienna: IAEA.

International Atomic Energy Agency (IAEA). (2000). Neutron Physics and Reactor Theory (IAEA Training Course Series No. 4). Vienna: IAEA.

European Commission. (1999). European Guidelines on Quality Criteria for Computed Tomography. Luxembourg: Office for Official Publications of the European Communities.