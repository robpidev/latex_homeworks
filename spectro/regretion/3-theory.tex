\section{Marco teórico}
La Ley de Beer-Lambert establece que la absorbancia $A$ de una disolución es directamente proporcional a la concentración $C$ del soluto y a la longitud del camino óptico $l$:
\begin{equation}
A = \varepsilon \, l \, C
\end{equation}
donde $\varepsilon$ es el coeficiente de absorción molar característico de cada sustancia. Esta relación lineal permite cuantificar analitos mediante una curva de calibración, obtenida a partir de mediciones experimentales.

En el caso del KMnO$_4$, su color violeta intenso lo hace ideal para estudios espectrofotométricos, pues presenta una absorción marcada en el rango visible. El método de mínimos cuadrados se emplea para encontrar la mejor recta que se ajusta a los datos experimentales, minimizando la suma de los cuadrados de los errores. El coeficiente de correlación $R^2$ evalúa la calidad del ajuste: valores cercanos a 1 indican una fuerte relación lineal entre concentración y absorbancia.

