
\section{Introducción}
El análisis de contaminantes presentes en cuerpos de agua es fundamental para evaluar la calidad ambiental y la salud pública. Entre las sustancias de interés, el permanganato de potasio (KMnO$_4$) se utiliza comúnmente como agente oxidante en procesos industriales y de tratamiento de aguas. Sin embargo, su presencia en exceso puede alterar las propiedades químicas del agua y representar un riesgo ecológico.

La espectrofotometría UV-Vis constituye una técnica eficaz para cuantificar concentraciones de compuestos en solución, basándose en la relación entre la absorbancia y la concentración según la Ley de Beer-Lambert. En esta investigación se busca establecer dicha relación para muestras de aguas residuales del río Piura.

