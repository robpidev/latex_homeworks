\section{Ventajas y limitaciones del método}

La espectroscopía Raman mejorada en superficie sin marcadores presenta numerosas ventajas que la convierten en una técnica atractiva para la detección sensible de ADN. Sin embargo, también posee ciertas limitaciones que deben considerarse para su correcta aplicación e interpretación de resultados.

\subsection{Ventajas del método}

Una de las principales ventajas de la técnica SERS sin marcadores es su elevada sensibilidad. Gracias al intenso realce electromagnético generado por las nanopartículas de plata, es posible detectar señales Raman de ADN en concentraciones extremadamente bajas, incluso cercanas al nivel de una sola molécula. Esta característica resulta especialmente valiosa en aplicaciones biomédicas y genéticas donde las muestras disponibles suelen ser limitadas.

Otra ventaja significativa es la ausencia de marcadores externos. Al no requerir fluoróforos ni etiquetas químicas, el método evita posibles alteraciones en la estructura del ADN y reduce la complejidad de la preparación de la muestra. Además, la detección directa de las señales vibracionales propias del ADN proporciona información molecular detallada y específica, lo que mejora la selectividad del análisis.

Asimismo, la espectroscopía SERS es una técnica rápida y no destructiva. La adquisición de los espectros Raman puede realizarse en tiempos cortos, lo que permite el análisis casi en tiempo real. Esta propiedad facilita su integración en sistemas de biosensado, diagnóstico clínico y análisis genético, así como su potencial aplicación en dispositivos portátiles.

\subsection{Limitaciones experimentales}

A pesar de sus ventajas, la técnica SERS sin marcadores presenta ciertas limitaciones. Una de las principales dificultades es la reproducibilidad de los sustratos SERS. La formación de \textit{hot spots} depende de la distribución, tamaño y morfología de las nanopartículas metálicas, lo que puede dar lugar a variaciones significativas en la intensidad de la señal Raman entre distintas mediciones.

Otra limitación importante está relacionada con la complejidad de la interpretación espectral. Las señales Raman del ADN pueden solaparse con bandas provenientes de otras moléculas presentes en la muestra o del propio sustrato metálico, lo que dificulta el análisis y la identificación precisa de las señales de interés. Esto requiere, en muchos casos, el uso de técnicas de análisis multivariante o métodos estadísticos avanzados.

Finalmente, aunque la plata es uno de los metales más eficientes para SERS, su estabilidad química puede verse afectada por procesos de oxidación, lo que puede degradar el rendimiento del sustrato con el tiempo. Esta limitación plantea desafíos adicionales para la aplicación de la técnica en entornos reales y a largo plazo, especialmente en dispositivos reutilizables o de uso clínico.

\subsection{Conclusiones}
En relación con el objetivo general, se concluye que la espectroscopía Raman mejorada en superficie sin marcadores constituye una técnica altamente eficaz para la detección sensible de ADN. El crecimiento de nanopartículas de plata mediado por ADN permite amplificar significativamente la señal Raman, estableciendo una relación directa entre la presencia de la biomolécula y la intensidad espectroscópica obtenida, incluso a concentraciones muy bajas.

\begin{enumerate}
    \item Respecto al primer objetivo específico, se demostró que el fundamento físico de la espectroscopía Raman y de la técnica SERS se basa en la interacción radiación--materia y en el realce electromagnético asociado a la resonancia de plasmones superficiales. Este realce incrementa el campo electromagnético local en las proximidades de nanopartículas metálicas, lo que conduce a un aumento considerable de la intensidad de la dispersión Raman del ADN.

    \item En cuanto al segundo objetivo específico, se concluye que el ADN desempeña un papel activo en el crecimiento de nanopartículas de plata, actuando como plantilla estructural y, en ciertos casos, como agente reductor. Esta interacción favorece la nucleación y organización de las nanopartículas, promoviendo la formación de regiones de alto campo electromagnético (\textit{hot spots}) esenciales para el efecto SERS.

    \item En relación con el tercer objetivo específico, se estableció que el mecanismo de detección sin marcadores permite identificar el ADN a partir de sus propias señales vibracionales, evitando el uso de fluoróforos u otras etiquetas químicas. Esta característica reduce la complejidad experimental y preserva la integridad estructural de la biomolécula, ofreciendo información molecular directa y específica.

    \item Finalmente, en concordancia con el cuarto objetivo específico, se identificaron como principales ventajas del método su alta sensibilidad, rapidez y carácter no destructivo. No obstante, también se reconocieron limitaciones importantes, como la reproducibilidad de los sustratos SERS, la estabilidad de las nanopartículas de plata y la complejidad en la interpretación espectral, aspectos que deben abordarse en investigaciones futuras.
\end{enumerate}
