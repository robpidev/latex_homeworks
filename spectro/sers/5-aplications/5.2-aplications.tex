\subsection{Biosensores basados en SERS}

La espectroscopía Raman mejorada en superficie sin marcadores ha impulsado el desarrollo de una nueva generación de biosensores altamente sensibles y selectivos para la detección de ADN. Un biosensor basado en SERS combina la especificidad molecular del reconocimiento biológico con el elevado realce de la señal Raman proporcionado por nanopartículas metálicas, permitiendo la identificación directa de biomoléculas sin la necesidad de etapas de marcaje.

En este tipo de biosensores, las nanopartículas de plata desempeñan un papel fundamental al actuar simultáneamente como sustrato plasmónico y como elemento amplificador de la señal espectroscópica. El crecimiento de nanopartículas mediado por ADN favorece la formación de estructuras nanométricas altamente activas desde el punto de vista SERS, lo que permite detectar señales vibracionales características de las bases nitrogenadas y del esqueleto fosfato del ADN.

Una ventaja significativa de los biosensores SERS sin marcadores es su capacidad para proporcionar información molecular detallada. A diferencia de biosensores basados en fluorescencia, que suelen ofrecer señales indirectas, los biosensores SERS permiten obtener espectros vibracionales específicos que actúan como huellas moleculares del ADN. Esta característica mejora la selectividad del sensor y reduce la probabilidad de interferencias o falsos positivos.

Asimismo, los biosensores basados en SERS destacan por su potencial para la detección rápida y en tiempo real. La adquisición de espectros Raman es un proceso rápido y no destructivo, lo que posibilita la monitorización continua de procesos biológicos o la detección inmediata de material genético. Estas propiedades hacen que los biosensores SERS sean especialmente atractivos para aplicaciones en diagnóstico biomédico, análisis clínico y monitoreo ambiental.

En conjunto, la integración de la espectroscopía SERS sin marcadores en el diseño de biosensores representa una estrategia prometedora para el desarrollo de dispositivos compactos, sensibles y altamente específicos, capaces de detectar ADN en condiciones donde otras técnicas convencionales presentan limitaciones.
