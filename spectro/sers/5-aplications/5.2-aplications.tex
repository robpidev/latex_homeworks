\subsection{Aplicaciones en distintas ramas y equipos}

La espectroscopía de microondas posee un amplio rango de aplicaciones en la investigación científica y tecnológica. Gracias a su alta resolución, permite obtener información precisa sobre las propiedades estructurales y dinámicas de las moléculas, así como sobre su interacción con campos electromagnéticos.

\subsubsection{Aplicaciones en química}
En química molecular, la espectroscopía rotacional es una herramienta esencial para determinar parámetros estructurales como la longitud de enlace, el momento dipolar y el ángulo de enlace.  
Ejemplos destacados incluyen:
\begin{itemize}
    \item Determinación de la longitud del enlace H–Cl en el ácido clorhídrico (HCl) a partir de su constante rotacional \( B \).
    \item Estudio de moléculas como H$_2$O y NH$_3$ para obtener su geometría angular.
    \item Análisis de moléculas orgánicas volátiles como el etanol (C$_2$H$_5$OH) o el formaldehído (H$_2$CO).
\end{itemize}
\textbf{Equipos utilizados:}
\begin{itemize}
    \item Espectrómetro de microondas de cavidad resonante.
    \item Espectrómetro de transformada de Fourier (FTMW).
    \item Celdas de descarga y sistemas criogénicos para generar especies reactivas.
\end{itemize}

\subsubsection{Aplicaciones en física}
En física cuántica y molecular, la espectroscopía de microondas permite comprobar modelos teóricos del rotor rígido y no rígido, estudiar acoplamientos espín-rotación y analizar efectos isotópicos.  
Ejemplos representativos:
\begin{itemize}
    \item Verificación de los niveles rotacionales en CO y N$_2$O.
    \item Determinación experimental del momento de inercia y comparación con resultados teóricos.
    \item Observación del desdoblamiento hiperfino en el radical OH.
\end{itemize}
\textbf{Equipos utilizados:}
\begin{itemize}
    \item Espectrómetros de cavidad resonante de alta precisión.
    \item Espectrómetros FTMW acoplados a fuentes láser.
\end{itemize}

\subsubsection{Aplicaciones en astronomía}
En astronomía, las transiciones rotacionales en el rango de microondas permiten identificar moléculas en el medio interestelar y estudiar la composición de atmósferas planetarias.  
Ejemplos:
\begin{itemize}
    \item Detección de CO, HCN y H$_2$O en nubes moleculares interestelares.
    \item Identificación de moléculas orgánicas complejas en regiones de formación estelar como Orión KL.
    \item Análisis de isótopos en moléculas como $^{13}$CO y C$^{18}$O.
\end{itemize}
\textbf{Equipos utilizados:}
\begin{itemize}
    \item Radiotelescopios como el \textbf{ALMA} y el \textbf{Green Bank Telescope (GBT)}.
    \item Espectrómetros acoplados a antenas parabólicas de alta sensibilidad.
\end{itemize}

\subsubsection{Aplicaciones en ingeniería y tecnología}
Los principios de la espectroscopía de microondas también se aplican en el desarrollo de sistemas tecnológicos de comunicación, sensores y diagnóstico.  
Ejemplos:
\begin{itemize}
    \item Diseño de sensores de humedad y densidad basados en absorción de microondas.
    \item Control de calidad industrial mediante medición dieléctrica.
    \item Aplicación de principios de dispersión de microondas en radares Doppler y sistemas médicos de diagnóstico.
\end{itemize}
\textbf{Equipos utilizados:}
\begin{itemize}
    \item Analizadores vectoriales de redes (VNA).
    \item Sensores dieléctricos y espectrómetros portátiles.
    \item Sistemas de radar de banda ancha y cámaras de microondas.
\end{itemize}

\subsubsection{Síntesis de aplicaciones}
La Tabla~\ref{tab:aplicaciones} resume las principales áreas de aplicación junto con ejemplos y equipos característicos.

\begin{table}[H]
\centering
\caption{Resumen de aplicaciones de la espectroscopía de microondas}
\label{tab:aplicaciones}
\begin{tabular}{|p{3cm}|p{5cm}|p{5cm}|}
\hline
\textbf{Área} & \textbf{Ejemplos de aplicación} & \textbf{Equipos utilizados} \\
\hline
Química & Geometría molecular de HCl, H$_2$O, NH$_3$ & FTMW, cavidad resonante \\
\hline
Física & Niveles rotacionales de CO, N$_2$O; acoplamiento espín-rotación & Cavidad resonante, espectroscopía láser \\
\hline
Astronomía & Detección de CO, HCN, H$_2$O en nubes moleculares & Radiotelescopios (ALMA, GBT) \\
\hline
Ingeniería & Sensores dieléctricos, radares, diagnóstico médico & Analizador de redes (VNA), sensores de microondas \\
\hline
\end{tabular}
\end{table}

En conjunto, estas aplicaciones evidencian la naturaleza interdisciplinaria de la espectroscopía de microondas y su papel clave en el avance de la física molecular, la química estructural, la radioastronomía y la ingeniería moderna.
