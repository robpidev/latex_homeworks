\subsection{Diagnóstico biomédico}

La espectroscopía Raman mejorada en superficie sin marcadores ha adquirido un papel relevante en el ámbito del diagnóstico biomédico debido a su elevada sensibilidad y a su capacidad para detectar material genético de forma directa. La identificación temprana de ADN asociado a enfermedades genéticas, infecciosas o cancerígenas constituye un desafío fundamental en la medicina moderna, y las técnicas SERS ofrecen una alternativa prometedora frente a métodos convencionales.

En el diagnóstico de enfermedades infecciosas, los sistemas SERS permiten la detección de secuencias específicas de ADN correspondientes a bacterias, virus u otros patógenos, incluso en concentraciones muy bajas. La amplificación de la señal Raman mediante nanopartículas de plata facilita la identificación rápida del agente causante, reduciendo los tiempos de análisis en comparación con técnicas que requieren procesos de amplificación genética, como la reacción en cadena de la polimerasa (PCR).

En el contexto de enfermedades genéticas y oncológicas, la detección de mutaciones o variaciones en la secuencia del ADN resulta crucial para el diagnóstico temprano y el seguimiento de la enfermedad. Los métodos SERS sin marcadores permiten analizar directamente las señales vibracionales del ADN, proporcionando información molecular detallada que puede utilizarse para distinguir entre estados normales y patológicos. Esta capacidad resulta especialmente valiosa en la identificación precoz de alteraciones genéticas asociadas al desarrollo del cáncer.

Otra ventaja importante de la espectroscopía SERS en el diagnóstico biomédico es su carácter no destructivo y su potencial para el análisis \textit{in situ}. La posibilidad de obtener información espectroscópica sin modificar químicamente la muestra reduce el riesgo de alterar el material biológico y facilita su aplicación en entornos clínicos. En conjunto, estas características posicionan a la espectroscopía SERS sin marcadores como una herramienta emergente con alto potencial para el desarrollo de métodos de diagnóstico biomédico rápidos, sensibles y específicos.
