\subsection{Equipos y principio de funcionamiento}

La espectroscopía de microondas utiliza una variedad de equipos diseñados para generar, detectar y analizar la radiación electromagnética en el rango de 1 a 300~GHz. A continuación se describen los principales dispositivos empleados en investigación y sus principios de operación.

\subsubsection{Espectrómetro de microondas de cavidad resonante}
Este tipo de espectrómetro utiliza una cavidad metálica cerrada en la cual se establece una onda estacionaria de microondas.  
Cuando una muestra gaseosa con moléculas polares se introduce en la cavidad, la absorción de radiación ocurre únicamente a frecuencias resonantes con las transiciones rotacionales.

\vspace{12pt}
\textbf{Principio de funcionamiento:}
\begin{itemize}
    \item Un generador de microondas (magnetrón o klystron) produce radiación coherente.
    \item La radiación es dirigida hacia la cavidad resonante donde se encuentra la muestra.
    \item Un detector de microondas mide la intensidad transmitida o reflejada.
    \item Las pérdidas de energía en resonancia corresponden a transiciones rotacionales.
\end{itemize}

Este tipo de espectrómetro fue ampliamente utilizado en los primeros estudios de HCl, CO y NH$_3$, permitiendo determinar con precisión sus constantes rotacionales.

\begin{figure}[H]
\centering
\begin{tikzpicture}[scale=1.1, every node/.style={font=\small}]
% Fuente de microondas
\node (fuente) at (0,0) [draw, rectangle, rounded corners, minimum width=2cm, minimum height=1cm, fill=blue!10] {Fuente de microondas};
\node[below=2pt of fuente] {Generador de señal};

% Modulador
\node (modulador) [right=2cm of fuente, draw, rectangle, rounded corners, minimum width=2cm, minimum height=1cm, fill=green!10] {Modulador};
\node[below=2pt of modulador] {Control de frecuencia};

% Cavidad resonante
\node (cavidad) [right=2cm of modulador, draw, rectangle, rounded corners, minimum width=2cm, minimum height=1cm, fill=yellow!20] {Cavidad resonante};
\node[below=2pt of cavidad] {Interacción molécula–campo};

% Detector
\node (detector) [right=2cm of cavidad, draw, rectangle, rounded corners, minimum width=2cm, minimum height=1cm, fill=red!10] {Detector};
\node[below=2pt of detector] {Registro del espectro};

% Flechas de conexión
\draw[->, thick] (fuente) -- (modulador);
\draw[->, thick] (modulador) -- (cavidad);
\draw[->, thick] (cavidad) -- (detector);
\end{tikzpicture}
\caption{Esquema general de un espectrómetro de microondas.}
\label{fig:espectrometro}
\end{figure}

\begin{figure}[H]
    \centering
    \includegraphics[width=0.8\textwidth]{img/microwave.png}    
    \caption{Espectrómetro de microondas con cavidad resonante. (\cite{quimicamoderna_microondas})}
\end{figure}


\subsubsection{Espectrómetro de Transformada de Fourier de microondas (FTMW)}
El espectrómetro FTMW (\textit{Fourier Transform Microwave Spectrometer}) utiliza pulsos breves de microondas que excitan las moléculas, y posteriormente mide la señal de emisión libre (\textit{free induction decay}) en el dominio del tiempo.  
Mediante una transformada de Fourier se obtiene el espectro en el dominio de la frecuencia.

\textbf{Principio de funcionamiento:}
\begin{itemize}
    \item Un pulso de microondas coherente excita las transiciones rotacionales.
    \item Las moléculas reemiten radiación coherente cuando regresan a su estado base.
    \item Una antena o guía de onda detecta esta señal de emisión libre.
    \item La transformada de Fourier convierte la señal temporal en el espectro de frecuencia.
\end{itemize}

Este método proporciona una resolución extremadamente alta (del orden de kHz) y permite estudiar mezclas complejas o especies reactivas generadas en celdas de descarga.

\begin{figure}[h]
    \centering
    \includegraphics[width=0.8\textwidth]{img/ftmw.png}
    \caption{Espectro de un FTMW. (\cite{kiel_ftmw_spectrometer})}
\end{figure}

\subsubsection{Radiotelescopios para observaciones astronómicas}
En astronomía, los radiotelescopios detectan la radiación de microondas emitida por moléculas en el espacio interestelar. Las antenas parabólicas concentran la radiación hacia receptores sensibles refrigerados criogénicamente.

\textbf{Principio de funcionamiento:}
\begin{itemize}
    \item La antena parabólica concentra las ondas de microondas hacia el receptor.
    \item El receptor (normalmente un \textit{mezclador heterodino}) convierte la señal a una frecuencia intermedia.
    \item Un analizador espectral o sistema digital extrae las líneas de emisión rotacionales.
    \item Los datos se procesan para determinar la intensidad, desplazamiento Doppler y abundancia molecular.
\end{itemize}

Ejemplos notables de equipos son el \textbf{Atacama Large Millimeter/submillimeter Array (ALMA)} en Chile y el \textbf{Green Bank Telescope (GBT)} en Estados Unidos.

\begin{figure}[H]
    \centering
    \includegraphics[width=0.8\textwidth]{img/alma.png}
    \caption{ALMA. (\cite{cooper_alma_space})}
\end{figure}

\begin{figure}[H]
    \centering
    \includegraphics[width=0.8\textwidth]{img/gbt.png}
    \caption{GBT. (\cite{nrao_gbt})}
\end{figure}



\subsubsection{Analizadores vectoriales de redes (VNA)}
Los analizadores de redes se utilizan para estudiar materiales o sistemas mediante la medición de parámetros de dispersión (\(S_{11}, S_{21}\)) de microondas incidentes.  
En el contexto espectroscópico, permiten estudiar la respuesta dieléctrica y la absorción en gases o sólidos.

\textbf{Principio de funcionamiento:}
\begin{itemize}
    \item El VNA genera una señal de microondas de frecuencia variable.
    \item Mide las señales reflejadas y transmitidas a través de la muestra.
    \item Calcula los parámetros \(S_{ij}\), que contienen información sobre la absorción y la fase.
    \item De estos parámetros se puede inferir la constante dieléctrica y las resonancias rotacionales.
\end{itemize}

Estos instrumentos son ampliamente empleados en ingeniería y en el desarrollo de sensores de microondas aplicados a biomedicina y materiales.

\begin{figure}[H]
    \centering
    \includegraphics[width=0.8\textwidth]{img/vna.png}
    \caption{VNA. (\cite{rohdeschwarz_lcr_vna})}
\end{figure}

\subsubsection{Sistemas láser acoplados a espectroscopía rotacional}
En investigaciones avanzadas, los espectrómetros de microondas pueden acoplarse a láseres de alta precisión para excitar o detectar estados rotacionales y vibracionales simultáneamente.  
Este enfoque es fundamental para estudiar moléculas con estructuras electrónicas complejas.

\begin{figure}[H]
    \centering
    \includegraphics[width=0.8\textwidth]{img/brown.png}
    \caption{Esquema de un espectrómetro de microondas de transformada de Fourier con pulso chirpado, introducido por Brown \textit{et al.} (2008). 
    Un pulso de microondas de 12\,GHz de ancho de banda (de 1\,$\mu$s de duración) es generado por un generador de formas de onda arbitrarias (AWG), convertido a frecuencias en el rango de 6.8--18.5\,GHz y amplificado hasta varios cientos de vatios (1), antes de ingresar a la cámara del haz molecular (2). 
    La decadencia de inducción libre coherente (FID) de las moléculas excitadas, que contiene la información espectral, es posteriormente registrada y analizada mediante un osciloscopio digital de alta velocidad. 
    Reproducido con permiso de \cite{brown2008}.}
    \label{fig:chirped_spectrometer}
\end{figure}


\textbf{Ejemplo:}
\begin{itemize}
    \item Sistemas \textit{Chirped Pulse FTMW}, que usan pulsos amplios en frecuencia (1–20~GHz) para cubrir grandes regiones espectrales.
    \begin{figure}
        \centering
        \includegraphics[width=0.5\textwidth]{img/cpftmw.png}
        \caption{Chiperd Pulse FTMW}
    \end{figure}
    \item Equipos \textit{Lamb-dip microwave} que permiten mediciones hiperfinas de alta resolución.
\end{itemize}

\vspace{1em}
En conjunto, estos equipos permiten cubrir desde la caracterización básica de moléculas simples hasta el estudio de sistemas astrofísicos y materiales avanzados, consolidando a la espectroscopía de microondas como una herramienta esencial en la física moderna.
