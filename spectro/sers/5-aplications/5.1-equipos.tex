\subsection{Detección de bajas concentraciones de ADN}

Una de las aplicaciones más relevantes de la espectroscopía Raman mejorada en superficie sin marcadores es la detección de ADN en concentraciones extremadamente bajas. La sensibilidad intrínseca de la técnica SERS permite la identificación de señales vibracionales características del ADN incluso cuando la cantidad de moléculas presentes es limitada, lo que resulta de gran interés en áreas como el diagnóstico temprano, la genética y el análisis biomolecular.

El elevado grado de sensibilidad se debe principalmente al intenso realce electromagnético generado en las proximidades de las nanopartículas de plata, especialmente en las regiones conocidas como \textit{hot spots}. En estas zonas, el campo electromagnético local se incrementa varios órdenes de magnitud, lo que conduce a un aumento significativo de la sección eficaz de dispersión Raman del ADN. Como consecuencia, señales que serían indetectables mediante espectroscopía Raman convencional pueden ser observadas claramente mediante SERS.

En sistemas donde el crecimiento de nanopartículas de plata es mediado por ADN, la formación de \textit{hot spots} está directamente relacionada con la presencia y concentración de la biomolécula. A bajas concentraciones de ADN, incluso un número reducido de cadenas puede inducir la nucleación y organización de nanopartículas metálicas, generando regiones localizadas de alto realce. Esta característica permite alcanzar límites de detección en el rango de concentraciones ultrabajas, llegando en algunos casos al nivel de una sola molécula.

Desde un punto de vista analítico, la relación entre la intensidad de la señal SERS y la concentración de ADN posibilita la detección cuantitativa en regímenes de baja concentración. La intensidad Raman registrada puede emplearse como un indicador directo de la cantidad de ADN presente, siempre que se mantengan condiciones experimentales controladas y reproducibles. Esta capacidad convierte a la técnica SERS sin marcadores en una herramienta altamente eficiente para la detección sensible de ADN, superando las limitaciones de métodos tradicionales que requieren etapas de amplificación o marcaje fluorescente.
