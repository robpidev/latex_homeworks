\subsection{Análisis genético}

La espectroscopía Raman mejorada en superficie sin marcadores representa una herramienta innovadora para el análisis genético, ya que permite estudiar directamente las características moleculares del ADN sin recurrir a procesos de marcaje o amplificación. Esta técnica ofrece la posibilidad de identificar y analizar secuencias genéticas a partir de sus señales vibracionales intrínsecas, lo que amplía las capacidades de las metodologías tradicionales empleadas en genética molecular.

En el análisis genético, la información obtenida mediante SERS se basa en la detección de bandas Raman asociadas a las vibraciones características de las bases nitrogenadas y del esqueleto fosfato del ADN. Diferencias sutiles en la composición o conformación molecular pueden dar lugar a variaciones en el espectro Raman, lo que permite discriminar entre distintas secuencias, estructuras o estados del material genético. Este enfoque resulta particularmente útil para el estudio de mutaciones, polimorfismos y cambios conformacionales del ADN.

El crecimiento de nanopartículas de plata mediado por ADN contribuye de manera decisiva a la sensibilidad del análisis genético mediante SERS. Al favorecer la formación de \textit{hot spots} plasmónicos, este mecanismo amplifica la señal Raman asociada al ADN, permitiendo la detección incluso cuando la cantidad de material genético disponible es limitada. Esta característica resulta ventajosa en estudios donde las muestras son escasas o difíciles de obtener.

Además, el análisis genético basado en SERS sin marcadores presenta un alto potencial para la integración en sistemas de diagnóstico rápido y automatizado. La posibilidad de obtener información espectroscópica en tiempos cortos y sin etapas complejas de preparación de la muestra facilita su aplicación en entornos clínicos y de investigación. En conjunto, estas propiedades posicionan a la espectroscopía SERS como una técnica complementaria y prometedora para el análisis genético avanzado.
