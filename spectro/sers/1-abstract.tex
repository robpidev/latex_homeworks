\begin{abstract}
La detección sensible y específica de ADN es un aspecto fundamental en aplicaciones biomédicas, genéticas y de biosensado. En este contexto, la espectroscopía Raman mejorada en superficie (SERS, por sus siglas en inglés) se presenta como una técnica analítica de gran potencial debido a su alta sensibilidad y a la posibilidad de obtener información molecular sin la necesidad de marcadores externos. En esta monografía se analizan los fundamentos físicos y el funcionamiento de la detección de ADN mediante SERS sin marcadores, basada en el crecimiento de nanopartículas de plata mediado por ADN. Se describen los principios de la dispersión Raman y del realce electromagnético asociado a los plasmones superficiales localizados en nanopartículas metálicas, así como el papel del ADN en la nucleación y crecimiento de dichas nanopartículas, lo que favorece la formación de regiones de alto campo electromagnético. Finalmente, se discuten las principales aplicaciones, ventajas y limitaciones del método, destacando su relevancia para el desarrollo de técnicas de detección rápida y de alta sensibilidad en el ámbito del biosensado molecular.

\textbf{Palabras clave:} espectroscopía Raman, SERS, detección de ADN, nanopartículas de plata, biosensores.
\end{abstract}
