\subsection{Espectroscopía Raman}

<<<<<<< HEAD
La espectroscopía Raman es una técnica vibracional no destructiva que permite obtener información molecular mediante la dispersión inelástica de la luz incidente, generando una huella espectral característica de la muestra \parencite[p.~428]{jenkins2016raman}.



\begin{figure}[h]
    \centering
    \includegraphics[width=0.5\linewidth]{img/raman.png}
    \caption{Esquema de dispersión Raman: esquema de dispersión Rayleigh, Stokes y anti-Stokes. Fuente: adaptado de Scancotec \parencite{scancotec2022raman}.}
    \label{fig:raman_scattering}
\end{figure}

=======
La espectroscopía Raman es una técnica vibracional no destructiva que permite obtener información molecular mediante la dispersión inelástica de la luz incidente, generando una huella espectral característica de la muestra \parencite[p.~248]{jenkins2016raman}.


La espectroscopía Raman es una técnica espectroscópica vibracional basada en la interacción de la radiación electromagnética con la materia, específicamente en los procesos de dispersión que ocurren cuando un haz de luz monocromática incide sobre una muestra. Como resultado de esta interacción, la radiación puede ser dispersada de forma elástica o inelástica, dando lugar a distintos fenómenos espectroscópicos que constituyen el fundamento de la técnica Raman.
>>>>>>> 6391fe6a2dd6864ef2eccdae1a59b1a22f27a240

\subsubsection{Dispersión elástica e inelástica}

En la espectroscopía Raman, la dispersión de la luz puede clasificarse en tres tipos: la dispersión Rayleigh, que es un proceso elástico donde el fotón dispersado conserva la misma energía que el fotón incidente; la dispersión Raman Stokes, en la cual el fotón pierde energía al excitar modos vibracionales de la molécula; y la dispersión Raman anti-Stokes, donde el fotón gana energía a partir de moléculas previamente excitadas vibracionalmente \parencite{scancotec2022raman}.

Cuando la radiación incidente interactúa con las moléculas de una muestra, la mayor parte de la luz es dispersada de manera elástica, proceso conocido como dispersión de Rayleigh. En este caso, la energía del fotón dispersado es igual a la energía del fotón incidente, por lo que no se produce un cambio en la frecuencia de la radiación. Este tipo de dispersión no aporta información directa sobre la estructura molecular del sistema.

En contraste, una pequeña fracción de la radiación incidente experimenta dispersión inelástica, fenómeno conocido como dispersión Raman. Durante este proceso, se produce un intercambio de energía entre el fotón incidente y los modos vibracionales de la molécula, lo que da lugar a un cambio en la frecuencia de la radiación dispersada. Dependiendo del sentido del intercambio energético, se distinguen dos contribuciones: las líneas Stokes, asociadas a la transferencia de energía del fotón a la molécula, y las líneas anti-Stokes, correspondientes a la transferencia de energía desde la molécula al fotón.

\subsubsection{Desplazamiento Raman}
El desplazamiento Raman corresponde a la diferencia de energía entre el fotón incidente y el fotón dispersado en un proceso de dispersión inelástica, asociado a los modos vibracionales característicos de la molécula \parencite[p.~428]{jenkins2016raman}.
Esta característica constituye una de las magnitudes fundamentales en el análisis espectroscópico. Este desplazamiento se expresa habitualmente en unidades de número de onda y se define como
\[
\Delta \tilde{\nu} = \tilde{\nu}_0 - \tilde{\nu}_s,
\]
donde $\tilde{\nu}_0$ representa el número de onda de la radiación incidente y $\tilde{\nu}_s$ el número de onda de la radiación dispersada. Una característica importante del desplazamiento Raman es que su valor es independiente de la longitud de onda del láser utilizado, ya que depende exclusivamente de las propiedades vibracionales de la molécula analizada.

El desplazamiento Raman permite identificar las energías asociadas a los modos vibracionales moleculares, por lo que actúa como una huella espectral característica de cada sustancia. Esta propiedad resulta especialmente relevante en la identificación de biomoléculas complejas, como el ADN.

\subsubsection{Relación con las vibraciones moleculares}

Desde un punto de vista físico, la dispersión Raman está directamente relacionada con las vibraciones moleculares y con la variación de la polarizabilidad inducida por el campo electromagnético incidente. Para que un modo vibracional sea Raman activo, es necesario que durante la vibración se produzca un cambio en la polarizabilidad de la molécula. Este criterio de selección explica por qué no todos los modos vibracionales observables en espectroscopía infrarroja son necesariamente visibles en espectroscopía Raman.

Las vibraciones moleculares observadas en un espectro Raman corresponden a movimientos colectivos de los átomos dentro de la molécula, tales como estiramientos y flexiones de enlaces químicos. En el caso del ADN, estas vibraciones están asociadas a los enlaces del esqueleto fosfodiéster, a las bases nitrogenadas y a las interacciones intermoleculares, lo que permite obtener información estructural detallada. En consecuencia, la espectroscopía Raman se convierte en una herramienta adecuada para el análisis vibracional de biomoléculas y sienta las bases teóricas para su aplicación en técnicas de mayor sensibilidad, como la espectroscopía Raman mejorada en superficie (SERS).
