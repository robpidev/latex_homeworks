\subsection{Crecimiento de nanopartículas de plata mediado por ADN}

El crecimiento de nanopartículas de plata mediado por ADN constituye una estrategia fundamental en el desarrollo de métodos SERS sin marcadores para la detección sensible de biomoléculas. En este enfoque, el ADN no solo actúa como la molécula objetivo a detectar, sino que además participa activamente en los procesos de nucleación, crecimiento y organización de las nanopartículas metálicas. Esta doble función permite establecer una relación directa entre la presencia del ADN y la intensidad de la señal Raman observada.

\subsubsection{El ADN como plantilla o agente reductor}

El ADN puede desempeñar el papel de plantilla estructural o de agente reductor durante la síntesis de nanopartículas de plata. La molécula de ADN presenta un esqueleto cargado negativamente debido a los grupos fosfato, lo que favorece la atracción electrostática de iones de plata (\(\mathrm{Ag}^+\)) presentes en la solución. Esta interacción facilita la nucleación de la plata metálica a lo largo de la cadena de ADN.

En determinadas condiciones experimentales, el ADN puede además actuar como agente reductor suave, promoviendo la conversión de iones \(\mathrm{Ag}^+\) en plata metálica (\(\mathrm{Ag}^0\)). De manera simplificada, este proceso puede representarse como
\[
\mathrm{Ag}^+ + e^- \rightarrow \mathrm{Ag}^0,
\]
donde los electrones necesarios para la reducción provienen de grupos funcionales presentes en las bases nitrogenadas del ADN o de especies químicas asociadas al entorno del sistema. Este mecanismo permite un crecimiento localizado y controlado de nanopartículas directamente asociado a la presencia del ADN.

\subsubsection{Crecimiento controlado de nanopartículas}

El crecimiento mediado por ADN conduce a la formación de nanopartículas de plata con tamaños y distribuciones espaciales que dependen de la concentración y conformación del ADN, así como de las condiciones físico-químicas del medio. El control del tamaño y la morfología de las nanopartículas resulta crucial, ya que estos parámetros determinan la posición y la intensidad de la resonancia de plasmones superficiales.

Desde un punto de vista óptico, el tamaño de las nanopartículas influye directamente en la frecuencia de resonancia plasmónica y en la intensidad del campo electromagnético local generado. Un crecimiento controlado favorece la aparición de estructuras nanométricas capaces de maximizar el acoplamiento plasmónico y, por ende, el realce de la señal Raman. De este modo, el ADN actúa como un modulador indirecto de las propiedades ópticas del sistema SERS.

\subsubsection{Formación de \textit{hot spots} SERS}

Uno de los aspectos más relevantes del crecimiento de nanopartículas mediado por ADN es la formación de regiones de campo electromagnético altamente intensificado, conocidas como \textit{hot spots}. Estas regiones suelen generarse en los espacios intersticiales entre nanopartículas cercanas o a lo largo de estructuras donde el ADN induce una organización específica de las partículas metálicas.

En estas zonas, el campo electromagnético local puede aumentar varios órdenes de magnitud debido al acoplamiento de plasmones superficiales entre nanopartículas adyacentes. Dado que la intensidad de la señal SERS depende aproximadamente de la cuarta potencia del campo eléctrico local,
\[
I_{\text{SERS}} \propto |E_{\text{loc}}|^4,
\]
la presencia de \textit{hot spots} resulta determinante para alcanzar una detección altamente sensible del ADN.

\subsubsection{Relación entre la concentración de ADN y la señal Raman}

La participación directa del ADN en el crecimiento y organización de las nanopartículas de plata establece una relación clara entre la concentración de ADN presente en el sistema y la intensidad de la señal Raman observada. A medida que la concentración de ADN aumenta, se incrementa el número de sitios de nucleación y la probabilidad de formación de \textit{hot spots}, lo que conduce a un mayor realce de la señal SERS.

De forma cualitativa, esta relación puede expresarse como
\[
I_{\text{SERS}} \propto C_{\text{ADN}} \, |E_{\text{loc}}|^4,
\]
donde \(C_{\text{ADN}}\) representa la concentración de ADN. Esta dependencia permite utilizar la intensidad de la señal Raman como una medida indirecta de la cantidad de ADN presente, lo que constituye la base de los métodos de detección sensible y cuantitativa basados en SERS sin marcadores.

En conjunto, el crecimiento de nanopartículas de plata mediado por ADN integra procesos físicos y químicos que permiten traducir la presencia de la biomolécula en una señal espectroscópica amplificada, consolidando este enfoque como una herramienta poderosa para el biosensado molecular.
