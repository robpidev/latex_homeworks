\subsection{Detección de ADN sin marcadores (Label-Free)}

La detección de ADN sin marcadores, conocida como detección \textit{label-free}, constituye una estrategia analítica en la cual la identificación de la biomolécula se realiza sin la incorporación de etiquetas externas, como fluoróforos, cromóforos o sondas químicas. En el contexto de la espectroscopía Raman mejorada en superficie, este enfoque resulta especialmente relevante, ya que permite aprovechar directamente las señales vibracionales intrínsecas del ADN, simplificando el procedimiento experimental y reduciendo posibles interferencias.

\subsubsection{Significado de la detección sin marcadores}

En los métodos tradicionales de detección de ADN, es común el uso de marcadores fluorescentes que se unen a la molécula objetivo con el fin de facilitar su detección. Aunque estos marcadores permiten una alta sensibilidad, su uso implica etapas adicionales de preparación de la muestra y puede alterar la estructura o el comportamiento natural del ADN. En contraste, la detección sin marcadores se basa en la observación directa de las propiedades físico-químicas inherentes de la biomolécula, sin modificarla químicamente.

En la técnica SERS \textit{label-free}, el ADN se detecta a partir de su interacción directa con superficies metálicas nanostructuradas, como las nanopartículas de plata. Esta interacción permite amplificar las señales Raman propias del ADN, evitando la necesidad de etiquetas externas y proporcionando una representación más fiel de la estructura molecular del sistema analizado.

\subsubsection{Señales Raman propias del ADN}

El ADN es una macromolécula compleja compuesta por un esqueleto fosfodiéster y por bases nitrogenadas, las cuales presentan modos vibracionales característicos. Estos modos vibracionales dan lugar a señales Raman específicas asociadas a enlaces químicos y a movimientos moleculares tales como estiramientos y flexiones. En un espectro Raman, estas señales se manifiestan como bandas bien definidas en regiones particulares del desplazamiento Raman.

En condiciones normales, las señales Raman del ADN son extremadamente débiles debido a la baja sección eficaz del proceso Raman. Sin embargo, en presencia de nanopartículas de plata, el campo electromagnético local se intensifica de manera significativa debido al efecto plasmónico, lo que permite amplificar las vibraciones características del ADN. Como resultado, es posible detectar directamente las bandas Raman asociadas a las bases nitrogenadas y al esqueleto molecular, incluso a concentraciones muy bajas.

\subsubsection{Ventajas frente a métodos con fluoróforos}

La detección de ADN sin marcadores mediante SERS presenta varias ventajas frente a los métodos basados en fluoróforos. En primer lugar, elimina la necesidad de pasos adicionales de marcaje químico, reduciendo el tiempo de preparación de la muestra y la complejidad experimental. En segundo lugar, evita problemas asociados a la fotodegradación y al fotoblanqueo de los fluoróforos, lo que mejora la estabilidad de la señal durante el análisis.

Adicionalmente, la detección \textit{label-free} permite obtener información vibracional detallada sobre la estructura molecular del ADN, lo que no es posible mediante técnicas puramente fluorescentes. Esta característica resulta especialmente valiosa para el desarrollo de biosensores basados en SERS, ya que combina alta sensibilidad con especificidad molecular, consolidando a la espectroscopía Raman mejorada en superficie como una herramienta prometedora para la detección directa de ADN.