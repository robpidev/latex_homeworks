\subsection{Espectroscopía Raman Mejorada en Superficie (SERS)}

La espectroscopía Raman mejorada en superficie (SERS, por sus siglas en inglés) es una técnica que consiste en incrementar la señal de dispersión Raman de las moléculas que se encuentran en contacto o muy próximas a superficies metálicas rugosas o nanoestructuradas, debido a la resonancia de los electrones del metal con el campo eléctrico de la luz láser incidente, lo que permite amplificar la señal Raman y reducir los efectos de la fluorescencia \parencite[p.~120]{varela2019sers}.

\subsubsection{Efecto plasmónico}

El efecto plasmónico o plasmon de superficie es un fenómeno físico que ocurre cuando la energía de la luz interactúa con la superficie de un metal bajo condiciones específicas, de tal manera que la energía de la luz queda confinada en la superficie del material debido al acoplamiento entre las ondas electromagnéticas de la luz y los electrones libres en el metal \parencite[p.~2]{gutierrez2022amplificacion}.

\begin{figure}[h]
    \centering
    \includegraphics[width=0.6\textwidth]{img/plasm.png}
    \caption{La luz incidente sobre una nanopartícula metálica induce la oscilación colectiva de los electrones de la banda de conducción, fenómeno conocido como plasmón de superficie localizado (LSPR). Fuente: Wikimedia Commons \parencite{wikimediaLSPR}.}
    \label{fig:lspr}
\end{figure}


La condición de resonancia plasmónica puede describirse, en el marco de la aproximación cuasiestática, mediante la función dieléctrica del metal $\varepsilon_m(\omega)$ y la del medio circundante $\varepsilon_d$. Para una nanopartícula esférica, la resonancia ocurre aproximadamente cuando
\[
\mathrm{Re}\left[\varepsilon_m(\omega)\right] = -2\varepsilon_d.
\]
Bajo esta condición, el campo electromagnético cerca de la superficie de la nanopartícula se intensifica notablemente, lo que resulta esencial para el fenómeno SERS.

\subsubsection{Incremento del campo electromagnético}

El incremento del campo electromagnético local es el principal responsable del realce de la señal Raman en SERS. La intensidad de la señal Raman dispersada es proporcional al cuadrado del momento dipolar inducido, el cual depende linealmente del campo eléctrico local. En consecuencia, la intensidad Raman $I_R$ puede expresarse de forma aproximada como
\[
I_R \propto |E_{\text{loc}}|^2 |E_{\text{sc}}|^2,
\]
donde $E_{\text{loc}}$ es el campo electromagnético local en la posición de la molécula y $E_{\text{sc}}$ es el campo asociado a la radiación dispersada. Considerando que ambos campos son amplificados por el efecto plasmónico, se obtiene la conocida relación
\[
I_R^{\text{SERS}} \propto |E_{\text{loc}}|^4.
\]
Esta dependencia de cuarta potencia explica por qué pequeñas variaciones en el campo electromagnético local pueden producir incrementos muy grandes en la señal Raman observada.

\subsubsection{Factores de realce}

El factor de realce en SERS, denotado usualmente como $G$, cuantifica el aumento de la intensidad Raman respecto a la espectroscopía Raman convencional y se define como
\[
G = \frac{I_{\text{SERS}}}{I_{\text{Raman}}}.
\]
Los valores típicos del factor de realce pueden variar desde $10^4$ hasta $10^8$, y en condiciones óptimas, especialmente en regiones denominadas \textit{hot spots}, pueden alcanzar valores del orden de $10^{10}$ o superiores. Estas regiones corresponden a zonas donde el campo electromagnético está fuertemente confinado, como los espacios entre nanopartículas metálicas cercanas o en regiones de alta curvatura de la superficie.

Además del mecanismo electromagnético, el factor de realce total puede incluir una contribución adicional asociada al realce químico, el cual está relacionado con la interacción electrónica entre la molécula y la superficie metálica. No obstante, el mecanismo electromagnético es el dominante y responsable de la mayor parte del realce observado en SERS.

\subsubsection{Importancia de los metales nobles (Ag, Au)}

La elección del material metálico es un aspecto crucial en la técnica SERS. Los metales nobles, en particular la plata (Ag) y el oro (Au), son los más utilizados debido a sus propiedades ópticas favorables en el rango visible y cercano al infrarrojo. Estos metales presentan una alta densidad de electrones libres y una función dieléctrica adecuada para soportar resonancias plasmónicas bien definidas con bajas pérdidas.

La plata destaca por proporcionar los mayores factores de realce electromagnético, ya que presenta menores pérdidas ópticas en el rango visible, lo que conduce a campos locales más intensos. Por su parte, el oro ofrece una mayor estabilidad química y biocompatibilidad, lo que resulta ventajoso en aplicaciones biomédicas. En el contexto de la detección de ADN, el uso de nanopartículas de plata mediadas por ADN combina un alto factor de realce con la posibilidad de controla
