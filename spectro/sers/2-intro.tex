\section{Introducción}

La espectroscopía de microondas constituye una de las técnicas más precisas y reveladoras en el estudio de la estructura molecular. Su desarrollo, a mediados del siglo XX, permitió por primera vez determinar con alta exactitud las distancias de enlace y los momentos de inercia de moléculas polares, abriendo una ventana directa hacia la comprensión de la dinámica rotacional de los sistemas cuánticos \parencite{banwell1994fundamentals}. A diferencia de otras ramas de la espectroscopía, como la óptica o la infrarroja, la espectroscopía de microondas se centra en las \textbf{transiciones entre niveles de energía rotacional}, cuya separación energética corresponde al rango de frecuencias comprendido entre 1 y 300~GHz \parencite{hollas2004modern}.

Desde el punto de vista físico, el estudio de estas transiciones proporciona información fundamental sobre la \textbf{distribución de masa} y la \textbf{simetría molecular}, permitiendo validar modelos teóricos derivados directamente de la mecánica cuántica. En particular, el análisis de los espectros rotacionales ofrece una conexión directa entre los parámetros observables (frecuencias de transición) y las magnitudes internas de la molécula (momentos de inercia y longitudes de enlace). Esta relación convierte a la espectroscopía de microondas en una herramienta esencial tanto en física molecular como en química cuántica y astrofísica \parencite{bernath2016spectra}.

El problema central que se aborda en esta monografía es la comprensión del origen cuántico de los \textbf{niveles rotacionales} y de las \textbf{reglas de selección} que determinan las líneas observadas en los espectros de microondas. Para ello, se parte de la \textbf{ecuación de Schrödinger para el rotor rígido lineal}, cuya solución permite deducir las expresiones de energía en función del número cuántico rotacional $J$, así como las condiciones que debe cumplir una molécula para exhibir transiciones activas en el dominio de las microondas (presencia de momento dipolar permanente) \parencite{atkins2018molecular}.

Los \textbf{objetivos} de este trabajo son: 
\begin{itemize}
    \item Derivar las expresiones teóricas de los niveles de energía rotacional a partir del modelo del rotor rígido.
    \item Analizar las reglas de selección que gobiernan las transiciones entre niveles.
    \item Extender el estudio al rotor no rígido, incluyendo las correcciones centrífugas.
    \item Aplicar los resultados al análisis del espectro rotacional de moléculas diatómicas típicas, como HCl y CO.
    \item Analizar el funcionamiento del \textbf{espectrómetro de microondas de pulsos chirpeados} (\textit{Chirped Pulse Fourier Transform Microwave Spectrometer}, CP-FTMW), identificando sus componentes principales y su papel en la detección de señales rotacionales.
    \item Identificar las aplicaciones prácticas de la espectroscopía de microondas en campos como la química cuántica, la detección molecular y la radioastronomía.
\end{itemize}

La presente monografía está organizada de la siguiente manera: en la\textbf{Sección~2}
se desarrolla el marco teórico, partiendo de la ecuación de Schrödinger para un rotor rígido y presentando las deducciones matemáticas que conducen a las expresiones de energía rotacional.
En la \textbf{Sección~3} se discuten las aplicaciones y ejemplos representativos.
Finalmente, la \textbf{Sección~4} contiene las conclusiones,
la \textbf{Sección~5} recomendaciones derivadas del estudio teórico.
y en los \textbf{anexos} algunos cálculos detallados.
