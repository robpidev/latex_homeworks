\section{Introducción}

La detección sensible y específica de ácido desoxirribonucleico (ADN) constituye un aspecto fundamental en diversas áreas de la ciencia y la tecnología, como la biomedicina, el diagnóstico molecular, la genética y el desarrollo de biosensores. La identificación de secuencias de ADN a bajas concentraciones es esencial para la detección temprana de enfermedades, el análisis genético y el monitoreo de procesos biológicos. No obstante, muchos de los métodos convencionales utilizados para este fin, como la reacción en cadena de la polimerasa (PCR) o las técnicas basadas en fluorescencia, requieren procedimientos complejos, el uso de marcadores químicos y un control experimental riguroso, lo que incrementa el costo y el tiempo de análisis.

La espectroscopía Raman se presenta como una técnica analítica capaz de proporcionar información molecular detallada a partir de las vibraciones características de las moléculas. Sin embargo, su aplicación directa está limitada por la baja intensidad de la señal Raman. Para superar esta limitación, se ha desarrollado la espectroscopía Raman mejorada en superficie (Surface-Enhanced Raman Spectroscopy, SERS), la cual permite un incremento significativo de la señal Raman mediante la interacción de las moléculas con superficies metálicas nanostructuradas, especialmente aquellas basadas en metales nobles como la plata.

En los últimos años, los métodos de detección sin marcadores (label-free) han cobrado especial relevancia debido a que permiten identificar biomoléculas sin la necesidad de etiquetas externas, reduciendo la complejidad experimental y posibles interferencias. En este contexto, el crecimiento de nanopartículas de plata mediado por ADN constituye una estrategia eficiente para la generación de regiones de alto campo electromagnético, conocidas como \textit{hot spots}, que potencian la sensibilidad de la técnica SERS y posibilitan la detección de ADN a muy bajas concentraciones.

En función de lo expuesto, la presente monografía se plantea los siguientes objetivos:

\subsection*{Objetivo general}
\begin{enumerate}
    \item Analizar el principio y la eficacia de la espectroscopía Raman mejorada en superficie sin marcadores para la detección sensible de ADN mediante el crecimiento de nanopartículas de plata mediado por ADN.
\end{enumerate}

\subsection*{Objetivos específicos}
\begin{enumerate}
    \item Explicar el fundamento físico de la espectroscopía Raman y de la espectroscopía Raman mejorada en superficie (SERS).
    \item Describir el papel del ADN en el crecimiento de nanopartículas de plata.
    \item Analizar el mecanismo de detección sin marcadores (\textit{label-free}) aplicado a la detección de ADN.
    \item Identificar las principales ventajas y limitaciones del método propuesto.
\end{enumerate}
