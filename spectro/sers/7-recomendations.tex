\section{Recomendaciones}

El presente trabajo ha permitido comprender las bases teóricas del espectro rotacional molecular y su relación con la estructura interna de las moléculas. Sin embargo, existen diversos aspectos que pueden abordarse en estudios posteriores para ampliar la comprensión del fenómeno.

\begin{itemize}
    \item Extender el análisis hacia modelos de \textbf{rotores no lineales} o de \textbf{simetría asimétrica}, lo cual permitiría incluir un mayor número de moléculas y comprender la complejidad de sus espectros rotacionales.
    \item Considerar los \textbf{efectos hiperfinos} y los \textbf{acoplamientos espín-rotación}, que introducen pequeñas correcciones observables en las transiciones y son relevantes en moléculas con núcleos con espín diferente de cero.
    \item Implementar el uso de \textbf{software de simulación espectroscópica} como \textit{PGOPHER} o \textit{SpecView} para la visualización y análisis de espectros rotacionales teóricos y experimentales, facilitando la comparación con datos reales.
    \item Relacionar el estudio de la \textbf{espectroscopía de microondas} con su aplicación en \textbf{radioastronomía}, donde la detección de moléculas interestelares mediante sus transiciones rotacionales permite obtener información sobre la composición y dinámica de las nubes moleculares del medio interestelar.
\end{itemize}

Estas recomendaciones apuntan a consolidar la conexión entre la teoría cuántica de la rotación molecular y las aplicaciones experimentales y observacionales de la espectroscopía moderna.
