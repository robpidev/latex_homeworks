\appendix
\section{Anexo: Deducciones complementarias}

\subsection*{1. Operador momento angular y sus propiedades}

El operador de momento angular orbital $\hat{\vec{L}}$ se define en mecánica cuántica como:
\begin{equation}
\hat{\vec{L}} = \hat{\vec{r}} \times \hat{\vec{p}},
\end{equation}
donde $\hat{\vec{p}} = -i\hbar \nabla$ es el operador momento lineal.  
En coordenadas cartesianas, las componentes de $\hat{\vec{L}}$ son:
\begin{equation}
\hat{L}_x = -i\hbar (y \partial_z - z \partial_y), \quad
\hat{L}_y = -i\hbar (z \partial_x - x \partial_z), \quad
\hat{L}_z = -i\hbar (x \partial_y - y \partial_x).
\end{equation}

Estas componentes satisfacen las relaciones de conmutación:
\begin{equation}
[\hat{L}_x, \hat{L}_y] = i\hbar \hat{L}_z, \quad
[\hat{L}_y, \hat{L}_z] = i\hbar \hat{L}_x, \quad
[\hat{L}_z, \hat{L}_x] = i\hbar \hat{L}_y,
\end{equation}
lo que demuestra que el momento angular es un generador de rotaciones.  
A partir de estas relaciones, se deduce que $\hat{L}^2 = \hat{L}_x^2 + \hat{L}_y^2 + \hat{L}_z^2$ conmuta con cada componente $\hat{L}_i$, permitiendo la existencia de funciones propias simultáneas:
\[
\hat{L}^2 Y_{J,M} = \hbar^2 J(J+1) Y_{J,M}, \qquad
\hat{L}_z Y_{J,M} = \hbar M Y_{J,M}.
\]

\subsection*{2. Masa reducida del sistema diatómico}

Considérese una molécula diatómica de masas $m_1$ y $m_2$ y coordenadas $\vec{r}_1$ y $\vec{r}_2$.  
El vector de posición relativa es $\vec{r} = \vec{r}_2 - \vec{r}_1$ y el centro de masas:
\begin{equation}
\vec{R} = \frac{m_1 \vec{r}_1 + m_2 \vec{r}_2}{m_1 + m_2}.
\end{equation}
El lagrangiano total puede separarse en una parte asociada al movimiento del centro de masas y otra al movimiento relativo.  
La masa efectiva que gobierna este movimiento relativo es la \textbf{masa reducida}:
\begin{equation}
\mu = \frac{m_1 m_2}{m_1 + m_2}.
\end{equation}
El momento de inercia del sistema respecto al centro de masas es entonces:
\[
I = \mu r^2.
\]

\subsection*{3. Armónicos esféricos y sus propiedades}

Las soluciones angulares de la ecuación de Schrödinger para un rotor rígido o para el átomo de hidrógeno son los \textbf{armónicos esféricos}:
\begin{equation}
Y_{J,M}(\theta, \phi) =
(-1)^M
\sqrt{\frac{(2J+1)}{4\pi}
\frac{(J-M)!}{(J+M)!}}
P_J^M(\cos \theta) e^{iM\phi},
\end{equation}
donde $P_J^M$ son los polinomios asociados de Legendre.  
Los armónicos esféricos cumplen:
\begin{equation}
\int_0^{2\pi}\int_0^\pi
Y_{J,M}^*(\theta, \phi) Y_{J',M'}(\theta, \phi)
\sin \theta\, d\theta\, d\phi =
\delta_{J,J'} \delta_{M,M'}.
\end{equation}

\begin{table}[h!]
\centering
\caption{Propiedades relevantes de los armónicos esféricos.}
\begin{tabular}{lll}
\hline
\textbf{Propiedad} & \textbf{Expresión} & \textbf{Comentario} \\
\hline
Ortogonalidad & $\displaystyle \int Y_{J,M}^* Y_{J',M'} d\Omega = \delta_{J,J'}\delta_{M,M'}$ & Estados distintos son ortogonales.\\
Norma & $\displaystyle \int |Y_{J,M}|^2 d\Omega = 1$ & Normalización unitaria.\\
Simetría & $Y_{J,-M} = (-1)^M Y_{J,M}^*$ & Propiedad de paridad compleja.\\
Número cuántico & $J = 0, 1, 2, \ldots$; $M = -J, \ldots, J$ & Describe los estados rotacionales.\\
\hline
\end{tabular}
\end{table}

\subsection*{4. Deducción del Hamiltoniano rotacional}

El operador Hamiltoniano para un sistema de dos partículas sometidas únicamente a fuerzas internas es:
\begin{equation}
\hat{H} = -\frac{\hbar^2}{2m_1} \nabla_1^2 - \frac{\hbar^2}{2m_2} \nabla_2^2 + V(|\vec{r}_1 - \vec{r}_2|).
\end{equation}
Transformando a coordenadas del centro de masas $(\vec{R}, \vec{r})$, se separa la ecuación de Schrödinger en dos partes:
\[
\hat{H} = -\frac{\hbar^2}{2M} \nabla_R^2 - \frac{\hbar^2}{2\mu} \nabla_r^2 + V(r),
\]
donde $M = m_1 + m_2$.  
La segunda parte describe el movimiento relativo.  
En coordenadas esféricas, el laplaciano se escribe:
\[
\nabla_r^2 = \frac{1}{r^2}\frac{\partial}{\partial r}\left(r^2 \frac{\partial}{\partial r}\right)
- \frac{\hat{L}^2}{\hbar^2 r^2}.
\]
Sustituyendo en la ecuación de Schrödinger, se obtiene:
\begin{equation}
\hat{H} = -\frac{\hbar^2}{2\mu} \left[
\frac{1}{r^2}\frac{\partial}{\partial r}\left(r^2 \frac{\partial}{\partial r}\right)
- \frac{\hat{L}^2}{\hbar^2 r^2}
\right] + V(r).
\end{equation}
Si la longitud del enlace se mantiene fija en $r = r_e$, el término radial desaparece y el Hamiltoniano se reduce a:
\begin{equation}
\hat{H}_{\text{rot}} = \frac{\hat{L}^2}{2I},
\end{equation}
con $I = \mu r_e^2$.  
Este operador describe la energía cinética rotacional pura de la molécula diatómica, tal como se presenta en la Ecuación~(6) del cuerpo principal del texto.

\subsection*{5. Deducción del operador momento angular en coordenadas esféricas}

En mecánica cuántica, el operador de momento angular total se define en términos del operador posición $\vec{r}$ y el operador momento lineal $\vec{p}$ como:
\[
\hat{\vec{L}} = \hat{\vec{r}} \times \hat{\vec{p}}.
\]
En el espacio de coordenadas, el operador momento lineal es:
\[
\hat{\vec{p}} = -i\hbar \nabla,
\]
por lo que:
\[
\hat{\vec{L}} = -i\hbar (\vec{r} \times \nabla).
\]

Para expresar $\hat{\vec{L}}$ en coordenadas esféricas $(r, \theta, \phi)$, se utiliza el operador nabla en su forma correspondiente:
\[
\nabla = \hat{r} \frac{\partial}{\partial r}
        + \hat{\theta} \frac{1}{r} \frac{\partial}{\partial \theta}
        + \hat{\phi} \frac{1}{r \sin\theta} \frac{\partial}{\partial \phi}.
\]
Luego, tomando el producto vectorial $\vec{r} \times \nabla$ y considerando que $\vec{r} = r \hat{r}$, se obtiene:

\[
\vec{r} \times \nabla = 
\hat{\theta}\left(-\frac{\partial}{\partial \phi}\right)
+ \hat{\phi}\left(\frac{\partial}{\partial \theta}\right).
\]

Por lo tanto, los componentes cartesianos del operador de momento angular en coordenadas esféricas son:

\[
\begin{aligned}
\hat{L}_x &= i\hbar \left(\sin\phi \frac{\partial}{\partial \theta}
 + \cot\theta \cos\phi \frac{\partial}{\partial \phi} \right), \\[6pt]
\hat{L}_y &= i\hbar \left(-\cos\phi \frac{\partial}{\partial \theta}
 + \cot\theta \sin\phi \frac{\partial}{\partial \phi} \right), \\[6pt]
\hat{L}_z &= -i\hbar \frac{\partial}{\partial \phi}.
\end{aligned}
\]

El operador del momento angular total se define como:
\[
\hat{L}^2 = \hat{L}_x^2 + \hat{L}_y^2 + \hat{L}_z^2,
\]
y su forma explícita en coordenadas esféricas es:
\[
\boxed{
\hat{L}^2 = -\hbar^2 \left[
\frac{1}{\sin\theta} \frac{\partial}{\partial \theta}
\left(\sin\theta \frac{\partial}{\partial \theta}\right)
+ \frac{1}{\sin^2\theta} \frac{\partial^2}{\partial \phi^2}
\right].
}
\]

Esta expresión aparece directamente en la ecuación de Schrödinger para el rotor rígido y es fundamental en la deducción de las funciones de onda esféricas armónicas $Y_{J,M}(\theta,\phi)$, que constituyen los autoestados simultáneos de $\hat{L}^2$ y $\hat{L}_z$:
\[
\hat{L}^2 Y_{J,M} = \hbar^2 J(J+1) Y_{J,M}, \qquad
\hat{L}_z Y_{J,M} = \hbar M Y_{J,M}.
\]


\subsection*{6. Operador momento dipolar y teoría de perturbaciones}

El \textbf{momento dipolar eléctrico} de una molécula se define clásicamente como:
\begin{equation}
\vec{\mu} = \sum_i q_i \vec{r}_i,
\end{equation}
donde $q_i$ son las cargas parciales de los átomos y $\vec{r}_i$ sus posiciones relativas al centro de masas.  
En el marco cuántico, se promueve a operador:
\begin{equation}
\hat{\vec{\mu}} = \sum_i q_i \hat{\vec{r}}_i.
\end{equation}

Cuando la molécula se somete a un campo electromagnético externo $\vec{E}(t)$, la energía de interacción viene dada por:
\begin{equation}
\hat{H}'(t) = -\hat{\vec{\mu}} \cdot \vec{E}(t).
\end{equation}
De acuerdo con la \textbf{teoría de perturbaciones dependiente del tiempo}, la probabilidad de transición entre un estado inicial $|i\rangle$ y uno final $|f\rangle$ es:
\begin{equation}
P_{i \to f} \propto 
\left| \langle f | \hat{\vec{\mu}} \cdot \vec{E} | i \rangle \right|^2
\delta(E_f - E_i - h\nu),
\end{equation}
lo que implica que solo se observan transiciones cuando la integral de momento de transición es diferente de cero.  
De aquí se deduce la regla de selección rotacional:
\[
\Delta J = \pm 1,
\]
ya que el operador dipolar se comporta como un tensor de rango 1, y su acción conecta estados angulares con diferencia unitaria de momento angular total.

Este análisis muestra cómo la teoría de perturbaciones proporciona el marco formal que justifica la interacción radiación–molécula y el origen cuántico de las líneas observadas en la espectroscopía de microondas.
