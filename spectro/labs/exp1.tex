\section{Determinación de la concentración de glucosa por un método enzimático colorimétrico}

\subsection{Objetivos}
\begin{enumerate}
    \item Construir la curva de calibración de glucosa utilizando disoluciones patrón y medir su absorbancia a 505 nm.
    \item Ajustar los datos a una línea recta mediante regresión lineal para establecer la relación entre absorbancia y concentración de glucosa.
    \item Determinar la concentración de glucosa en los tubos preparados con las muestras problema (tubos 6 a 11) utilizando la ecuación de la recta de calibración.
    \item Calcular la concentración de glucosa en las muestras originales (A y B) corrigiendo por el factor de dilución de cada tubo.
\end{enumerate}

\subsection{Materiales}
\begin{enumerate}
  \item Dos muestras problema para la determinación de la concentración de glucosa.
  \item Disolución patrón de glucosa, concentración $10\,\mathrm{mg/dL}$.
  \item Disolución salina de NaCl, concentración $0.9\,\mathrm{g/L}$.
  \item Reactivo cromógeno (R), que contiene:
    \begin{itemize}
      \item Glucosa oxidasa (EC~1.1.3.4).
      \item Peroxidasa (EC~1.11.1.7).
      \item Fenol.
      \item 4-aminofenazona (PubChem CID: 6009; InChIKey: RMMXTBMQSGEXHJ-UHFFFAOYSA-N).
      \item Tampón Tris $92\,\mathrm{mM}$, $\mathrm{pH}=7.4$.
    \end{itemize}
  \item Espectrofotómetro ajustable a una longitud de onda de $505\,\mathrm{nm}$.
  \item Micropipetas de volumen variable P200 y P1000.
  \item Tubos de ensayo.
\end{enumerate}


\begin{figure}[H]
    \centering
    \includegraphics[width=0.8\textwidth]{img/e2.png}
    \caption{Materiales del laboratorio virtual.}
    \label{fig:e1}
\end{figure}


\subsection{Procedimiento}
\begin{enumerate}
  \item Encender el espectrofotómetro y dejarlo estabilizar durante unos minutos.
        Ajustar la longitud de onda a $\lambda = 505\,\mathrm{nm}$.

  \item Rotular doce tubos de ensayo del 0 al 11, correspondientes al blanco,
        los patrones de glucosa y las muestras problema.

  \item Preparar el blanco (tubo 0) añadiendo $1000\,\mu\mathrm{L}$ de disolución
        salina de NaCl $0.9\,\mathrm{g/L}$.

  \item Preparar los tubos patrón de glucosa (tubos 1 a 5) pipeteando los volúmenes
        indicados de disolución patrón de glucosa ($10\,\mathrm{mg/dL}$) y completando
        hasta un volumen total de $1000\,\mu\mathrm{L}$ con disolución salina.

  \item Preparar los tubos de las muestras problema (tubos 6 a 11) añadiendo los
        volúmenes correspondientes de las muestras A y B, y completando cada tubo
        hasta $1000\,\mu\mathrm{L}$ con disolución salina.

  \item Añadir a cada tubo un volumen fijo de reactivo cromógeno (R), asegurando que
        el mismo volumen se agregue a todos los tubos, incluido el blanco.

  \item Mezclar suavemente el contenido de cada tubo mediante agitación manual o
        con un vortex, evitando la formación de burbujas.

  \item Incubar las mezclas durante el tiempo indicado por el método (por ejemplo,
        $10$--$15$ minutos) a temperatura ambiente para permitir el desarrollo del
        color.

  \item Ajustar el espectrofotómetro con el blanco (tubo 0) a absorbancia cero.

  \item Medir la absorbancia de los tubos patrón (1 a 5) a $\lambda = 505\,\mathrm{nm}$
        y registrar los valores obtenidos.

  \item Construir la curva de calibración graficando la absorbancia en función de la
        concentración de glucosa de los patrones y obtener la recta de aproximación.

  \item Medir la absorbancia de las muestras problema (tubos 6 a 11) a
        $\lambda = 505\,\mathrm{nm}$.

  \item Calcular la concentración de glucosa en cada tubo de muestra utilizando la
        ecuación de la recta de calibración.

  \item Corregir la concentración obtenida en cada tubo por el factor de dilución
        correspondiente para determinar la concentración de glucosa en las muestras
        originales A y B.
\end{enumerate}

\subsection{Datos para la calibración}
\begin{table}[h]
\centering
\begin{tabular}{ccc}
\toprule
Tubo & Patrón de glucosa ($\mu$L) & Disolución salina ($\mu$L) \\
\midrule
0 & 0   & 1000 \\
1 & 30  & 970  \\
2 & 90  & 910  \\
3 & 150 & 850  \\
4 & 300 & 700  \\
5 & 500 & 500  \\
\bottomrule
\end{tabular}
\caption{Preparación de los tubos patrón para la curva de calibración de glucosa (tubos 0–5).}
\end{table}

\subsubsection{Cálculo de la concentración de clucosa}

La concentración del patrón madre de glucosa es:

\[
C_0 = 10 \, \text{mg/dL}
\]

El volumen total en cada tubo es constante:

\[
V_{\text{total}} = 1000 \, \mu\text{L}
\]

La concentración de glucosa en cada tubo patrón se calcula mediante:

\[
C_i = C_0 \frac{V_{\text{patrón},i}}{V_{\text{total}}}
\]

Sustituyendo los valores experimentales:

\[
C_i = 10 \frac{V_{\text{patrón},i}}{1000} \quad (\text{mg/dL})
\]

La relación entre la absorbancia y la concentración está dada por la recta de calibración:

\[
A = a C + b
\]

donde \(a\) es la pendiente y \(b\) la ordenada al origen.  
Despejando la concentración:

\[
C = \frac{A - b}{a}
\]

\vspace{0.5cm}

\begin{table}[h]
\centering
\begin{tabular}{cccc}
\toprule
Tubo & Patrón de glucosa ($\mu$L) & Disolución salina ($\mu$L) & Concentración (mg/dL) \\
\midrule
0 & 0   & 1000 & 0.00 \\
1 & 30  & 970  & 0.30 \\
2 & 90  & 910  & 0.90 \\
3 & 150 & 850  & 1.50 \\
4 & 300 & 700  & 3.00 \\
5 & 500 & 500  & 5.00 \\
\bottomrule
\end{tabular}
\caption{Preparación de los tubos patrón y concentraciones de glucosa utilizadas para la curva de calibración.}
\end{table}


\subsubsection{Resultados de Absorbancia}
\begin{table}[h]
\centering
\begin{tabular}{ccc}
\toprule
Tubo & Concentración de glucosa (mg/dL) & Absorbancia (505 nm) \\
\midrule
0 & 0.00 & 0.000 \\
1 & 0.30 & 0.058 \\
2 & 0.90 & 0.159 \\
3 & 1.50 & 0.261 \\
4 & 3.00 & 0.519 \\
5 & 5.00 & 0.835 \\
\bottomrule
\end{tabular}
\caption{Absorbancia medida a 505 nm en función de la concentración de glucosa para los tubos patrón.}
\end{table}


\subsubsection{Regresión lineal de la curva de calibración}

A partir de los datos experimentales de concentración de glucosa \(C\) y
absorbancia \(A\), se realiza un ajuste lineal del tipo:

\begin{equation*}
A = aC + b
\end{equation*}

El número de puntos experimentales es:

\begin{equation*}
n = 6
\end{equation*}

Los valores promedio de las variables son:

\begin{equation*}
\bar{x} = \frac{1}{n}\sum_{i=1}^{n} x_i = 1.7833 \, \text{mg/dL}
\end{equation*}

\begin{equation*}
\bar{y} = \frac{1}{n}\sum_{i=1}^{n} y_i = 0.3053
\end{equation*}

Las sumas necesarias para la regresión lineal se definen como:

\begin{equation*}
S_{xx} = \sum_{i=1}^{n} (x_i - \bar{x})^2 = 18.0683
\end{equation*}

\begin{equation*}
S_{yy} = \sum_{i=1}^{n} (y_i - \bar{y})^2
\end{equation*}

\begin{equation*}
S_{xy} = \sum_{i=1}^{n} (x_i - \bar{x})(y_i - \bar{y}) = 3.0169
\end{equation*}

La pendiente de la recta de calibración se obtiene mediante:

\begin{equation*}
a = \frac{S_{xy}}{S_{xx}} = 0.16697
\end{equation*}

La ordenada al origen está dada por:

\begin{equation*}
b = \bar{y} - a\bar{x} = 0.00756
\end{equation*}

Por lo tanto, la ecuación final de la recta de calibración es:

\begin{equation*}
A = 0.16697\,C + 0.00756
\end{equation*}

El coeficiente de determinación se calcula como:

\begin{equation*}
R^2 = \frac{S_{xy}^2}{S_{xx} S_{yy}} = 0.9995
\end{equation*}


\begin{figure}[H]
    \centering
    \includegraphics[width=0.8\textwidth]{img/e12.png}
    \caption{Regresión lineal de la curva de calibración.}
\end{figure}



\subsection{Muestras}

\begin{table}[h]
\centering
\begin{tabular}{ccc}
\toprule
Tubo & Muestra A ($\mu$L) & Disolución salina ($\mu$L) \\
\midrule
6 & 80 & 920 \\
7 & 40 & 960 \\
8 & 20 & 980 \\
\bottomrule
\end{tabular}
\caption{Preparación de los tubos correspondientes a la muestra A.}
\end{table}

\begin{table}[h]
\centering
\begin{tabular}{ccc}
\toprule
Tubo & Muestra B ($\mu$L) & Disolución salina ($\mu$L) \\
\midrule
9  & 200 & 800 \\
10 & 400 & 600 \\
11 & 700 & 300 \\
\bottomrule
\end{tabular}
\caption{Preparación de los tubos correspondientes a la muestra B.}
\end{table}

\subsubsection{Cálculo de la concentración de glucosa a partir de la absorbancia}
La concentración de glucosa en cada tubo correspondiente a las muestras problema se
obtiene despejando la concentración a partir de la ecuación de la recta:

\begin{equation}
C_{\text{tubo}} = \frac{A - b}{a}
\end{equation}

Asi para los datos se tiene

\begin{table}[h]
\centering
\begin{tabular}{ccc}
\toprule
Tubo & Muestra A ($\mu$L) & Concentración en el tubo (mg/dL) \\
\midrule
6 & 80 & 7.76 \\
7 & 40 & 3.91 \\
8 & 20 & 1.91 \\
\bottomrule
\end{tabular}
\caption{Concentración de glucosa en los tubos correspondientes a la muestra A.}
\end{table}

\begin{table}[h]
\centering
\begin{tabular}{ccc}
\toprule
Tubo & Muestra B ($\mu$L) & Concentración en el tubo (mg/dL) \\
\midrule
9  & 200 & 0.06 \\
10 & 400 & 0.02 \\
11 & 700 & 0.09 \\
\bottomrule
\end{tabular}
\caption{Concentración de glucosa en los tubos correspondientes a la muestra B.}
\end{table}

\subsubsection{Cálculo de la concentración de glucosa en las muestras originales}

La concentración de glucosa en la muestra original se obtiene aplicando la ecuación:

\begin{equation}
\label{eq:conc_muestra}
C_{\text{muestra}} = C_{\text{tubo}} \frac{V_{\text{total}}}{V_{\text{muestra}}}
\end{equation}

donde:
\begin{itemize}
    \item $C_{\text{tubo}}$ es la concentración de glucosa calculada en el tubo de ensayo a partir de la curva de calibración,
    \item $V_{\text{total}} = 1000 \, \mu\text{L}$ es el volumen total del tubo,
    \item $V_{\text{muestra}}$ es el volumen de muestra añadido al tubo.
\end{itemize}

\subsection{Resultados}

\paragraph{Muestra A (tubos 6--8)}

\begin{table}[h]
\centering
\begin{tabular}{cccc}
\toprule
Tubo & Muestra A ($\mu$L) & Concentración en el tubo (mg/dL) & Concentración real (mg/dL) \\
\midrule
6 & 80 & 7.76 & 97.0 \\
7 & 40 & 3.91 & 97.8 \\
8 & 20 & 1.91 & 95.5 \\
\bottomrule
\end{tabular}
\caption{Concentración de glucosa en los tubos y concentración final en la muestra A.}
\end{table}

Promedio de la muestra A: $\bar{C}_A \approx 96.8 \, \text{mg/dL}$

\paragraph{Muestra B (tubos 9--11)}

\begin{table}[h]
\centering
\begin{tabular}{cccc}
\toprule
Tubo & Muestra B ($\mu$L) & Concentración en el tubo (mg/dL) & Concentración real (mg/dL) \\
\midrule
9  & 200 & 0.063 & 0.315 \\
10 & 400 & 0.021 & 0.0525 \\
11 & 700 & 0.093 & 0.133 \\
\bottomrule
\end{tabular}
\caption{Concentración de glucosa en los tubos y concentración final en la muestra B.}
\end{table}

Promedio de la muestra B: $\bar{C}_B \approx 0.167 \, \text{mg/dL}$


\subsection{Conclusiones}
\begin{enumerate}
    \item \textbf{Construcción de la curva de calibración:} La curva de calibración permitió correlacionar de manera directa la absorbancia medida con la concentración de glucosa en los tubos patrón.  
    \item \textbf{Ajuste lineal:} El ajuste de los datos a una línea recta fue adecuado, obteniendo la ecuación $A = 0.16697\,C + 0.00756$, que sirve para estimar concentraciones en tubos con muestras problema.
    \item \textbf{Determinación en los tubos problema:} Las concentraciones calculadas en los tubos 6 a 11 muestran que las muestras se encuentran dentro del rango medible por la curva de calibración, permitiendo su cuantificación precisa.
    \item \textbf{Concentración en las muestras originales:} Aplicando el factor de dilución, se obtuvieron las concentraciones finales de glucosa: $\bar{C}_A \approx 96.8 \, \text{mg/dL}$ y $\bar{C}_B \approx 0.167 \, \text{mg/dL}$, mostrando que la muestra A es significativamente más concentrada que la B.
\end{enumerate}