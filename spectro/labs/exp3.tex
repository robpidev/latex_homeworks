\section{Determinación de la concentración de proteínas por un método colorimétrico: método de Lowry}

\subsection{Objetivos}

\begin{enumerate}
    \item Determinar la concentración de proteínas en muestras problema mediante el método colorimétrico de Lowry, utilizando seroalbúmina bovina como proteína patrón.
    \item Establecer una curva de calibración a partir de soluciones patrón de albúmina y emplearla para calcular la concentración de proteínas en las muestras de partida A y B a partir de sus valores de absorbancia.
\end{enumerate}

\subsection{Materiales}

\begin{enumerate}
    \item Dos muestras problema en las que se desea determinar la concentración de proteínas.
    \item Disolución patrón de seroalbúmina bovina (BSA), concentración 2 g/L.
    \item Reactivo C, preparado mezclando los reactivos A, B1 y B2 en proporción 100 : 1 : 1 (v/v).
    \item Reactivo A: carbonato sódico (Na$_2$CO$_3$) al 2\% en hidróxido sódico (NaOH) 0.1 M.
    \item Reactivo B1: sulfato de cobre pentahidratado (CuSO$_4\cdot$5H$_2$O) al 1\%.
    \item Reactivo B2: tartrato sódico-potásico al 2\%.
    \item Reactivo de Folin--Ciocalteau diluido 1/4 en agua.
\end{enumerate}

\begin{figure}[h]
    \centering
    \includegraphics[width=0.8\textwidth]{img/e3.png}
    \caption{Materiales del laboratorio virtual.}
\end{figure}


\begin{figure}[h]
    \centering
    \includegraphics[width=0.8\textwidth]{img/e32.png}
    \caption{Compuestos reales.}
\end{figure}

\subsection{Procedimiento}

\begin{enumerate}
    \item Se conecta el espectrofotómetro y se ajusta la longitud de onda a 580 nm.
    \item Empleando micropipetas P200 y P1000, se preparan los tubos de ensayo con un volumen total de 400~$\mu$L según la Tabla~\ref{tab:lowry_preparacion}.
    \item A cada tubo se añaden 2 mL del reactivo C y se deja reaccionar durante 10 minutos virtuales.
    \item Posteriormente, se añaden 200~$\mu$L del reactivo de Folin--Ciocalteau diluido a cada tubo y se espera 15 minutos virtuales para el desarrollo del color.
    \item El contenido del tubo 0 se transfiere a la cubeta del espectrofotómetro y se ajusta el blanco pulsando ``A = 0''.
    \item Se mide la absorbancia de los tubos restantes (1 a 9) y se registran los valores obtenidos.
\end{enumerate}

\begin{table}[h]
\centering
\begin{tabular}{c c c c}
\hline
Tubo & BSA ($\mu$L) & Muestra ($\mu$L) & Agua ($\mu$L) \\
\hline
0 & 0   & -- & 400 \\
1 & 30  & -- & 370 \\
2 & 60  & -- & 340 \\
3 & 90  & -- & 310 \\
4 & 120 & -- & 280 \\
5 & 150 & -- & 250 \\
6 & --  & 120 (A) & 280 \\
7 & --  & 200 (A) & 200 \\
8 & --  & 120 (B) & 280 \\
9 & --  & 200 (B) & 200 \\
\hline
\end{tabular}
\caption{Preparación de tubos para la determinación de proteínas por el método de Lowry.}
\label{tab:lowry_preparacion}
\end{table}

\subsection{Datos experimentales}

Las absorbancias se miden a una longitud de onda de $\lambda = 580$ nm. Los valores obtenidos se muestran en la Tabla~\ref{tab:lowry_abs}.

\begin{table}[h]
\centering
\begin{tabular}{c c}
\hline
Tubo & Absorbancia (580 nm) \\
\hline
1 & 0.233 \\
2 & 0.445 \\
3 & 0.612 \\
4 & 0.798 \\
5 & 0.850 \\
6 & 0.306 \\
7 & 0.458 \\
8 & 0.760 \\
9 & 0.436 \\
\hline
\end{tabular}
\caption{Valores de absorbancia obtenidos para el método de Lowry.}
\label{tab:lowry_abs}
\end{table}

\subsubsection{Curva de calibración con proteína patrón}

La concentración de proteína en cada tubo patrón se calcula a partir de la disolución de BSA de 2 g/L y el volumen total de 400~$\mu$L.

\begin{table}[h]
\centering
\begin{tabular}{c c c}
\hline
Tubo & Concentración de proteína (g/L) & Absorbancia (580 nm) \\
\hline
0 & 0.00 & 0.000 \\
1 & 0.15 & 0.233 \\
2 & 0.30 & 0.445 \\
3 & 0.45 & 0.612 \\
4 & 0.60 & 0.798 \\
5 & 0.75 & 0.850 \\
\hline
\end{tabular}
\caption{Datos de calibración para el método de Lowry empleando albúmina bovina como patrón.}
\label{tab:lowry_calibracion}
\end{table}

Para la curva de calibración del método de Lowry se ajusta una recta de la forma:

\begin{equation*}
A = m\,c + b
\end{equation*}

donde $A$ es la absorbancia y $c$ la concentración de proteína (g/L).  
El ajuste se realiza mediante el método de mínimos cuadrados, definiendo las siguientes magnitudes:

\begin{equation*}
\bar{c} = \frac{1}{n}\sum c_i
\qquad
\bar{A} = \frac{1}{n}\sum A_i
\end{equation*}

\begin{equation*}
S_{xx} = \sum (c_i - \bar{c})^2
\qquad
S_{xy} = \sum (c_i - \bar{c})(A_i - \bar{A})
\end{equation*}

A partir de los datos de calibración (tubos 1 a 5):

\[
\begin{array}{c c}
c \,(\text{g/L}) & A \\
0.15 & 0.233 \\
0.30 & 0.445 \\
0.45 & 0.612 \\
0.60 & 0.798 \\
0.75 & 0.850 \\
\end{array}
\]

con $n = 5$, se obtienen los valores medios:

\begin{equation*}
\bar{c} = 0.45
\qquad
\bar{A} = 0.588
\end{equation*}

A continuación, se calculan las sumas:

\begin{equation*}
S_{xx} = (0.15-0.45)^2 + (0.30-0.45)^2 + (0.45-0.45)^2 + (0.60-0.45)^2 + (0.75-0.45)^2
= 0.225
\end{equation*}

\begin{gather*}
S_{xy} =
(0.15-0.45)(0.233-0.588) +
(0.30-0.45)(0.445-0.588) +\\
(0.45-0.45)(0.612-0.588) +\\
(0.60-0.45)(0.798-0.588) +
(0.75-0.45)(0.850-0.588)
= 0.236
\end{gather*}

La pendiente de la recta se calcula como:

\begin{equation*}
m = \frac{S_{xy}}{S_{xx}} = \frac{0.236}{0.225} = 1.05
\end{equation*}

La ordenada al origen se obtiene mediante:

\begin{equation*}
b = \bar{A} - m\,\bar{c} = 0.588 - (1.05)(0.45) = 0.12
\end{equation*}

Por lo tanto, la ecuación de la recta de calibración es:

\begin{equation*}
A = 1.05\,c + 0.12
\end{equation*}

Para $R^2$ Se tiene:

\begin{equation*}
S_{yy} = \sum (A_i - \bar{A})^2
\end{equation*}

Usando $\bar{A} = 0.588$, se obtiene:

\begin{equation*}
S_{yy} =
(0.233-0.588)^2 +
(0.445-0.588)^2 +
(0.612-0.588)^2 +
(0.798-0.588)^2 +
(0.850-0.588)^2
= 0.259
\end{equation*}


\begin{equation*}
R^2 = \frac{S_{xy}^2}{S_{xx}\,S_{yy}}
\end{equation*}

Sustituyendo los valores obtenidos:

\begin{equation*}
R^2 = \frac{(0.236)^2}{(0.225)(0.259)} = 0.96
\end{equation*}

Por lo que el es un buen ajuste.
\begin{figure}[H]
    \centering
    \includegraphics[width=0.8\textwidth]{img/e33.png}
    \caption{Curva de calibración del método de Lowry: absorbancia a 580 nm en función de la concentración de proteína (BSA).}
    \label{fig:lowry_calibracion}
\end{figure}

\subsection{Concentración real del patrón y concentración en los tubos de calibración}

La disolución patrón de seroalbúmina bovina (BSA) tiene una concentración real de:

\[
c_{\text{real}} = 2 \; \text{g/L}
\]

La concentración de proteína en cada tubo de calibración se calcula considerando la dilución producida al preparar un volumen total de 400~$\mu$L:

\begin{equation*}
c_{\text{tubo}} = c_{\text{real}} \cdot \frac{V_{\text{patrón}}}{V_{\text{total}}}
\end{equation*}

donde $V_{\text{total}} = 400~\mu\text{L}$.

\begin{table}[h]
\centering
\begin{tabular}{c c c c}
\hline
Tubo & Volumen de BSA ($\mu$L) & Concentración real (g/L) & Concentración en el tubo (g/L) \\
\hline
0 & 0   & 2.00 & 0.00 \\
1 & 30  & 2.00 & 0.15 \\
2 & 60  & 2.00 & 0.30 \\
3 & 90  & 2.00 & 0.45 \\
4 & 120 & 2.00 & 0.60 \\
5 & 150 & 2.00 & 0.75 \\
\hline
\end{tabular}
\caption{Concentración real del patrón de BSA y concentración efectiva en los tubos de calibración.}
\label{tab:calibracion_real_tubo}
\end{table}

\subsection{Cálculo de la concentración de proteínas en las muestras}

La recta de calibración obtenida mediante mínimos cuadrados es:

\begin{equation*}
A = 1.058\,c + 0.112
\end{equation*}

Despejando la concentración de proteína en el tubo:

\begin{equation*}
c_{\text{tubo}} = \frac{A - 0.112}{1.058}
\end{equation*}

La concentración real de la muestra se obtiene corrigiendo por el factor de dilución aplicado durante la preparación del tubo:

\begin{equation*}
c_{\text{real}} = c_{\text{tubo}} \cdot \frac{V_{\text{total}}}{V_{\text{muestra}}}
\end{equation*}

donde $V_{\text{total}} = 400~\mu$L.

\subsubsection{Muestra A}

\begin{table}[h]
\centering
\begin{tabular}{c c c c}
\hline
Tubo & $V_{\text{muestra}}$ ($\mu$L) & $c_{\text{tubo}}$ (g/L) & $c_{\text{real}}$ (g/L) \\
\hline
6 & 120 & 0.184 & 0.613 \\
7 & 200 & 0.328 & 0.655 \\
\hline
\end{tabular}
\caption{Concentración de proteínas en el tubo y concentración real para la muestra A.}
\label{tab:muestraA_final}
\end{table}

\subsubsection{Muestra B}

\begin{table}[h]
\centering
\begin{tabular}{c c c c}
\hline
Tubo & $V_{\text{muestra}}$ ($\mu$L) & $c_{\text{tubo}}$ (g/L) & $c_{\text{real}}$ (g/L) \\
\hline
8 & 120 & 0.613 & 2.043 \\
9 & 200 & 0.307 & 0.613 \\
\hline
\end{tabular}
\caption{Concentración de proteínas en el tubo y concentración real para la muestra B.}
\label{tab:muestraB_final}
\end{table}

\subsection{Conclusiones}
\begin{enumerate}
    \item Determinar la concentración de proteínas en muestras problema mediante el método colorimétrico de Lowry, utilizando seroalbúmina bovina como proteína patrón.
    \item Establecer una curva de calibración a partir de soluciones patrón de albúmina y emplearla para calcular la concentración de proteínas en las muestras de partida A y B a partir de sus valores de absorbancia.
\end{enumerate}