\section{RESUMEN}
El laboratorio virtual de espectrofotometría UV-VIS se desarrolló mediante el uso de un simulador interactivo, el cual permitió comprender el funcionamiento del espectrofotómetro y aplicar sus principios en el análisis cuantitativo de sustancias bioquímicas. A través del simulador, se obtuvieron datos de absorbancia necesarios para la construcción de curvas de calibración y la determinación de la concentración de muestras problema, aplicando la ley de Beer-Lambert.
Durante la práctica se realizaron ensayos colorimétricos para la determinación de glucosa, proteínas mediante el método de Lowry y creatinina, midiendo la absorbancia a longitudes de onda específicas para cada analito. Los resultados evidenciaron una relación directamente proporcional entre la absorbancia y la concentración dentro del rango de trabajo, lo que confirmó la utilidad de la espectrofotometría UV-VIS como una técnica confiable y ampliamente empleada en el análisis bioquímico. Asimismo, el laboratorio virtual permitió reforzar los conceptos teóricos y desarrollar habilidades analíticas sin la necesidad de un entorno experimental físico.
\textbf{Palabras clave: absorbancia, concentración, espectrofotometría UV-VIS, simulador virtual, análisis bioquímico.}
\section{INTRODUCCIÓN}
Dentro de las técnicas analíticas instrumentales se encuentra la espectrofotometría UV-visible, la cual es una de las más empleadas en el ámbito bioquímico debido a su robustez, sencillez y confiabilidad, además de requerir una instrumentación de costo relativamente bajo. Esta técnica permite realizar tanto mediciones directas como aquellas acopladas a otros procesos analíticos, tales como la cromatografía, electroforesis y análisis de fluidos. Su importancia radica en la capacidad de identificar sustancias mediante su espectro de absorción y determinar cuantitativamente la concentración de compuestos presentes en solución, así como en el seguimiento y monitoreo de reacciones químicas y enzimáticas (Skoog et al., 2020).

A nivel mundial, diversas investigaciones han implementado la espectrofotometría UV-VIS como herramienta analítica fundamental en estudios bioquímicos y clínicos. Esta técnica ha sido utilizada para la cuantificación de proteínas mediante métodos colorimétricos, así como para la determinación de metabolitos de interés biológico. Por ejemplo, Flores Ramos y Ruiz Soto (2017) emplearon espectrofotometría UV-VIS para la cuantificación de proteínas, reportando parámetros analíticos como alta linealidad y límites de detección adecuados para el análisis de muestras biológicas. Asimismo, otros estudios han aplicado la espectrofotometría directa para la determinación de hemoglobina libre en plasma, permitiendo el monitoreo de procesos hemolíticos sin el uso de reactivos altamente tóxicos (C. et al., 2020).
En el contexto latinoamericano, la espectrofotometría UV-visible ha sido ampliamente utilizada en investigaciones orientadas al análisis clínico y bioquímico. En Colombia, se ha aplicado esta técnica para la determinación cuantitativa de carboxihemoglobina en muestras de sangre, empleando longitudes de onda específicas que garantizan resultados confiables y reproducibles (Manrique et al., 2016). De igual manera, se ha utilizado para la evaluación del potencial irritante ocular de sustancias químicas y para la cuantificación de proteínas totales mediante métodos como Bradford, seguidos de análisis espectrofotométricos en el rango visible (Torres Cabra et al., 2013).
Por lo anteriormente expuesto, el objetivo del presente laboratorio virtual fue conocer y comprender los principios básicos de la espectrofotometría UV-VIS, así como aplicar esta técnica para la cuantificación de distintas sustancias bioquímicas y el registro de espectros de absorción, mediante el adecuado manejo de un simulador virtual.

\section{MARCO TEÓRICO}

\subsection{Espectrofotometría UV-VIS}
La espectrofotometría UV-VIS es una técnica analítica ampliamente utilizada para la cuantificación de sustancias químicas y bioquímicas en solución. Su principio se basa en la absorción selectiva de radiación electromagnética por parte de una sustancia a una longitud de onda determinada. Según Skoog et al. (2020), “la espectrofotometría de absorción molecular mide la atenuación de la radiación incidente como resultado de la absorción por moléculas en fase líquida”.
La relación entre absorbancia y concentración está descrita por la ley de Beer-Lambert, la cual establece que “la absorbancia de una solución es directamente proporcional a la concentración del analito y a la longitud del trayecto óptico” (Harris, 202). Este principio es la base para la construcción de curvas de calibración empleadas en el laboratorio virtual UV-VIS.

\subsection{Glucosa}
La glucosa es un monosacárido de seis átomos de carbono que desempeña un papel central en el metabolismo energético celular. Es la principal fuente de energía para los tejidos, especialmente el sistema nervioso central y los eritrocitos. La glucosa se obtiene a partir de la digestión de carbohidratos y de procesos metabólicos como la glucogenólisis y la gluconeogénesis. Su regulación es fundamental para mantener el equilibrio metabólico del organismo (Murray et al., 2021).
En los ensayos espectrofotométricos, la glucosa puede determinarse mediante métodos enzimáticos que producen un compuesto coloreado cuya intensidad es proporcional a la concentración de glucosa presente en la muestra.

\subsection{Glucemia}
La glucemia se define como la concentración de glucosa en la sangre y constituye un indicador clave del estado metabólico y endocrino. En condiciones normales, los valores de glucemia se mantienen dentro de un rango estrecho gracias a la acción coordinada de hormonas como la insulina y el glucagón. Alteraciones persistentes en la glucemia pueden estar asociadas a patologías metabólicas, siendo la diabetes mellitus una de las más relevantes desde el punto de vista clínico y bioquímico (American Diabetes Association, 2024).

\subsection{Método de Lowry}
El método de Lowry es una técnica colorimétrica utilizada para la cuantificación de proteínas totales en muestras biológicas. Se basa en la reacción de los enlaces peptídicos con iones cúpricos en medio alcalino y en la posterior reducción del reactivo de Folin–Ciocalteu por aminoácidos aromáticos, generando un complejo coloreado azul. La absorbancia medida mediante espectrofotometría es proporcional a la concentración de proteínas presentes. A pesar del desarrollo de métodos más modernos, el método de Lowry sigue siendo ampliamente empleado por su alta sensibilidad (Sapan et al., 2020).

\subsection{Creatinina}
La creatinina es un producto final del metabolismo de la creatina fosfato en el músculo. Delanaye et al. (2022) indican que “la creatinina sérica es uno de los biomarcadores más utilizados para evaluar la función renal debido a su producción relativamente constante y su eliminación por filtración glomerular” .
En análisis bioquímicos, la creatinina puede cuantificarse mediante métodos colorimétricos cuya absorbancia es directamente proporcional a su concentración, aplicando los principios de la espectrofotometría UV-VIS.
