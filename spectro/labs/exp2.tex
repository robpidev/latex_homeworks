\section{Determinación de la concentración de glucosa por un método enzimático colorimétrico}

\subsection{Objetivos}

\begin{enumerate}
    \item Determinar la concentración de glucosa en muestras de suero sanguíneo mediante un método espectrofotométrico basado en la reacción enzimática glucosa oxidasa--peroxidasa y la ley de Lambert--Beer.
    \item Calcular la glucemia de los sueros problema (A y B) a partir de la comparación de sus absorbancias con un control patrón de glucosa, considerando las diluciones aplicadas durante el procedimiento experimental.
\end{enumerate}


\subsection{Materiales}

\begin{enumerate}
    \item Dos muestras de suero en las que se quiere determinar la concentración de glucosa.
    \item Disolución control de glucosa (o patrón), con concentración verificada de 100 mg/dL.
    \item Disolución salina: NaCl 0.9 g/L.
    \item Reactivo cromógeno (R), que contiene:
    \begin{enumerate}
        \item Glucosa oxidasa (EC 1.1.3.4).
        \item Peroxidasa (EC 1.11.1.7).
        \item Fenol.
        \item 4-aminofenazona (PubChem CID: 6009; InChIKey: RMMXTBMQSGEXHJ-UHFFFAOYSA-N).
        \item Tampón Tris 92 mM, pH = 7.4.
    \end{enumerate}
\end{enumerate}

\begin{figure}[h]
    \centering
    \includegraphics[width=0.8\textwidth]{img/e2.png}
    \caption{Materiales del laboratorio virtual.}
    \label{fig:e1}
\end{figure}

\subsection{Procedimiento}

\begin{enumerate}
    \item Conectar el espectrofotómetro y ajustar la longitud de onda a 505 nm.

    \item Tener en cuenta las siguientes notas importantes:
    \begin{enumerate}
        \item El control de glucosa se analiza por triplicado (tubos 1, 2 y 3).
        \item Las diluciones del suero no se realizan manualmente en el laboratorio virtual; sin embargo, se aplican internamente. Aunque la pipeta tome la muestra del frasco sin diluir, el contenido de los tubos 7 y 10 se considera diluido 1/2, y el de los tubos 8 y 11 diluido 1/5.
    \end{enumerate}

    \item Empleando las micropipetas de volumen variable P200 y P1000, preparar en tubos de ensayo las mezclas indicadas.

    \item Una vez preparados todos los tubos, añadir 2 mL del reactivo cromógeno a cada uno y esperar 15 minutos virtuales para el desarrollo de la reacción.

    \item Transferir con una pipeta Pasteur el contenido del tubo n.º 0 a la cubeta del espectrofotómetro, introducir la cubeta en el equipo y pulsar el botón ``A = 0'' para ajustar el blanco. Retirar la cubeta y vaciarla.

    \item Transferir el contenido del tubo n.º 1 a la cubeta, introducirla en el espectrofotómetro y anotar el valor de absorbancia en el cuaderno de laboratorio. Si se desea, pulsar el botón correspondiente para registrar la lectura en la pantalla inferior.

    \item Repetir el procedimiento descrito en el paso anterior para el resto de los tubos.

    \item Construir una tabla con todas las medidas de absorbancia obtenidas. Opcionalmente, utilizar el botón correspondiente para copiar el listado de valores.
\end{enumerate}

\subsection{calibración}

\begin{table}[h]
\centering
\begin{tabular}{c c c}
\hline
Tubo & Concentración de glucosa (mg/dL) & Absorbancia (505 nm) \\
\hline
1 & 100 & 0.322 \\
2 & 100 & 0.321 \\
3 & 100 & 0.321 \\
\hline
\end{tabular}
\caption{Tabla de calibración espectrofotométrica (tubos 0 a 3) con valores de absorbancia medidos a 505 nm.}
\label{tab:calibracion_abs}
\end{table}

Con estos datos la media de la absorbancia es:

\begin{equation*}
\bar{A}_{\text{control}} = \frac{0.322 + 0.321 + 0.321}{3} = \frac{0.964}{3} = 0.321
\end{equation*}

La concentración de glucosa se calcula por comparación con el control patrón, según:

\begin{equation*}
c = \frac{A}{\bar{A}_{\text{control}}} \times c_{\text{control}}
\end{equation*}

donde $\bar{A}_{\text{control}} = 0.321$ y $c_{\text{control}} = 100$ mg/dL.

\begin{table}[h]
\centering
\begin{tabular}{c c c}
\hline
Tubo & Absorbancia (505 nm) & Concentración de glucosa (mg/dL) \\
\hline
1 & 0.322 & 100.3 \\
2 & 0.321 & 100.0 \\
3 & 0.321 & 100.0 \\
\hline
\end{tabular}
\caption{Concentración de glucosa calculada para los tubos control (1 a 3).}
\label{tab:control_conc}
\end{table}

\subsection{Cálculo de la concentración de glucosa en las muestras problema}

La concentración de glucosa se calcula empleando la ley de Lambert--Beer:

\begin{equation*}
A = \varepsilon \cdot L \cdot c
\end{equation*}

donde $A$ es la absorbancia, $\varepsilon$ es el coeficiente de extinción molar, $L$ es la longitud del camino óptico y $c$ es la concentración. Dado que $\varepsilon$ y $L$ se mantienen constantes para todas las mediciones, la concentración de glucosa se determina por comparación con el control patrón:

\begin{equation*}
c = \frac{A}{\bar{A}_{\text{control}}} \cdot c_{\text{control}}
\end{equation*}

donde $\bar{A}_{\text{control}} = 0.321$ y $c_{\text{control}} = 100$ mg/dL.  
En todos los casos se emplea un volumen de muestra de 20~$\mu$L, al que se añaden 2~mL de reactivo cromógeno.

\subsubsection{Suero A}

\begin{table}[H]
\centering
\begin{tabular}{c c c c}
\hline
Tubo & Volumen de muestra ($\mu$L) & Absorbancia (505 nm) & Concentración de glucosa (mg/dL) \\
\hline
6 & 20 & 0.488 & 152.0 \\
7 & 20 & 0.249 & 77.6 \\
8 & 20 & 0.085 & 26.5 \\
\hline
\end{tabular}
\caption{Volumen empleado, absorbancias medidas y concentraciones de glucosa calculadas para el suero A.}
\label{tab:sueroA}
\end{table}

La concentración del suero de partida se calcula como:

\begin{equation*}
c_{\text{A}} =
\frac{
152.0
+ (77.6 \times 2)
+ (26.5 \times 5)
}{3}
=
\frac{152.0 + 155.2 + 132.5}{3}
=
146.6 \ \text{mg/dL}
\end{equation*}


\subsubsection{Suero B}

\begin{table}[H]
\centering
\begin{tabular}{c c c c}
\hline
Tubo & Volumen de muestra ($\mu$L) & Absorbancia (505 nm) & Concentración de glucosa (mg/dL) \\
\hline
9  & 20 & 0.480 & 149.5 \\
10 & 20 & 0.231 & 71.9 \\
11 & 20 & 0.085 & 26.5 \\
\hline
\end{tabular}
\caption{Volumen empleado, absorbancias medidas y concentraciones de glucosa calculadas para el suero B.}
\label{tab:sueroB}
\end{table}

La concentración del suero de partida se calcula como:

\begin{equation*}
c_{\text{B}} =
\frac{
149.5
+ (71.9 \times 2)
+ (26.5 \times 5)
}{3}
=
\frac{149.5 + 143.8 + 132.5}{3}
=
141.9 \ \text{mg/dL}
\end{equation*}

\subsection{Conclusiones}

\begin{itemize}
    \item Se cumplió el objetivo de determinar la concentración de glucosa en muestras de suero sanguíneo mediante un método espectrofotométrico basado en la reacción enzimática glucosa oxidasa--peroxidasa y la aplicación de la ley de Lambert--Beer.
    \item Se alcanzó el objetivo de calcular la glucemia de los sueros problema A y B a partir de la comparación de sus absorbancias con un control patrón de glucosa, teniendo en cuenta las diluciones aplicadas durante el procedimiento experimental.
\end{itemize}
