\documentclass[12pt, a4paper]{article}
\usepackage{amsmath}
\usepackage{amssymb}
\usepackage[utf8]{inputenc}    % Soporte para caracteres especiales
\usepackage{amsfonts}
\usepackage{graphicx}
\usepackage{float}
\usepackage[right=3cm, left=2.5cm, top=3cm, bottom=3.5cm]{geometry}
%\usepackage{hyperref}          % Enlaces y referencias
%\usepackage[style=apa]{biblatex}
%\usepackage{csquotes}          % Manejo de citas (opcional, útil para biblatex)
%\addbibresource{references.bib}
%\usepackage[spanish]{babel}
\title{Ejercicios de IR}
\author{Torres Tarrillo, Rober E.}

\begin{document}

\maketitle

\textbf{Ejercicio 3:} En el espectro IR se registran:

\begin{enumerate}
  \item Señal intensa en 1725~cm$^{-1}$
  \item Dos bandas débiles alrededor de 2720 y 2820~cm$^{-1}$
  \item Bandas de C–H en la zona de 2950~cm$^{-1}$
\end{enumerate}

\textbf{Solución:}

\begin{enumerate}
  \item La señal intensa en 1725~cm$^{-1}$ corresponde al \textbf{estiramiento del grupo carbonilo} (C=O).
  \item Las dos bandas débiles a 2720 y 2820~cm$^{-1}$ son el \textbf{doblete característico del grupo –CHO}, debido a la combinación de vibraciones del enlace C–H del grupo aldehído.
  \item Las bandas alrededor de 2950~cm$^{-1}$ corresponden a \textbf{estiramientos C–H alifáticos}.
\end{enumerate}

\textbf{Conclusión:}

El conjunto de señales indica la presencia de un \textbf{grupo aldehído (–CHO)}.  

\textbf{Función química:} Aldehído

\textbf{Ejemplo de compuesto:} (CH$_3$CH$_2$CHO) o (C$_6$H$_5$CHO).

\vspace{48pt}
\textbf{Espectro 3: }

\begin{figure}[H]
    \begin{center}
        \includegraphics[width=\textwidth]{f1.jpg}
    \end{center}
\end{figure}

\textbf{Solución:} Se puede observar lo siguiente

\begin{figure}[H]
    \begin{center}
        \includegraphics[width=\textwidth]{f2.jpg}
    \end{center}
\end{figure}

En el espectro IR se registran las siguientes señales:

\begin{enumerate}
  \item Una ancha y débil entre 3100 y 3000~cm$^{-1}$ con $\%T = 70$
  \item Una delgada e intensa en 1658~cm$^{-1}$ con $\%T = 5$
  \item Una delgada y media en 1599~cm$^{-1}$ con $\%T = 28$
  \item Una delgada y media en 1450~cm$^{-1}$ con $\%T = 30$
  \item Una delgada e intensa en 1350~cm$^{-1}$ con $\%T = 15$
  \item Una delgada e intensa en 1280~cm$^{-1}$ con $\%T = 4$
  \item Una delgada y media en 1010~cm$^{-1}$ con $\%T = 70$
  \item Una delgada y media en 960~cm$^{-1}$ con $\%T = 35$
  \item Una delgada en 760~cm$^{-1}$ con $\%T = 15$
  \item Una delgada en 680~cm$^{-1}$ con $\%T = 35$
\end{enumerate}

\textbf{Análisis de las señales:}

\begin{itemize}
  \item \textbf{3100--3000~cm$^{-1}$:} Estiramientos C--H de compuestos aromáticos o vinílicos. La banda ancha pero débil sugiere la presencia de un anillo aromático.
  \item \textbf{1658~cm$^{-1}$:} Banda intensa de estiramiento C=O conjugado, característica de una \textbf{amida} o de un \textbf{carbonilo $\alpha,\beta$-insaturado}.
  \item \textbf{1599 y 1450~cm$^{-1}$:} Vibraciones C=C típicas de un \textbf{anillo aromático}.
  \item \textbf{1350~cm$^{-1}$:} Posible estiramiento simétrico del grupo \textbf{nitro (NO$_2$)} o una deformación C--H intensa.
  \item \textbf{1280 y 1010~cm$^{-1}$:} Bandas intensas de estiramiento \textbf{C--O}, compatibles con grupos éter o éster aromático.
  \item \textbf{960, 760 y 680~cm$^{-1}$:} Bandas de deformación fuera del plano del C--H, típicas de un \textbf{anillo aromático monosustituido o disustituido}.
\end{itemize}

\textbf{Conclusión:}

El conjunto de señales indica la presencia de los siguientes grupos funcionales:

\begin{itemize}
  \item \textbf{Anillo aromático} (bandas en 3100--3000, 1599, 760 y 680~cm$^{-1}$)
  \item \textbf{Grupo carbonilo conjugado} (C=O a 1658~cm$^{-1}$)
  \item \textbf{Enlaces C--O} (1280 y 1010~cm$^{-1}$)
  \item \textbf{Posible grupo nitro (NO$_2$)} (1350~cm$^{-1}$)
\end{itemize}

\textbf{Posibles compuestos:}

\begin{enumerate}
  \item \textbf{Benzamida (benzamida):}
    \begin{itemize}
      \item Fórmula condensada: $\mathrm{C_6H_5CONH_2}$
    \end{itemize}

  \item \textbf{Nitrobenceno:}
    \begin{itemize}
      \item Fórmula condensada: $\mathrm{C_6H_5NO_2}$
    \end{itemize}

  \item \textbf{Nitrotolueno (genérico):}
    \begin{itemize}
      \item Fórmula condensada: $\mathrm{CH_3C_6H_4NO_2}$ 
    \end{itemize}

  \item \textbf{Benzoato de metilo (éster aromático):}
    \begin{itemize}
      \item Fórmula condensada: $\mathrm{C_6H_5COOCH_3}$
    \end{itemize}
\end{enumerate}

\end{document}