Calcula la variación de energía que experimenta el eletrón
del átomo de hidrógeno cuando pasa del primer al quinto
nivel. ¿Esta energía es desprendida o absorbida?

\vspace{12pt}
\textit{SOLUCIÓN.} La energía en eV está dada por

\begin{equation*}
    \Delta E = -13.6 \left(
        \frac{1}{n_f^2} - \frac{1}{n_i^2}{} 
    \right)
\end{equation*}

Reemplazando obtenemos

\begin{equation*}
    \Delta E = -13.6 \left(
        \frac{1}{5^2} - \frac{1}{1^2}
    \right)
    = 13.056 \times \frac{1.602\times10^{-19}\mathrm{J}}{1\mathrm{eV}}{}
    = 2.08\times10^{16}\;\mathrm{J}
\end{equation*}

El signo positivo indica que la energía se desprende.
\hfill $\blacksquare$