Si la energía de la primer órbita de Bohr es $-13.6$ eV.
¿Cuál es la energía de laa quinta órbita en eV y en J?.

\vspace{12pt}
\textit{SOLUCIÓN.} Se tiene que
\begin{equation*}
    E_n = -\frac{13.6}{n^2}\;\mathrm{eV}
\end{equation*}

Por lo que para $n = 5$ remplazando obtenemos la energía en eV
\begin{equation*}
    E_5 = -\frac{13.6}{25}\;\mathrm{eV} = 0.544\;\mathrm{eV}
\end{equation*}

Convirtiendo esto a J obtenemos
\begin{equation*}
   E_5 = 0.544\;\mathrm{eV}\cdot
   \frac{1.6\times10^{-19}\;\mathrm{J}}{1\mathrm{eV}}
   = 8.704\times10^{20}\;\mathrm{J} 
\end{equation*}