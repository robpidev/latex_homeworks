Calcula la velocidad del electrón en la primera
órbita ($n=1$) del átomo de hidrógeno.

\vspace{12pt}
\textit{SOLUCIÓN.} Se tiene que la velocidad y
el radio de la n-sima orbita respectivamente está dada por:

\begin{equation*}
    v_n = \frac{e}{\sqrt{4\pi\epsilon_0 m r_n}},
    \quad r_n = \frac{n^2 h^2 \epsilon_0}{\pi m e^2}
\end{equation*}

remplazando para $n = 1$, se tiene
\begin{align*}
    &r_1 = \frac{h^2\epsilon_0}{\pi m e^2}
    = \frac{6.626\times10^{-34}\cdot8.84\times10^{-12}}
        {\pi\cdot9.1\times10^{-31}}\cdot1.602\times10^{-19}
    = 5.92\times10^{-12}\;\mathrm{m}\\
    &v_1 = \frac{
        1.602\times10^{-19}\;\mathrm{C}
    }{
    \sqrt{
        4\pi\times8.87\times10^{-12}\;\mathrm{F}
        \cdot 9.1\times10^{-31}\;\mathrm{kg}
        \cdot 5.29\times10^{-11}\;\mathrm{m}
    }
    }=2.188\times10^6 \frac{m}{s}
\end{align*}
\hfill $\blacksquare$
