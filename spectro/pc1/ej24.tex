Un electrón en un átomo de hidrógeno pasa del nivel $n=5$
al nivel $n=2$. Calcula la energía del fotón emitido en Joules
(j) y su frecuencia ($nu$) en Hz.

\vspace{12pt}
\textit{SOLUCIÓN.} Se tiene que la variación de energía
podemos calcularla mediante

\begin{equation*}
    \Delta E = -13.6 \left(
        \frac{1}{n_f^2} - \frac{1}{n_i^2}
    \right)\;\mathrm{eV}
\end{equation*}

Reemplazando con $n_i = 5$ y $n_f = 2$ obtenemos

\begin{align*}
    &\Delta E = -13.6 \left(
        \frac{1}{2^2} - \frac{1}{5^2}
    \right)\;\mathrm{eV}
    = -2.856 \;\mathrm{eV}
    = -2.856 \cdot \frac{
        1.602\times10^{-19}\;\mathrm{J}
    }{\mathrm{1 eV}}\\
    \Rightarrow&\Delta E= 4.569\times10^{-19}\;\mathrm{J}
\end{align*}

Usando la variación de energía con la ecuación $E = h\nu$
obtenemos la frecuencia

\begin{equation*}
    \nu = \frac{E}{h} = \frac{4.569\times10^{-19}\;\mathrm{J}}{6.626\times10^{-34}\;\mathrm{Js}} 
    = 6.896\times10^{14}\;\mathrm{Hz}{}
\end{equation*}
\hfill $\blacksquare$