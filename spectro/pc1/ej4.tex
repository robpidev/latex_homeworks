Hallar los números cuánticos de los siguientes orbitales:
(a) $\mathrm{3p^5}$, (b) $\mathrm{4s^1}$,  (c) $\mathrm{3d^8}$ y (d) $\mathrm{5f^{13}}$.

\vspace{12pt}
\textit{SOLUCIÓN.} Para cada caso debemos hacer la distribución
electronica, para (a) se tiene:

\begin{equation*}
    \mathrm{
        \overset{\textstyle
                \frac{ \uparrow \downarrow}{0}
        }{1s^2}
        \quad
        \overset{\textstyle
                \frac{ \uparrow \downarrow}{0}
        }{2s^2}
        \quad
        \overset{\textstyle
                \frac{ \uparrow \downarrow}{-1}
                \frac{ \uparrow \downarrow}{0}
                \frac{ \uparrow \downarrow}{1}
        }{2p^6}
        \quad
        \overset{\textstyle
                \frac{ \uparrow \downarrow}{0}
        }{3s^2}
        \quad
        \overset{\textstyle
                \frac{ \uparrow \downarrow}{0}
        }{3s^2}
        \quad
        \overset{\textstyle
                \frac{ \uparrow \downarrow}{-1}
                \frac{ \uparrow \downarrow}{0}
                \frac{ \uparrow }{1}
        }{3p^5}
        }
\end{equation*}

por lo qué los números cuánticos serán: $n = 3, l = 1, m = 0, s = -1/2$.

Para (b)

\begin{equation*}
   \mathrm{
        \overset{\textstyle
                \frac{ \uparrow \downarrow}{0}
        }{1s^2}
        \quad
        \overset{\textstyle
                \frac{ \uparrow \downarrow}{0}
        }{2s^2}
        \quad
        \overset{\textstyle
                \frac{ \uparrow \downarrow}{-1}
                \frac{ \uparrow \downarrow}{0}
                \frac{ \uparrow \downarrow}{1}
        }{2p^6}
        \quad
        \overset{\textstyle
                \frac{ \uparrow \downarrow}{0}
        }{3s^2}
        \quad
        \overset{\textstyle
                \frac{ \uparrow \downarrow}{-1}
                \frac{ \uparrow \downarrow}{0}
                \frac{ \uparrow \downarrow}{1}
        }{3p^6}
        \quad
        \overset{\textstyle
                \frac{ \uparrow}{0}
        }{4s^1}
        \quad
    }
\end{equation*}

Por lo qué los números cuánticos serán: $n = 4, l = 0, m = 0, s = +1/2$.

Para (c)
\begin{equation*}
    \mathrm{
        \overset{\textstyle
                \frac{ \uparrow \downarrow}{0}
        }{1s^2}
        \quad
        \overset{\textstyle
                \frac{ \uparrow \downarrow}{0}
        }{2s^2}
        \quad
        \overset{\textstyle
                \frac{ \uparrow \downarrow}{-1}
                \frac{ \uparrow \downarrow}{0}
                \frac{ \uparrow \downarrow}{1}
        }{2p^6}
        \quad
        \overset{\textstyle
                \frac{ \uparrow \downarrow}{0}
        }{3s^2}
        \quad
        \overset{\textstyle
                \frac{ \uparrow \downarrow}{-1}
                \frac{ \uparrow \downarrow}{0}
                \frac{ \uparrow \downarrow}{1}
        }{3p^6}
        \quad
        \overset{\textstyle
                \frac{ \uparrow \downarrow}{0}
        }{4s^2}
        \quad
        \overset{\textstyle
                \frac{ \uparrow \downarrow}{-2}
                \frac{ \uparrow \downarrow}{-1}
                \frac{ \uparrow \downarrow}{0}
                \frac{ \uparrow }{1}
                \frac{ \uparrow }{2}
        }{3d^{8}}
        \quad
    }
\end{equation*}

Por lo qué los números cuánticos serán $n = 4, l = 2, m = 0, s = -1/2$.

Por ultimo para (f) se tiene
\begin{align*}
    \mathrm{
        \overset{\textstyle
                \frac{ \uparrow \downarrow}{0}
        }{1s^2}
        \quad
        \overset{\textstyle
                \frac{ \uparrow \downarrow}{0}
        }{2s^2}
        \quad
        \overset{\textstyle
                \frac{ \uparrow \downarrow}{-1}
                \frac{ \uparrow \downarrow}{0}
                \frac{ \uparrow \downarrow}{1}
        }{2p^6}
        \quad
        \overset{\textstyle
                \frac{ \uparrow \downarrow}{0}
        }{3s^2}
        \quad
        \overset{\textstyle
                \frac{ \uparrow \downarrow}{-1}
                \frac{ \uparrow \downarrow}{0}
                \frac{ \uparrow \downarrow}{1}
        }{3p^6}
        \quad
        \overset{\textstyle
                \frac{ \uparrow \downarrow}{0}
        }{4s^2}
        \quad
        \overset{\textstyle
                \frac{ \uparrow \downarrow}{-2}
                \frac{ \uparrow \downarrow}{-1}
                \frac{ \uparrow \downarrow}{0}
                \frac{ \uparrow \downarrow}{1}
                \frac{ \uparrow \downarrow}{2}
        }{3d^{10}}
        \quad
        \overset{\textstyle
                \frac{ \uparrow \downarrow}{-1}
                \frac{ \uparrow \downarrow}{0}
                \frac{ \uparrow \downarrow}{1}
        }{4p^6}
        \quad
        \overset{\textstyle
                \frac{ \uparrow \downarrow}{0}
        }{5s^2}
}\\\mathrm{
        \overset{\textstyle
                \frac{ \uparrow \downarrow}{-2}
                \frac{ \uparrow \downarrow}{-1}
                \frac{ \uparrow \downarrow}{0}
                \frac{ \uparrow \downarrow}{1}
                \frac{ \uparrow \downarrow}{2}
        }{4d^{10}}
        \quad
        \overset{\textstyle
                \frac{ \uparrow \downarrow}{-1}
                \frac{ \uparrow \downarrow}{0}
                \frac{ \uparrow \downarrow}{1}
        }{5p^6}
        \quad
        \overset{\textstyle
                \frac{ \uparrow \downarrow}{0}
        }{6s^2}
        \quad
        \overset{\textstyle
                \frac{ \uparrow \downarrow}{-3}
                \frac{ \uparrow \downarrow}{-2}
                \frac{ \uparrow \downarrow}{-1}
                \frac{ \uparrow \downarrow}{0}
                \frac{ \uparrow \downarrow}{1}
                \frac{ \uparrow \downarrow}{2}
                \frac{ \uparrow \downarrow}{3}
        }{4f^{14}}
        \quad
        \overset{\textstyle
                \frac{ \uparrow \downarrow}{-2}
                \frac{ \uparrow \downarrow}{-1}
                \frac{ \uparrow \downarrow}{0}
                \frac{ \uparrow \downarrow}{1}
                \frac{ \uparrow \downarrow}{2}
        }{5d^{10}}
        \quad
        \overset{\textstyle
                \frac{ \uparrow \downarrow}{-1}
                \frac{ \uparrow \downarrow}{0}
                \frac{ \uparrow \downarrow}{1}
        }{6p^6}
        \quad
        \overset{\textstyle
                \frac{ \uparrow \downarrow}{0}
        }{7s^2}
}\\\mathrm{
        \overset{\textstyle
                \frac{ \uparrow \downarrow}{-3}
                \frac{ \uparrow \downarrow}{-2}
                \frac{ \uparrow \downarrow}{-1}
                \frac{ \uparrow \downarrow}{0}
                \frac{ \uparrow \downarrow}{1}
                \frac{ \uparrow \downarrow}{2}
                \frac{ \uparrow }{3}
        }{5f^{13}}
        \quad
        }
\end{align*}

Por lo qué los números cuánticos serán: $n = 7, l = 3, m = 2, s = -1/2$