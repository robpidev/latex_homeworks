una línea del espectro de hidrógeno correspondiente a la serie de
Balmer tiene una longitud de onda igual a $434.05$ $\mathrm{nm}$.

\begin{enumerate}
    \item ¿Cuál es el valor de $n$ correspondiente al nivel superior
    que interviene en la transición?
    \item Calcula el potencial de ionización del hidrógeno en J/mol.
    \item Calcular la frecuencia de la radiación que tendría que incidir
    sobre un átomo de hidrógeno en estado fundamental, si queremos que
    arranque un electrón este posea una energía cinética de $3.10^{-19}$ J.
\end{enumerate}

\vspace{12pt}
\textit{SOLUCIÓN.} (a), para la serie de Balmer se tiene

\begin{equation}\label{eq:balmer}
    \frac{1}{\lambda} = R\left(
        \frac{1}{2^2} - \frac{1}{n^2}
    \right)
\end{equation}

Despejando

\begin{equation*}
    \frac{1}{\lambda R} = \frac{1}{2^2} - \frac{1}{n^2}
    \implies n^2 = \frac{4\lambda R}{\lambda R - 4}
    \implies n = 2\sqrt{\frac{\lambda R}{\lambda R - 4}}
\end{equation*}

Remplazando los valores de las constantes
$\lambda = 4.3502\times10^{-7}$ m y $R = 1.097\times10^{7}$ $\mathrm{m^{-1}}$

\begin{equation*}
    n = 2\sqrt{\frac{4.3502\times10^{-7} \times 1.097\times10^{7}}{4.3502\times10^{-7} \times 1.097\times10^{7} - 4}}
    \approx 5
\end{equation*}

(b) Se pide calcular la energía de ionización en J/mol,
está es la mínima energía para arrancar un electrón,
llevándolo del estado fundamental ($n=1$) al infinito ($n=\infty$),
por lo qué remplazando en la eq-\ref{eq:balmer} tenemos:

\begin{equation*}
    \frac{1}{\lambda} = R
\end{equation*}

Remplazando en la energía

\begin{equation*}
    E = \frac{hc}{\lambda} = hcr
\end{equation*}

Entonces la energía en J/mol es
\begin{equation*}
   E = \frac{6.626\times10^{-34} \mathrm{Js}
   \cdot 3 \times 10^8 \mathrm{m/s}\cdot 1.097\times10^7 \mathrm{m^{-1}}}{1}
    \cdot 6.023\times10^{23} \mathrm{atom/mol}
\end{equation*}

\begin{equation*}
    \therefore E = 1.312\times10^6\quad\mathrm{J/mol}
\end{equation*}

(c) %TODO: FALTA Este inciso

\hfill $\blacksquare$