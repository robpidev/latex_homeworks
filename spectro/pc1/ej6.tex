Determina la configuración electrónica del ion $\mathrm{Fe}^{+3}$.
explica por qué se eliminan primero los electrones del orbital
4s y n del 3d. 

\vspace{12pt}
\textit{SOLUCIÓN.} Cómo es ion positivo a perdido $3$ electrones
se elimina primero de $4s$ en vez de $3d$ dado que siempre que
se pierda electrones lo hará de la ultima capa, por
lo qué para el átomo neutro Fe tiene $26$ electrones,
como a perdido $3$ tendrá $23$ así la distribución electrónica será:

\begin{equation*}
    \mathrm{
        \overset{\textstyle
                \frac{ \uparrow \downarrow}{0}
        }{1s^2}
        \quad
        \overset{\textstyle
                \frac{ \uparrow \downarrow}{0}
        }{2s^2}
        \quad
        \overset{\textstyle
                \frac{ \uparrow \downarrow}{-1}
                \frac{ \uparrow \downarrow}{0}
                \frac{ \uparrow \downarrow}{1}
        }{2p^6}
        \quad
        \overset{\textstyle
                \frac{ \uparrow \downarrow}{0}
        }{3s^2}
        \quad
        \overset{\textstyle
                \frac{ \uparrow \downarrow}{-1}
                \frac{ \uparrow \downarrow}{0}
                \frac{ \uparrow \downarrow}{1}
        }{3p^6}
        \quad
        \overset{\textstyle
                \frac{ }{0}
        }{4s^0}
        \quad
        \overset{\textstyle
                \frac{ \uparrow}{-2}
                \frac{ \uparrow}{-1}
                \frac{ \uparrow}{0}
                \frac{ \uparrow}{1}
                \frac{ \uparrow}{2}
        }{3d^{10}}
        \quad
        }
\end{equation*}
\hfill $\blacksquare$