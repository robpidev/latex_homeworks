El electrón de un átomo de hidrógeno se ha excitado al nivel
correspondiente a $n=2$. Otro átomo de hidrógeno tiene un electrón
en el nivel $n = 3$. Al volver cada electrón a su estado
fundamental $(n = 1)$, ¿Cuál de ellos emitirá fotones de mayor energía?
¿Y de mayor frecuencia? ¿y de mayor longitud de onda?

\vspace{12pt}
\textit{SOLUCIÓN.} Cálculo para el átomo con $n = 2$ al nivel $n = 1$.
Se tiene que:

\begin{equation*}
    \Delta E = E_i - E_f = 13.6
    \left(
        \frac{1}{n_f^2} - \frac{1}{n_i^2}
    \right)
\end{equation*}

por lo que remplazando es

\begin{equation*}
    \Delta E = 13.6
    \left(
        \frac{1}{2^2} - \frac{1}{2^2}
    \right)
    = 13.6 \left(1 - \frac{1}{4}\right)
    = 10.2\quad\mathrm{eV}
\end{equation*}

de la misma forma para el segundo átomo con $n = 3$ al nivel $n = 1$, se tiene

\begin{equation*}
    \Delta E = 13.6
    \left(
        \frac{1}{1^2} - \frac{1}{3^2}
    \right)
    = 13.6 \left(1 - \frac{1}{9}\right)
    = 12.09\quad\mathrm{eV}
\end{equation*}

Como se puede Observar, el segundo átomo emite fotones de mayor energía.
Dado qué $E = h\nu$, entonces también el segundo átomo 
tendrá mayor frecuencia (ya que son directamente proporcionales).
Finalmente como

\begin{equation*}
    E = h\nu = h\frac{c}{\lambda} = \frac{hc}{\lambda}
    \implies \lambda = \frac{hc}{E} 
\end{equation*}

Por lo qué quien tiene mayor longitud de onda es el átomo que
tiene menor energía, es decir el primero.

\hfill $\blacksquare$