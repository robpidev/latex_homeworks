Determina la energía del fotón emitido cuando
un electrón en el ion de $\mathrm{He}^{+}$ 
cae del nivel $n = 3$ al $n = 1$.

\vspace{12pt}
\textit{SOLUCIÓN.} 
\[
E_n = - \frac{Z^2 R_H}{n^2}
\]

donde \(R_H = 13.6 \, \text{eV}\), y \(Z=2\) para el ion \(\mathrm{He}^+\).

\[
E_1 = - \frac{(2)^2 (13.6 \, \text{eV})}{1^2} = -54.4 \, \text{eV}
\]

\[
E_3 = - \frac{(2)^2 (13.6 \, \text{eV})}{3^2} 
= - \frac{54.4}{9} \, \text{eV} 
\approx -6.044 \, \text{eV}
\]

\[
\Delta E = E_1 - E_3 = (-54.4) - (-6.044) = -48.356 \, \text{eV}
\]

Como es emisión, tomamos el valor absoluto:

\[
E_\gamma = 48.36 \, \text{eV}
\]

En joules, usando \(1 \, \text{eV} = 1.602 \times 10^{-19} \, \text{J}\):

\[
E_\gamma = 48.36 \times 1.602 \times 10^{-19} \, \text{J}
\]

\[
E_\gamma \approx 7.75 \times 10^{-18} \, \text{J}
\]

\hfill $\blacksquare$