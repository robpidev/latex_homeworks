Calcular la longitud de onda que emite un electrón
en el átomo de hidrógeno cuando pasa de una órbita
$n = 54$ hasta la órbita $n = 3$.

\vspace{12pt}
\textit{SOLUCIÓN.} de la eq~\eqref{eq:balmer} se tiene

\begin{equation*}
    \frac{1}{\lambda} = R \left(
        \frac{1}{3^2} -  \frac{1}{5^2}
    \right)
    = 7.8\times10^5\;\mathrm{m^{-1}}
    \implies \lambda = 1.281\times10^{-6} m
\end{equation*}