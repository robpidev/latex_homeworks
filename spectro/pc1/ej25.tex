¿Qué longitud de onda de radiación debe absorber
un átomo de hidrógeno para que su electrón slate del estado
fundamental ($n=1$) al estado ($n=2$)?

\vspace{12pt}
\textit{SOLUCIÓN.} Se tiene que la variación de energía
podemos calcularla mediante la eq~\eqref{eq:balmer} 

\begin{equation*}
    \frac{1}{\lambda} = 1.097\times10^7\left(
        1 - \frac{1}{4}
    \right)\;\mathrm{m^{-1}}
    = 8.227\times10^6\;\mathrm{m^{-1}}
    \implies \lambda = 1.215\times10^{-7}\;\mathrm{m}
\end{equation*}
\hfill $\blacksquare$