Demuestra que el radio de la órbita de Bohr
es proporcional al cuadrado del número cuántico principal
($r_n \propto n^2$) y calcula el radio de la tercera
órbita ($n=3$) de un átomo de hidrógeno.

\vspace{12pt}
\textit{SOLUCIÓN.} Se tiene que el radio de la n-sima orbita respectivamente está dada por:

\begin{equation*}
    r_n = \frac{n^2 h^2 \epsilon_0}{\pi m e^2}
    = a_0 n^2
\end{equation*}

donde $a_0$ es el radio de Bohr, y es constante, con esto
$r_n \propto n^2$. Ahora para calcular $r$ para $n=3$, solo remplazamos 

\begin{equation*}
    r_3 = \frac{h^2\epsilon_0}{\pi m e^2}(3)^2
    = 4.763\times10^{-10}\;\mathrm{m}
\end{equation*}
\hfill $\blacksquare$