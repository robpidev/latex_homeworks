Para cada caso ¿Cuál es el número total de
electrones que poseen los siguientes valores cuánticos:
(a) $n = 3, l = 1$.
(b) $n = 3, l = 1, m = -1$
?

\vspace{12pt}
\textit{SOLUCIÓN.} Para ambos casos debemos hacer la distribución
electronica, por lo que para (a) se tiene que el último nivel es $n = 3$
y $l = 1$ que corresponde a p:

\begin{equation*}
    \mathrm{1s^2 2s^2 2p^6 3s^2 3p^6 }
\end{equation*}

Por lo qué el número de electrones total (máximo) será

\begin{equation*}
    2 + 2 + 6 + 2 + 6 = 18
\end{equation*}

De la misma forma hagamos para (b), como $m = -1$ esto es

\begin{equation*}
    \mathrm{
        \overset{\textstyle\underline{\uparrow \downarrow}}{1s^2}
        \quad
        \overset{\textstyle\underline{\uparrow \downarrow}}{2s^2}
        \quad
        \overset{\textstyle
            \underline{\uparrow \downarrow}
            \;
            \underline{\uparrow \downarrow}
            \;
            \underline{\uparrow \downarrow}
        }{2p^6}
        \quad
        \overset{\textstyle\underline{\uparrow \downarrow}}{3s^2}
        \quad
        \overset{\textstyle
            \underline{\uparrow \downarrow}
            \;
            \underline{\uparrow}
            \;
            \underline{\uparrow}
        }{2p^4}
    }
\end{equation*}

Por lo qué el número de electrones total (máximo) para estos
número cuánticos será

\begin{equation*}
    2 + 2 + 6 + 2 + 4 = 16
\end{equation*}
\hfill $\blacksquare$