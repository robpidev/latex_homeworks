Demostrar que la energía cinética del electrón en la
órbita $n$ del átomo de hidrógeno es

\begin{equation*}
    E_n = \frac{13.6}{n^2}\;\mathrm{eV}
\end{equation*}

Y verificar para $n=3$.

\vspace{12pt}
\textit{SOLUCIÓN.} Se tiene las siguientes ecuaciones
de energía cinética, velocidad con respecto al radio
y radio respectivamente:

\begin{equation*}
    T = \frac{1}{2}mv^2,\quad
    v^2 = \frac{e^2}{4\pi\epsilon_0 m r},\quad
    r = \frac{n^2 h^2 \epsilon_0}{\pi m e^2}
\end{equation*}

Remplazando la rapidez y y el radio en la energía cinética

\begin{equation*}
    T = \frac{1}{2}m
    \frac{e^2}{4\pi\epsilon_0 m r}
    \frac{\pi m e^2}{n^2 h^2 \epsilon_0}
    = \frac{1}{n^2} \frac{me^4}{8h^2\epsilon_0}
\end{equation*}

Calculando el último valor de la fracción de la derecha
tenemos 

\begin{equation*}
    T = \frac{1}{n^2}
    \left(
        \frac{
            9.1\times10^{-31}\;\mathrm{kg}
            {\left(\cdot 1.602\times10^{-19}\;\mathrm{C}\right)}^4
        }{
            8\times{(6.626\times10^{-34}\;\mathrm{Js})}^2
            {\left(\cdot 8.854\times10^{-12}\;\mathrm{F/m}\right)}^2
        }
    \right)
    = \frac{2.16\times10^{-18}}{n^2}\;\mathrm{J}
\end{equation*}

Convirtiendo a eV finalmente

\begin{equation*}
    T = \frac{2.16\times10^{-18}}{n^2}\;\mathrm{J}
    \cdot \frac{1\;\mathrm{eV}}{1.602\times10^{-6}\;\mathrm{J}}
    = \frac{13.6}{n^2}\;\mathrm{eV}
\end{equation*}

Ahora calculemos para $n=3$

\begin{equation*}
    T = \frac{13.6}{3^2}\;\mathrm{eV}
    = 1.51\;\mathrm{eV}
\end{equation*}
\hfill $\blacksquare$