Se sabe que la velocidad del electrón en la primera órbita
del átomo de hidrógeno es de apropiadamente 
$v_1 = 2.188\times10^6 \frac{m}{s}$ y el radio
de la primera órbita es $r_1 = 5.29\times10^{-11}\;\mathrm{m}$.
Usa esta información para verificar el segundo postulado de
Bohr para $n = 1$

\vspace{12pt}
\textit{SOLUCIÓN.} Para $L = mrv$, se tiene

\begin{equation*}
    L  = 9.1\times10^{-31}\;\mathrm{kg}\cdot
    5.29\times10^{-11}\;\mathrm{m}\cdot
    2.19\times10^6\;\mathrm{m/s}
    = 1.055\times10^{-34}\;\mathrm{Js}
\end{equation*}

Mientras que para $L = n\hbar$ se obtiene

\begin{equation*}
    L = 1\cdot 1.055\times10^{-34}\;\mathrm{Js}
    = 1.055\times10^{-34}\;\mathrm{Js}
\end{equation*}

En ambos casos obtenemos que $L = 1.055\times10^{-34}\;\mathrm{Js}$,
por lo que se cumple el segundo postulado de Bohr.
\hfill $\blacksquare$
