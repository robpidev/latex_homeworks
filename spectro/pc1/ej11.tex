Dadas las siguientes configuraciones electrónicas:
(A) $\mathrm{1s^2 2s^2 2p^6 3s^2 3p^4}$,
(B) $\mathrm{1s^2 2s^2}$,
(C) $\mathrm{1s^2 2s^2 2p^6 3s^2 3p^4}$.

Indique, razonadamente:
\begin{enumerate}
    \item El grupo y perdido en los que se hallan A, B y C.
    \item Los iones están más estables que formarán A, B y C.
\end{enumerate}

\vspace{12pt}
\textit{SOLUCIÓN.} (a) Para A, se tiene $n = 3$, por lo qué pertenece 
al periodo 3, cómo termina p, pertenece al grupo A, ademas tiene
6 electrones en su ultimo nivel, por lo qué está en el grupo VI A.

Para B, de la misma forma $n = 2$, entonces periodo, termina en s,
entonces grupo A, tiene 2 electrones en el último nivel, entonces
pertenece al II A.

Para C, $n = 2$, entonces periodo 2, termina en p, entonces grupo A,
tienes 8 electrones en su ultimo nivel, entonces pertenece al VIII A.

(b) (A) ión mas estable $\mathrm{S^{2-}}$; (B) ión mas estable Be;
(C) ión más estable Ne.

\hfill $\blacksquare$