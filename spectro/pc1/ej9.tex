Para el sodio (Na $z = 11$) Responda las siguientes preguntas

\begin{enumerate}
    \item ¿Cuál es la configuración electrónica?
    \item ¿En qué periodo se encuentra?
    \item ¿Cuántos electrones de valencia tiene?
    \item ¿A qué grupo pertenece?
    \item ¿Cuáles son los números cuánticos del último electrón?
\end{enumerate}

\vspace{12pt}
\textit{SOLUCIÓN.}

\begin{enumerate}
    \item 
\begin{equation*}
    \mathrm{
        \overset{\textstyle
                \frac{ \uparrow \downarrow}{0}
        }{1s^2}
        \quad
        \overset{\textstyle
                \frac{ \uparrow \downarrow}{0}
        }{2s^2}
        \quad
        \overset{\textstyle
                \frac{ \uparrow \downarrow}{-1}
                \frac{ \uparrow \downarrow}{0}
                \frac{ \uparrow \downarrow}{1}
        }{2p^6}
        \quad
        \overset{\textstyle
                \frac{ \uparrow }{0}
        }{3s^1}
        \quad
        }
\end{equation*}
    \item Como $n = 3$ se encuentra en el periodo 3.
    \item Cómo se tiene un electron en el ultimo nivel,
        tiene un electrón de valencia.
    
    \item Al grupo IA
        
    \item $n = 3, l = 0, m = 0, s = +1/2$
\end{enumerate}

\hfill $\blacksquare$