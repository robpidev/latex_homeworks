Calcula el momento angular de un electrón en la
segunda orbita ($n = 2$) de un átomo de hidrógeno.
usando el segundo postulado de Bohr.

\vspace{12pt}
\textit{SOLUCIÓN.} Se tiene que el momento angular y
la rapidez de la n-sima orbita respectivamente está dada por:

\begin{equation}
\label{eq:angular:momentum}
    L = mrv, \quad v = \frac{nh}{2\pi m r}
    \implies L = mr\left(
        \frac{nh}{2\pi mr} 
    \right)
    = \frac{nh}{2\pi} = n\hbar
\end{equation}

Por lo que remplazando con $n = 1$ y la constante de Plank
reducida, finalmente obtenemos

\begin{equation*}
    L = 2\cdot 1.054\times10^{-34}\;\mathrm{Js} = 2.109\times10^{-34}
    \;\mathrm{Js}
\end{equation*}
\hfill $\blacksquare$


