La configuración electrónica del elemento 12 Mg 
establece que los números cuánticos principal, secundario y magnético del último electrón son,
respectivamente

\vspace{12pt}
\textit{SOLUCIÓN.} Haciendo la distribución
electrónica se tiene:

\begin{equation*}
    \mathrm{
        \overset{\textstyle
                \frac{ \uparrow \downarrow}{0}
        }{1s^2}
        \quad
        \overset{\textstyle
                \frac{ \uparrow \downarrow}{0}
        }{2s^2}
        \quad
        \overset{\textstyle
                \frac{ \uparrow \downarrow}{-1}
                \frac{ \uparrow \downarrow}{0}
                \frac{ \uparrow \downarrow}{1}
        }{2p^6}
        \quad
        \overset{\textstyle
                \frac{ \uparrow \downarrow}{0}
        }{3s^2}
        \quad
        }
\end{equation*}

Por lo qué los números cuántos serán
$n = 3, l = 0, m = 0$

\hfill $\blacksquare$