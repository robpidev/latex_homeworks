Escriba la configuración electrónica completa del Ion $Fe^{3+}$ ($Z = 26$)
y determina los números cuánticos de su último electrón (el de mayor energía
en el ión).

\vspace{12pt}
\textit{SOLUCIÓN.} El número electrones del ion $Fe^{3+}$ es $26 - 3 = 2$,
por lo qué su distribución electrónica es:

\begin{equation*}
    \mathrm{
        \overset{\textstyle
                \frac{ \uparrow \downarrow}{0}
        }{1s^2}
        \quad
        \overset{\textstyle
                \frac{ \uparrow \downarrow}{0}
        }{2s^2}
        \quad
        \overset{\textstyle
                \frac{ \uparrow \downarrow}{-1}
                \frac{ \uparrow \downarrow}{0}
                \frac{ \uparrow \downarrow}{1}
        }{2p^6}
        \quad
        \overset{\textstyle
                \frac{ \uparrow \downarrow}{0}
        }{3s^2}
        \quad
        \overset{\textstyle
                \frac{ \uparrow \downarrow}{-1}
                \frac{ \uparrow \downarrow}{0}
                \frac{ \uparrow \downarrow}{1}
        }{3p^6}
        \quad
        \overset{\textstyle
                \frac{ \uparrow \downarrow}{0}
        }{4s^2}
        \quad
        \overset{\textstyle
                \frac{ \uparrow }{-2}
                \frac{ \uparrow }{-1}
                \frac{ \uparrow }{0}
                \frac{  }{1}
                \frac{  }{2}
        }{3d^{3}}
        \quad
        }
\end{equation*}

Para qué sea más estable se tiene

\begin{equation*}
   \mathrm{
        \overset{\textstyle
                \frac{ \uparrow \downarrow}{0}
        }{1s^2}
        \quad
        \overset{\textstyle
                \frac{ \uparrow \downarrow}{0}
        }{2s^2}
        \quad
        \overset{\textstyle
                \frac{ \uparrow \downarrow}{-1}
                \frac{ \uparrow \downarrow}{0}
                \frac{ \uparrow \downarrow}{1}
        }{2p^6}
        \quad
        \overset{\textstyle
                \frac{ \uparrow \downarrow}{0}
        }{3s^2}
        \quad
        \overset{\textstyle
                \frac{ \uparrow \downarrow}{-1}
                \frac{ \uparrow \downarrow}{0}
                \frac{ \uparrow \downarrow}{1}
        }{3p^6}
        \quad
        \overset{\textstyle
                \frac{ }{0}
        }{4s^0}
        \quad
        \overset{\textstyle
                \frac{ \uparrow}{-2}
                \frac{ \uparrow}{-1}
                \frac{ \uparrow}{0}
                \frac{ \uparrow}{1}
                \frac{ \uparrow}{2}
        }{3d^{5}}
        \quad
        } 
\end{equation*}

\hfill $\blacksquare$