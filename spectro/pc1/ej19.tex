Calcula la energía de ionización del átomo
de hidrógeno de Joules (J) y en electronvoltios (eV).
La energía de ionización es la energía mínima
necesario para arrancar un electron del átomo,
llevándolo del estado fundamental ($n = 1$) al infinito
($n = \infty$).

\vspace{12pt}
\textit{SOLUCIÓN.} Utilizando la eq~\ref{eq:balmer}
y remplazando con $n_i = 1$ y $n_f = \infty$

\begin{equation*}
    \frac{1}{\lambda} = R\left(
        1 - \lim_{n_f\to\infty} \frac{1}{n_f^2} 
    \right) = R
\end{equation*}

remplazando en $E = hc/\lambda$ tenemos
\begin{equation*}
    E = 6.626\times10^{-34}\;\mathrm{Js}
    \cdot 3\times10^8\;\mathrm{m/s}
    \cdot 1.097\times10^7\;\mathrm{m^{-1}}
    \implies E = 2.18\times10^{-18}\;\mathrm{J}
\end{equation*}

Convirtiendo a eV

\begin{equation*}
    E = 2.18\times10^{-18}\;\mathrm{J}
    \cdot \frac{1\;\mathrm{eV}}{1.602\times10^{-6}\;\mathrm{J}}
    = 13.63\;\mathrm{eV}
\end{equation*}