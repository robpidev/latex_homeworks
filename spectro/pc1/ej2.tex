el sodio, elemento del grupo I y periodo 3
tiene una configuración electrónica: $\mathrm{1s^2 2s^2 2p^6 3s^1}$.
Cuántos electrones del átomo de sodio tienen: (a) $m = -1$, (b) $l = -1$.

\vspace{12pt}
\textit{SOLUCIÓN.} Agreguemos los electrones a la configuración electrónica.

\begin{equation*}
    \mathrm{
        \overset{\textstyle \frac{\uparrow\downarrow}{0}}{1s^2}
        \quad
        \overset{\textstyle \frac{\uparrow\downarrow}{0}}{2s^2}
        \quad
        \overset{\textstyle 
            \boxed{
                \frac{\uparrow\downarrow}{-1}
            }
            \frac{\uparrow\downarrow}{0}
            \frac{\uparrow\downarrow}{1}
        }{2p^6}
        \quad
        \overset{\textstyle\frac{\uparrow}{0}}{3s^1}
    }
\end{equation*} 

Como se puede observar el único que cumple con (a) es el par de electrones
encerrados en el recuadro, i.e 2 electrones tienen $m = -1$.

Por otra parte el número cuántico $l \geq 0$, por lo qué el número
de electrones que pueden tomar (b) $l = -1$ es 0.


\hfill $\blacksquare$