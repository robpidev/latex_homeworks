Determina qué especie es más pequeña:
$\mathrm{
    Al^{3+}, Mg^{2+}, Na^+, Ne, F^-}
$.
Todos  son iso-electrónicos. Justifica en base
a la distribución electrónica y la carga nuclear efectiva.

\vspace{12pt}
\textit{SOLUCIÓN.} Cómo son iso-electrónicos,
entonces tienen el mismo número de electrones, así se tiene qué

\begin{align*}
    &\mathrm{Al^{+3}} \longrightarrow \#e = 13 - 3 = 10 \quad \text{electrones}\\
    &\mathrm{Mg^{+2}} \longrightarrow \#e = 12 - 2 = 10 \quad \text{electrones}\\
    &\mathrm{Na^{+}} \longrightarrow \#e = 11 - 1 = 10 \quad \text{electrones}\\
    &\mathrm{Na^+} \longrightarrow \#e = 11 - 1 = 10 \quad \text{electrones}\\
    &\mathrm{Ne} \longrightarrow \#e = 10 \quad \text{electrones}\\
    &\mathrm{F^{-}} \longrightarrow \#e = 9 + 1 = 10 \quad \text{electrones}\\
\end{align*}

Todos tiene la misma configuración electrónica $\mathrm{1s^2 2s^2 2p^6}$,

A mayor $z$, mayor carga nuclear efectiva, así sq tiene que

\begin{align*}
    &\mathrm{Al} \longrightarrow z = 13 \\
    &\mathrm{Mg} \longrightarrow z = 12 \\
    &\mathrm{Na} \longrightarrow z = 11 \\
    &\mathrm{Ne} \longrightarrow z = 10 \\
    &\mathrm{F} \longrightarrow z = 9 \\
\end{align*}

Por lo que la especie más pequeña es $Al^{+3}$

\hfill $\blacksquare$

\vspace{12pt}