\section*{Ejercicio 1}

\begin{figure}[H]
    \centering
    \includegraphics[width=0.8\textwidth]{img/1.jpg}
\end{figure}

\textit{Solución.} Como el ejercicio no tiene un compuesto
para $m/z = 104,\quad \mathrm{M^+}$ menos probable 
según el gráfico, Voy a proponer el siguiente compuesto
y así poder hacer la fragmentación

\[
\chemfig{CH3-CH2-CH2-CH2-CH2-SH}
\]

Este compuesto cumple con lo requerido, pues

\[
5(12) + 12(1) + 32 = 104
\]

\begin{figure}[H]
    \centering
    \includegraphics[width=0.8\textwidth]{img/2.jpg}
\end{figure}

\subsection*{Fragmentaciones principales}

\begin{enumerate}
    \item \textbf{Pérdida de \ce{SH^.}}  
    El tiol terminal pierde un radical \ce{SH^.} (33 uma), formando el catión pentilo:
    \[
    \ce{C5H12S ->[EI] C5H11+ + SH^.}
    \]
    \[
    104 - 33 = \textbf{71 (m/z)}
    \]
    \ce{C5H11+} es el ion \textbf{pentilio}.

    \item \textbf{Clivaje $\alpha$ al azufre.}  
    La ruptura del enlace C–C cerca del azufre genera el catión butilio:
    \[
    \ce{CH3-CH2-CH2-CH2-CH2-SH ->[EI] C4H9+ + CH2SH^.}
    \]
    \[
    4(12) + 9(1) = \textbf{57 (m/z)}
    \]

    \item \textbf{Formación del catión propilio (\ce{C3H7+}).}  
    Es el ion más estable y el \textbf{pico base} en el espectro:
    \[
    \ce{C5H12S ->[reordenamiento] C3H7+ + resto}
    \]
    \[
    3(12) + 7(1) = \textbf{43 (m/z)}
    \]

    \item \textbf{Deshidrogenación secuencial:}
    \[
    \ce{C3H7+ ->[-H] C3H6+ ->[-H] C3H5+}
    \]
    Generando los picos:
    \[
    \textbf{m/z = 42 (C3H6+), 41 (C3H5+)}
    \]

    \item \textbf{Fragmentos de menor masa:}  
    \[
    \ce{C2H3+ (m/z = 27)}, \quad \ce{C2H5+ (m/z = 29)}
    \]

    \item \textbf{Fragmentos con azufre retenido:}
    \[
    \ce{C4H5S+ (m/z = 85)}, \quad \ce{C4H4S+ (m/z = 84)}
    \]
\end{enumerate}

\subsection*{Resumen de picos observados}

\begin{center}
\begin{tabular}{c c c}
\hline
\textbf{m/z} & \textbf{Ion propuesto} & \textbf{Comentario} \\
\hline
104 & \ce{M+} & Ion molecular débil \\
85, 84 & \ce{C4H5S+}, \ce{C4H4S+} & Fragmentos con azufre \\
71 & \ce{C5H11+} & Pérdida de \ce{SH^.} \\
57 & \ce{C4H9+} & Clivaje $\alpha$ al azufre \\
43 & \ce{C3H7+} & Pico base (catión propilio) \\
42 & \ce{C3H6+} & Deshidrogenación \\
41 & \ce{C3H5+} & Deshidrogenación \\
29 & \ce{C2H5+} & Fragmento etilo \\
27 & \ce{C2H3+} & Fragmento vinilo \\
\hline
\end{tabular}
\end{center}

\subsection*{Estructura del 1-pentanotiol}

\[
\chemfig{CH3-CH2-CH2-CH2-CH2-SH}
\]