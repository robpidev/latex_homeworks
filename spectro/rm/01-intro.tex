\section{Introducción}

La resonancia magnética (RMN) constituye una de las herramientas más importantes en el diagnóstico médico contemporáneo. Desde su desarrollo en la década de 1970, esta técnica ha revolucionado la forma en que los profesionales de la salud observan el interior del cuerpo humano, permitiendo obtener imágenes detalladas de los tejidos blandos sin la necesidad de procedimientos invasivos ni el uso de radiación ionizante. Su aplicación se ha extendido ampliamente en áreas como la neurología, la cardiología, la oncología y la ortopedia, convirtiéndose en un pilar esencial para el diagnóstico temprano y el seguimiento de diversas patologías.

En la actualidad, la resonancia magnética adquiere una relevancia cada vez mayor debido a la necesidad de métodos diagnósticos más seguros, precisos y no invasivos. Comprender los beneficios que ofrece y los riesgos que puede implicar resulta fundamental tanto para los profesionales de la salud como para los pacientes, ya que permite un uso responsable y ético de esta tecnología. Además, su importancia trasciende el ámbito clínico, al involucrar también aspectos físicos, tecnológicos y económicos que influyen en su implementación en los sistemas de salud.

El objetivo principal de este trabajo es analizar los beneficios y los riesgos de la resonancia magnética en el cuerpo humano, considerando sus fundamentos físicos, su impacto en el diagnóstico médico y las posibles limitaciones que presenta. Asimismo, se busca promover una reflexión sobre la necesidad de equilibrar la precisión diagnóstica con la seguridad del paciente.

El contenido de esta monografía se organiza en varias secciones. En primer lugar, se presenta el marco teórico, donde se explican los principios físicos y tecnológicos que sustentan la resonancia magnética. Posteriormente, se desarrollan los apartados dedicados a sus beneficios clínicos y a los riesgos potenciales asociados con su uso. Finalmente, se exponen las conclusiones generales y las recomendaciones que surgen del análisis realizado.
