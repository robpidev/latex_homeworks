\section{Recomendaciones}

A partir del análisis realizado, se proponen las siguientes recomendaciones orientadas tanto a la práctica clínica como a la investigación y desarrollo de la resonancia magnética:

\subsection{Buenas prácticas para la realización de exámenes de RMN}

Se recomienda que los exámenes de resonancia magnética sean efectuados únicamente por personal capacitado en el manejo de equipos de alto campo magnético. 
Es esencial verificar que el paciente no porte objetos metálicos ni dispositivos electrónicos implantados (como marcapasos o prótesis ferromagnéticas) antes de ingresar a la sala de exploración.  
Asimismo, se debe realizar una adecuada calibración de los gradientes y bobinas de radiofrecuencia para garantizar imágenes precisas y evitar artefactos.

Es conveniente mantener un protocolo de comunicación con el paciente, explicando el procedimiento y el ruido característico del escáner, lo que ayuda a reducir la ansiedad y los movimientos involuntarios durante la adquisición de datos.

\subsection{Precauciones para pacientes y personal médico}

El personal médico debe seguir las normas internacionales de seguridad magnética (IEC 60601-2-33) y las recomendaciones de la \textit{Food and Drug Administration} (FDA).  
En especial, debe controlarse la exposición a campos magnéticos variables y la temperatura corporal del paciente, ya que la radiofrecuencia puede inducir un calentamiento local de los tejidos descrito por la tasa de absorción específica (SAR, \textit{Specific Absorption Rate}):

\begin{equation}
\text{SAR} = \frac{\sigma |E|^2}{\rho}
\end{equation}

donde $\sigma$ es la conductividad eléctrica del tejido, $E$ la amplitud del campo eléctrico inducido y $\rho$ la densidad del material biológico.  
Es recomendable mantener la SAR por debajo de los límites establecidos por la FDA (4\,W/kg en promedio para cuerpo entero).

Además, deben existir zonas de seguridad claramente delimitadas (Zonas I a IV según la normativa ASTM), restringiendo el acceso a personal no autorizado durante la operación del escáner.

\subsection{Recomendaciones para la investigación futura}

Desde la perspectiva científica, se sugiere continuar con el desarrollo de nuevas técnicas de resonancia magnética que optimicen la señal–ruido ($S/N$) y reduzcan los tiempos de adquisición.  
El empleo de imanes de campo ultraalto ($B_0 > 7\,\text{T}$) y algoritmos avanzados de reconstrucción mediante inteligencia artificial ofrecen perspectivas prometedoras para mejorar la resolución espacial y temporal.

Asimismo, la investigación interdisciplinaria entre físicos, ingenieros biomédicos y médicos permitirá explorar nuevas modalidades como la \textit{resonancia magnética cuántica}, la \textit{hiperpolarización de espines nucleares} y la integración con técnicas ópticas o de espectroscopía de terahercios, ampliando el rango de aplicación diagnóstica y terapéutica.

En conclusión, la resonancia magnética seguirá evolucionando como una herramienta esencial de la medicina moderna, en la cual la física desempeña un papel central para garantizar su precisión, seguridad y desarrollo futuro.
