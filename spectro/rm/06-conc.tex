\section{Conclusiones}

La resonancia magnética se consolida como una de las herramientas más poderosas en el diagnóstico médico contemporáneo. 
Su fundamento físico, basado en la interacción de los momentos magnéticos nucleares con un campo magnético estático $B_0$ y pulsos de radiofrecuencia, permite obtener imágenes de alta resolución sin recurrir a radiación ionizante. 
El principio de Larmor, $\omega_0 = \gamma B_0$, sustenta la selectividad y precisión de esta técnica, lo que la convierte en un método de gran sensibilidad para estudiar tejidos blandos, procesos metabólicos y actividad cerebral.

Los beneficios de la resonancia magnética incluyen su carácter no invasivo, su excelente contraste entre tejidos y la posibilidad de realizar estudios funcionales y espectroscópicos. 
Aplicaciones en neurología, cardiología, oncología y ortopedia demuestran su relevancia clínica y su constante expansión hacia nuevos campos como la resonancia magnética funcional (RMf) y la espectroscopia por RM.

No obstante, el uso de esta tecnología implica ciertas limitaciones y riesgos que deben considerarse. 
El fuerte campo magnético representa un peligro potencial para pacientes portadores de implantes metálicos o marcapasos. 
Asimismo, el uso de contrastes basados en gadolinio puede generar reacciones adversas en individuos con insuficiencia renal, y el alto costo del equipo restringe su acceso en muchos entornos hospitalarios.

En balance, los beneficios superan ampliamente los riesgos cuando la resonancia magnética se emplea de forma responsable y con criterios clínicos adecuados. 
Desde el punto de vista físico, la técnica constituye una aplicación magistral de los principios del magnetismo nuclear y la interacción spin–campo, evidenciando cómo la física moderna puede contribuir directamente al bienestar humano. 
La optimización de tiempos de adquisición, la reducción de ruido y el desarrollo de imanes más eficientes seguirán mejorando la precisión y seguridad de este método, consolidando su papel en la medicina del futuro.
