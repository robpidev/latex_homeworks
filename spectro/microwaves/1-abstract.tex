\begin{abstract}
La presente monografía aborda de manera teórica los fundamentos cuánticos de la \textbf{espectroscopía de microondas}, técnica espectroscópica que permite el estudio de los \textbf{niveles rotacionales de moléculas polares}. A partir de la ecuación de Schrödinger para el \textit{rotor rígido lineal}, se deducen las \textbf{energías de los niveles rotacionales} y se analiza la aparición de las líneas espectrales mediante las reglas de selección permitidas por el momento dipolar. 

Se desarrolla la expresión general para la \textbf{constante rotacional} $B = \frac{h}{8\pi^2 I c}$, relacionándola con el \textbf{momento de inercia molecular} y las \textbf{longitudes de enlace}, mostrando cómo la espectroscopía de microondas constituye una herramienta precisa para determinar parámetros estructurales a nivel cuántico. Además, se extiende el modelo al \textbf{rotor no rígido}, incorporando correcciones centrífugas que explican la separación no uniforme de las líneas espectrales observadas experimentalmente.

El trabajo incluye la deducción de las \textbf{reglas de selección rotacionales} $(\Delta J = \pm 1)$, el análisis del espectro rotacional de moléculas diatómicas como HCl y CO, así como la discusión de las diferencias entre \textbf{rotores lineales, simétricos y asimétricos}. Finalmente, se presentan aplicaciones relevantes en la \textbf{astrofísica molecular}, la \textbf{identificación espectral de compuestos gaseosos} y la \textbf{determinación de constantes moleculares fundamentales}. 

El enfoque es enteramente teórico, sin componente experimental, y se acompaña de las deducciones matemáticas necesarias para comprender la naturaleza cuántica de las transiciones rotacionales en el dominio de las microondas.
\end{abstract}