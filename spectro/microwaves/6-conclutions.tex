\section{Conclusiones}

En el desarrollo de esta monografía se han cumplido los objetivos planteados, obteniéndose los siguientes resultados y conclusiones específicas:

\begin{enumerate}
    \item \textbf{Derivar las expresiones teóricas de los niveles de energía rotacional a partir del modelo del rotor rígido.}\\
    Se obtuvo que los niveles de energía rotacional de una molécula diatómica se expresan como 
    \[
        E_J = B J (J + 1),
    \]
    donde $B = \dfrac{h}{8\pi^2 I c}$ es la constante rotacional. Esta deducción demuestra que la cuantización del momento angular molecular conduce a una estructura discreta de niveles energéticos, coherente con la naturaleza cuántica del movimiento rotacional.

    \item \textbf{Analizar las reglas de selección que gobiernan las transiciones entre niveles.}\\
    Se dedujo la regla de selección $\Delta J = \pm 1$ a partir de las propiedades del operador momento dipolar y de la interacción de la molécula con la radiación electromagnética. Esta condición explica por qué sólo ciertas transiciones son observables en el espectro de microondas y sustenta la interpretación de las líneas espectrales registradas experimentalmente.

    \item \textbf{Extender el estudio al rotor no rígido, incluyendo las correcciones centrífugas.}\\
    La introducción de un término correctivo de la forma 
    \[
        E_J = B J (J + 1) - D [J(J + 1)]^2
    \]
    permitió considerar el alargamiento del enlace molecular debido a la fuerza centrífuga. Esta corrección es esencial para describir con precisión las desviaciones observadas en los espectros experimentales de moléculas livianas o altamente excitadas rotacionalmente.

    \item \textbf{Aplicar los resultados al análisis del espectro rotacional de moléculas diatómicas típicas, como HCl y CO.}\\
    Mediante valores experimentales de $B$ y de las masas atómicas, se calcularon los momentos de inercia y las longitudes de enlace de moléculas como HCl y CO. Los resultados concuerdan con los valores reportados en la literatura, evidenciando la potencia del modelo rotacional para determinar parámetros estructurales moleculares a partir de observaciones espectrales.

    \item \textbf{Conocer los principales equipos utilizados en espectroscopía de microondas.}\\
    Se a logrado conocer los principales equipos utilizados así como su funcionamiento tales como CP-FTMW, FTMW, VNA.

    \item \textbf{Identificar las aplicaciones prácticas de la espectroscopía de microondas.}\\
    Las técnicas de microondas tienen aplicaciones en química cuántica, caracterización de enlaces, estudio de isotopólogos, control de reacciones y detección ambiental. Además, en radioastronomía se utilizan para identificar moléculas interestelares, permitiendo correlacionar observaciones experimentales con modelos teóricos de dinámica molecular.
\end{enumerate}

En conjunto, este trabajo evidencia cómo los fundamentos cuánticos de la rotación molecular, junto con el desarrollo tecnológico de los equipos espectroscópicos, hacen de la espectroscopía de microondas una herramienta clave en la física moderna y la ciencia molecular.
