\subsection{Reglas de selección}

Las transiciones rotacionales observables en espectroscopía de microondas ocurren cuando una molécula interacciona con la radiación electromagnética de una frecuencia que coincide con la diferencia de energía entre dos niveles rotacionales. Sin embargo, no todas las transiciones entre niveles cuánticos están permitidas. La probabilidad de transición está determinada por la \textbf{regla de selección}, que depende de la simetría del sistema y de la naturaleza del operador de interacción con el campo electromagnético.

\subsubsection*{Consideraciones de momento dipolar}

El acoplamiento entre la molécula y el campo de radiación proviene del término de interacción dipolar:
\begin{equation}
\hat{H}' = - \vec{\mu} \cdot \vec{E},
\end{equation}
donde $\vec{\mu}$ es el operador momento dipolar eléctrico de la molécula y $\vec{E}$ el campo eléctrico de la onda electromagnética incidente.  
La probabilidad de transición entre un estado inicial $\psi_i$ y uno final $\psi_f$ está dada, según la teoría de perturbaciones de primer orden, por:
\begin{equation}
P_{i \rightarrow f} \propto \left| \langle \psi_f | \vec{\mu} \cdot \vec{E} | \psi_i \rangle \right|^2.
\end{equation}

Para que la transición sea permitida, la integral de momento de transición debe ser diferente de cero:
\begin{equation}
\langle \psi_f | \vec{\mu} | \psi_i \rangle \neq 0.
\end{equation}
Esta condición impone que el operador de momento dipolar debe conectar funciones de onda cuyas simetrías difieran de manera compatible con los componentes espaciales de $\vec{\mu}$.  
En el caso de una molécula diatómica lineal, las componentes relevantes son $\mu_x$, $\mu_y$ y $\mu_z$, que se comportan como los armónicos esféricos $Y_{1,1}$, $Y_{1,-1}$ y $Y_{1,0}$, respectivamente \parencite[p.~120]{bernath2016spectra}.

\subsubsection*{Derivación de la regla de selección}

El operador $\vec{\mu}$ actúa como un tensor de rango uno. Por lo tanto, de las propiedades de los armónicos esféricos y de los coeficientes de Clebsch–Gordan, se deduce que solo son posibles las transiciones entre niveles rotacionales con números cuánticos $J$ que difieran en una unidad:
\begin{equation}
\Delta J = \pm 1.
\end{equation}

Esto se obtiene al evaluar el elemento de matriz angular:
\begin{equation}
\langle J', M' | \mu_q | J, M \rangle \propto 
\begin{pmatrix}
J & 1 & J' \\
-M & q & M'
\end{pmatrix},
\end{equation}
donde el símbolo de 3-$j$ de Wigner sólo es distinto de cero si se cumple la condición triangular:
\begin{equation}
|J' - J| \le 1 \le J' + J.
\end{equation}
De esta relación se desprende directamente que:
\[
\Delta J = J' - J = \pm 1.
\]
El caso $\Delta J = 0$ no contribuye al espectro rotacional porque no produce cambio en el momento dipolar efectivo y, por tanto, no emite ni absorbe radiación electromagnética en el rango de microondas \parencite[p.~91]{banwell1994fundamentals}.

\subsubsection*{Moléculas polares y actividad rotacional}

La existencia de un \textbf{momento dipolar permanente} es condición indispensable para que una molécula sea activa en el dominio de microondas.  
Cuando una molécula polar rota, la orientación de su momento dipolar cambia con el tiempo, generando una componente oscilante del campo eléctrico que puede acoplarse con la radiación incidente. En contraste, las moléculas no polares (como $\text{N}_2$, $\text{O}_2$ o $\text{CO}_2$) poseen simetría que cancela el momento dipolar neto, por lo que $\vec{\mu} = 0$ y el término de interacción $\vec{\mu} \cdot \vec{E}$ desaparece.  
Consecuentemente, estas moléculas no presentan líneas en el espectro rotacional de microondas, aunque sí pueden ser activas en la región del infrarrojo debido a transiciones vibracionales que inducen dipolos temporales \parencite[p.~93]{hollas2004modern}.

\subsubsection*{Resumen de la selección rotacional}

En resumen, las condiciones para que una transición rotacional sea observable son:
\begin{itemize}
    \item La molécula debe poseer un momento dipolar eléctrico permanente ($\mu \neq 0$).
    \item La transición debe cumplir la regla de selección $\Delta J = \pm 1$.
    \item La frecuencia de la radiación absorbida o emitida está dada por:
    \[
    \nu = 2B(J + 1),
    \]
    donde $B$ es la constante rotacional definida en la Ecuación~(9).
\end{itemize}

Estas reglas explican la estructura característica del espectro de microondas, constituido por una serie de líneas igualmente espaciadas en frecuencia, cuya separación es proporcional a la constante rotacional y, por tanto, inversamente proporcional al momento de inercia molecular.
