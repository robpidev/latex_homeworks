\subsection{Fundamentos cuánticos de la rotación molecular}

El estudio de la rotación molecular se fundamenta en los principios de la mecánica cuántica aplicados al movimiento rotacional de un sistema diatómico. En este contexto, la molécula se considera un \textbf{rotor rígido lineal}, es decir, un sistema de dos masas puntuales unidas por un enlace de longitud fija $r_e$, que rota libremente en el espacio alrededor de su centro de masas \parencite[p.~95]{atkins2018molecular}.

\subsubsection*{Postulados básicos}

El movimiento rotacional está gobernado por el operador de momento angular $\hat{L}$, cuyo cuadrado total $\hat{L}^2$ y componente proyectada sobre el eje $z$ ($\hat{L}_z$) cumplen las relaciones de conmutación:
\begin{equation}
[\hat{L}_x, \hat{L}_y] = i\hbar \hat{L}_z, \qquad
[\hat{L}^2, \hat{L}_z] = 0.
\end{equation}
De estas propiedades se deduce que existen funciones propias comunes de $\hat{L}^2$ y $\hat{L}_z$, las cuales se designan mediante los números cuánticos $J$ y $M$, respectivamente. Estos valores satisfacen:
\begin{equation}
\hat{L}^2 Y_{J,M}(\theta, \phi) = \hbar^2 J(J+1) Y_{J,M}(\theta, \phi),
\end{equation}
\begin{equation}
\hat{L}_z Y_{J,M}(\theta, \phi) = \hbar M Y_{J,M}(\theta, \phi),
\end{equation}
donde $Y_{J,M}(\theta, \phi)$ son las \textbf{funciones esféricas armónicas}, base natural para describir estados angulares \parencite[p.~77]{atkins2018molecular}.

\subsubsection*{Hamiltoniano rotacional}

El Hamiltoniano que describe la energía cinética rotacional de una molécula diatómica rígida está dado por:
\begin{equation}
\hat{H}_{\text{rot}} = \frac{\hat{L}^2}{2I},
\end{equation}
donde $I = \mu r_e^2$ es el momento de inercia respecto al centro de masas, y $\mu = \frac{m_1 m_2}{m_1 + m_2}$ representa la masa reducida del sistema \parencite[p.~112]{bernath2016spectra}.  
Este operador actúa únicamente sobre las coordenadas angulares $(\theta, \phi)$, pues la distancia internuclear $r_e$ se asume constante bajo la aproximación de rigidez.

\subsubsection*{Ecuación de Schrödinger rotacional}

Al sustituir el Hamiltoniano en la ecuación de Schrödinger estacionaria, se obtiene:
\begin{equation}
\hat{H}_{\text{rot}} \psi(\theta, \phi) = E \psi(\theta, \phi),
\qquad
\Rightarrow \qquad
\frac{\hat{L}^2}{2I} \psi(\theta, \phi) = E \psi(\theta, \phi).
\end{equation}
El operador $\hat{L}^2$ en coordenadas esféricas se expresa como:
\begin{equation}
\hat{L}^2 = -\hbar^2
\left[
\frac{1}{\sin \theta} \frac{\partial}{\partial \theta}
\left(\sin \theta \frac{\partial}{\partial \theta}\right)
+ \frac{1}{\sin^2 \theta} \frac{\partial^2}{\partial \phi^2}
\right].
\end{equation}
La ecuación diferencial resultante es idéntica a la ecuación angular del átomo de hidrógeno y sus soluciones normalizadas son las funciones esféricas armónicas $Y_{J,M}(\theta, \phi)$, con $J = 0, 1, 2, \ldots$ y $M = -J, -J+1, \ldots, J$.

\subsubsection*{Niveles de energía rotacional}

Al aplicar el operador $\hat{L}^2$ sobre su función propia, se obtiene la energía asociada a cada nivel rotacional:
\begin{equation}
E_J = \frac{\hbar^2}{2I} J(J+1),
\end{equation}
que usualmente se expresa en unidades de energía por mol, o en términos de número de onda (cm$^{-1}$).  
Definiendo la constante rotacional $B$ como:
\begin{equation}
B = \frac{h}{8\pi^2 I c},
\end{equation}
la energía rotacional se escribe:
\begin{equation}
E_J = B J(J+1),
\end{equation}
donde $h$ es la constante de Planck y $c$ la velocidad de la luz.  
Esta forma demuestra que los niveles de energía están espaciados de manera no uniforme, con incrementos crecientes en función de $J$ \parencite[p.~90]{banwell1994fundamentals}.

\subsubsection*{Degeneración de los niveles}

Cada nivel caracterizado por el número cuántico $J$ posee una degeneración $g_J = 2J + 1$, correspondiente a los posibles valores del número cuántico magnético $M$.  
Las funciones de onda $Y_{J,M}(\theta, \phi)$ describen la orientación angular de la molécula en el espacio, y su ortogonalidad garantiza la independencia de los distintos estados rotacionales.  

En el contexto espectroscópico, las transiciones entre estos niveles obedecen la regla de selección:
\begin{equation}
\Delta J = \pm 1,
\end{equation}
válida únicamente para moléculas con momento dipolar eléctrico permanente, condición esencial para la observación de espectros de microondas.
