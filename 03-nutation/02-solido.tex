\section{Sólido rígido}
\subsection{Velocidad en un sólido rígido}
Un sólido rígido es un sistema de partículas en la que la distancia
entre partículas permanece constante. La velocidad de una partícula
ubicada en el punto $P(x, y, z)$, en un sólido que tiene una velocidad
angular $\mathbf{\omega}$ y velocidad $\mathbf{v_0}$ en el punto $O(x_0, 
y_0, z_0)$, estará dada por

\begin{figure}[h]
    \centering 
    \includegraphics[scale=0.5]{img/1.png}
    \caption{Gráfico de un sólido rígido y sus velocidades}
\end{figure}

\begin{equation}\label{eq:velocidad_angular}
    \mathbf{v_p} = \mathbf{v_0} + \bm{\omega}\times\mathbf{r}
\end{equation}

La velocidad angular $\bm{\omega}$ es libre i.e es independiente
de donde se elija el punto $O$, siempre será la misma.

\subsection{Energía Cinética}
Para un sistema de partículas la energía cinética está dada por

\begin{equation}
    T = \frac{1}{2} \sum_{\alpha = 0}^{n}m_\alpha \mathbf{v}_\alpha
\end{equation}

Tomando $\mathbf{v_\alpha} = \mathbf{v_0} + \bm{\omega} \times \mathbf{r}$,
con $\mathbf{v_0} = \v{V}$ es la velocidad del centro de masas, se deduce que la
energía cinética es
\begin{equation}
    T = \frac{1}{2}MV^2 + \frac{1}{2}\bm{\omega}\mathbb{I}\bm{\omega}
\end{equation}

donde $\mathbb{I}$ es el tensor de inercia de cuyas componentes
están dadas por

\begin{equation}\label{eq:tensor_inercia}
    I = \begin{bmatrix}
        \sum m_\a(y^2 + z^2) && -\sum m_\a xy && -\sum m_\a xz \\
        -\sum m_\a xy && \sum m_\a (x^2 + z^2) && -\sum m_\a yz \\
        -\sum m_a xz && -\sum m_\a yz && \sum m_\a (x^2 + y^2) \\
    \end{bmatrix} 
\end{equation}

Las componentes $I_{ii}$ son los momentos de inercia con respecto a los
ejes, como es un tensor en $\mathbb{R}$ y simétrico, entonce siempre
existe una base donde es diagonal.

\begin{equation*}
    I^0 = \begin{bmatrix}
        I_x & 0 & 0\\
        0 & I_y & 0\\
        0 & 0 & I_z \\
    \end{bmatrix}
\end{equation*}

\begin{itemize}
    \item si $I_x \neq I_y \neq I_z$ se le llama peonza o trompo asimétrico.
    \item Si $I_x = I_y \neq I_z$ se llama peonza o trompo simétrico.
    \item Si $I_x = I_y = I_z$ se le llama peonza o trompo esférico.
\end{itemize}

% Momento angular
\subsection{Momento Angular}
El momento lineal de un sistema de partículas se define como
\begin{equation}
    \v{P} = \sum_{\a}\v{p} = \sum_{\a}m_a\v{v}_\a = \v{P_{cm}}
\end{equation}

Si derivamos con respecto al tiempo obtenemos
\begin{equation}
    \frac{d\v{P}}{dt} = M\frac{d\v{V}}{dt}
\end{equation}

Y el momento angular en un sistema de partículas es

\begin{equation}
    \v{L_o} = \sum_{\a}\v{r_\a}\times \v{p_\a}
\end{equation}

De esta ultima ecuación se deduces que
\begin{equation}
    \frac{d\v{L}}{dt} = \sum_{\a}\v{r}_\a \times \v{F}^{(ext)} = \v{N}^{(ext)}
\end{equation}

haciendo
\begin{gather*}
    \v{r_\a = R + r'_\a}\\
    \v{v_\a = v_0 + \bm{\omega}\times r'_\a}
\end{gather*}

Y remplazando en (5) se deduce que
\begin{equation}
    \v{L_0 = R\times P} + \sum_\a m_\a \v{r'_\a \times (\bm{\o} \times r'_\a)}
    = v{L_{cm} + L_{spin}}
\end{equation}

Simplificando esta ultima ecuación para $\v{L_{spin}}$ se tiene
\begin{equation}
    \v{L_{spin} = \mathbb{I}_{cm}\bm{\o}}
\end{equation}

Donde $\mathbb{I}$ es el tensor de inercias ya encontrado. Como
$I_{cm}$ es diagonal entonces el momento angular está dado por
\begin{equation}
    \v{L} = (I_x\o_x, I_y\o_y, I_z\o_z)
\end{equation}

% Ecuaciones de Euler
\subsection{Ecuaciones de Euler}
Sea $d\v{A}/dt$ la velocidad de variación de un vector $\v{A}$ con
respecto al sistema fijo. Si el vector $\v{A}$ no cambia en un sistema que gira, su
velocidad de variación relativa al sistema fijo será debida únicamente a
la rotación, y se tendrá entonces,
\begin{equation}
    \frac{d\v{A}}{dt} = \bm{\o}\times \v{A}
\end{equation}

En el caso general, debe agregarse al segundo miembro de esta igualdad
la velocidad de variación del vector $\v{A}$ con respecto al sistema
móvil, designando esta velocidad por $d'\v{a}/dt$ se obtiene:
\begin{equation*}
    \frac{d\v{A}}{dt} = \frac{d'\v{A}}{dt} + \bm\o \times \v{A}
\end{equation*}

Usando esta forma general tanto para $P$ como para $L$ se obtiene
\begin{equation*}
    \frac{d'\v{P}}{dt} + \bm{\o} \times \v{P} = \v{F},
    \frac{d'\v{L}}{dt} + \bm{\o} \times \v{L} = \v{N}
\end{equation*}

Estas ultimas ecuaciones en componentes será
\begin{align}
    M(\frac{dV_1}{dt}+\o_2V_3 - \o_3V2) = F_1\\
    M(\frac{dV_2}{dt}+\o_3V_1 - \o_1V3) = F_2\\
    M(\frac{dV_3}{dt}+\o_1V_2 - \o_2V3) = F_3\\
\end{align}

Si los ejes, coinciden con los ejes principales de inercia, se puede poner
$N_i = I_i\o_i$ remplazando en la segunda ecuación

\begin{align}
    I_1 \frac{d\o_1}{dt} + (I_3 - I_2)\o_2\o_3 = N1\\
    I_2 \frac{d\o_3}{dt} + (I_1 - I_3)\o_3\o_1 = N2\\
    I_3 \frac{d\o_3}{dt} + (I_2 - I_1)\o_1\o_2 = N3\\
\end{align}

Si se tiene una rotación Libre $\v{N} = 0$ por lo que se tiene

\begin{align}
    I_1 \frac{d\o_1}{dt} + (I_3 - I_2)\o_2\o_3 = 0\\
    I_2 \frac{d\o_3}{dt} + (I_1 - I_3)\o_3\o_1 = 0\\
    I_3 \frac{d\o_3}{dt} + (I_2 - I_1)\o_1\o_2 = 0\\
\end{align}

