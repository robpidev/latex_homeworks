\section{Nutación de la Tierra}
Dado que la tierra es una peonza simétrica
entonces se tiene que $I_x = I_y$, esta
está rotando alrededor de los ejes $x, y$ y $z$
y sus respectivas velocidades angulares
$X(t) = \o_x(t), Y(t)=\o_y(t), \o_z = const$.
Las Ecuaciones de Euler del movimiento de $X$ y
$Y$ se reducen a

\begin{equation}
    \dot X = -aY, \dot Y = aX
\end{equation}

donde $a = [(I_z - I_x)/I_z]\o_z$, Los valores
iniciales $X$ y $Y$ ambos no son 0, por lo que el
eje de simetría no está alineado, para la tierra
solo es alrededor de 15 metros, medidos dese los
ángulos polares. Ahora se resolverá las ecuaciones
diferenciales acopladas, usando las transformadas de Laplace
se tiene


\begin{equation*}
    \mathcal{L}(\dot X = -aY), \mathcal{L}(\dot Y = aX)
\end{equation*}

Lo que nos da
\begin{equation*}
    sX(s) - X(0) = -ay(s), sY(s) - Y(0) = aX(s)
\end{equation*}

Eliminando $Y(s)$, tenemos
\begin{gather*}
    s^2X(s) - sX(0) + aY(0) = -a^2 X(s)\\
    \Rightarrow X(s) = X(0) \frac{s}{s^2 + a^2} - Y(0)\frac{a}{s^2+a^2}
\end{gather*}

Tomando la transformada inversa tenemos la solución
\begin{equation}
    X(t) = X(0)\cos(at) - Y(0)\sin(at)
\end{equation}

De manera similar, se obtiene la solución par $Y(t)$
\begin{equation}
    Y(t) = X(0)\sin(at) + Y(0)\cos(at)
\end{equation}

Tomando $Y(0)$ las ecuaciones quedarán
\begin{equation}
    X(t) = X(0)\cos(at), Y(t) = X(0)\sin(at)
\end{equation}

Las cuales serán ecuaciones paramétricas de rotación
de (X, Y) en una orbita particular de radio $X(0)$, 
Con algunas velocidades angulares $a$ en el sentido
contrario de las agujas del reloj.

Si se toma otros valores iniciales de $X(0), Y(0)$, 
El eje de rotación tenderá a moverse de forma oscilante
sobre el eje de precesión como se muestra en la figura siguiente.

\begin{figure}
    \centering
    \includegraphics[scale=0.5]{img/2.png}
\end{figure}