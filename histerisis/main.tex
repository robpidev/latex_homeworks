\documentclass[12pt, a4paper]{article}
\usepackage{amsmath}
\usepackage{amssymb}
\usepackage{amsfonts}
\usepackage[right=3cm, left=2.5cm, top=3cm, bottom=3.5cm]{geometry}
% graphix
\usepackage{graphicx}
%\usepackage[style=apa]{biblatex}
%\addbibresource{references.bib}

\title{Ciclo de histérisis}
\author{Torres Tarrillo Rober}

\begin{document}

\maketitle
\section{Relación lineal entre la magnetización $\mathbf{M}$ y el campo
magnético externo $\mathbf{H}$}
En una extensa clase de materiales existe una relación
lineal aproximada entre $\mathbf{M}$ y $\mathbf{H}$:

\begin{equation}
\mathbf{M} = \chi_m\mathbf{H}
\end{equation}

donde $\chi_m$ se llama suceptibilidad magnética. En algunos
materiales también dpende de la masa $\chi_{m,masa}$ o masa molar $\chi_{m,molar}$
y está relacionado por

\begin{equation}
    \chi_m = \chi_{m,masa} d
\end{equation}
\begin{equation}
    \chi_m = \chi_{m, molar} \frac{d}{A}
\end{equation}

Donde $d$es la densidad del material y $A$ masa molar del material.

Una relación lineal entre $\mathbf{M}$ y $\mathbf{H}$ también
implica una relación lineal entre $\mathbf{B}$ y $\mathbf{H}$:

\begin{equation}
\mathbf{B} = \mu\mathbf{H}
\end{equation}

de aquí
\begin{equation}
    \mu = \mu_0 (1 + \chi_m)
\end{equation}

ademas en el vacio $\mu = \mu_0$

\section{Materiales ferromagnéticos}
Losferromagnéticosforman otra clase de material magnético. Dicho material
se caracteriza por una posible magnetización permanente y por el
hecho de que su presencia tiene generlamente un efecto muy significativo
sobre la inducción magnética.

Los materiales ferromagnéticos no son lineales, por lo que las ecuaciones
anteriores no son aplicables. Por lo qué de la ecuación (4) podemos
definir a $\mu = \mu(\mathbf{M})$ y está se calculará a partir de aproximaciones
dado qué es complicado tener una función analítica, ademas varía
de cero a infinito.

\subsection{Dominio ferromagnético}
es una región microscópica dentro de un material ferromagnético en la que los momentos magnéticos de los átomos están alineados en la misma dirección debido a interacciones entre ellos. Esto crea un magnetismo colectivo en esa región.
En un material ferromagnético sin magnetizar, los dominios magnéticos están orientados de forma aleatoria, cancelándose entre sí y resultando en un magnetismo neto nulo. Sin embargo, cuando se aplica un campo magnético externo, estos dominios tienden a alinearse en la dirección del campo, lo que incrementa el magnetismo del material.
En la figura 2 se muestra el siclo de histérisis y se da como sigue:
\begin{figure}[h]
    \centering
    \includegraphics[scale=0.5]{f1.png}
    \caption{Dominio ferromagnético}
\end{figure}

\section{Siclo de histérisis}
El ciclo de histéresis describe cómo la magnetización de un material ferromagnético cambia cuando se somete a un campo magnético externo variable.

\begin{enumerate}
    \item Inicio (Punto O):
        \begin{itemize}
            \item El material no está magnetizado.
            \item No hay campo magnético externo $\mathbf{H = 0}$ y tampoco magnetización $\mathbf{B = 0}$
        \end{itemize}
    
    \item Imposición del campomagnético ($O\to a$):
    \begin{itemize}
        \item Un campo magnético externo $\mathbf{H}$ comienza a aplicarse.
        \item Los dominios magnéticos empiezan a alinearse con el campo, aumentando la magnetización del material ($\mathbf{B}$).
        \item El material se magnetiza gradualmente hasta alcanzar la magnetización de saturación en el punto $a$, donde prácticamente todos los dominios están alineados.
    \end{itemize}
    
    \item Reducción del campo magnético ($a\to c$):
    \begin{itemize}
        \item $\mathbf{H}$ se reduce desde su valor máximo hacia cero.
        \item Aunque $\mathbf{H = 0}$, algunos dominios permanecen alineados, lo que deja una magnetización remanente ($\mathbf{B}_4$
        punto $C$).
        \item Este comportamiento muestra que el material retiene magnetización incluso sin un campo externo.
    \end{itemize}

    \item Aplicación de un campo en sentido opuesto ($c\to d$):
    \begin{itemize}
        \item $\mathbf{H}$ se invierte y aumenta en la dirección opuesta.
        \item La magnetización disminuye progresivamente hasta llegar a $\mathbf{B = 0}$ en el punto $d$.
        \item El valor de $\mathbf{H}$ necesario para desmagnetizar el material se llama coercitividad ($\mathbf{H}_c$).
    \end{itemize}

    \item Saturación inversa ($d \to e$):
    \begin{itemize}
        \item Al continuar aumentadon $H$ en la dirección opuesta, los dominios se realinean completamente en esa dirección.
        \item El material alcanza la saturación inversa $e$.
    \end{itemize}
    
    \item Reducción del campo magnético inverso ($e \to g$):
    \begin{itemize}
        \item $\mathbf{H}$ se reduce nuevamente hacia cero y algunos dominios
        permanecen alineados en la dirección inversa.
        \item Esto da lugar a una magnetización remantente inversa ($\mathbf{-B}_r$).
    \end{itemize}

    \item Reaplicación del campo original ($g \to a$):
    \begin{itemize}
        \item si $\mathbf{H}$  se invierte otra vez hacia su dirección inicial, la magnetización pasa por un comportamiento simétrico al ciclo anterior, regresando finalmente al punto 
        $a$
    \end{itemize}

    \end{enumerate}

\begin{figure}[ht]
    \centering
    \includegraphics[scale=0.5]{f2.png}
    \caption{Ciclo de histérisis}
\end{figure}
    
El ancho de la curva de histéresis $-Hc, Hc$ no indica si tiene
    una magnetización muy grande o pequeña cuando no se esta aplicando un
    campo externo como se puede ovservar en la figura 3.


\begin{figure}[ht]
    \centering
    \includegraphics[scale=0.5]{f3.png}
    \caption{Curvas de otro material}
\end{figure}

Para desmagnetizar un material ferromagnético se necesita un campo

\section{Conversión de energía en calor en un sistema que existe histérisis}
Este calor es adicional al calor producido por la existencia de una conductividad.

Para la variación de la energía magnética se tiene
\begin{equation}
    \delta U_m = \sum_{j}i_j\delta\Phi_j = \sum_{j}i_j\oint_{C_j}\delta\vec{A}_j\cdot d \vec{s}_j
\end{equation}

siendo $\delta \vec{A}_j$ el cambio correspondiente del potencial vectorial en la parte
$j$ del sistema. Esto se puede escribir como

\begin{equation}
    \delta U_m = \int_V \vec{J}_f \cdot \delta \vec{A}d\tau = \int_V \nabla\times \vec{H}\cdot \delta \vec{A}d\tau
\end{equation}

Pero $\nabla \times \vec{A} = \delta \vec{B}$ por lo que
$\nabla \times (\vec{A} + \delta \vec{A}) = \vec{B} + \delta \vec{B}$
por lo que $\nabla\times\delta\vec{A} = \delta \vec{B}$. Por otra parte
\begin{equation}
    \vec{A}\cdot(\nabla\times\vec{H}) = \vec{H}\cdot\vec{B} - \nabla\cdot(\vec{A}\times \vec{H})
\end{equation}

por lo que se obtiene

\begin{equation}
    \delta U_m = \int_V \vec{H}\cdot \delta \vec{B}d\tau
\end{equation}

para sistemas lineales no hay problema en resolver esta integral, 
pero para sistemas qué no son lineales, se tiene que hacer mediante métodos
númericos.

Si $\delta w_m$ es la energía requerida de una fuente externa por unidad de volumen
de material, se puede expresar el integrando (9) como $\delta w_m d\tau$, con
lo que se tiene

\begin{equation}
    \delta w_m = \vec{H} \cdot \delta \vec{B}
\end{equation}

Esto es el area del recuadro en la figura (4). Integrando se tiene

\begin{figure}[ht]
    \centering
    \includegraphics[scale=0.5]{f4.png}
    \caption{Interpretación de $H\delta B$ como una superficie}
\end{figure}

\begin{equation}
    w_m = \oint_{ciclo} \vec{H}\cdot \delta \vec{B} = \oint_{ciclo} H\delta B
\end{equation}

Esto es númericamente igual al área encerrada por la curva de histéricis

esto viene ha ser el harea sombre de la figura (4)



\section{Referencias}
Reitsz, J., Milford, F. y Chisty, R. (Sin fecha). "Fundamentos de la teoría electromagnética", 4ta ed. Addison-Wesley Iberoamericana.

Wangsness, R. (2001). "Campos electromagnéticos". México: Limusa.

\end{document}