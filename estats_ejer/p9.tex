La radiación electromagnética dentro de una cavidad cerrada y en equilibrio a una
puede estudiarse considerando que los modos permitidos
de las ondas resonantes dentro de la cavidad tienen energías
$\e_n = nh\nu (n = 0, 1, 2,\ldots)$, es decir,
que el campo de radiación constituye un conjunto
de cuantos de energía $h\nu$, o ``O gas de fotones''.
En el lengjuaje cuántico, los fotones son bosones cuyo n'umero no tiene por
qu'e conservarse.
$(a)$ Calcular la función de partición del sistema. (Recordar
que las ondas electromagnéticas tienes dos direcciones
de polarización transversales para cada frecuencia).
$(b)$ Hallar la energía total $U$ (ley de Stefan-Boltzmann)
y la presión de radiación en una cavidad de volumen $V$.
$(c)$ Determinar la ley de Planck para la densidad espectral de
radiación del cuerpo negro $\rho(\nu)$. Es decir, la energía
por unidad de volumen de radiación electromagnética entre
$\nu$ y $\nu + d\nu$

\sol\ $(a)$ Para un gas de partículas en cierto volumen, en equilibrio a temperatura T, la hipótesis de 
interacción es despreciablemente pequeña. Podemos escribir la energía total del gas 
cuando esta en cierto estado R de la siguiente manera. 

\begin{equation}
    \e_r = \sum_r n_r\e_r
\end{equation}

Si se conoce el numero total de partículas entonces lo que se debe cumplir es lo 
siguiente: 

\begin{equation*}
    \sum_r n_r = N
\end{equation*}

Por lo que

\begin{equation*}
    \sum_r e^{-\beta e_r}
\end{equation*}

En este caso $\e_r = \e_n$ asi se tiene

\begin{equation*}
    Z = \sum_r e^{-\beta \e_n} = \sum_{n = 0}^{\infty} e^{-\beta nh\nu}
    = \frac{1}{1 - e^{-\beta h\nu}}
\end{equation*}

Puesto que, hay un total de N posibilidades de valores para las distintas frecuencias 
podemos expresarlos en función de cada una de estas es decir 

\begin{equation*}
    Z = \prod_N \frac{1}{1 - e^{-\beta h\nu}}
\end{equation*}

$(b)$ Ahora encontramos la energía total

\newcommand{\p}[1]{\frac{\partial}{\partial\beta}#1}

\begin{equation*}
    U = - \p \ln Z = - \p \ln \prod \frac{1}{1 - e^{-\beta h\nu}}
    = - \p \sum_N \ln \frac{1}{1 - e^{-\beta h\nu}}
\end{equation*}

haciendo $x = \beta h\nu \implies dx = \beta h\ d\nu$ y cambiando
la sumatoria por una integral tenemos

\begin{equation*}
    U = \frac{8\pi}{c^3h^3\beta^4}\int_{0}^{\infty}
    \left(
        \frac{x^3}{e^x - 1}dx
    \right)
\end{equation*}

El resultado de esta integral es $\pi^4/15$ por lo que obtenemos

\begin{equation*}
    U = \frac{8\pi}{c^3h^3\beta^4}\int_{0}^{\infty}
    \left(
        \frac{\pi^4}{15}
    \right)
\end{equation*}

Remplazando $\beta$ y dentro de un volumen $V$ la energía se tiene

\begin{equation*}
  U = \sigma T^4 V  
\end{equation*}

donde $\sigma$ es conocida como la constante de Stefan-Boltzman
y es

\begin{equation*}
    \sigma = \frac{8\pi^5k^4}{15c^3h^3}
\end{equation*}

La relación entre la presión y la energía total para un gas de fotones en equilibrio térmico 
está dada por: 

\begin{equation*}
    P = \frac{1}{3}{U}{V} = \frac{1}{3}\frac{\sigma T^4 V}{V}
    = \frac{\sigma T^4}{3} = \frac{\sigma T^4}{3}
\end{equation*}

$(c)$
\begin{equation*}
    U_\omega(\omega + d\omega) - \omega U_\omega = U\omega d\omega
    = \frac{h\omega^{-\beta h\omega}}{1 - e^{-\beta h \omega}}D(\omega)d\omega
\end{equation*}

donde $D(\omega)$ es la densidad de modos por $V/\pi^2c^3$ reemplazando
nos queda

\begin{equation*}
    U\omega d\omega
    = \frac{h\omega^{-\beta h\omega}}{1 - e^{-\beta h \omega}}
    \left(
        \frac{V}{\pi^2c^3}
    \right)
    d\omega
\end{equation*}

usando la relación $\nu = \frac{\omega}{2\pi}$ y simplificando
la expresión final sería

\begin{equation*}
    \frac{\hbar \omega^3}{pi^2c^3} \frac{1}{e^{\beta \hbar \omega} - 1}d\omega
\end{equation*}

La cual vendría a ser la Ley de Radiación de Planck.

\solend\