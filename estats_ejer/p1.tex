Considere un sistema formado por 2 partículas, cada una
de las cuales puede estar en cualquiera de 3 estados cuánticos
de energías respectivas $0$, $\varepsilon$, $3\varepsilon$.
El sistema está en contacto con un foco térmico
a temperatura $T$. $(a)$ Escriba la función de partición $Z$
si las partículas obedecen la estadística de Maxwell-Boltzmann
y son considerados distinguibles. $(b)$ Cuál es la expresión
de la función de partición $Z$ Si las partículas son bosones.
$(c)$ Cuál es la función de partición $Z$ si las partículas son
Fermiones.

\definecolor{solcolor}{HTML}{E91E63} % 
\newcommand{\e}{\varepsilon}

\sol $(a)$ la función de partición para la estadística de
Maxwell-Boltzmann está dada por
\begin{equation*}
    Z = {\left(
        \sum_r e^{-\beta \e_r}
    \right)} ^ N 
\end{equation*}

Se tiene que el número de partículas es $N = 2$, Como las estadística
de Maxwell-Boltzmann es para partículas distinguibles entonces
los estados posibles son las combinaciones de las distintas energías
incluyendo las que pueden ocupar a la vez, entonces e tiene:

\begin{equation*}
    Z = {\left(1 + e^{-\beta \e} + e^{-3\beta}\right)}^2  =         1 + 2e^{-\beta \e} + 2e^{-3\beta \e} + e^{-2\beta \e}
        + 2e^{-4\beta \e} + e^{-6\beta\e}
\end{equation*}

$(b)$ Para bosones la función de partición está dado por
\begin{equation*}
    Z = \sum_r \Omega(\e_r) e^{-\beta \e_r}
\end{equation*}

las energías son $0,\e, 2\e, 3\e, 4\e, 6\e$, por lo que se
tendrá que $Z$ es

\begin{equation*}
    Z = 1 + e^{-\beta \e} + e^{-2\beta \e} + e^{-3\beta \e}
    + e^{-4\beta\e} + e^{-6\beta\e}
\end{equation*}

(c) Si fuesen fermiones se tiene que la función de partición es:

\begin{equation*}
    Z = \sum_r \Omega(\e_r) e^{-\beta \e_r}
\end{equation*}

Pero 2 partículas no pueden ocupar el mismo estado a la vez, por
lo qué las energías posibles serán: $1\e, 3\e, 4\e$, entonces la
la función de partición es

\begin{equation*}
    Z = e^{-\beta\e} + e^{-3\beta\e} + e^{-4\beta\e}
\end{equation*}

\solend