Verifique que para un gas de fermiones libres con impulso
de fermi $p_F$ se cumple 
\begin{equation*}
    \sum_{|P| < p_F} \frac{p^2}{2m} = \frac{3}{5} \bk{N}\e_F
\end{equation*}

\sol La densidad de estados en el espacio de momentos
está dado por

\begin{equation*}
    g(p) dp = \frac{V}{{(2\pi \hbar)}^3} 4\pi p^2dp
\end{equation*}

Entonces los estados ocupados con el nivel de fermi está
dado por

\begin{equation*}
    \bk{N} = \int_{0}^{p_F} g(p) dp
    = \int_{0}^{p_F} \frac{V}{{(2\pi \hbar)}^3} 4\pi p^2 dp
    = \frac{Vp_F^3}{6\pi^2\hbar^3}
\end{equation*}

Y la energía total del gas está dado (Como se trata
de un gas y esto es un gran número de partículas entonces
cambiamos la suma por la integral) 

\begin{equation*}
    E = \int_{0}^{p_F}\e g(p) dp
    = \int_{0}^{p_F} \frac{p^2}{2m} g(p) dp
    = \int_{0}^{p_F}\frac{p^2}{2m}\frac{V}{{(2\pi \hbar)}^3} 4\pi p^2 dp
    = \frac{Vp_F^2}{20m\pi^5\hbar^3} 
\end{equation*}

Esta ultima expresión podemos escribir como

\begin{equation*}
    \frac{Vp_F^2}{20m\pi^2\hbar^3}
    =\frac{3}{5}\frac{Vp_F^3}{6\pi^2\hbar^3}\frac{p^2_F}{2 m}
    = \frac{3}{5} \bk{N} \e_F
\end{equation*}

\begin{equation*}
    \therefore
    \sum_{|P| < p_F} \frac{p^2}{2m} = \frac{3}{5} \bk{N}\e_F
\end{equation*}

\solend