Muestre que para un gas ideal de Fermi la energía libre de Helmholtz
por partícula a bajas temperaturas está dada por

\begin{equation*}
    f = \lim_{N \to \infty} \frac{F}{N}
    = \frac{3}{5}\e_F
    \left[
        1 - \frac{5\pi^2}{12}
        {\left(
            \frac{kT}{\e_F}
        \right)}^2+\cdots
    \right]
\end{equation*}

\sol\ Se tiene que $F = U - TS$ y $ F = U_0 + U - T (k \ln Z + U/T)$
por lo que

\begin{equation*}
    F = U_0 - kT \ln Z
\end{equation*}

ahora encontremos $U_0$

\begin{align*}
    U_0 &= \frac{V}{h^3} 4 \pi \int_{0}^{\e_F}\rho d\rho
    = \frac{4\pi V}{h^3}\int_{0}^{\e_F} \e (2m) \e{(2m)}^{1/2} \frac{1}{2}
    \e^{-1/2}d\e \\
    &= \frac{4\pi V}{h^3}\int_{0}^{\e_F}\e^{3/2}d\e
    = \frac{5}{2}\left(
        \frac{2\pi V  {(2m)}^{3/2}}{h^3}
    \right)\e^{5/2}
\end{align*}

Ahora encontrando la función de partición

\begin{equation*}
    \ln Z
    = \frac{2\pi V {(2m)}^{3/2}}{h^3}
    \int_{0}^{\infty} \e^{1/2} \ln(1 + e^{-\beta(\e - \mu)})d\e
\end{equation*}

Integrando por partes
\begin{equation*}
    u = \ln \left(
        1 + e^{-\beta(\e - \mu)}
    \right)
    \implies
    du = \frac{
        -\beta e^{-\beta(\e - \mu)}
    }{1 + e^{-\beta(\e - \mu)}}
    = \frac{-\beta}{e^{\beta(\e - \mu)} + 1} d\e
\end{equation*}

\begin{equation*}
    dv = e^{1/2}de \implies v = \frac{2}{3}\e^{3/2}
\end{equation*}

Entonces la función de partición será

\begin{equation*}
    \ln Z = \frac{2\pi V}{\hbar^3} {(2m)}^{3/2}
    \left[
        \frac{2}{3} \e^{3/2} \ln (1 + e^{-\beta(\e - \mu)})
        + \frac{2\beta}{3}
        \int_{0}^{\infty} \frac{e^3/2}{e^{\beta(\e - \mu)} + 1} d\e
    \right]
\end{equation*}

aproximando esta ultima integral

\begin{equation*}
    \ln Z = \frac{2\pi V}{\hbar^3} {(2m)}^{3/2} \frac{2\beta}{3}
    \left[
        \int_{0}^{\mu}\e^{3/2}d\e
        + \frac{\pi^6}{6}
        {\left(\frac{1}{\beta}\right)}^2
        \frac{3}{2}\mu^{1/2} + \cdots
    \right]
\end{equation*}

\begin{equation*}
    \ln Z = \frac{2\pi V}{\hbar^3} {(2m)}^{3/2} \frac{2\beta}{3}
    \left[
        \frac{2}{5}\e^{5/2} + 
        \frac{\pi^6}{6}
        {\left(\frac{1}{\beta}\right)}^2
        \frac{3}{2}\e_F^{1/2} + \cdots
    \right]
\end{equation*}

Calculando $N$

\begin{equation*}
    N = \frac{V}{h^3}\frac{4\pi}{2}{(2m)}^{3/2} \int_{0}^{\e_F} \e^{1/2}d\e
    = \frac{V}{h^3}\frac{4\pi}{2}{(2m)}^{3/2} \frac{2}{3}\e_F^{3/2}
\end{equation*}

Ahora a $F = U_0 - kT\ln Z$ dividiendo por $N$ tenemos

\begin{equation*}
    \frac{F}{N} = \frac{U_0}{N} - \frac{kT \ln Z}{N}
\end{equation*}

Reemplazando y simplificando finalmente obtenemos

\begin{equation*}
    \frac{F}{N} = 3/5
    \left[
        1 - \frac{3\pi^2}{12}{\left(
            \frac{kT}{\e_F}^2 + \cdots
        \right)}
    \right]
\end{equation*}

\solend\