$(a)$ Considere un gas de partículas libres en una caja
cúbica de lado $L$. Las partículas pueden ser Bosones,
Fermiones o partículas clásicas. Muestre que
la función densidad de estados $D(E)$, es decir el
número de niveles por unidad de energía en el rango de
$E$, $E + dE$, es 

\begin{equation*}
    D(E) = \frac{m^{3/2}}{\sqrt{2}\pi^2\hbar^3} V E ^ {1/2}
\end{equation*}

$(b)$ Utilice el resultado anterior para demostrar que
partículas de Maxwell-Boltzmann la energía media es

\begin{equation*}
    \bk{E} = \frac{3}{2} kT
\end{equation*}

\sol\ $(a)$ Sea un gas ideal cuántico en una caja de volumen
$V$ con $V = L^3$, entonces el hamiltoniano está dado por:

\begin{equation*}
    H = \frac{p^2}{2m} = - \frac{\hbar^2}{2m}\Delta^2
\end{equation*}

donde

\begin{equation*}
    \vec{\Delta} = \left(
        \frac{\partial}{\partial x},
        \frac{\partial}{\partial y},
        \frac{\partial}{\partial z},
        \right)
\end{equation*}

Utilizando la ecuación de Schrodinger tenemos la solución

\begin{equation*}
    \psi_{\vec{k}}(\vec{r})
    = \sqrt{\frac{8}{L^3}}
    \sin(k_x x)
    \sin(k_y y)
    \sin(k_z z)
\end{equation*}

Para cumplir con las condiciones iniciales se debe cumplir que

\begin{equation*}
    k_x = \frac{n_x \pi}{L},
    k_y = \frac{n_y \pi}{L},
    k_z = \frac{n_z \pi}{L},
\end{equation*}

por lo que la ecuación de onda es:

\begin{equation*}
    \psi_{\vec{k}}(\vec{r})
    = \sqrt{\frac{8}{L^3}}
    \sin(\frac{n_x \pi}{L}x)
    \sin(\frac{n_y \pi}{L}y)
    \sin(\frac{n_z \pi}{L}z)
\end{equation*}

aplicando el hamiltoniano para calcular la energía

\begin{equation*}
    H\psi_{\vec{k}}(\vec{r})
    = - \frac{\hbar^2}{2m}\vec{\nabla}^2\phi_{\vec{k}}(\vec{r})
    = \frac{\hbar^2}{2m}(k_x^2 + k_y^2 + k_z^2)\psi_{\vec{k}}(\vec{r})
    = \epsilon_{\vec{k}}\psi_{\vec{k}}(\vec{r})
\end{equation*}

\begin{equation*}
    \implies \epsilon_{\vec{k}}
    = \frac{\hbar^2}{2m}(k_x^2 + k_y^2 + k_z^2)
    = \frac{\hbar^2\pi^2}{2mL^2}(n_x^2 + n_y^2 + n_z^2)
    = \epsilon_{\vec{n}}
\end{equation*}

ahora definimos
\begin{equation*}
    k^2 = k_x^2 + k_y^2 + k_z^2 \wedge n^2 = n_x^2 + n_y^2 + n_z^2
\end{equation*}

por lo que podemos escribir de forma compacta

\begin{equation*}
    \epsilon_{\vec{k}} = \frac{\hbar^2}{2m}k^2 = 
     \frac{\hbar^2\pi^2}{2mL^2}n^2
     = \epsilon_{\vec{k}} 
\end{equation*}

Ahora calculamos la densidad de estados como una función de energía
para eso utilizamos la función delta de Dirac así tenemos

\begin{align*}
    D(E) &=
    \int_{0}^{\infty}dn_x
    \int_{0}^{\infty}dn_y
    \int_{0}^{\infty}dn_z\\
    &= \frac{1}{8}\int_{0}^{\infty}4\pi n^2 dn
    \delta\left(E - \frac{\hbar^2\pi^2}{2mL^2}n^2\right)\\
    &= \frac{\pi}{4}
    \left(
       \frac{2mL^2}{\hbar^2\pi^2}^{3/2}
    \right)
    \int_{0}^{\infty}x^{1/2 dx}\delta(E - x)\\
    &= \frac{V}{4\pi^2}{(\frac{2m}{\hbar^2})}^{3/2}E^{1/2}
\end{align*}

\begin{equation*}
    \therefore
    D(E) = \frac{m^{3/2}}{\sqrt{2}\pi^2\hbar^3}VE^{1/2}
\end{equation*}


$(b)$ Cálculo de la energía media del sistema. Primero calculamos
la función de partición

\begin{equation*}
    Z = \int_{0}^{\infty}D(E)e^{-\beta E}dE
    = \frac{m^{3/2}}{\sqrt{2}\pi^2\hbar^3}V
    \int_{0}^{\infty}E^{1/2} e^{-\beta E}dE
    = \frac{m^{3/2}}{\sqrt{2}\pi^2\hbar^3}V\frac{1}{\beta^{3/2}}\frac{\sqrt{\pi}}{2}
\end{equation*}

\begin{equation*}
    \therefore
    \bk{E} = -\frac{\partial}{\partial \beta}\ln Z = \frac{3}{2}kT
\end{equation*}

\solend\