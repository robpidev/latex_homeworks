Hallar el potencial químico a $T = 0$ (el nivel de Fermi) de un gas de fermiones de masa $m$ en términos de densidad
$n = N/V_D$ para dimensiones $D = 1, 2, 3$.
¿Qué ocurre en cada uno de estos casos con el potencial
químico al variar la temperatura, suponiendo que se mantiene fijo
el número de partículas?

\sol\

\solend\