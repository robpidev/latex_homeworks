\section{Transistor de efecto de campo semiconductor de óxido metálico}
La ecuación 10 describe el control, a través de
la tensión aplicada a la puerta, de la concentración de electrón
en la capa de inversión. Estas cargas forman una fina capa
de electrones móviles rodeados por un lado por el óxido y por
el otro por una región de carga espacial. A continuación,
la capa de carga se limita entre dos regiones aislantes. Las
cargas pueden mantenerse en movimiento si se proporcionan dos electrodos
adecuados en los bordes de la capa. Estos electrodos son dos regiones $n^+$
implantadas en los bordes de la capa de inversión. Eventualmente,
el dispositivo es una estructura de tres electrodos, donde la capa
de inversión se convierte en un canal conductor, cuya resistividad está
controlada por el voltaje aplicado al condensador MOS\@. Los electrodos en
los bordes del canal se llaman drenaje y fuente, mientras que
el electrodo aplicado al metal del MOS se llama puerta. El
electrodo metálico está separado del semiconductor de tipo p por una gruesa
capa de óxido. De esta manera, la capacitancia asociada es
tan pequeña que hace que cualquier efecto de campo sea insignificante.
La configuración contiene un cuarto electrodo aplicado al sustrato, conocido como
cuerpo. Los electrodos $n^+$ introducen dos uniones PN entre los
electrodos y la masa. Siendo $V_d, V_s$ $V_b$ los voltajes en la fuente,
drenaje y  los electrodos del cuerpo. En condiciones normales de funcionamiento,
$V_D - V_B > 0$ y $V_S 0 V_B > 0$,
de modo que ambas uniones están polarizadas inversamente y solo una corriente
insignificante fluye hacia el contacto con el cuerpo. Dicho dispositivo es
el transistor de efecto de campo semiconductor de óxido metálico (MOS
FET).

Es interesante observar que un canal conductor puede obtenerse acumulando cargas mayoritarias
o minoritarias. En ambos casos, la conductividad entre la fuente y
el drenaje puede ser regulada por el voltaje de la compuerta.
Sin embargo, el caso de inversión es mucho más interesante,
porque la corriente en el canal de inversión fluye a través de
una franja estrecha confinada entre dos aislantes. Como consecuencia, la
conductividad entre los contactos solo se debe al canal. En caso
de acumulación, tanto la fuente como el drenaje deben ser contactos
óhmicos para la mayoría de las cargas. Sin embargo, la
mayor parte del semiconductor (donde $p = N_A$) también contribuye
a la conducción entre la fuente y el drenaje. Dado que la
capa de acumulación y la masa están en paralelo, si el
semiconductor es mucho más grueso que la capa de acumulación, entonces
la conductividad siempre está dominada por la masa y el efecto de
campo no es observable. Esta configuración se está volviendo popular en
el campo de la electrónica molecular, donde se utilizan capas delgadas
de semiconductores moleculares. Otra ventaja importante de utilizar el dispositivo en
modo de inversión radica en el aislamiento intrínseco del resto del sustrato
que proporciona la capa de agotamiento
Este es un tema importante para la integración de más dispositivos en
el mismo sustrato.

\begin{figure}[ht]
    \centering
    \includegraphics[width=\textwidth]{img/f2.png}
    \caption{Configuración básica de un MOSFET}
\end{figure}

en un MOSFET, las cargas en los canales se pueden variar
a alta velocidad. Esta diferencia se debe a los pozos n+,
que actúan como reservorios de electrones, proporcionando, con gran eficiencia
todos los electrones necesarios para modular la carga en el canal.
La Figura 2 ilustra la configuración básica del MOSFET\@.
La mayoría de las características del dispositivo son el resultado de la
diferencia entre las funciones de trabajo del metal y del semiconductor 
($V_{fb}$), el dopaje del semiconductor ($\phi_p$) y el espesor
del óxido ($C_{ox}$). La combinación de estas tres cantidades define
dos categorías de dispositivos conocidos como MOSFET de mejora y agotamiento.
En un MOS-FET de mejora, en $V_G = 0$ no
se forma el canal. En particular, en un semiconductor de
tipo p, el canal se forma y luego se modula en
$V_G > 0$, mientras que el signo de $V_G$ se invierte
cuando se considera un semiconductor de tipo n. En un MOSFET
de agotamiento, por el contrario, el canal existe incluso a
$V_G = 0$, por lo que el voltaje de la puerta
puede aumentar, disminuir o incluso apagar la capa de inversión.

Cuando $V_G < V_{fb}$, la interfaz está en modo de acumulación
de modo que el camino desde la fuente hasta el drenaje forma
una serie de dos uniones PN opuestas (una configuración llamada back
to-back). En esta condición, el canal es análogo a un BJT en la región de corte,
de modo que la corriente desde la fuente
hasta el drenaje es la corriente inversa de las dos uniones PN\@.

A medida
que VG aumenta, se cumple la condición de que $V_{fb} < V_G < V_T$. En este intervalo
la región del canal se agota de las cargas móviles. Al aumentar el $V_G$, se
forma una población débil de electrones. Pero, por lo demás, esto no es la
capa de inversión. Una pequeña corriente llamada corriente sub-umbral fluye
entre la fuente y el drenaje. Finalmente, cuando $V_G > V_T$ se cumple la condición
de inversión fuerte, de modo que el canal de electrones está completamente
formado. En el modo de inversión fuerte, la capa de agotamiento permanece
constante y cualquier aumento en $V_G$ da como resultado un aumento en la concentración
de electrones del canal.

Vamos a derivar ahora la relación entre la corriente
que fluye a través del canal y el par de voltajes $V_{DS}$ (aplicados al canal)
y VG (aplicados a la puerta). En principio, $V_G$ establece la concentración
de electrones y, por lo tanto, la conductividad del canal, mientras que $V_{DS}$
impulsa la corriente de deriva. En realidad, la situación es un poco más
complicada porque el $V_{DS}$ se distribuye a lo largo del canal, de modo que
en cada punto, se encuentra un voltaje adicional $V_C(y)$ (Fig. 3). Este voltaje
es aditivo a $V_G$, por lo que afecta la densidad de carga en el canal:

\begin{equation}
    Q_n(y) = - C_{ox}(V_G - V_{fb}-\pp + V_C(y))
    + \sqrt{2\e_s q N_A (\pp + V_C(y))}
\end{equation}

\begin{figure}[ht]
    \centering
    \includegraphics[width=\textwidth]{img/f3.png}
    \caption{
        Comportamiento del VDS a lo largo del canal. Se
        establece un sistema de coordenadas $x-y$ para describir el
        comportamiento de la carga a lo largo del canal y hacia el
        bulto. Gracias al voltaje agregado al canal, la capa de
        inversión debajo del óxido no es uniforme, sino que se vuelve
        más delgada a medida que se acerca al contacto del drenaje. Por
        esta razón, incluso si las cargas móviles disminuyen, la
        concentración de carga es prácticamente constante, por lo que se
        puede despreciar la corriente de difusión. Eventualmente, por
        encima del umbral, la corriente del MOSFET se puede describir
        solo en términos de corriente de deriva
    }
\end{figure}

Por encima del voltaje umbral, se forma el canal de inversión y la
unión entre las regiones $n^+$ y el canal es despreciable. Por lo
tanto, la corriente se presta a la corriente que fluye a través de
un semiconductor dopado homogéneamente, y está hecha de un
componente de deriva y difusión. El mecanismo es análogo al del
transporte BJT de zona activa en la base, donde para disminuir el
tiempo de tránsito, se crea un campo eléctrico, ya sea debido al
dopaje graduado o a la banda prohibida graduada. En ambos casos, una
tensión limitada superior al doble de la tensión térmica (56 mV) es
suficiente para que la contribución de la corriente de deriva sea
mayor que la de difusión. Por lo tanto, es sencillo suponer que,
dado que el $V_{DS}$ es mayor que el voltaje térmico, prevalece el
componente de corriente de deriva.

Para calcular el efecto de $V_{DS}$ sobre las cargas en el canal,
supongamos que el potencial a lo largo del canal de coordenadas
(coordenada y) cambia menos rápidamente que a través del canal
(dirección de la coordenada-x):

\newcommand{\dip}[2]{\frac{\partial#1}{\partial#2}}
\let\D\Delta{}

\begin{equation}
    ||\dip{\phi}{y}|| \ll ||\dip{\phi}{x}||
\end{equation}

Esta condición es la aproximación de canal graduado. Bajo esta
aproximación, para cada $\D y$ se tiene $ \D x$ correspondiente es constante. En la
práctica, el canal se puede cortar en porciones donde la carga es
constante.

Cabe destacar que esta condición se cumple cuando la longitud del
canal ($L$) es mucho mayor que la capa de agotamiento. La consecuencia
principal de la aproximación del canal graduado es que la relación
$Q_n =-C_{ox}(V_G - V_T)$ es válida en cada intervalo $dy$.

Para calcular la distribución de $V_{DS}$, es necesario calcular el
perfil de resistencia del canal. La resistencia diferencial ($dR$) de
una pequeña porción del canal (desde $y$ hasta $dy$) es:
$d R(y) = \rho (y) dy / A$. La resistividad $\rho$ depende de la
concentración de electrones en la posición $y$:

\begin{equation}
    \rho(y) = \frac{1}{\sigma(y)} = \frac{1}{\mu_n Q(y)}
\end{equation}

donde $\mu_n$ es el canal de movilidad del electrón.
donde $\mu_n$ es el canal de movilidad electrónica. El canal se encuentra
en la interfaz entre el semiconductor y el óxido, en una región
super poblada de defectos con respecto al volumen. Por lo tanto,
podemos esperar que la movilidad en la interfaz sea menor que la de
la masa. El valor real de la movilidad, sin embargo, depende de las
condiciones en las que se cultiva el dispositivo y no se puede
calcular teóricamente. Pero de lo contrario, se da una estimación
aproximada como $\mu_n \approx 1/2 \mu_{bulk}$, donde $\mu_n$ es un
parámetro desconocido del dispositivo.



La unidad de medida para la concentración de carga en la Ec. 13
es Ccm-3, mientras que en la Ec. 10 es Ccm-2. Es fácil deducir
que, dada una porción del canal, su volumen es $Ady$ y el área debajo
del óxido es $\omega dy$, donde $\omega$ es el dimensión lateral del
canal y A es el área de la rebanada

\begin{figure}[ht]
    \centering
    \includegraphics[width=\textwidth]{img/f4.png}
    \caption{
        El volumen y las concentraciones de carga superficial
        se ilustran en un corte del canal
    }
\end{figure}

Finalmente, tenemos: $Q(y)A = Q_n(y)\omega$. La relación entre las
concentraciones volumétricas y superficiales se muestra en la Fig 4

The differential resistance of a slice dy of the channel at a y position is:

\begin{equation}
    dR = \frac{1}{\mu_n Q(y)}\frac{dy}{A}
        = \frac{dy}{\mu_n Q_n(y) \omega}
\end{equation}

La corriente $I_D$ a traves del canal es una corriente de deriva

\newcommand{\dif}[2]{\frac{d#1}{d#2}}

\begin{equation}
    I_d = \dif{V_c}{R}
\end{equation}
que es obviamente constante a lo largo del canal e independiente de $y$.

La concentración de carga en el canal viene dada por la Ec. 11
como función de $V_C$. Sin embargo, $dR$ es una función de $y$. Dado que
VC crece monótonamente a lo largo de $y$, las variables $y$ y $V_C$ se
pueden intercambiar reemplazando y por $V_C(y)$ y $Q_n(y)$ por $Q_n(V_C)$. Por
lo tanto, la ecuación 9.15 se puede integrar con respecto a y de 0 a
$L$ y con respecto a VC de VS a $V_D$:

\begin{equation}
    \omega \int_{V_S}{V_D}Q_n(V_c)d V_c
    = I_D \int_0^L \frac{dy}{\mu_n}
\end{equation}

Asumiendo que $\mu_n$ permanece constante a lo largo del canal, obtenemos
la corriente de drenado

\begin{equation}
    I_d = \frac{\mu_n \omega}{L} \int_{V_s}^{V_D} Q_n(V_c) d V_c
\end{equation}

donde
$Q_n = - C_{ox}[V_G - V_{fb} - 2\pp + V_C - V_B] + \sqrt{2\e_s q N_A (2||\phi_p) + V_c}$

El cálculo de la integral se puede simplificar sin tener en cuenta
la dependencia de la carga de la capa de agotamiento en $V_C$.
Obviamente, el tamaño de la capa de agotamiento cambia a lo largo
del canal. Sin embargo, la incertidumbre sobre el valor real de la
movilidad en el canal ($\mu_n$) hace que la aproximación sea tolerable. A
nivel práctico, lo importante aquí es calcular una relación
funcional plausible entre corrientes y voltajes, dejando la
evaluación de los parámetros reales del dispositivo a la calibración
experimental. Teniendo en cuenta la definición de voltaje umbral
Ec. 10, la carga se puede escribir como:

\begin{equation}
    Q_n = - C_{ox}(V_G - V_T - V_C)
\end{equation}

La carga móvil disminuye a medida que el $V_C$ aumenta hacia el
contacto de drenaje. Como consecuencia, el $V_{DS}$ no se distribuye
uniformemente a lo largo del canal, sino que es más intenso hacia el
contacto con el drenaje. La corriente de drenaje se obtiene
integrando la Ec .17

\begin{equation}
    I_d = - \mu_n C_{ox} \frac{\omega}{L}
    \left[
        (V_G - V_T)V_{DS} - \frac{V^2_{DS}}{2}
    \right]
\end{equation}

El signo negativo indica que la corriente fluye desde el desagüe
hasta la fuente. Esta ecuación se denomina ecuación MOSFET de canal
largo. Describe una familia de funciones parabólicas $I_d /(V_{DS})$,
donde $V_G$ es un parámetro. Cabe destacar que la forma parabólica
implica la existencia de un tracto de resistencia diferencial
negativo. En realidad, cuando $V_C = VG - VT$, la carga en el canal se
vuelve nula y el modelo, basado en la continuidad de la carga en el
canal, ya no es válido. La condición de carga nula implica que la
resistencia en ese tracto se vuelve infinita. En otras palabras, en
$V_{DS} = V_G - V_T$ se interrumpe el canal. Esta condición se cumple en la
parte superior de la función parabólica, cuando la corriente alcanza
su valor máximo.

\begin{equation}
    I_{d\max} = - \mu_n C_{ox} \frac{\omega}{L} \frac{{(V_G - V_T)}^2}{2}
\end{equation}

El voltaje de la fuente de drenaje al que se obtiene la corriente de
drenaje máxima se denomina voltaje de saturación. Corresponde a la
condición en la que Qn es nulo en el contacto de drenaje. En VDS >
$V_{DS sat}$ cualquier aumento adicional en la tensión cae en el borde de
la región, donde Qn = 0.Entonces, la parte del canal sigue siendo sesgada en
$V_{DS sat}$, de modo que la corriente en los canales permanece fija en
$(I_d\max)$. La región agotada de electrones móviles es la región de
pellizco. El pellizco es muy estrecho y está sesgado en 
$V_{DS}-V_{DSsat}$. Tal voltaje a través de una pequeña región da lugar a un
gran campo eléctrico, que puede arrastrar la corriente desde el extremo del canal hacia el
contacto de drenaje.

\begin{figure}[ht]
    \centering
    \includegraphics[width=0.5\textwidth]{img/f5.png}
    \caption{Corriente de drenaje frente a voltaje de drenaje de
    fuente con voltaje de compuerta como parámetro. El modelo es
    correcto hasta que se alcanza el valor máximo de la parábola.
    Después de eso, la conductividad de un tracto de canal cerca del
    contacto de drenaje se vuelve cero y el modelo ya no es
    válido}
\end{figure}

\begin{figure}[ht]
    \centering
    \includegraphics[width=\textwidth]{img/f6.png}
    \caption{
        Izquierda: comportamiento de la carga en el canal en
        función de la tensión de compensación. Más allá de $V_G - V_T$, el
        modelo predice una carga positiva imposible en el canal.
        Derecha: forma del canal para aumentar el VDS y la distribución
        del VDS a lo largo del canal
    }
\end{figure}

\begin{figure}[ht]
    \centering
    \includegraphics[width=\textwidth]{img/f7.png}
    \caption{
        Izquierda: Características del MOSFET\@. El desglose no es
        visible en el gráfico. Sin embargo, hay que tener en cuenta
        que en los $V_{DS}$ grandes la corriente de drenaje sufre un
        fuerte aumento debido al efecto avalancha. Derecha:
        dependencia de la corriente de saturación frente a la
        tensión de la puerta. Tenga en cuenta que el voltaje umbral
        define el inicio de la corriente
    }
\end{figure}

Para obtener la corriente total, la corriente de difusión se añade
posteriormente a la corriente de deriva:

\begin{equation}
    I_{tot} = -\mu_n \frac{\omega}{L}C_{ox}
    \left[
        \left(
            V_G - V_T + \frac{kT}{q}
        \right) + \frac{V^2_{DS}}{2}
    \right]
\end{equation}

\subsection{Modulación de longitud de canal}
Con el régimen de in-saturación de MOSFET, cada aumento adicional en
VDS cae a través de la región de pellizco. Por lo tanto, el tamaño
de la región de pellizco aumenta con VDS y, en consecuencia, la
longitud del canal de electrones disminuye. Dado que la corriente de
drenaje es inversamente proporcional a la longitud del canal, el
aumento de VDS eventualmente induce un aumento en la corriente de drenaje.

\subsection{Tiempo de Transito}
El tiempo de respuesta del MOSFET depende principalmente del tiempo
que se tarda en cargar la capa de agotamiento y del tiempo de
tránsito de los electrones en el canal. El tiempo necesario para
cargar la capa de agotamiento se trató en la sección de unión PN\@. En
cuanto al tiempo de tránsito, al ser una característica peculiar del
MOSFET, merece un poco más de atención. La velocidad de los
electrones es variable a lo largo del canal porque su concentración
es variable. De hecho, una velocidad variable asegura que la
corriente de drenaje se mantenga constante a lo largo de todo el
canal. El tiempo de tránsito se puede calcular de la siguiente
manera:

\begin{equation}
    T_{tr} = \int_{0}^{L} \frac{1}{v(y)}dy
    = - \int_{0}^{L} \frac{1}{\mu_n \E_(y)}dy
\end{equation}

haciendo los cálculos respectivos llegamos a

\begin{equation}
    T_{tr} = \frac{4}{3}\frac{L^2}{\mu_n(V_G - V_T)}
\end{equation}

El tiempo de tránsito depende en gran medida de la longitud del
canal y de la movilidad. La disminución de la longitud del canal y
el aumento de la movilidad son las posibles acciones a tomar para
reducir el tiempo de tránsito y, en consecuencia, el tiempo de
respuesta del
dispositivo.

\subsection{MOSFET de canal corto}
En la sección anterior, se ha derivado el llamado modelo de canal
largo del MOSFET\@. La condición de longitud de canal largo garantiza
la aproximación de canal graduado, que es la base para la derivación
del modelo anterior. Además, la corriente se ha calculado asumiendo
una movilidad constante en el canal. Pronto veremos que esta es una
de las condiciones que deben eliminarse cuando se considera un
MOSFET de canal corto.

La longitud del canal es inversamente
proporcional a la corriente de la fuente de drenaje y al
tiempo de tránsito. Por lo tanto, una disminución en la
longitud del canal es una opción óptima para aumentar el rendimiento
del dispositivo. La reducción de la longitud del canal, sin embargo,
tiene una serie de consecuencias adicionales, no siempre positivas,
que deben ser consideradas. Veremos que para mantener el
rendimiento, se debe establecer un escalado de la longitud del canal
en paralelo con un escalado de los otros parámetros del MOSFET\@. A
continuación se introducen algunas de las consecuencias del escalado
de la longitud del
canal.


\subsection{Modulación de voltaje de umbral}
Luego $n^+$ regiones dopadas permiten que el drenaje y los contactos de la
fuente creen una unión PN con un sustrato de tipo p. Como
consecuencia, aparecen dos capas de agotamiento adicionales,
superpuestas a la capa de agotamiento de MOS, en las interfaces
contacto-sustrato. En condiciones normales, las uniones PN están
polarizadas inversamente, de modo que las capas de agotamiento
adicionales tienden a expandirse hacia el sustrato y, también, hacia
el canal. En un dispositivo de canal largo, el tamaño de estas
regiones se puede despreciar con respecto al tamaño de la capa de
agotamiento de MOS\@. Sin embargo, cuando la distancia entre el
drenaje y la fuente es pequeña, las contribuciones de estas dos
regiones a la capa de agotamiento total ya no son despreciables.
En las capas de agotamiento pertinentes a la unión PN, la carga del
aceptor no está controlada por $V_G$, sino por $V_D$ y $V_S$\@. Como
consecuencia, la carga de la capa de agotamiento controlada por VG
disminuye. Esta carga es el último término del umbral de voltaje en (9),
de modo que si la carga de la capa de agotamiento
controlada por $V_G$ disminuye, entonces, como se muestra en la
siguiente ecuación, también el voltaje umbral se vuelve más pequeño
(con respecto a un dispositivo similar donde solo ha cambiado la
longitud del canal):

\begin{equation}
    V_T = V_{fb} - \pp + \frac{Q^*_D}{C_{ox}} 
\end{equation}

donde $Q^*_d$ es la porción de la carga de la capa de agotamiento
realmente controlada por $V_{GS}$. Dado que la fuente normalmente está
conectada a tierra, la modulación de voltaje umbral se produce en el
contacto de drenaje. $V_T$ tiene un comportamiento exponencial con la
longitud del canal. La disminución de $Q^*_d$ puede ser compensada, o
aumentada, por la capacitancia de óxido. Dejando intactos los
materiales, una reducción en el espesor del óxido puede
contrarrestar adecuadamente la disminución del voltaje umbral. Como
se anticipó en la sección anterior, para mantener las propiedades
del dispositivo, el escalado de la longitud del canal debe
complementarse con el escalado de otras cantidades. En este caso,
para mantener el mismo voltaje umbral, también se debe reducir el
espesor del óxido.