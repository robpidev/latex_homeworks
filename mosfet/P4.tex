\section{Configuración CMOS}
El MOSFET descrito en las secciones anteriores se ha fabricado en un
sustrato de tipo p, donde el canal está hecho de electrones. Por
supuesto, la configuración complementaria también se puede fabricar
en un sustrato de tipo n y un canal hecho de agujeros. Los
diferentes dispositivos se indican como n-mos y p-mos
respectivamente. Las características de p mos son opuestas a las de
los n-mos. Por ejemplo, el voltaje umbral es negativo y se necesita
un voltaje de puerta más negativo para activar el canal. En la
práctica, las condiciones de formación del canal son $V_G - V_T > 0$
para un n-mos y $V_G - V_T < 0$ para p-mos, con la consideración de que $V_T$,
en el caso de p-mos, es negativo fig. 8

\begin{figure}[ht]
    \centering
    \includegraphics[width=\textwidth]{img/f8.png}
    \caption{Esquema constructivo de la estructura CMOS}
\end{figure}

Los dos dispositivos complementarios se pueden formar sobre el mismo
sustrato, donde se crea un pozo dopado de forma opuesta a la del
sustrato. El pozo alberga el segundo dispositivo, como se muestra en
la Fig. 9 Esta oportunidad da lugar a un dispositivo compuesto
llamado CMOS (MOS complementario). El aspecto más interesante del
CMOS es el hecho de que se puede utilizar como un inversor que
genera funciones lógicas, funcionando como bloque de construcción
primario de circuitos lógicos.

\begin{figure}[ht]
    \centering
    \includegraphics[width=\textwidth]{img/f9.png}
    \caption{Inversor como generador de funciones fabricado con
    MOSFETs complementarios, y física de implementación con una
    estructura CMOS}
\end{figure}

