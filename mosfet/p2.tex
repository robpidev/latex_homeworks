\section{Modulación de la carga del canal y la tensión umbral}
Antes de discutir las propiedades de los MOSFET, es necesario evaluar
la relación entre la carga en la capa de inversión, el
canal de los MOSFET y el voltaje aplicado al MOS ($V_G$).
La discusión considera un MOS hecho de semiconductor de tipo ap sesgado
en una inversión fuerte. En esta condición, la capa de
agotamiento alcanza su máxima extensión:

\begin{equation}\label{eq:depletion:length}
    x_{d \max} = \sqrt{
        \frac{2\e_s (2||\phi_p||)}{qN_A}
    }
\end{equation}

En cuanto a la carga almacenada en la capa de agotamiento, es

\begin{equation}\label{eq:depletion:charge}
    Q_d = - \sqrt{2\e_s q N_a (2 ||\phi_p||)}
\end{equation}

Para escribir la relación entre ($V_G$) y la carga total
en el semiconductor, consideremos el perfil potencial entre la puerta y
la mayor parte del semiconductor (Figura 1).

El voltaje de banda plana ($V_G = V_{fb}$) corresponde a
una caída de potencial nula a través del semiconductor. Vfb es
negativo y corresponde a la diferencia entre las funciones de trabajo de
la puerta metálica y el semiconductor. Para calcular la carga en
el semiconductor, consideremos primero el campo eléctrico en el óxido:

\begin{figure}[h]
    \centering
    \includegraphics[width=0.6\textwidth]{img/f1.png}
    \caption{Distribution of the potential in the MOS structure in strong inversion mode}
\end{figure}

\begin{gather}
    \Eox = \f{\ox{V}}{\ox{x}} = \frac{V_G - V_B - V_{fb} - V_{si}}{\ox{x}}
\end{gather}

En la interfaz óxido-semiconductor, la continuidad del desplazamiento eléctrico
requiere que: $\e_{ox}\Eox = \e_{si}\E_{si0}$ donde $\E_{si0}$ es el campo
eléctrico en la interfaz en el lado del silicio. Reemplazando Eox
con Eq. (9.3), el campo eléctrico en
la superficie de silicio es:

\begin{equation}
    \e_{si}\E_{si0}
    = \f{e_{ox}}{x_{ox}}
    (V_G - V_B - V_{fb}- V_{si})
    = C_{ox}(V_G - V_B - V_{fb} - V_{si})
\end{equation}

Gracias al teorema de Gauss, el campo eléctrico en el semiconductor
depende de las cargas totales en el semiconductor:

\begin{equation}
    \E_{sib} - \E_{si0}
    = \f{Q_n + Q_d}{\e_s}
\end{equation}

donde $\E_{sib}$ es el campo eléctrico a granel, $Q_n$ son los
electrones móviles y $Q_d$ son los aceptores fijos. Dado que el
campo eléctrico del cuerpo es nulo, el campo en la interfaz
es $-\E_{si0} = (Q_n + Q_d)/\e_s$, a partir del
cual se calcula la carga móvil:

\begin{equation}
    Q_n = -\e_s\E_{si0} - Q_d
\end{equation}

Reemplazando el campo eléctrico en la interface en (4) y $Q_d$
con la ecuación 2 obtenemos

\begin{equation}
    Q_n = - C_{ox}\left[
        V_G - V_B - V_{fb} - 2||\phi_p||
    \right]
    + \sqrt{2\e_s q N_A (2||\phi_p||)}
\end{equation}

La carga móvil en la interfaz está hecha de electrones, entonces
la carga existe cuando $Q_n$ es negativo. Esta condición ocurre si
el primer término de la Ec. 7
es mayor que el segundo. La condición $Q_n = 0$ es
particularmente importante porque define el inicio de la formación de la capa
de inversión. El voltaje $V_G$ al que se cumple esta condición
es el voltaje umbral ($V_T$). Ajuste $Q_n = 0$ en Eq. 7 conduce a:

\newcommand{\pp}{2||\phi_p||}

\begin{equation}
    V_T = V_{fb} + 2||\phi_p|| + \frac{1}{C_{ox}}
    \sqrt{2\e_s q N_A (2||\phi_p||)}
\end{equation}

considerando que $(V_G - V_B) - V_{fb} = V_{ox} + V_{si}$ en la
inversion fuere $\pp$. Esto es $V_G = V_T$ es equivalente a

\begin{equation}
    V_T = V_{fb} + \pp + V_{ox}
\end{equation}

Comparando la Ec. 8 con la Ec
9, encontramos que el voltaje aplicado al óxido también
es responsable de la carga de la capa de agotamiento
($V_{ox}= Q_d / C_{ox}$).
Dada la definición de $V_T$, se obtiene
una expresión concisa para la carga en la capa de inversión:

\begin{equation}\label{eq:charge:inersion-layer}
    Q_n = - C_{ox}(V_G - V_T)
\end{equation}

Esta relación fundamental establece que la carga, y por lo tanto
la conductividad, del canal está controlada por el voltaje aplicado al
MOS\@. Vale la pena señalar que todas las características de MOS
contribuyen al voltaje umbral, que depende de la diferencia entre las
funciones de trabajo del metal y el semiconductor ($V_{fb}$), de
la concentración de dopaje ($\phi_p$) y del espesor del óxido
($C_{ox}$).