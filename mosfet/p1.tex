\section{Introducción}
Las uniones metal-óxido-semiconductor son el núcleo de una
familia de dispositivos de tres terminales, donde la conductividad de
la capa de inversión es modulada por el voltaje aplicado a la
estructura MOS\@. Estos dispositivos se conocen como transistores de efecto de
campo (FET). El concepto básico que subyace a la tecnología
FET se basa en la evidencia de que las cargas móviles que
se encuentran en el lado del conductor pueden acumularse o agotarse a
través de una aplicación de voltaje en una dirección ortogonal al flujo
de corriente. La idea del FET, como alternativa a los
triodos, surgió en los años veinte. La primera patente se
presentó en 1925, aunque fue la siguiente patente en 1934 la
que proporcionó una descripción más detallada del dispositivo. Posteriormente,
el concepto atrajo a W. Shockley, quien, en 1948,
desarrolló un prototipo que no funcionó correctamente debido a una densidad demasiado
alta de defectos en la interfaz óxido-semiconductor. El fracaso,
sin embargo, instó a Shockley a dirigir su atención hacia el
cruce de PN\@. El primer FET funcional se fabricó solo en
1957, en el Laboratorio Bell, utilizando silicio. Este logro
fue el resultado de la madurez de la tecnología de semiconductores,
que permitió la fabricación de una interfaz óxido-semiconductor suficientemente limpia
Por lo tanto, el FET de silicio comenzó a conocerse como
Transistor de efecto de campo semiconductor de óxido metálico (MOSFET).
Los primeros dispositivos comerciales se produjeron en el Bell Lab después de
1960. 