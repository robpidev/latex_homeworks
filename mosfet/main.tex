\documentclass[12pt, a4paper]{article}
\usepackage{amsmath}
\usepackage{amssymb}
\usepackage[utf8]{inputenc}    % Soporte para caracteres especiales
\usepackage{amsfonts}
\usepackage[right=3cm, left=2.5cm, top=3cm, bottom=3.5cm]{geometry}
\usepackage{graphicx}
%\usepackage{hyperref}          % Enlaces y referencias
%\usepackage[style=apa]{biblatex}
%\usepackage{csquotes}          % Manejo de citas (opcional, útil para biblatex)
%\addbibresource{references.bib}
\usepackage[spanish]{babel}

\newcommand{\e}{\epsilon}
\newcommand{\Eox}{\mathcal{E}_{ox}}
\newcommand{\E}{\mathcal{E}}
\newcommand{\ox}[1]{#1_{ox}}
\newcommand{\f}[2]{\frac{#1}{#2}}


\title{9. Transistores de efecto de campo}
\author{
        Torres Tarrillo, Rober E.\\
        Carcamo Calderon, Andri J.
    }

\begin{document}



\maketitle
\section{El modelo vectorial del momento angular acoplado}
El modelo vectorial de momentos angulares acoplados es un intento
de representar pictóricamente las características de los momentos
angulares acoplados que se a deducido de las relaciones de
conmutación.

Las características que deben expresar los diagramas vectoriales
de momentos acoplados son los siguientes:

\begin{itemize}
    \item La longitud del vector que representa el momento angular total
    es $\{{j(j+1)\}}^{1/2}$, con $j$ uno de los valores permitidos de
    las series de Clebsch-Gordan.

    \item Este vector debe estar en un ángulo indeterminado en un
    cono alrededor del eje $z$ (porque $j_x$ y $j_y$ no se pueden
    especificar si se ha especificado $j_z$).

    \item Las longitudes de los vectores de momento angular
    contribuyentes son ${\{j_1 (j_1 + 1)\}}^{1/2}$ y ${\{j_2 (j_2 + 1)}\}1/2$.
    Estas longitudes tienen valores definidos incluso cuando se
    especifica $j$.

    \item La proyección del momento angular total en el eje $z$ es
    $m_j$; En la imagen acoplada (en la que se especifica $j$),
    los valores de $m_{j1}$ y $m_{j2}$ son indefinidos, pero su suma es igual
    a $m_j$.

    \item En la imagen desacoplada (en la que no se especifica $j$),
    se pueden especificar los componentes individuales $m_{j1}$ y
    $m_{j2}$, y su suma es igual a $m_j$.
\end{itemize}

Las figuras 1 y 2 muestran estos puntos. La figura 1 muestra uno
de los estados de la imagen desacoplada: se especifican tanto
$m_{j1}$ como $m_j2$, pero no hay indicación de la orientación
relativa de $j1$ y $j2$, aparte del hecho de que se encuentran
sobre sus respectivos conos. POr lo tanto, el momento angular total
es indeterminado, ya que podría ser cualquiera de las resultantes
que muestran (a) o (b) o cualquier intermedia. (Nótese,
sin embargo, que la componente $z$ del momento angular) total
está bien definida ya que $m_j = m_{j1} + m_{j2}$. La figura 2
muestra uno de los estados de la imagen acoplada. Ahora bien,
la resultante, el momento angular total, tienen una magnitud
bien definida y la resultante en el eje $z$, pero las componentes
individuales $m_{j1}$ y $m_{j2}$ son indeterminadas.
el modelo vectorial es una visualización de orientaciones posibles
pero no específicos.


\begin{figure}[h!]
    \centering
    \includegraphics[width=0.46\linewidth]{img/1}
    \caption{Dos posibles estados del momento angular total que
    puede surgir de dos momentos contribuyentes especificados
    con números cuánticos $j_1$ y $j_2$. Las orientaciones relativas
    de los momentos contribuyentes en sus conos determinan
    la magnitud total.}

\end{figure}

\begin{figure}[h!]
        \centering
        \includegraphics[width=0.4\linewidth]{img/2}
        \caption{Si los dos momentos contribuyentes están
        bloqueados juntos de modo que dan lugar a un total
        especificado, las proyecciones de los momentos
        contribuyentes abarcan un rango (como se representa en las
        barras verticales) y, aunque se puede especificar
        su suma, no se puede especificar sus valores individuales.}
\end{figure}

Un ejemplo importante, es el caso de dos partículas con espín $s = 1/2$, como
dos electrones. Para cada partícula, $s = 1/2$ y $m_s = ± 1/2$.
En la imagen desacoplada, los electrones pueden estar en cualquiera
de los cuatro estados

\begin{gather*}
    \alpha_1\alpha_2 \qquad \alpha_1\beta_2 \qquad \beta_1\alpha_2
    \qquad \beta_1\beta_2
\end{gather*}

Estos cuatro estados se ilustran en la figura 3 los
momentos individuales se encuentra en posiciones no
especificadas en sus conos y el momento indeterminado.

Consideremos ahora la imagen acoplada. La condición
del triángulo (o la serie de Clebsch-Gorban) nos dice
que el espín total $S$ (Utilizaremos letras mayúsculas
para denotar los momentos angulares de las colecciones
de partículas) pueden tomar los valores 1 y 0.
Cuando $S = 0$, solo hay un valor posible
de su componente $z$, llamado 0, correspondiente
a $M_s = 0$, a este lo llamaremos singlete. Cuando
$S = 1$, $M_s = +1, 0, -1$, al cual llamaremos triplete.

\begin{figure}[h]
    \centering
    \includegraphics[width=0.4\linewidth]{img/3.png}
    \caption{Los cuatro estados desacoplados de un sistema que
    consta de dos partículas de espín $1/2$ (como electrones),
    representados por los conos en los que se encuentran
    los espines individuales}
\end{figure}

El modelo vectorial del triplete se puede ver en en la figura
4. Los conos se han dibujado a escala y varios puntos
deberían ser evidentes. Una es que para llegar a una
resultante correspondiente a $s = 1$ (de longitud 
$2^{1/2}$), utilizando vectores componentes
correspondientes a $s = 1/2$ (de longitud $1/2 \times 3^{1/2}$),
los vectores deben estar en un ángulo definido entre sí.
De hecho, deben estar en el mismo plano vertical, como
se muestra en la figura, ya que solo esa orientación
da como resultado un vector de longitud correcta. Nótese
que aunque se dice que los espines son paralelos 
en un estado de triplete (y se representa $\uparrow \uparrow$),
en realidad están en un ángulo agudo. Los
dos espines forma el mismo ángulo entre si en los tres estados;
que es necesario para que tengan el mismo resultado.

\begin{figure}[h]
    \centering
    \includegraphics[width=0.4\linewidth]{img/4.png}
    \caption{Tres de cuatro estados acoplados
    de un sistema formado por dos partículas de espín ${-1/2}$}
\end{figure}

El modelo vectorial del singlete debe representar un
estado en el que los vectores de momento angular de espín
se suman para dar una resultante cero (Figura 5).
De la figura se deduce claramente que los dos espines
son verdaderamente antiparallel $\uparrow \downarrow$
en este estado. Al igual que los estados tripletes,
solo se fija la orientación relativa de los vectores;
la orientación absoluta alrededor del eje $z$ es
completamente indeterminada.

\begin{figure}[h]
    \centering
    \includegraphics[width=0.4\linewidth]{img/5.png}
    \caption[short]{Estado resultante de dos partículas de
    espín $-1/2$. Este caso corresponde a S = 0}
\end{figure}
\section{El oscilador armónico simple}
El hamiltoniano en una dimensión está dado por
\begin{equation}
    H = \frac{1}{2}Kx^2 + \frac{p^2}{2m}
    = \frac{1}{2}Kx^2 - \frac{\hbar^2}{2m}\frac{d^2}{dx^2}
\end{equation}

donde $m$ es la masa de la partícula y $K$ la constante elástica.
Se tiene una solución no trivial a la ecuación de Schrödinger para
$n = 0,1,\ldots,\infty$ con energía.

\begin{equation}
    \label{eq:sho:energy}
    E_n = \hbar\omega\left(n + \frac{1}{2}\right)
\end{equation}

La frecuencia angular $\omega$ es identico al valor classico.
\begin{equation}
    \label{eq:angular:frecuency}
    \omega = \sqrt{\frac{K}{m}}
\end{equation}

Como la función de partición está dada por

\begin{equation*}
    Z = \sum_{n=0}^{\infty} \exp(-\beta E_n)
\end{equation*}

Con $\beta = 1/k_B T$ remplazando se tiene

\begin{equation}
   Z  = \sum_{0}^{\infty} \exp(-\beta\hbar\omega(n + 1/2))
      = \frac{\exp(-\beta\hbar\omega/2)}{1 - \exp(-\beta\hbar\omega)}
\end{equation}

Donde se a utilizado el hecho que
\begin{equation*}
    \sum_{n = 0}^{\infty} x^{-n}= \frac{1}{1 - x}
\end{equation*}

Utilizando física estadística para calcular la energía tenemos

\begin{align}
    U = - \frac{\partial}{\partial} \ln{Z}
     = \frac{1}{2}\hbar\omega + \frac{\hbar\omega}{\exp(\beta\hbar\omega) - 1}
\end{align}

Comparando con la energía media
\begin{gather}
    \langle E_n \rangle = \frac{1}{2}\hbar\omega + \hbar\omega\langle n \rangle
\end{gather}

Se puede ver que el número promedio de excitaciones del oscilador armónico es
\begin{gather}
    \langle n \rangle = \frac{1}{\exp(\beta\hbar\omega) - 1}
\end{gather}
\section{Gas de fotones}

\subsection{Cuerpo Negro}
En física, la expresión ``cuerpo negro'' se refiere a un objeto que absorbe
toda la radiación que incide sobre él y no refleja nada. Es, por supuesto,
una idealización, pero una que se puede aproximar muy bien en el laboratorio.

Un cuerpo negro no es realmente negro. Aunque no refleja la luz, puede
irradiar luz derivada de su energía térmica. Esto es, por supuesto,
necesario si el cuerpo negro ha de estar alguna vez en equilibrio térmico
con otro objeto.

Planck investigó las consecuencias de la suposición de que la luz sólo podía
aparecer en cantidades discretas dadas por la cantidad

\begin{equation}
    \label{eq:wave:energy}
    \Delta \epsilon_\omega = h\nu = \hbar\omega
\end{equation}

Dónde $\nu$ es la frecuencia, $\omega = 2\pi\nu$, $h$ constante de Planck
y $\hbar = h/2\pi$

\subsection{Modelo de solución}
Consideraremos una cavidad cúbica con dimensiones $L\times L\times L$.
Los lados están hechos de metal y contiene radiación electromagnética,
pero no importa. La radiación solo puede entrar y salir de la cavidad a
través de un orificio muy pequeño en un lado. Dado que la radiación debe
ser reflejada por las paredes muchas veces antes de regresar al agujero,
podemos suponer que ha sido absorbida en el camino, haciendo de este
objeto, o al menos del agujero, un cuerpo negro. Lo único que hay dentro
de la cavidad es radiación electromagnética a temperatura T.

Deseamos encontrar el espectro de frecuencia de la energía almacenada
en esa radiación, que también nos dará el espectro de frecuencia de la luz
emitida por el agujero.

\subsection{Dos tipos de cuantización}
Al analizar el modelo simple descrito en la sección anterior,
debemos ser conscientes de los dos tipos de cuantificación que
entran en el problema. Como resultado de las condiciones de contorno
debidas a las paredes metálicas del contenedor, se cuantifican las
frecuencias de las ondas estacionarias permitidas. Se trata de un efecto
totalmente clásico, similar a la cuantización de la frecuencia en las
vibraciones de una cuerda de guitarra.

La segunda forma de cuantización se debe a la mecánica cuántica,
que especifica que la energíaalmacenada en una onda electromagnética
con frecuencia angular $\omega$ viene en múltiplos de $\hbar \omega$
(generalmente llamados
fotones). La teoría de la electrodinámica nos da la ecuación de onda
para el campo eléctrico $\vec{E}(\vec{r})$ esta dado por

\begin{equation}
    \nabla^2 \vec E (\vec{r}, t)
    = \frac{1}{c^2} \partial_{t}^2 \vec{E}(\vec{r}, t)
\end{equation}

donde $c$ es la velocidad de la luz y $\partial_q^n = \partial^n/\partial_{q^n}$

Tomando en cuanta que

\begin{equation}
    \vec \nabla \equiv
    \left(\partial_x, \partial_y, \partial_z \right)
\end{equation}

Entonces

\begin{equation}
    \nabla^2 = \vec\nabla\cdot \vec\nabla
    = \partial_x^2 + \partial_y^2 + \partial_z^2
\end{equation}

Usando la simetría del modelo podemos encontrar la sigueinte solución

\begin{gather}
    \label{eq:electric:field:solution}
    E_x(\vec{r}, t) = E_{x,0}\sin(\omega t) \cos(k_x x) \sin(k_y y) \sin(k_z z)\\
    E_y(\vec{r}, t) = E_{y,0}\sin(\omega t) \sin(k_x x) \cos(k_y y) \sin(k_z z)\\
    E_z(\vec{r}, t) = E_{z,0}\sin(\omega t) \sin(k_x x) \cos(k_y y) \cos(k_z z)
\end{gather}

Dónde $E_{x_i, 0}$, son las aplitudes del campo electromagnético en la cavidad
correspondiente.

Por las condiciones de contotorno las ecuaciones anteriores las ecuaciones
~\ref{eq:electric:field:solution} a 11 se tiene que

\begin{gather}
    k_x L = n_x\pi\\
    k_y L = n_y\pi\\
    k_z L = n_z\pi
\end{gather}

Donde $n_x, n_y, n_z \in \mathbf{Z}^+$, ya que para los negativos tienen la
misma solución.

Utilizando la relación entre la frecuencia y número de onda
\begin{equation}
    k_x^2 + k_y^2 + k_z^2 = \frac{\omega^2}{c^2}
\end{equation}

Y remplazando las ecuaciones de 12 a 14

\begin{equation}
   {\left(\frac{n_x \pi}{L}\right)}^2
   + {\left(\frac{n_y \pi}{L}\right)}^2
   + {\left(\frac{n_z \pi}{L}\right)}^2
   = \frac{\omega^2}{c^2}
\end{equation}

tomando en centa la $\vec{n}$-dependencia para $\omega$ podemos escribir

\begin{equation}
    \omega_{\vec{n}}
    = (n_x^2 + n_y^2 + n_z^2){\left(\frac{\pi c}{L}\right)}^2
\end{equation}

Por lo que obtiene

\begin{equation}
    \omega_{\vec{n}}
    = \frac{\pi c}{L} \sqrt{n_x^2 + n_y^2 + n_z^2}
    = \frac{n \pi c}{L} = \omega_n
\end{equation}

Donde $n = |{\vec{n}}|$

\subsection{Espectro de energía del cuerpo negro}
El primer paso para calcular el espectro de energía del cuerpo negro es
encontrar la densidad de estados. A partir de las soluciones a la ecuación
de onda en la Sección anterior, expresamos los modos individuales en términos
de los vectores $\vec{n}$. Nótese que la densidad de puntos en el espacio 
$\vec{n}$ es uno, ya que los componentes de $\vec{n}$ son todos enteros.

\begin{equation}
    P_{\vec{n}}(\vec{n}) = 1
\end{equation}

Par encontrar la dencidad de estados $P_\omega(\omega)$,
solo se necesita integrar la ecuación anterior en el $\vec{n}$-espacio

\begin{equation}
    P_\omega(\omega) = 1 \frac{1}{8}
    \int_{0}^{\infty} 4\pi n^2 dn \delta(\omega - nc\pi / L) 
    = \pi {\left(\frac{L}{c\pi}\right)}^3 \omega^2
\end{equation}

El factor de 2 es para las dos polarizaciones de la radiación electromagnética,
y el factor de 1/8 corrige para contar los valores positivos y negativos
de los componentes de n. Dado que cada fotón con frecuencia $\omega$ tiene
energía $\hbar\omega$, la energía promedio se puede encontrar sumando todos los
números de fotones ponderados por el factor de Boltzmann $\exp(-\beta \omega \hbar)$.
Dado que esta suma es formalmente idéntica a la del oscilador armónico simple, podemos
escribir la respuesta.

\begin{equation}
    \langle \epsilon_\omega \rangle
    = \frac{\hbar\omega}{\exp(\beta\hbar\omega) - 1}
\end{equation}

Asi con las ecuaciones anteriores se tiene la dencidad de energía

\begin{equation}
    u_\omega = \frac{1}{V} \pi {\frac{L}{c\pi}}^3
    \omega^2 \frac{\hbar\omega}{\exp(\beta\hbar\omega) - 1}
    = \frac{\hbar}{\pi^2c^3}\omega^3(\exp(\beta\hbar\omega) - 1)^{-1}
\end{equation}

\subsection{Energía Total}
La energía mecánica cuántica total en la radiación de la cavidad a la temperatura $T$ se da sumando la energía media en cada modo.

\begin{equation}
    U = 2 \sum_{\vec{n}} \langle \epsilon_{\vec{n}} \rangle
    = 2 \sum_{\vec{n}} \hbar\omega_{\vec{n}}
    \frac{1}{\exp(\beta\hbar\omega) - 1}
\end{equation}

Esta ecuación se restringe al espacio octantante positivo para evitar
la doble contabilidad, y el factor de dos explicaciones para la doble
polarización asociada con todos los modos espaciales.
Volvemos a utilizar el hecho de que el espectro de frecuencias es
continuo para escribir la suma en una integral, que podemos evaluar
explícitamente utilizando la densidad de estados, $P_\omega(\omega)$, Encontrada

\begin{align}
    U &= \int_{0}^{\infty}\langle \epsilon\omega \rangle  P_\omega(\omega) d\omega\\
    &= \pi {\left(\frac{L}{c\pi}\right)}^3
    \int_{0}^{\infty}\langle \epsilon_\omega \rangle \omega^3 d\omega\\
    &=\pi {\left(\frac{L}{c\pi}\right)}^3
    \int_{0}^{\infty} \frac{\hbar\omega}{exp(\beta\hbar\omega - 1)} \omega^2 d\omega
\end{align}

Tomando $x = \beta \hbar \omega$

\begin{equation}
    U = \pi \beta^{-1}{\left(\frac{L}{\beta\hbar\pi c}\right)}
    \int_{0}^{\infty}dx \frac{x^3}{e^x - 1}
\end{equation}

Esta ultima integral es conocida y es $\pi^4/15$

Por lo tanto la energía está dada por

\begin{equation}
    U = u {\left(\frac{\pi^2}{15\hbar^3c^3}\right)}V^3{\left(k_B T\right)}^4
\end{equation}

\section{Conclusiones}
Quizás la ocurrencia más famosa de radiación de cuerpo negro es en la
radiación de fondo del universo, que fue descubierta en 1964 por Arno
Penzias (físico alemán que se convirtió en ciudadano estadounidense, 1933-,
Premio Nobel 1978) y Robert Wilson (astrónomo estadounidense, 1936-, Premio
Nobel 1978). Poco después del Big Bang, el universo contenía radiación
electromagnética a una temperatura muy alta. Con la expansión del universo,
el gas de los fotones se enfrió, de la misma manera que un gas de partículas
se enfriaría a medida que aumentara el tamaño del recipiente. Las mediciones
actuales muestran que la radiación de fondo del universo se describe a una temperatura de 2.725K.

Con esta ultima ecuación se puede calcular el resto de los potenciales termodinámicos
segun sea requerido.
\section{Ley de circuitos de Ampére}
Los campos deducidos anteriormente satisfacen la siguiente ecuación
\begin{equation}
    \label{corriente:densidad:eq}
    \mathbf{\nabla} \cdot \mathbf{J} = 0
\end{equation}

Por lo que se puede se puede deducir una ecuación importante para el rotacional
de la ecuación~\ref{campo:magnetico:densidad}

\begin{gather*}
    \mathbf{\nabla_2\times B(r_2)}
    =\frac{\mu_0}{4\pi}
    \int_V\frac{\mathbf{J(r_1)\times\nabla_2(r_2 - r_1)}}{\mathbf{|r_2 - r_1|}^\text{3}}
    dv_1
\end{gather*}

Ahora utilizando
$\mathbf{A\times\nabla\times B = \nabla_B(A\cdot B) - (A\cdot\nabla)\times B}$,
y como $\nabla_2$ solo afecta a $\mathbf{r_2}$

\begin{gather*}
    \mathbf{\nabla_2\times B(r_2)}
    =  \frac{\mu_0}{4\pi}\int_V\left[
        \mathbf{
            J(r_1)\left(
                \nabla_2\cdot\frac{r_2 - r_1}{|r_2-r_1|^\text{3}}
            \right)
            -
            J(r_1)\cdot\nabla_2\frac{r_2-r_1}{|r_2-r_1|^\text{3}}
        }
    \right]dv_1
\end{gather*}

El primer término se puede expresar en función de la delta de Dirac,
el segundo término intercambiando los vectores y por simetría obtenemos

\begin{gather*}
    \mathbf{\nabla_2\times B(r_2)}
    =  \frac{\mu_0}{4\pi}\int_V\left[
        \mathbf{J(r_1)}4\pi\delta(\mathbf{r_2 - r_1})
        -
        \mathbf{J(r_1)\cdot\nabla_1\frac{\mathbf{r_1 - r_2}}{\mathbf{|r_1-r_2|^3}}}
    \right]
    dv_1
\end{gather*}

Al integrar el primer término obtenemos $\mu_0\mathbf{J(r_2)}$,
para el segundo término podemos usar el hecho que

\begin{gather*}
    \nabla_1\cdot\left(
        \mathbf{J(r_1)}\frac{x_1 - x_2}{|\mathbf{r_1 - r_2}|^3}
    \right)
    = \frac{x_1-x_2}{\mathbf{|r_1 - r_2|}^3}\nabla_1\cdot\mathbf{J}
    +\mathbf{J}\cdot\nabla_1\frac{x_1-x_2}{|\mathbf{r_1 - r_2}|^3}
\end{gather*}

de la misma forma par $y$ y $z$, al sumar estos términos se anula, esta es la
forma diferencial de la ley de Ampére.

\begin{equation}
    \label{ley:apere:diferencial}
    \mathbf{\nabla_1\times B(r_2)} = \mu_0\mathbf{J(r_2)}
\end{equation}

Integrando sobre una superficie $S$ tenemos

\begin{gather*}
    \int_S\mathbf{\nabla_1\times B(r_2)}\cdot d\mathbf{S} = \int_S\mu_0\mathbf{J(r_2)}\cdot d\mathbf{S}
\end{gather*}

Utilizando el teorema de Stokes tenemos
\begin{equation}
    \label{ley:apere:integral}
    \oint_c \mathbf{B}\cdot d\mathbf{l} = \mu_0 \int_S \mathbf{J}\cdot d\mathbf{S}
\end{equation}

esta es la ley de Ampére la cual es análoga a la ley de Gauss, no tiene significado
físico como tal, pero no ayuda a calcular los campos magnéticos.


\section{Referencias}

Reitsz, J., Milford, F. y Chisty, R. (Sin fecha). "Fundamentos de la teoría electromagnética", 4ta ed. Addison-Wesley Iberoamericana.

Young, H. y Freedman, A. (2013). Física universitaria con física moderna(13ava Ed, Vol. 2). México: PEARSON.

Wangsness, R. (2001). "Campos electromagnéticos". México: Limusa.



\section{Referencias}
\begin{itemize}
    \item Di Natale, C. (2023). Introduction to Electronic Devices. Springer.
    \item Veritasium en español. (2025, febrero 7). Por qué es casi imposible hacer luz LED azul [Video]. YouTube. https://youtu.be/KzTm5UmF0Xk
    \item Mentalidad de Ingeniería. (2025, febrero 10). Cómo funciona un MOSFET - Guía completa, entiéndelo como un PRO [Video]. YouTube. https://www.youtube.com/watch?v=4Dnx5YR9k-8
\end{itemize}

\end{document}